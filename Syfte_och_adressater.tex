\section{Syfte och adressater}
\label{sec:syfte}

% Detta inledande avsnitt introducerar adressaterna och det konstitutionella syftet.

Detta dokument är en formell framställan riktad till:

\begin{itemize}
  \item Justitiedepartementet och justitieministern,
  \item Utrikesdepartementet och utrikesministern,
  \item Statsrådsberedningen, såsom samordnande instans för regeringens politik,
  \item Justitieombudsmannen (JO),
  \item Konstitutionsutskottet (KU),
  \item Högsta domstolen (HD).
\end{itemize}

\begin{spacing}{1.2}
Syftet är att formellt uppmärksamma och rättsligt analysera den svenska regeringens agerande i relation till dess skyldigheter enligt:

\begin{itemize}
  \item \textit{Regeringsformen (RF) 1 kap. 10 § – om respekt för internationella åtaganden,}
  \item \textit{FN-stadgan – särskilt artikel 2 om våldsförbud och artikel 1 om fredliga syften,}
  \item \textit{Konventionen om förebyggande och bestraffning av brottet folkmord (1948).}
\end{itemize}

\vspace{0.5cm}
\textit{
Framställan aktualiserar frågan om statligt ansvar för passivitet och medverkan i folkrättsbrott – särskilt i ljuset av pågående brott mot mänskligheten i Gaza och regeringens val att avstå från tydlig rättslig och moralisk positionering.
}

\vspace{0.5cm}
\textit{
Eftersom Konstitutionsutskottets uppgift är att pröva huruvida regeringen eller enskilda statsråd har brutit mot grundlagen, och eftersom en sådan prövning ytterst kan kräva rättslig prövning av Högsta domstolen enligt 13 kap. 3 § regeringsformen, tillställs denna skrivelse även Högsta domstolen.
}

\vspace{0.5cm}
\textit{
Denna framställan bör därför behandlas som en konstitutionell angelägenhet av särskild vikt.
}
\end{spacing}

