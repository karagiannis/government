





\section{Begärda åtgärder}
% Lista konkreta krav: diplomatiska sanktioner, avbryt handel, rättsutredning, KU-granskning.
% Visa att dessa krav ligger inom ramen för Sveriges skyldigheter – inte önskemål.

%\section*{Regeringen har inga befogenheter att formulera egna rättsprinciper}
%\addcontentsline{toc}{section}{Regeringen har inga befogenheter att formulera egna rättsprinciper}

Det är djupt oroande att Sveriges regering i sitt officiella språkbruk tycks föreställa sig att den kan ”balansera” folkrätt, militär ockupation och systematiskt våld med egna moraliska avvägningar – som om Utrikesdepartementet vore en gymnasieklass i etik.

Men Sveriges regering är inte en moralfilosofisk tankesmedja. Den har inga befogenheter att formulera egna rättsprinciper – än mindre att ersätta gällande internationell rätt med känslomässiga symmetrier.

Utrikespolitiken är inte ett tyckande. Den är – eller borde vara – bunden av:

\begin{itemize}
  \item \textbf{Regeringsformen (RF) 1 kap. 10 §}, där det uttryckligen anges att Sverige ska respektera sina internationella åtaganden,
  \item \textbf{FN-stadgan}, inklusive förbudet mot anfallskrig och kollektiv bestraffning,
  \item \textbf{Internationella domstolens prejudikat}, som fastslår att Israel är ockupationsmakt – inte ett offer för yttre aggression.
\end{itemize}

Att i detta rättsläge hävda att Israel har ”rätt till självförsvar” gentemot en befolkning det själv ockuperar är inte bara juridiskt ohållbart – det är ett kompetensöverskridande.

Internationell rätt är inte ”vår tolkning”. Det är inte ett perspektiv bland andra. Det är ett förpliktande ramverk. Om regeringen börjar ersätta rätt med moral, har vi i praktiken avskaffat rättsstaten.

\vspace{1em}
\textit{Makten må utgå från folket – men rättsstatens giltighet gör det inte.}

Även om folkopinionen formats av desinformation, psykologisk krigsföring och inverterad verklighetsbeskrivning, får regeringen inte ta sin legitimitet ur folkets affekt. Den får det ur konstitutionen – och ur Sveriges internationella rättsliga åtaganden.

Inte ens om regeringen, på formellt laglig väg, skulle välja att utträda ur folkmordskonventionen och andra traktat, kvarstår det rättsliga faktum: dessa normer utgör \textit{jus cogens} – tvingande internationell rätt – och kan inte avtalas bort.

Om regeringen försöker, innebär det inte att rättsordningen förändrats. Det innebär att rättsstaten har övergetts.

Och när så sker – vilket humanistiska och juridiska eliter världen över förstår – uppstår ett nytt ansvar: att stoppa rättens sammanbrott.

\vspace{1em}
Regeringen kommer då att kalla dessa försök till försvar för ”extremism” eller ”terrorism” – men det är förväntat. Så reagerar alltid den makt som tappat sin juridiska legitimitet.






\section*{Statens skyldigheter enligt folkrätten}
\addcontentsline{toc}{section}{Statens skyldigheter enligt folkrätten}
\subsection*{Folkmordskonventionen väger tyngre än lojalitet mot EU}
\addcontentsline{toc}{subsection}{Folkmordskonventionen väger tyngre än lojalitet mot EU}

Att Sveriges regering hänvisar till framtida domstolsprövningar som ursäkt för sin passivitet är inte en neutral hållning – det är ett moraliskt svek.

Enligt artikel I i FN:s konvention om förebyggande och bestraffning av brottet folkmord (1948) är varje fördragsslutande stat skyldig att \textbf{förhindra} och \textbf{bestraffa} folkmord – oavsett om någon internationell domstol ännu fällt en dom.\footnote{\url{https://www.ohchr.org/en/instruments-mechanisms/instruments/convention-prevention-and-punishment-crime-genocide}}

Regeringen låtsas att inte se helheten. Den påstår att åtgärder måste ske ”tillsammans med andra EU-länder och likasinnade”, som om Sveriges rättsstatliga skyldigheter vore beroende av konsensus inom en politisk union. Men Sveriges folkrättsliga åtaganden står över lojaliteten till EU:s utrikespolitiska linje. Folkmordskonventionen utgör bindande internationell rätt – inte ett politiskt vägval.

Om regeringen föreställer sig att dess väljarmandat – eller dess alliansstrategi – ger rätt att ignorera folkmordskonventionen, har den allvarligt missförstått grundlagen.  
Regeringsformens portalparagraf (1 kap. 2 §) slår fast att Sverige ska respektera folkrätten. Detta gäller särskilt skyldigheten att agera preventivt vid risk för folkmord.

\lagrum{1 kap. 2 § 1 st. RF\quad Den offentliga makten ska utövas med respekt för alla människors lika värde och för den enskilda människans frihet och värdighet.}

Skyldigheten att respektera mänskliga rättigheter har även en grund i Europakonventionen, som har ställning som svensk lag.

\lagrum{2 kap. 19 § RF\quad Lag eller annan föreskrift får inte meddelas i strid med Sveriges åtaganden på grund av den europeiska konventionen angående skydd för de mänskliga rättigheterna och de grundläggande friheterna.}

\lagrum{1 § Lag (1994:1219)\quad Som lag här i landet ska gälla den europeiska konventionen den 4 november 1950 angående skydd för de mänskliga rättigheterna och de grundläggande friheterna – med de ändringar som gjorts genom ändringsprotokollen nr 11, 14 och 15 till konventionen, och med de tillägg som gjorts genom tilläggsprotokollen nr 1, 4, 6, 7, 13 och 16 till konventionen.}
Detta gäller särskilt skyldigheten att agera preventivt vid risk för folkmord.



\subsection*{Om folkrättens selektiva tillämpning och kritisk rättsteori}
\addcontentsline{toc}{subsection}{Om folkrättens selektiva tillämpning och kritisk rättsteori}

Folkrättens auktoritet vilar på dess anspråk på universell och likvärdig tillämpning. I teorin är dess normer opartiska. I praktiken är tillämpningen ofta asymmetrisk.

Modern forskning inom kritisk rättsteori har visat hur internationell rätt tenderar att tillämpas olika beroende på staters maktposition, allianser och – inte sällan – geopolitisk eller rasifierad identitet.  

Detta har beskrivits som en form av “rasialiserad folkrättstillämpning”\footnote{Se t.ex. Antony Anghie, \textit{Imperialism, Sovereignty and the Making of International Law}, Cambridge University Press, 2004.}, där vissa staters rätt till självförsvar eller territoriell kontroll aldrig ifrågasätts, medan andra stater och folk kontinuerligt misstänkliggörs. Koloniala mönster reproduceras i nutida diplomati och rättstillämpning – inte genom explicita hierarkier, utan genom selektiv tolerans mot folkrättsbrott beroende på avsändare.

I detta ljus blir Sveriges selektiva agerande i fallet Israel-Palestina svårt att försvara.

Samma regering som varit pådrivande i sanktioner mot Rysslands aggression mot Ukraina och i fördömanden av andra auktoritära stater, har valt att uttrycka sig med största försiktighet när det gäller en mångårig ockupationsmakt som är militärt och politiskt allierad med väst. 

Detta riskerar inte bara att skada Sveriges internationella anseende – det bidrar till att normalisera ett rättssystem där mänskliga rättigheter blir beroende av geostrategiska överväganden. En sådan ordning är oförenlig med den rättsstatliga principen om allas likhet inför lagen.



\vspace{1em}

Denna asymmetriska tillämpning är inte ett akademiskt problem – den får direkta konsekvenser när stater som Sverige avstår från att ingripa trots övertygande tecken på pågående folkrättsbrott.\footnote{Utöver Anghie har även forskare som Makau Mutua, Martti Koskenniemi och Sundhya Pahuja bidragit till förståelsen av hur internationell rätt används selektivt – ofta till förmån för globala maktcentra.}





\subsection*{Att vägra ingripa är inte neutralitet – det är medskyldighet}
\addcontentsline{toc}{subsection}{Att vägra ingripa är inte neutralitet – det är medskyldighet}

Det är därför inte möjligt att undandra sig konventionsansvaret genom att hänvisa till att ”dessa budskap framförs i våra egna kontakter med Israel och vi gör det tillsammans med andra EU-länder och likasinnade”.  
Det vore som om en polis, trots att han bevittnar ett pågående övergrepp, väljer att inte ingripa – med ursäkten att det vore ”opassande” att stöta sig med gärningsmannen, eftersom de tillhör samma yrkeskår.

På motsvarande sätt väljer Sveriges regering att inte använda sina statliga maktmedel – såsom ekonomiska sanktioner, vapenembargo eller diplomatiska markeringar – trots folkrättsbrott av högsta allvarlighetsgrad.

Detta är inte återhållsamhet. Det är \textbf{medskyldighet}.

Enligt artikel I i folkmordskonventionen\footnote{\url{https://www.ohchr.org/en/instruments-mechanisms/instruments/convention-prevention-and-punishment-crime-genocide}} är Sverige folkrättsligt skyldigt att inte bara bestraffa, utan även \textit{förhindra} folkmord – oavsett om någon dom ännu har fallit. Detta gäller i synnerhet när avsikten (dolus specialis) är öppet uttalad och inte hypotetisk.

Att regeringen trots detta avstår från handling får allvarliga konsekvenser:

\begin{itemize}
  \item Det underminerar den internationella rättsordningen och respekten för bindande konventioner.
  \item Det smutsar ner Sveriges rykte som humanitär rättsstat – särskilt i ögonen på det Globala Syd.
  \item Det skadar medborgarnas tilltro till demokratins möjligheter och till samhällets moraliska kompass.
  \item Det göder konspirationsteorier om internationell maktutövning – till exempel föreställningar om inflytande från mäktiga judiska nätverk.
  \item Det skapar en grogrund för radikalisering, polarisering och social fragmentering.
\end{itemize}

Om denna regering vill försvaga demokratin, urholka samhällsgemenskapen och dra Sveriges internationella anseende i smutsen – då är den inslagna vägen utan tvekan den rätta.






\subsection*{Från moralisk förebild till passiv möjliggörare – ekonomiska implikationer}
\addcontentsline{toc}{subsection}{Från moralisk förebild till passiv möjliggörare – ekonomiska implikationer}

Sverige vann under 1970- och 80-talen stor respekt i det Globala Syd genom sin kompromisslösa hållning mot apartheidregimen i Sydafrika. Genom att tydligt stödja ANC, kräva sanktioner och motsätta sig USA:s och Storbritanniens eftergiftspolitik, sågs Sverige som en röst för rättvisa, jämlikhet och folkrätt.

Detta moraliska kapital öppnade dörrar till affärsmöjligheter, biståndssamarbeten och diplomatiskt inflytande. Den svenska hållningen uppfattades som trovärdig – därför att vi var beredda att betala ett politiskt och ekonomiskt pris för våra principer.

I dag är situationen den omvända.

Varför skulle stater i Afrika, Asien eller Latinamerika vilja handla med Sverige – om vi uppfattas som en \textit{folkmordets medlöpare}?  
Om vi varken vågar använda våra maktmedel – eller ens vår röst – när ett folk systematiskt utplånas med uttalad avsikt, vad är då kvar av vår trovärdighet?

Varför skulle någon i det Globala Syd vilja samarbeta med ett europeiskt land som:

\begin{itemize}
  \item svikit den internationella rättsordningen,
  \item underminerat FN-systemets auktoritet, och
  \item rullat tillbaka de normer som syftade till att bygga en \textit{civiliserad} värld?
\end{itemize}

När kvalitet och pris är likvärdigt – varför välja Sverige, när Kina eller Brasilien erbjuder samma villkor utan moralisk dubbelmoral?

\textit{"Ni har redan visat oss att principer inte betyder något. Varför skulle vi då gynna er?"}



\section*{Krav på åtgärder}
\addcontentsline{toc}{section}{Krav på åtgärder}
\subsection*{Sanktioner mot Israel måste spegla Sveriges praxis mot andra stater}
\addcontentsline{toc}{subsection}{Sanktioner mot Israel måste spegla Sveriges praxis mot andra stater}

Om Sveriges regering menar allvar med sitt åtagande att försvara mänskliga rättigheter, krävs mer än halvhjärtade markeringar mot enskilda israeliska ministrar. Det räcker inte att – inom ramen för EU-samarbetet – diskutera selektiva sanktioner, samtidigt som man protesterar mot Internationella brottmålsdomstolens arresteringsorder mot Israels premiärminister.

Att motverka folkmord kräver samma kompromisslösa hållning som Sverige tidigare intog mot apartheidregimen i Sydafrika – och som i dag riktas mot Rysslands aggression mot Ukraina.

Om regeringen vill signalera att palestinska liv har samma människovärde som ukrainska, måste följande åtgärder vidtas utan dröjsmål:

\begin{itemize}
  \item Inför fullskaliga sanktioner mot Israel – inklusive vapenembargo, import- och exportförbud samt frysning av tillgångar för ansvariga befattningshavare.
  \item Avbryt statliga besök, samarbetsavtal och militära utbyten med Israel tills folkrätten respekteras.
  \item Förbjud israelisk statspropaganda i Sverige – inklusive sändningstillstånd, riktade kampanjer i sociala medier och pressutspel från IDF – på samma sätt som rysk propaganda via RT förbjöds.
  \item Stoppa legitimeringen av Israels narrativ genom krav på opartiskhet i public service. Sveriges Radio och andra statsunderstödda medier får inte okritiskt återge IDF:s kommunikéer som om de vore oberoende källor.
\end{itemize}

Regeringen kan inte hävda att den värnar rättsstat och mänskliga rättigheter om den samtidigt tillämpar dubbla måttstockar.

Det är inte neutralitet – det är hyckleri.\\
Det är inte återhållsamhet – det är moralisk feghet.\\
Det är inte diplomati – det är \textbf{tyst medverkan}.

Sverige ska inte dalta med en ockupationsmakt som kränker folkrätten – lika lite som vi gjorde det med apartheidregimen i Sydafrika eller Rysslands invasion av Ukraina.

\textbf{Att erkänna palestiniers människovärde kräver handling – inte retorik.}


Underlåtenhet att vidta dessa åtgärder riskerar att tolkas som konstitutionell underlåtenhet och som passiv medverkan i folkrättsbrott – med följd att regeringen kan komma att hållas ansvarig inför såväl riksdagens kontrollorgan som internationella rättsliga forum.

