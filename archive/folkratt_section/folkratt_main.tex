%filnamn:folkratt_main.tex
% Genomgång av Sveriges folkrättsliga åtaganden och konventionsbundenhet

\section{Folkrättens bindande ramverk}

% Sammanfatta relevanta rättskällor: FN-stadgan, Folkmordskonventionen, Genèvekonventionerna, sedvanerätt.
% Förklara varför folkrätten inte kan kringgås av politiska hänsyn.
% Avsluta med Sveriges konstitutionella skyldigheter enligt Regeringsformen 10:1 och 13:3.


Regeringen skriver i sin kommuniké från den 27 maj 2025:\footnote{\url{https://www.government.se/statements/2025/05/statement-from-the-ministry-for-foreign-affairs-on-summoning-of-israeli-ambassador/}}

\begin{quote}
\textit{“When the Ambassador was summoned, it was stressed that Israel has the right to defend itself. However, that right must be exercised in accordance with international law.”}
\end{quote}

Vid en första anblick framstår detta som ett balanserat uttalande. Men den retoriska konstruktionen döljer en djupare förskjutning. Vad som presenteras som ett villkorat erkännande av folkrätten, fungerar i själva verket som en rättslig täckmantel för folkrättsbrott.

FN-stadgan artikel 51\footnote{\url{https://www.un.org/en/about-us/un-charter/full-text}} ger endast rätt till självförsvar vid ett väpnat angrepp – riktat mot en suverän medlemsstat. Rätten gäller alltså inte en ockupationsmakt som utövar kontroll över ett territorium och dess civilbefolkning.

\lagrum{Artikel 51, FN-stadgan\quad Nothing in the present Charter shall impair the inherent right of individual or collective self-defence if an armed attack occurs against a Member of the United Nations...}

Att hävda att Israel har rätt att försvara sig mot Gaza är därför inte enbart en politisk förenkling – det är en rättslig förfalskning av folkrättens grundprinciper. Det ger det strukturella våldet en rättslig fernissa, trots att folkrätten bygger på ansvar – inte på retoriska undantag.

Internationella domstolen (ICJ) fastslog den 19 juli 2024 att Israels ockupation av Gaza, Västbanken och östra Jerusalem är olaglig, och att den utgör apartheid.\footnote{\url{https://mondoweiss.net/2024/07/in-a-historic-ruling-icj-declares-israeli-occupation-unlawful-calls-for-settlements-to-be-evacuated-and-for-palestinian-reparations/}} Detta är inte en tolkning. Det är ett rättsligt bindande konstaterande.

Trots detta fortsätter den svenska regeringen att referera till Israels \enquote{rätt till självförsvar} som om Gaza vore en angripande stat, inte ett ockuperat territorium. Uttalandet neutraliserar den folkrättsliga kontexten och bekräftar därigenom en rättsvidrig världsbild.

Därmed gör sig regeringen skyldig till mer än en vilseledande formulering. Den medverkar – om än indirekt – till att undergräva den internationella rättsordningen. Och detta sker inte av juridisk okunskap, utan som del av en förutsägbar, politiskt motiverad ordkonstruktion:

\begin{quote}
Ett sätt att säga något – men samtidigt inte mena det.  
Detta är inte ansvarsutkrävande. Det är ett medvetet undandragande av ansvar, maskerat som neutral diplomati.
\end{quote}

\medskip

\textbf{Frågan är därför inte om Israel har rätt till självförsvar – utan var den rätten får utövas.}

För att bedöma giltigheten i regeringens påstående krävs en tydlig förståelse av folkrättens territoriella begränsning: självförsvar enligt FN-stadgan artikel 51 får endast utövas inom en stats eget internationellt erkända territorium. Ockupationsmaktens rättigheter är däremot reglerade av helt andra folkrättsliga instrument.


