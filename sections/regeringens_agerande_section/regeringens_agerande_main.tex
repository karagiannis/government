%filnamn:regeringens_agerande_main.tex
% Beskrivning av regeringens faktiska agerande eller passivitet

\section{Sveriges folkrättsliga ansvar}
% Redogör för Sveriges positiva skyldigheter enligt internationell rätt.
% Bevisa att tystnad, fortsatt vapenhandel eller selektiv diplomati utgör medansvar.
% Redogör för tidigare exempel där Sverige agerat kraftfullt mot folkrättsbrott (t.ex. Ryssland, Myanmar, Iran).

\subsection*{Israel saknar rätt till försvar utanför israelisk mark}
\addcontentsline{toc}{subsection}{Israel saknar rätt till försvar utanför israelisk mark}

Oavsett legaliteten i själva ockupationen kan en ockupationsmakt aldrig åberopa självförsvar mot den befolkning som står under dess kontroll. Enligt folkrätten får en ockupationsmakt endast vidta åtgärder som är strikt nödvändiga för att upprätthålla allmän ordning och säkerhet, i enlighet med artikel 43 i Haagreglementet:\footnote{\url{https://ihl-databases.icrc.org/en/ihl-treaties/hague-regulations-1899/article-43}}

\lagrum{Article 43\quad The authority of the legitimate power having in fact passed into the hands of the occupant, the latter shall take all the measures in his power to restore, and ensure, as far as possible, public order and safety, while respecting, unless absolutely prevented, the laws in force in the country.}

Samtidigt är ockupationsmakten skyldig att skydda civilbefolkningen från våld och hot enligt artikel 27 i den fjärde Genèvekonventionen:\footnote{\url{https://ihl-databases.icrc.org/en/ihl-treaties/gciv-1949/article-27}}

\lagrum{Article 27\quad Protected persons are entitled, in all circumstances, to respect for their persons, their honour, their family rights, their religious convictions and practices, and their manners and customs. They shall at all times be humanely treated, and shall be protected especially against all acts of violence or threats thereof.}

Därtill får ockupationsmakten inte ersätta det lokala rättssystemet, utan endast temporärt administrera det för upprätthållande av civil ordning, enligt artikel 64:\footnote{\url{https://ihl-databases.icrc.org/en/ihl-treaties/gciv-1949/article-64}}

\lagrum{Article 64\quad The penal laws of the occupied territory shall remain in force, with the exception that they may be repealed or suspended by the Occupying Power in cases where they constitute a threat to its security or an obstacle to the application of the present Convention...}

Att hävda rätt till militärt självförsvar inom ett territorium där man själv är ockupant innebär att man juridiskt förväxlar våld med skydd, och därigenom ger repressionen ett falskt moraliskt anspråk. Det är att kriminalisera motstånd – och samtidigt rättfärdiga fortsatt förtryck.

Självförsvar enligt FN-stadgans artikel 51 är territoriellt begränsat till internationellt erkänd, egen mark:

\lagrum{Article 51\quad Nothing in the present Charter shall impair the inherent right of individual or collective self-defence if an armed attack occurs against a Member of the United Nations...}

Detta innebär att självförsvar inte kan åberopas av en stat som befinner sig utanför sitt eget erkända territorium – särskilt inte mot den civilbefolkning som staten i egenskap av ockupationsmakt har skyldighet att skydda.

Detta är också exakt den folkrättsliga position Sverige intar gentemot Ryssland och som man framhåller som självklart i officiella sammanhang: Ryssland kan inte hävda självförsvar på ukrainsk mark, eftersom Ukraina inte är ryskt territorium.  

Att däremot medge Israel rätt till självförsvar på ockuperad palestinsk mark är ett flagrant avsteg från denna princip. Det är en inkonsekvent rättstillämpning – där likabehandlingsprincipen offras till förmån för politisk opportunism.


Att därtill tillfoga formuleringar som \enquote{men det måste ske i enlighet med internationell rätt} är utan betydelse. Ty det är själva användningen av begreppet \textit{självförsvar} i detta sammanhang som utgör ett folkrättsbrott.

När en regering offentligt erkänner rätten till självförsvar för en stat som agerar utanför sitt eget territorium – i strid med folkrätten – så förskjuts hela den rättsliga tyngdpunkten. Diskussionen handlar då inte längre om \textit{huruvida} våldet är tillåtet, utan \textit{hur mycket} våld som kan anses proportionerligt.

Detta medför två rättsliga och moraliska implikationer:

\begin{enumerate}
  \item \textbf{Konkludent handlande i folkrättslig mening.}  
  Genom att använda termen \enquote{självförsvar} utan förbehåll, agerar regeringen i strid med den folkrättsliga huvudprincipen att självförsvar enligt FN-stadgan endast får utövas på egen, internationellt erkänd mark.

  Det utgör ett tyst – och därmed konkludent – godkännande av ett pågående folkrättsbrott, såsom olaglig ockupation eller övervåld, eftersom formuleringen ger dessa handlingar en legal fernissa.

 \item \textbf{Medverkan eller stämpling i straffrättslig mening.}  
Om en regering genom officiella uttalanden uppmuntrar, normaliserar eller skyddar ett folkrättsbrott, kan detta – i ett internationellt rättsligt sammanhang – jämställas med det som i svensk straffrätt motsvarar stämpling till brott.

\lagrum{23 kap. 2 § 2 st BrB\quad  I de fall det särskilt anges döms för stämpling till brott. Med stämpling förstås att någon i samråd med någon annan beslutar gärningen eller att någon söker anstifta någon annan eller åtar eller erbjuder sig att utföra den.}

Ett exempel: Om Sveriges statsminister skulle säga \enquote{Ryssland har rätt att försvara sig på ukrainsk mark, men det måste ske enligt internationell rätt}, så har man redan legitimerat det första skottet. Diskussionen handlar därefter inte om \textit{att} Ryssland får använda våld – utan \textit{hur mycket} våld som är acceptabelt. På så sätt förskjuts skuldbedömningen från själva olagligheten till våldets omfattning, vilket innebär att man rättsligt och moraliskt gör sig medskyldig till brottet.
\end{enumerate}

\subsubsection*{Sammanfattning}
\noindent En ockupationsmakt har ingen rätt till självförsvar mot den befolkning som står under dess kontroll. Att erkänna sådan rätt är att förskjuta rättens tyngdpunkt från olagligt våld till "godtagbar nivå av våld", och därmed aktivt legitimera förtryck. 

Regeringens uttalanden står därmed i direkt motsättning till både Sveriges officiella folkrättsliga linje i andra konflikter och till grundläggande principer i internationell humanitär rätt.


\section{Regeringens tystnad, undfallenhet och rättsstridiga stöd}

\subsection*{Tystnaden inför apartheid är inte neutral}


Alla former av affärstransaktioner och diplomatiska förbindelser bidrar till att upprätthålla en apartheidregim.  
Det som i dag erkänns av Internationella domstolen (ICJ) och av världens främsta folkrättsliga institutioner benämns ännu inte av Sveriges regering\footnote{\url{https://www.amnesty.org/en/latest/news/2022/02/israels-apartheid-against-palestinians/}}\footnote{\url{https://www.hrw.org/news/2021/04/27/israel-apartheid-against-palestinians}}.  

Tystnaden är inte neutral – den är ett aktivt val.


\subsection*{Regeringens selektiva moral och dubbla måttstockar}
\addcontentsline{toc}{subsection}{Regeringens selektiva moral och dubbla måttstockar}

\subsubsection*{Regeringens uttalanden och brist på åtgärder}
% Texten om statsministerns uttalande, från "Regeringen skriver vidare..." till "...fortsatt handel och diplomati som om inget hänt."

Regeringen skriver vidare:

\begin{quote}
\textit{“The terrorist organisation Hamas bears heavy responsibility for the current situation. The hostages must be released – unconditionally and immediately.”}
\end{quote}

Regeringen anser sig inta det moraliska överläget men har aldrig sanktionslagt Israel för dess folkrättsbrott. Ingen konsekvens har utdelats trots årtionden av övergrepp. Istället har Sverige fortsatt handel och diplomati som om inget hänt och har därmed inte gjort sin bekärda andel för upprätthållandet av internationell lag och ordning som skulle kunnat ge regeringen den moraliska höjden att utpeka en folkligt förankrad befrielserörelse såsom terroristorganisation – så till vida att denna inte beaktat och respekterat regeringens försök att upprätthålla internationell lag och ordning.

Detta är inte enbart en moralisk inkonsekvens. Det är ett folkrättsligt avtalsbrott i förhållande till Sveriges åtaganden enligt bland annat folkmordskonventionen, FN-stadgan och sedvanerättens tvingande normer.

Genom ett historiskt mönster av passivitet och konkludent handlande har Sverige inte bara underlåtit att agera – utan de facto bidragit till att ge Israel immunitet. Oavsett regim har Sverige fortsatt diplomatiskt och ekonomiskt stöd utan att utkräva ansvar för de upprepade folkrättsbrotten: illegala bosättningar, bombningar av civila och systematisk rasdiskriminering har ursäktats eller ignorerats.

Den som konsekvent tolererar statsterror förlorar rätten att fördöma motstånd mot den. Sverige har därmed spelat en \textit{aiding and abetting}-roll – och saknar rättslig trovärdighet när det gäller att peka ut andra för terror.


\subsubsection*{Regeringens underminering av folkrättens institutioner}
\addcontentsline{toc}{subsubsection}{Regeringens underminering av folkrättens institutioner}

Regeringen har konsekvent agerat för att skydda Israel från rättsligt ansvarsutkrävande. Den svenska statsministern har öppet opponerat sig mot Internationella brottmålsdomstolens (ICC) arresteringsorder mot Israels premiärminister – ett beslut grundat i omfattande bevisning om brott mot mänskligheten. Denna hållning utgör inte bara ett politiskt ställningstagande, utan en direkt underminering av det mest centrala internationella rättsliga verktyget för ansvar efter folkrättsbrott.

Samtidigt har regeringen ignorerat den samstämmiga bedömningen från världens främsta människorättsorganisationer\footnote{\url{https://www.amnesty.org/en/documents/mde15/8668/2024/en/}},  
Human Rights Watch\footnote{\url{https://www.hrw.org/report/2024/12/19/extermination-and-acts-genocide/israel-deliberately-depriving-palestinians-gaza}},  
FN:s särskilda rapportör Francesca Albanese\footnote{\url{https://www.ohchr.org/en/documents/country-reports/ahrc5573-report-special-rapporteur-situation-human-rights-palestinian}},  
och ledande folkmordsforskare\footnote{\url{https://ifpnews.com/top-scholars-israel-genocide-gaza/}},  
vilka alla konstaterat att ett folkmord pågår i Gaza. 

Den svenska regeringen har därmed förbrukat varje anspråk på att vara en neutral aktör eller försvarare av folkrätten. Genom att inte agera i enlighet med sina skyldigheter under folkmordskonventionen och FN-stadgan har Sverige inte enbart misslyckats i sitt förebyggande ansvar – det har valt sida i ett pågående folkrättsbrott.

\lagrum{Artikel I, Folkmordskonventionen\quad De fördragsslutande parterna bekräftar att folkmord, vare sig det begås i fredstid eller krigstid, är ett brott enligt internationell rätt som de åtar sig att förebygga och bestraffa.\footnote{\url{https://www.ohchr.org/en/instruments-mechanisms/instruments/convention-prevention-and-punishment-crime-genocide}}}

Denna artikel innebär uttryckligen att:
\begin{itemize}
  \item Staten inte enbart förbjuds att själv begå folkmord,
  \item utan är skyldig att förebygga det – även utanför sitt eget territorium,
  \item och att underlåtenhet att ingripa kan medföra internationellt ansvar.
\end{itemize}

Detta fastslogs entydigt av Internationella domstolen (ICJ) i målet \textit{Bosnia and Herzegovina v. Serbia and Montenegro} (2007):\footnote{\url{https://www.icj-cij.org/public/files/case-related/91/091-20070226-JUD-01-00-EN.pdf}}

\begin{quote}
\textit{“A State may incur responsibility not only for its own acts but also by aiding or assisting another State in the commission of an internationally wrongful act.”}
\end{quote}

(se domens punkt 420 ff.)

Vidare fastslås i FN-stadgan att medlemsstater inte får vara passiva inför grova människorättsbrott:

\lagrum{Article 1(3), FN-stadgan\quad To achieve international co-operation in solving international problems of an economic, social, cultural, or humanitarian character, and in promoting and encouraging respect for human rights...\footnote{\url{https://www.un.org/en/about-us/un-charter/full-text}}}

\lagrum{Article 56, FN-stadgan\quad All Members pledge themselves to take joint and separate action in co-operation with the Organization for the achievement of the purposes set forth in Article 55.\footnote{\url{https://www.un.org/en/about-us/un-charter/full-text}}}

Regeringens systematiska ignorans, dess aktiva försvar av förövaren och dess tystnad inför samstämmiga larm från rättsliga och humanitära organ visar att alla gränser nu är passerade. Detta handlar inte längre om att "utreda" eller "bevaka utvecklingen". Sveriges regering har, genom sitt agerande och sin underlåtenhet, trätt över den gräns som skiljer neutralitet från medverkan.

Detta dokument är därför inte en förfrågan, utan en anmälan.

\subsubsection*{Underlåtenhet att förebygga folkrättsbrott}
\addcontentsline{toc}{subsubsection}{Underlåtenhet att förebygga folkrättsbrott}

Den svenska regeringen har inte enbart förhållit sig passiv till folkmordskonventionens krav. Den har aktivt brutit mot dess anda och bokstav. Genom att frysa finansieringen av UNRWA – det FN-organ som ansvarar för livsnödvändigt bistånd till palestinska flyktingar – i enlighet med Israels påtryckningar, har Sverige bidragit till att avväpna det internationella systemet för humanitärt skydd.

Istället har regeringen omdirigerat biståndet till en israeliskt kontrollerad distributionsstruktur, underställd militär logistik. Detta har lett till ett system där civila i Gaza tvingas hämta mat och mediciner inom snäva, av militären fastställda tidsfönster – under hot om beskjutning. Det är ett system designat för kontroll, inte skydd. Enligt Euro-Med Human Rights Monitor har minst 60 civila skjutits ihjäl vid dessa hjälppunkter under tre dagar.

Detta utgör inte en avvikelse – det är en följd. En följd av att Sveriges regering valt att medverka till ett militärt organiserat biståndssystem som i praktiken upphäver Genèvekonventionens grundprinciper om opartiskhet, humanitet och civilas särskilda skyddsbehov.

Regeringen har därmed aktivt undergrävt FN:s auktoritet, legitimerat ett dödligt distributionssystem och svikit sitt ansvar att stå upp för den humanitära rätten. Det handlar inte om missriktad välvilja. Det handlar om medverkan till ett systematiskt brott mot folkrätten.

Vidare föreligger trovärdiga rapporter om att Israel beväpnar och skyddar väpnade grupper i Gaza – grupper vars ledare tidigare fängslats av Hamas för bland annat narkotikabrott och terrorism. Syftet har varit att destabilisera samhället inifrån och sabotera hjälpsystemet. Att Sveriges regering förblir tyst trots vetskap om detta är inte längre moraliskt förkastligt – det är rättsligt förpliktigande.

Vi konstaterar:

\begin{itemize}
  \item Att regeringens agerande utgör brott mot Sveriges skyldigheter enligt folkmordskonventionen.
  \item Att Sveriges modell för biståndsdistribution i Gaza innebär ett medvetet avsteg från humanitär folkrätt.
  \item Att tystnad inför rapporter om israeliskt stöd till kriminella och jihadistiska grupper kan medföra medansvar.
\end{itemize}

Vi kräver därför:

\begin{enumerate}
  \item Att utrikesministern offentligt kommenterar uppgifterna om svenskt stöd till en folkrättsvidrig biståndsmodell.
  \item Att Konstitutionsutskottet omedelbart utreder huruvida Utrikesdepartementet fullgjort sina förpliktelser enligt folkmordskonventionen och FN-stadgan.
\end{enumerate}

Detta är inte en begäran om förklaring. Det är en formell anklagelse. Sverige har inte bara brutit mot sina skyldigheter – det har gjort det med öppen blick och kallt beräknande.

\subsubsection*{Varför Israel stödjer jihadistgrupper}
\addcontentsline{toc}{subsubsection}{Varför Israel stödjer jihadistgrupper}

Frågan varför Israel aktivt stödjer salafistiska och jihadistiska grupper i Gaza – inklusive element med koppling till ISIS – måste förstås utifrån en strategisk logik, inte som ett säkerhetspolitiskt misslyckande.

\textbf{1. Hamas har förändrats – och det hotar Israels narrativ}

Sedan åtminstone 2006 har Hamas genomgått en djupgående politisk omorientering. I sitt policyprogram från 2017 samt i det gemensamma avtalet med Fatah 2021 erkände rörelsen:

\begin{itemize}
  \item internationell rätt som ramverk,
  \item PLO:s överordnade roll som palestinskt paraplyorgan,
  \item en tvåstatslösning enligt 1967 års gränser, med östra Jerusalem som huvudstad,
  \item och ett fredligt, folkligt motstånd som metod.
\end{itemize}

Denna förändring dokumenteras bland annat av Wikipedia, som redogör för hur Hamas i sitt nya policydokument 2017 explicit accepterade en palestinsk stat inom 1967 års gränser och bygger vidare på tidigare initiativ såsom Prisoners’ Document (2006)\footnote{\url{https://en.wikipedia.org/wiki/Palestinian_Prisoners\%27_Document}}, Mecka-avtalet (2007)\footnote{\url{https://en.wikipedia.org/wiki/Fatah–Hamas_Mecca_Agreement}} och avtalet 2020\footnote{\url{https://en.wikipedia.org/wiki/2020_Palestinian_reconciliation_agreement}}.

Akademiskt har detta analyserats som en genuin förändring, inte endast en kosmetisk fasad. Professor Neve Gordon\footnote{\url{https://en.wikipedia.org/wiki/Neve_Gordon}} och professor Menachem Klein\footnote{\url{https://www.972mag.com/hamas-fatah-elections-israel-arrogance/}} menar att denna utveckling syftade till att uppnå internationell legitimitet, demokratisk försoning och en väg mot ett återförenat palestinskt ledarskap.

Men enligt Klein underminerades dessa fredssträvanden aktivt av Israel, som i stället destabiliserade processen för att kunna upprätthålla narrativet att ”det saknas en trovärdig fredspartner”.

\textbf{2. Israel behöver oresonliga fiender för att rättfärdiga sin politik}

Att Israel historiskt har funnit strategiskt värde i att understödja radikala element är väl dokumenterat. Tidigare premiärminister Ehud Barak har själv erkänt att Israel i decennier aktivt stött Hamas i syfte att försvaga PLO – en splittringsstrategi som nu återanvänds, med än farligare inslag: salafistiska och jihadistiska klaner, inklusive grupper ledda av personer tidigare fängslade av Hamas för narkotikabrott och religiös extremism.

\textbf{Syftet är inte att bekämpa terror, utan att bevara kaos.} Ett extremistdominerat Gaza fungerar som en permanent motbild till ”fred”, och förstärker den israeliska statens centrala narrativ: att det inte existerar någon legitim, förhandlingsbar palestinsk motpart. Detta narrativ möjliggör fortsatt ockupation, kollektiv bestraffning och territoriell expansion – under förevändning av säkerhetsbehov.

Professor Norman Finkelstein har i detta sammanhang lyft den israeliska administrationens begrepp \textit{“The Palestinian peace offensive”} – en intern varningssignal för när motståndet uppfattas som \textit{för rationellt}. Ett måttfullt Hamas som respekterar internationell rätt, erkänner 1967 års gränser och söker val utgör ett långt större hot mot den israeliska långsiktiga strategin än en beväpnad jihadist. Fienden får inte bli trovärdig – den måste vara grotesk.

\textbf{Därför understöds extremism – inte trots dess brutalitet, utan på grund av dess politiska användbarhet.} Detta är inte ett misstag, inte en olycklig följd. Det är en avsiktlig, väldokumenterad realpolitisk strategi med djup historisk kontinuitet.

\medskip

\textbf{Att Sveriges regering ignorerar detta mönster – och därmed i praktiken legitimerar det – är inte en fråga om tolkning, utan om ansvar.} Genom att ensidigt fördöma Hamas, utan att erkänna rörelsens dokumenterade förvandling till en potentiell politisk aktör, förstärker Sverige ett förljugat narrativ som aktivt motverkar fred.

Detta är inte en underrättelsemiss. Det är ett systemfel.

\textbf{Vi frågar därför: Hur kan Sveriges regering, med tillgång till all tillgänglig dokumentation, agera såsom om konflikten endast handlade om terrorbekämpning – och inte om ockupation, kontroll och medvetet sabotage av varje fredsinitiativ som hotar status quo?}

\subsubsection*{Sammanfattning: Sveriges folkrättsliga skyldigheter är bindande – inte valfria}
\addcontentsline{toc}{subsubsection}{Sammanfattning: Sveriges folkrättsliga skyldigheter är bindande – inte valfria}

Den svenska regeringens agerande måste nu bedömas i ljuset av sina folkrättsliga förpliktelser. Vad som ovan visats – ockupationens karaktär, sabotaget mot fredsprocesser, beväpning av extremistgrupper och vägran att ingripa – utgör inte enbart politiska eller moraliska tillkortakommanden. Det är fråga om konkreta rättsbrott genom underlåtenhet att uppfylla bindande konventionsplikt.

\medskip

Enligt artikel I i Konventionen om förebyggande och bestraffning av brottet folkmord är Sverige skyldigt att inte bara avstå från folkmord, utan också att aktivt förebygga och straffa det. Denna skyldighet har erkänts av Internationella domstolen (ICJ) som en \textit{erga omnes}-förpliktelse – det vill säga en skyldighet som gäller gentemot hela det internationella samfundet.

Därtill är Sverige bundet av sedvanerättens princip om \textit{non-assistance in wrongful acts}, som förbjuder:

\begin{itemize}
  \item att bistå aktörer som begår folkrättsbrott,
  \item att ekonomiskt eller politiskt dra nytta av sådana brott,
  \item att förhålla sig passiv när man har kännedom om brott och en rättslig skyldighet att agera.
\end{itemize}

Detta gäller i synnerhet vid:

\begin{itemize}
  \item folkmord (Genocide Convention),
  \item brott mot mänskligheten (Romstadgan),
  \item grova krigsbrott (Genèvekonventionerna),
  \item apartheid (FN:s apartheidkonvention).
\end{itemize}

Att i detta läge – där Israel systematiskt förvägrar Gazas befolkning skydd enligt humanitär rätt – kräva att det palestinska folket ska avstå från motstånd, utan att samtidigt ingripa mot förövaren, är inte en neutral hållning. Det är ett rättsbrott.

Att kalla varje handling av motstånd “terrorism” medan man själv skyddar, finansierar eller legitimerar ockupationsmakten är att delta i det rättsvidriga status quo.

\medskip

Endast den stat som själv uppfyller sina rättsliga skyldigheter kan moraliskt och juridiskt fördöma andras svar. Regeringens vägran att göra detta innebär att Sverige:

\begin{itemize}
  \item förnekar det palestinska folket varje legitimt alternativ till självförsvar,
  \item avstår från att använda diplomatiska, rättsliga och ekonomiska påtryckningsmedel för att förhindra folkrättsbrott,
  \item och därigenom gör sig medskyldig genom underlåtenhet att reagera.
\end{itemize}

Det är alltså inte den som slår tillbaka i desperation som bär huvudansvaret – utan den regering som, trots kännedom om brotten, vägrar att ingripa.

\bigskip

\subsubsection*{Avtalet med Elbit Systems – svensk medverkan}
\addcontentsline{toc}{subsubsection}{Avtalet med Elbit Systems – svensk medverkan}

Regeringens skuld stannar inte vid demoniseringen av en folkligt förankrad befrielserörelse. Den sträcker sig vidare genom total vägran att tillämpa internationell rätt gentemot Israel – trots överväldigande bevis på tidigare krigsbrott och folkrättsbrott (före den 7 oktober 2023) – och kulminerar i aktiv medverkan.

Den 27 oktober 2023 – samma dag som Israels markinvasion av Gaza inleddes – undertecknade Sveriges regering ett militärt samarbetsavtal med den israeliska vapentillverkaren Elbit Systems. Genom detta har Sverige aktivt bidragit till att legitimera och stödja en krigförande stats vapenindustri mitt under ett pågående folkmord.

Detta trots att det sedan länge är dokumenterat att Elbit och andra israeliska försvarskoncerner använder Gaza som testarena för nya vapensystem. Dessa vapen säljs därefter internationellt som “battle tested”.\footnote{\url{https://www.youtube.com/watch?v=78rs9_FrgmA}} Journalisten Yotam Feldman har visat detta i dokumentären \textit{The Lab}.

Enligt Euro-Med Human Rights Monitor hade Israel, redan inom den första månaden av angreppet, fällt en mängd sprängmedel över Gaza motsvarande två Hiroshimabomber.\footnote{\url{https://euromedmonitor.org/en/article/5908/Israel-hits-Gaza-Strip-with-the-equivalent-of-two-nuclear-bombs}} Trots detta valde Sveriges regering att investera i denna vapenapparat – samtidigt som den fördömde det palestinska motståndet och förteg ockupationsmaktens rättsbrott.

Detta agerande – att i affektens skugga inleda militärt samarbete med en regim som systematiskt bryter mot internationell rätt – utgör inte bara ett moraliskt svek. Det är en rättsstridig handling i sig. Det är medverkan till folkrättsbrott.


\subsubsection*{Sammanfattning: Sveriges folkrättsliga skyldigheter är bindande – inte valfria}
\addcontentsline{toc}{subsubsection}{Sammanfattning: Sveriges folkrättsliga skyldigheter är bindande – inte valfria}

Den svenska regeringens agerande måste nu bedömas i ljuset av sina folkrättsliga förpliktelser. Vad som ovan visats – ockupationens karaktär, sabotaget mot fredsprocesser, beväpning av extremistgrupper och vägran att ingripa – utgör inte enbart politiska eller moraliska tillkortakommanden. Det är fråga om konkreta rättsbrott genom underlåtenhet att uppfylla bindande konventionsplikt.

\medskip

Enligt artikel I i Konventionen om förebyggande och bestraffning av brottet folkmord är Sverige skyldigt att inte bara avstå från folkmord, utan också att aktivt förebygga och straffa det. Denna skyldighet har erkänts av Internationella domstolen (ICJ) som en \textit{erga omnes}-förpliktelse – det vill säga en skyldighet som gäller gentemot hela det internationella samfundet.

Därtill är Sverige bundet av sedvanerättens princip om \textit{non-assistance in wrongful acts}, som förbjuder:

\begin{itemize}
  \item att bistå aktörer som begår folkrättsbrott,
  \item att ekonomiskt eller politiskt dra nytta av sådana brott,
  \item att förhålla sig passiv när man har kännedom om brott och en rättslig skyldighet att agera.
\end{itemize}

Detta gäller i synnerhet vid:

\begin{itemize}
  \item folkmord (Genocide Convention),
  \item brott mot mänskligheten (Romstadgan),
  \item grova krigsbrott (Genèvekonventionerna),
  \item apartheid (FN:s apartheidkonvention).
\end{itemize}

Att i detta läge – där Israel systematiskt förvägrar Gazas befolkning skydd enligt humanitär rätt – kräva att det palestinska folket ska avstå från motstånd, utan att samtidigt ingripa mot förövaren, är inte en neutral hållning. Det är ett rättsbrott.

Att kalla varje handling av motstånd “terrorism” medan man själv skyddar, finansierar eller legitimerar ockupationsmakten är att delta i det rättsvidriga status quo.

\medskip

Endast den stat som själv uppfyller sina rättsliga skyldigheter kan moraliskt och juridiskt fördöma andras svar. Regeringens vägran att göra detta innebär att Sverige:

\begin{itemize}
  \item förnekar det palestinska folket varje legitimt alternativ till självförsvar,
  \item avstår från att använda diplomatiska, rättsliga och ekonomiska påtryckningsmedel för att förhindra folkrättsbrott,
  \item och därigenom gör sig medskyldig genom underlåtenhet att reagera.
\end{itemize}

Det är alltså inte den som slår tillbaka i desperation som bär huvudansvaret – utan den regering som, trots kännedom om brotten, vägrar att ingripa.

\bigskip



\subsection{Sveriges aktiva medverkan}
  \subsubsection*{Avtalet med Elbit Systems – svensk medverkan}
\addcontentsline{toc}{subsubsection}{Avtalet med Elbit Systems – svensk medverkan}

Regeringens skuld stannar inte vid demoniseringen av en folkligt förankrad befrielserörelse. Den sträcker sig vidare genom total vägran att tillämpa internationell rätt gentemot Israel – trots överväldigande bevis på tidigare krigsbrott och folkrättsbrott (före den 7 oktober 2023) – och kulminerar i aktiv medverkan.

Den 27 oktober 2023 – samma dag som Israels markinvasion av Gaza inleddes – undertecknade Sveriges regering ett militärt samarbetsavtal med den israeliska vapentillverkaren Elbit Systems. Genom detta har Sverige aktivt bidragit till att legitimera och stödja en krigförande stats vapenindustri mitt under ett pågående folkmord.

Detta trots att det sedan länge är dokumenterat att Elbit och andra israeliska försvarskoncerner använder Gaza som testarena för nya vapensystem. Dessa vapen säljs därefter internationellt som “battle tested”.\footnote{\url{https://www.youtube.com/watch?v=78rs9_FrgmA}} Journalisten Yotam Feldman har visat detta i dokumentären \textit{The Lab}.

Enligt Euro-Med Human Rights Monitor hade Israel, redan inom den första månaden av angreppet, fällt en mängd sprängmedel över Gaza motsvarande två Hiroshimabomber.\footnote{\url{https://euromedmonitor.org/en/article/5908/Israel-hits-Gaza-Strip-with-the-equivalent-of-two-nuclear-bombs}} Trots detta valde Sveriges regering att investera i denna vapenapparat – samtidigt som den fördömde det palestinska motståndet och förteg ockupationsmaktens rättsbrott.

Detta agerande – att i affektens skugga inleda militärt samarbete med en regim som systematiskt bryter mot internationell rätt – utgör inte bara ett moraliskt svek. Det är en rättsstridig handling i sig. Det är medverkan till folkrättsbrott.
