


\section{Bevisning av folkmordsavsikt}
%\addcontentsline{toc}{section}{Bevisning av folkmordsavsikt}
\subsection{Långvarig normalisering av folkmordsretorik}
%\addcontentsline{toc}{subsection}{Långvarig normalisering av folkmordsretorik}

David Sheen\footnote{\url{https://www.youtube.com/watch?v=9GbKsAvuBDM}} vittnade år 2014 i den tyska förbundsdagen om hur folkmordsretorik, som länge existerat
i det israeliska samhällets marginaler, efter omkring 2010 i allt högre grad började uttryckas öppet och
ogenerat. Hets mot palestinier och andra icke-judar hade sedan tidigare förekommit inom både religiösa
och akademiska kretsar, i rabbiners predikningar och tidningsartiklar – men ofta i kodifierad form eller
utan offentlig erkänsla. Från 2010-talet började dessa budskap yttras utan förbehåll, inte längre som
undantag eller skandaler, utan som en del av ett normaliserat offentligt samtal. Sheen beskrev inte enskilda
beslutsfattares policy, utan en bred social atmosfär där allt fler samhällsaktörer – politiker, medier,
religiösa ledare, akademiker och medborgare – tillsammans medverkar till att göra eliminatorisk retorik
socialt acceptabel. Vad som tidigare hade varit fördolt, filtrerat eller ursäktat släpps nu fram i ljuset, och
försvaras utan skam.


Ett centralt exempel är Ayelet Shaked, som inför valet 2015 postade ett inlägg på Facebook där hon kallade
palestinska mödrar för dem som "föder små ormar" och förklarade att de också var legitima mål i krig. Inlägget
var inte ett hinder – det blev en språngbräda till hennes utnämning som justitieminister. Shakeds
framgång visade enligt Sheen att folkmordsretorik inte bara tolereras, utan premieras politiskt.\footnote{\url{https://electronicintifada.net/blogs/ali-abunimah/israeli-lawmakers-call-genocide-palestinians-gets-thousands-facebook-likes}}

På sociala medier postade tusentals israeler bilder på sig själva med texter som "Hämnas!", "Döda araber!",
eller "Hat mot araber är inte rasism – det är värderingar". Många använde sina riktiga namn, sina egna
Facebook-profiler – utan skam, utan fruktan. Denna uppmaning till våld återkom också på israeliska granater,
där soldater ritade meddelanden inför bombningarna av Gaza: "För att Backstreet Boys-konserten ställdes in",
eller i kodspråk: "Utplåna Amalek". Just referensen till "Amalek" är, enligt Sheen, ett religiöst kodord för
folkmord – en teologisk uppmaning att förinta män, kvinnor, barn och boskap i enlighet med Gamla testamentets
beskrivning av fiender till det israelitiska folket.

I israeliska medier kunde man samtidigt läsa opinionstexter med rubriker som \textit{"When Genocide is
Permissible"} – ett inlägg som publicerades i \textit{Times of Israel}, men togs bort efter internationellt ramaskri.
Dock publicerades samma dag en annan artikel med samma budskap, men uttryckt i kodspråk, och den fick ligga
kvar. Det är, enligt Sheen, just i övergången från explicit till kodad folkmordsretorik som staten söker
försvara det ohållbara – att folkrätten negligeras till förmån för ett offentligt samtal där utplåning av en
civilbefolkning tillåts bli politiskt legitimt.

Några artiklar om israeliska socialmedia poster som speglar det folkliga sentimentet för mer än 10 år sedan:
\begin{itemize}
\item  \textit{Palestinian woman attacked – Israeli police officer Ariel Shapiro posts on Facebook: “It’s a shame that the Arab whore didn’t die”}\footnote{\url{https://occupiedpalestine.wordpress.com/2013/02/28/palestinian-woman-attacked-israeli-police-officer-ariel-shapiro-posts-on-fb-its-a-shame-that-the-arab-whore-didnt-die/}}

\item \textit{“Castrate them!” “Burn them!” “Bullet in the head!”: Facebook Israelis react to photo of Palestinian kids}\footnote{\url{https://electronicintifada.net/blogs/ali-abunimah/castrate-them-burn-them-bullet-head-facebook-israelis-react-photo-palestinian}}

\item \textit{“May all Arabs die!” Israelis Express Joy at Facebook over Palestinian Deaths}\footnote{\url{https://occupiedpalestine.wordpress.com/2013/03/18/may-all-arabs-die-israelis-express-joy-at-facebook-over-palestinian-deaths/}}

\item \textit{Bombing of Gaza children gives me “orgasm”: Israelis celebrate slaughter on Facebook}\footnote{\url{https://electronicintifada.net/blogs/patrick-strickland/bombing-gaza-children-gives-me-orgasm-israelis-celebrate-slaughter-facebook?}}

\item \textit{Former Israeli soldier describes how IDF troops do not view Palestinians as 'human beings'}\footnote{\url{https://www.independent.co.uk/news/world/middle-east/former-israeli-soldier-describes-how-idf-troops-do-not-view-palestinians-as-human-beings-9828056.html}}

\item \textit{Israeli Professor Calls for Palestinian Genocide}\footnote{\url{https://www.richardsilverstein.com/2014/10/14/israeli-professor-calls-for-palestinian-genocide/}}

\item \textit{Yosef: Gentiles exist only to serve Jews}\footnote{\url{https://archive.is/Xhnn}}

\item \textit{Rabbi Ginsburg: "If a Jews needs a liver, can he take the liver from an innocent non-Jew to save him?"}\footnote{\url{https://archive.org/details/jewishfundamenta0000shah/page/42/mode/2up}}

\item \textit{Reprint of Yochanan Gordon’s “When Genocide is Permissible”}\footnote{\url{https://mondoweiss.net/2014/08/yochanan-genocide-permissible/}}

\item \textit{Israeli Minster Calls for “Civil Targeted Killings” of BDS Leaders}\footnote{\url{https://www.richardsilverstein.com/2016/03/30/israeli-minster-calls-for-civil-targeted-killings-of-bds-leaders/}}

\item \textit{Israeli officer: I was right to shoot 13-year-old child}\footnote{\url{https://www.theguardian.com/world/2004/nov/24/israel}}

\item \textit{“You want to kill him but he’s crying”: More Israeli soldiers’ confessions of crimes against children}\footnote{\url{https://electronicintifada.net/blogs/adri-nieuwhof/you-want-kill-him-hes-crying-more-israeli-soldiers-confessions-crimes-against}}

\item \textit{Snipers with children in their sights}\footnote{\url{https://www.theguardian.com/world/2005/jun/28/comment.israelandthepalestinians}}

\item \textit{Israeli rabbi says killing civilians in Gaza is allowed}\footnote{\url{https://occupiedpalestine.wordpress.com/2014/07/23/gazaunderattack-israeli-rabbi-says-killing-civilians-in-gaza-is-allowed/}}

\item \textit{Choose a kid at random, “aim at his body”: Israeli soldiers confess their violence}\footnote{\url{https://electronicintifada.net/blogs/adri-nieuwhof/choose-kid-random-aim-his-body-israeli-soldiers-confess-their-violence}}

\item \textit{CNN camera catches Israeli soldier who fired at, killed Palestinian teen}\footnote{\url{https://electronicintifada.net/blogs/ali-abunimah/cnn-camera-catches-israeli-soldier-who-fired-killed-palestinian-teen}}

\item \textit{https://electronicintifada.net/blogs/ali-abunimah/cnn-camera-catches-israeli-soldier-who-fired-killed-palestinian-teen}\footnote{\url{}}

\end{itemize}


Under Israels tidigare invasioner av Gaza, särskilt 2009 och 2014, har israeliska trupper enligt
vittnesmål från den israeliska veteranorganisationen Breaking the Silence fått explicita order att
skjuta civila.\footnote{\url{https://normanfinkelstein.substack.com/p/two-or-three-quotes-that-you-use}}%
\footnote{\url{https://electronicintifada.net/blogs/patrick-strickland/israeli-soldiers-ordered-kill-civilians-gaza-says-breaking-silence}} 
En rad äldre artiklar om dödade barn i Gaza vittnar om att sådana handlingar har förekommit i
decennier – utan rättsliga konsekvenser. Särskilt belysande är den amerikanske journalisten
Chris Hedges text \enquote{Why Israel Lies}, där han som ögonvittne beskriver följande:

\begin{quote}
    \textit{\enquote{I saw small boys baited and killed by Israeli soldiers in the Gaza refugee
    camp of Khan Younis. The soldiers swore at the boys in Arabic over the loudspeakers of
    their armored jeep. The boys, about 10 years old, then threw stones at an Israeli vehicle
    and the soldiers opened fire, killing some, wounding others. I was present more than once
    as Israeli troops drew out and shot Palestinian children in this way. Such incidents, in
    the Israeli lexicon, become children caught in crossfire.}}
\end{quote}


\begin{itemize}
    \item \textit{2005: Not guilty. The Israeli captain who emptied his rifle into a Palestinian schoolgirl}\footnote{\url{https://www.theguardian.com/world/2005/nov/16/israel2}}
    
    \item \texit{2004: Girl’s life ended by Israeli bullets}\footnote{\url{https://electronicintifada.net/content/girls-life-ended-israeli-bullets/5241}}

    \item \textit{2005: Snipers with children in their sights - Palestinian civilians have been killed by the army with impunity}\footnote{\url{https://www.theguardian.com/world/2005/jun/28/comment.israelandthepalestinians}}

    \item \textit{2004: Girl shot in UNRWA school dies}\footnote{url{https://electronicintifada.net/content/girl-shot-unrwa-school-dies/1912}}

    \item \textit{2013: One Palestinian child has been killed by Israel every 3 days for the past 13 years}\footnote{\url{https://occupiedpalestine.wordpress.com/2013/06/04/one-palestinian-child-has-been-killed-by-israel-every-3-days-for-the-past-13-years/}}

    \item \textit{2014: Children killed by Israeli soldiers “hiding” near schools, says Human Rights Watch}\footnote{\url{https://electronicintifada.net/blogs/ali-abunimah/children-killed-israeli-soldiers-hiding-near-schools-says-human-rights-watch}}
    
    \item \textit{2018: Snipers ordered to shoot children, Israeli general confirms}\footnote{\url{https://electronicintifada.net/blogs/ali-abunimah/snipers-ordered-shoot-children-israeli-general-confirms}}
    
    \item \textit{2018: Israel lies that boy shot in head “fell off bike”}\footnote{\url{Israel lies that boy shot in head “fell off bike”}}

\end{itemize}


\textbf{“There are no uninvolved civilians”} har sedan dess etablerats som ett normativt uttryck i det israeliska civilsamhället och återfinns i uttalanden från både militära befälhavare och politiska företrädare. Genom att förneka existensen av oskyldiga palestinska civila ges varje form av våld ett rättfärdigande ramverk.

Begreppet \textit{“The world’s first livestreamed genocide”} har myntats av människorättsorganisationer och folkmordsforskare för att beskriva den samtidiga brutaliteten och transparensen i Israels angrepp. Trots detta har den svenska regeringen accepterat varje israelisk bortförklaring om att man enbart riktar sig mot \enquote{Hamas} – och därigenom godtagit en folkrättsvidrig doktrin där hela civilbefolkningen utpekas som legitim måltavla.




%\section*{Om folkmord – när intentionen är uttalad}
%\addcontentsline{toc}{section}{Om folkmord – när intentionen är uttalad}

Utrikesdepartementet har i svar till svenska medborgare förklarat att det inte ankommer på regeringen 
att ta ställning till huruvida Israel begår folkmord, utan att sådana bedömningar måste invänta slutsatser 
från ICJ och ICC (Svar från ud.mena.brevsvar@gov.se):

\textit{Kraven på respekt för folkrätten inklusive den internationella humanitära rätten har varit – och fortsätter att vara – ett av regeringens nyckelbudskap. Dessa budskap framförs i våra egna kontakter med Israel och vi gör det tillsammans med andra EU-länder och likasinnade. Ibland sker det offentligt och ibland på annat sätt. Hur regler respekteras och om krigsförbrytelser begåtts måste bedömas från fall till fall utifrån den internationella humanitära rätten. Det är inte regeringens roll att göra sådana bedömningar. För regeringen är det är centralt att eventuellt överträdelser av den internationella humanitära rätten och möjliga krigsförbrytelser utreds och att ansvarsutkrävande säkerställs. Både ICC och ICJ har pågående utredningar om situationen i Palestina. Hittills har de kommit fram till att Israel måstes göra mer för att skydda den civila befolkningen. Det är något regeringen också har framfört till Israel.}



Utrikesdepartementets linje är att det inte är regeringens roll att ta ställning till huruvida Israel begår folkmord, utan att sådana bedömningar ska överlämnas till internationella domstolar. Men om detta vore regeringens övergripande princip, skulle Sverige exempelvis ha tvingats förhålla sig neutralt till Rysslands invasion av Ukraina – i väntan på en formell dom från ICJ. Det gjorde man inte. Redan efter några dagar fördömde Sverige invasionen som ett brott mot FN-stadgan och internationell rätt.  

Samma mönster gäller andra folkrättsöverträdelser. Sverige tog tydligt ställning mot USA:s invasion av Irak 2003. Man fördömde apartheidregimen i Sydafrika långt innan internationella domstolar hade fällt bindande utslag. Regeringen har också kritiserat förtryck i Syrien, Iran, Belarus och Kina, utan att hänvisa till att man måste “vänta på ICC”.

Det visar att Sverige mycket väl kan – och faktiskt redan gör – folkrättsliga och moraliska bedömningar i realtid. Men när det gäller Israel gör man plötsligt avkall på denna förmåga, som om det skulle krävas en internationell prästvigning för att skilja rätt från fel.

``Det är inte regeringens roll att göra sådana bedömningar'' - Att gömma sig bakom framtida domstolsprövningar är inget juridiskt ställningstagande – det är ett moraliskt abdikerande. Och det sker i strid med Sveriges skyldigheter enligt folkmordskonventionen, som inte bara kräver straff, utan uttryckligen kräver att stater \textit{förhindrar} folkmord innan det sker.




%\subsection*{“Frågan om folkmord är en komplex juridisk fråga”}
%\addcontentsline{toc}{subsection}{“Frågan om folkmord är en komplex juridisk fråga”}

En återkommande ursäkt för att undvika att benämna det som sker i Gaza som folkmord är påståendet att detta vore en särskilt svårbedömd brottsrubricering – som om dess juridiska status vore mer oklar än andra internationella förbrytelser. Denna hållning är vilseledande.

Folkmord är ett tydligt definierat brott i internationell rätt. Den rättsliga svårigheten ligger inte i att förstå vad folkmord \textit{är}, utan i att bevisa gärningsmannens \textit{avsikt att förinta} en folkgrupp i sin helhet eller delvis – den så kallade \textit{dolus specialis}.

\textbf{Artikel II i FN:s konvention om förebyggande och bestraffning av brottet folkmord (1948):}\\
\textit{“I denna konvention avses med folkmord någon av följande gärningar, begångna i avsikt att förinta, helt eller delvis, en nationell, etnisk, raslig eller religiös grupp som sådan.”}\footnote{\url{https://www.ohchr.org/en/instruments-mechanisms/instruments/convention-prevention-and-punishment-crime-genocide}}

Att bevisa denna särskilda avsikt – \textit{dolus specialis} – är i regel en av de mest utmanande delarna i ett folkmordsfall. Förövarna är sällan benägna att uttrycka sina intentioner öppet; istället förnekar de vanligtvis att någon sådan avsikt alls föreligger. 

I Israels fall gäller dock det omvända. Här har intentionen inte dolts – tvärtom har den uttryckts öppet, upprepat och från högsta ort.

\subsection{Minst ett decennium av föregående uppvigling till folkmord}
Den israeliske 


\subsection{Internationella domstolen bekräftar folkmordsrisken}
%\addcontentsline{toc}{subsection}{Internationella domstolen bekräftar folkmordsrisken}

Just detta låg till grund för Sydafrikas stämning av Israel inför Internationella domstolen (ICJ) i Haag. I sitt beslut den 26 januari 2024 fann domstolens majoritet att det föreligger en \textit{“plausible risk”} för folkmord i Gaza och ålade Israel att omedelbart vidta åtgärder för att förhindra att brottet fullbordas.\footnote{\url{https://www.icj-cij.org/sites/default/files/case-related/192/192-20240126-ord-01-00-en.pdf}}

Domstolen uttryckte särskilt oro över officiella israeliska uttalanden som uppmanade till att “utrota” Gazas invånare, samt över militära åtgärder som tydligt antyder att sådana uttalanden inte är retorik utan strategi.

Detta rättsliga ställningstagande har därefter förstärkts av nya bevis och rapporter från FN-organ och människorättsorganisationer, vilka bekräftar att den israeliska offensiven fortsatt i strid med ICJ:s interimistiska föreläggande.

\subsection{Israels rättsstat ignorerar systematisk uppvigling till folkmord}
%\addcontentsline{toc}{subsection}{Israels rättsstat ignorerar systematisk uppvigling till folkmord}

Parallellt med den internationella kritik som riktats mot Israels politiska ledarskap, har även prominenta israeliska samhällsaktörer larmat om en inhemsk rättslig kollaps.

Den 3 januari 2024 publicerade \textit{The Guardian} ett öppet brev, undertecknat av flera av Israels mest respekterade akademiker, tidigare diplomater, journalister och parlamentariker. De anklagar landets justitieväsende för att ignorera vad de kallar en “utbredd och flagrant” uppvigling till folkmord och etnisk rensning, särskilt riktad mot Gazas civilbefolkning.\footnote{\url{https://web.archive.org/web/20250305043036/https://www.theguardian.com/world/2024/jan/03/israeli-public-figures-accuse-judiciary-of-ignoring-incitement-to-genocide-in-gaza}}

I brevet konstateras att:
\begin{quote}
\textit{“För första gången vi kan minnas har explicita uppmaningar att begå ohyggliga brott mot miljontals civila blivit en legitim och regelbunden del av det israeliska offentliga samtalet. Idag är sådana uttalanden vardagsmat i Israel.”}
\end{quote}

Under juridisk representation av människorättsadvokaten Michael Sfard listar de 11 sidor av uttalanden från regeringsmedlemmar, Knesset-ledamöter, journalister, militära befäl, opinionsbildare och kändisar – alltifrån krav på användning av kärnvapen till bibelbaserade anspelningar om att utplåna Gazas befolkning såsom “Amalek”.

Brevet konstaterar att myndigheterna inte agerat mot dessa brott, samtidigt som de drivit hundratals utredningar mot arabiska medborgare i Israel för tal som tolkats som stöd till Hamas – ofta från anonyma konton med begränsad räckvidd.

Sfard uttrycker särskild oro för att denna retorik har spridits från marginalerna till mitten av samhället:
\begin{quote}
\textit{“Jag kunde aldrig föreställa mig att jag skulle behöva skriva ett sådant brev. Denna typ av språk har lämnat ytterkanterna och blivit mainstream i en utsträckning som är ofattbar.”}
\end{quote}

Brevet sändes \textit{innan} Sydafrika lämnade in sin stämning till Internationella domstolen, men innehåller citat och exempel som sedermera införlivats i den sydafrikanska argumentationen om “incitement to genocide”.

\textit{“Detta är precis den jordmån där omoraliska monster växer – och växer.”}, avslutar signatärerna.



\subsection{Israelisk jurist lämnar in ICC-ärende om uppvigling till folkmord}
%\addcontentsline{toc}{subsection}{Israelisk jurist lämnar in ICC-ärende om uppvigling till folkmord}

Den 10 december 2024 lämnade den fransk-israeliske juristen Omer Shatz in en formell anmälan till Internationella brottmålsdomstolen (ICC) i Haag, med krav på att åtta högt uppsatta israeliska politiker, militärer och opinionsbildare ska åtalas för uppvigling till folkmord (\textit{incitement to genocide}).\footnote{\url{https://www.statewatch.org/news/2024/december/case-filed-at-icc-to-prosecute-israeli-officials-for-incitement-to-genocide/}}

Anmälan riktar sig bland annat mot premiärminister Benjamin Netanyahu, försvarsminister Yoav Gallant, president Isaac Herzog samt journalisten Zvi Yehezkeli. Den åberopar just den punkt där Internationella domstolen (ICJ) i januari 2024 konstaterade att Israel har en folkrättslig skyldighet att förhindra och bestraffa offentlig uppmaning till folkmord i Gaza.

Trots detta har den israeliska regeringens juridiska rådgivare meddelat Högsta domstolen i Israel att inga brottsutredningar kommer att inledas – i direkt trots mot ICJ:s föreläggande.

Juristens inlagor argumenterar därför att ICC är skyldig att agera i statens ställe. Enligt anmälan uppfyller ICJ:s bevisstandard om "plausible risk" samtidigt ICC:s tröskelvärde för arrestering: \textit{"reasonable grounds to believe"}.

De misstänkta namnges enligt följande:

\begin{itemize}
  \item Benjamin Netanyahu, premiärminister
  \item Yoav Gallant, tidigare försvarsminister
  \item Isaac Herzog, president
  \item Bezalel Smotrich, finansminister
  \item Itamar Ben-Gvir, säkerhetsminister
  \item Israel Katz, försvarsminister
  \item Giora Eiland, f.d. generalmajor
  \item Zvi Yehezkeli, journalist
\end{itemize}

Dokumentet understryker att brottet uppvigling till folkmord är självständigt enligt Romstadgan och inte kräver att ett faktiskt folkmord fullbordats.

\textit{Bilaga:} Den fullständiga stämningsansökan finns att läsa i originalversion här: \url{https://www.statewatch.org/media/4123/icc-communication-gaza-incitement-genocide-dec-2024.pdf}


\subsection{“A textbook case of genocide”}
%\addcontentsline{toc}{subsection}{“A textbook case of genocide”}

Just för att Israels intentioner inte bara är dokumenterade utan uttryckta offentligt – och från högsta politiska och militära nivå – fastslår den israelisk-amerikanske folkmordsforskaren Raz Segal att fallet Gaza utgör ett \textit{textbook case of genocide}, det vill säga ett skolboksexempel på folkmord.

\begin{quote}
“This is a textbook case of genocide. [...] Israel’s intent to commit genocide is not hidden. It’s completely out in the open, declared by high-level officials, ministers, the president, and military leaders.”
\end{quote}

\footnote{\url{https://www.youtube.com/watch?v=AUeEnjULHe0}; se även: \url{https://www.democracynow.org/2023/10/16/raz_segal_textbook_case_of_genocide}}

Detta är inte ett isolerat utlåtande från en enskild akademiker – det vilar på en bred och växande internationell forskarkonsensus.


\subsection{Konsenus råder därför bland folkmordsexperter världen över}
%\addcontentsline{toc}{subsection}{Konsenus råder därför bland folkmordsexperter världen över}


Konsenus råder därför bland folkmordsexperter världen över, inklusive israeliska som tidigare motsatt sig detta.\footnote{\url{https://ifpnews.com/top-scholars-israel-genocide-gaza/}}

Enligt en granskning från den nederländska dagstidningen \textit{NRC} råder idag närmast total enighet bland världens ledande folkmordsexperter om att Israels agerande i Gaza utgör folkmord. NRC:s undersökning bygger på intervjuer med sju framstående forskare från sex olika länder och kompletteras med en genomgång av den senaste forskningen inom området. 

Israels folkmordsforskare Raz Segal, verksam vid Stockton University i USA, uttalar sig utan förbehåll:

\begin{quote}
“Can I name someone whose work I respect who does not think it is genocide? No, there is no counterargument that takes into account all the evidence,” Israeli researcher Raz Segal told NRC.
\end{quote}

Segals uttalande är centralt: det handlar inte bara om hans egen ståndpunkt, utan om frånvaron av kvalificerad, evidensbaserad opposition inom det akademiska fältet.

Denna uppfattning bekräftas av professor Ugur Umit Ungor, verksam vid Amsterdams universitet och NIOD Institute for War, Holocaust and Genocide Studies:

\begin{quote}
“While there are certainly researchers who say it is not genocide, I don’t know them,” said Professor Ugur Umit Ungor.
\end{quote}

Ungor framhåller alltså inte bara konsensus bland kollegor, utan att han personligen inte känner till en enda sakkunnig inom fältet som motsätter sig bedömningen.

För att styrka detta empiriskt genomförde NRC en genomgång av 25 vetenskapliga artiklar publicerade i den ledande tidskriften \textit{Journal of Genocide Research}. Resultatet är tydligt:

\begin{quote}
“All eight academics from the field of genocide studies see genocide or at least genocidal violence in Gaza.”
\end{quote}

Detta är anmärkningsvärt i ett akademiskt fält som annars präglas av metodologisk försiktighet och där det ofta finns olika tolkningsmodeller. NRC:s artikel sammanfattar:

\begin{quote}
“Contrary to public opinion, leading genocide researchers are surprisingly unanimous: the Benjamin Netanyahu government, they say, is in that process – according to the majority, even in its final stages. That is why most researchers no longer speak only of ‘genocidal violence’, but of ‘genocide’.”
\end{quote}

Särskilt intressant är att även forskare som tidigare varit skeptiska till folkmordsbegreppet i detta sammanhang nu ändrat uppfattning. Ett exempel är Shmuel Lederman från Open University of Israel:

\begin{quote}
“Lederman initially opposed the use of the genocide label. However, following Prime Minister Benjamin Netanyahu’s dismissal of the ICJ’s ruling, the continued closure of land crossings to Gaza and a letter by 99 US health workers stating that the death toll in Gaza exceeded 100,000, he was convinced that Israel’s actions do in fact constitute genocide.”
\end{quote}

Det faktum att till och med tidigare kritiker har ändrat ståndpunkt understryker allvaret i utvecklingen – när de juridiska och humanitära indikatorerna blivit för starka för att ignoreras.

Slutligen bör ett särskilt fokus riktas mot Melanie O’Brien, ordförande för \textit{International Association of Genocide Scholars}. Det är inte bara hennes bedömning som är intressant, utan även hennes institutionella roll. Forskare i sådana ledarpositioner tillskrivs vanligen särskild integritet, metodologisk skärpa och fältets förtroende:

\begin{quote}
“Melanie O’Brien, president of the International Association of Genocide Scholars, told NRC that Israel’s deliberate denial of food, water, shelter and sanitation was the key factor in her determination that the military campaign was a genocide.”
\end{quote}

O’Brien lyfter här fram att även utan direkt dödande kan förstörelsen av livets nödvändiga förutsättningar – såsom vatten, föda, skydd och sanitet – utgöra folkmord i folkrättslig mening. Det handlar om att skapa levnadsförhållanden som syftar till att fysiskt förinta en grupp, helt eller delvis.

Sammanfattningsvis visar detta avsnitt att det inte längre råder någon meningsfull akademisk oenighet om huruvida Israels agerande i Gaza utgör folkmord. De mest framstående experterna i världen, inklusive israeliska forskare och ledande företrädare för internationella sammanslutningar, är överens om att kriterierna för folkmord enligt FN:s konvention är uppfyllda.

Men folkrättens ansvarsfördelning stannar inte vid beslutsfattare, soldater eller domstolens expertutlåtanden. Enligt folkrätten kan även stater hållas ansvariga för ett folkmord när det visar sig att hela samhällen – inte bara en ledning – aktivt eller passivt bär, sprider eller legitimerar tanken att en skyddad grupp ska förintas.


Det är därför folkrättsligt avgörande att undersöka vad det israeliska folket anser. 
\textbf{Om föreställningen om palestiniernas eliminering har förvandlats från marginell avvikelse till majoritetsuppfattning, medför det en förskjutning i folkrättens tillämpning från individuell till strukturell skuld.} 
Det implicerar inte att varje individ bär juridiskt ansvar – men att statens samhälleliga infrastruktur genomsyras av stöd för utplåningstanken.


Mot bakgrund av detta blir nästa fråga oundviklig: Har folkmordsideologin förankrats i samhället i stort? Om tanken på palestiniernas utplåning inte längre möter motstånd, utan snarare betraktas som legitim eller nödvändig, implicerar det en samhällsdiskurs som i sig bär folkmordsretorik – något som måste återspeglas i gallupundersökningar.


Det är i detta sammanhang vi nu vänder oss till de mest aktuella opinionsundersökningarna.


\subsection{Opinionsundersökningar visar utbrett stöd för eliminatorisk politik mot palestinier}
%\addcontentsline{toc}{subsection}{Opinionsundersökningar visar utbrett stöd för eliminatorisk politik mot palestinier}

Folkrättens förbud mot folkmord gäller inte enbart dem som bär vapen eller ger order. En stat kan hållas ansvarig när dess samhälle utvecklar en bred acceptans – eller till och med entusiasm – för systematiskt våld mot en skyddad grupp. Det är därför av avgörande betydelse att granska de samhälleliga attityderna. Om det visar sig att idén om palestiniernas utplåning inte längre är en marginalföreteelse, utan snarare en brett omfattad samhällsåsikt, förändras den folkrättsliga bedömningen i grunden.

Den israeliske journalisten Jonathan Ofir rapporterar i \textit{Mondoweiss} att eliminatoriskt tänkande – det vill säga uppfattningen att palestinier bör utplånas – inte längre är en ”fringe opinion”, utan i dag delas av en majoritet av Israels judiska befolkning.\footnote{\url{https://mondoweiss.net/2025/05/poll-shows-israeli-belief-that-palestinians-should-be-eradicated-is-no-longer-a-fringe-opinion/}}

Bakgrunden är en opinionsundersökning genomförd vid Penn State University, där 65\% av judiska israeler instämmer i att det finns en samtida inkarnation av det bibliska folket Amalek – den grupp som enligt Gamla testamentet skulle utplånas ”till sista barnet och sista oxen”. 

\begin{quote}
“One of the poll’s results shows that 65\% of Jewish Israelis agree that a present-day incarnation of the ‘Amalek’ exists.”
\end{quote}

Av dem som instämmer i detta menar 93\% att det även gäller dagens palestinier. Detta innebär att en absolut majoritet stöder förintande åtgärder i religiöst motiverad bemärkelse.

\begin{quote}
“93\% of those who believe in that ‘reincarnation’ of the Amalek also answer that ‘eradicating its memory’ also applies to Palestinians today.”
\end{quote}

Men stödet sträcker sig bortom religiösa referensramar. På frågan om huruvida Israels armé bör agera såsom israeliterna gjorde vid intagandet av Jeriko – där alla invånare dödades – svarade 47\% jakande. Än mer alarmerande är att 82\% stödjer tvångsfördrivning av hela Gazas befolkning, och 56\% vill även fördriva palestinska medborgare inom Israel.

Detta är inte enbart stöd för etnisk rensning – det är ett uttryck för etablerad beredskap för folkmord.

Ofir noterar att detta tankegods inte är begränsat till ultraortodoxa eller högerextrema grupper. Tvärtom är stödet mycket utbrett även bland sekulära judar:

\begin{quote}
“On the question of forced expulsion from the Gaza Strip, the percentage among seculars is 70\%. Among ‘traditional,’ it’s 91\%, and among the ‘religious’ ultra-orthodox, or Haredim, a whopping 97\%.”
\end{quote}

Särskilt oroande är att yngre judiska israeler uppvisar de mest folkmordstoleranta attityderna. Endast 9\% av israeler under 40 år förkastar idéerna om utplåning eller tvångsfördrivning av palestinier. Det är just denna åldersgrupp som också utgör ryggraden i det israeliska militärapparatet.

\begin{quote}
“Jewish Israelis under 40 are more genocidal. Only 9\% of those under 40 rejected the ideas of expulsion and extermination presented to them.”
\end{quote}

Med andra ord: i den demografiska grupp som idag bär vapen i Gaza, instämmer 91\% i idén att palestinierna bör utplånas.

Sammanfattningsvis visar dessa data att folkmordsideologin inte är isolerad till ledarskiktet – den delas av det israeliska samhället i stort. Den har stöd från både sekulära och religiösa grupper, och frodas bland unga. Detta implicerar inte att varje individ aktivt förespråkar folkmord, men det demonstrerar att de sociala, politiska och militära förutsättningarna för ett folkmord är uppfyllda. Det rör sig inte längre om en extrem minoritet – det är en dominerande samhällshållning.





\subsection{Exempel på offentlig uppvigling till folkmord enligt folkmordskonventionen från civila och politiska företrädare}
%\addcontentsline{toc}{subsection}{Exempel på offentlig uppvigling till folkmord enligt folkmordskonventionen från civila och politiska företrädare}

Som komplettering till de kvantitativa opinionsundersökningarna finns ett växande antal videoklipp och dokumenterade uttalanden, där israeler – både privatpersoner och folkvalda företrädare – uttrycker stöd för utplåning av Gaza eller hela det palestinska folket.

Dessa uttalanden är inte avgörande för att fastställa statsansvar i sig, men de utgör \textbf{empiriskt stöd} för att eliminatoriskt tänkande genomsyrar såväl civil som politisk diskurs. De visar på en folklig och institutionell acceptans av extrema åtgärder, vilket förstärker bedömningen att Israel som stat befinner sig i ett strukturellt folkrättsbrott\footnote{Se FN:s konvention om förebyggande och bestraffning av brottet folkmord (1948), särskilt artikel III c: “direct and public incitement to commit genocide”}.

\vspace{1em}
\textbf{Exempel (videoklipp och källor):}
\begin{itemize}
    \item \textbf{100 israeliska läkare kräver att Gazas sjukhus bombas:} I ett öppet brev – publicerat i israeliska medier – uppmanar cirka 100 läkare och professorer det israeliska försvaret att \textit{"förinta terroristnästen och Hamas-högkvarter i Gazas sjukhus"}. Brevet beskriver sjukhusen som legitima mål och förklarar att civila som befinner sig där inte har något skydd, eftersom "de valt att omvandla sjukhus till terrorbostäder".\footnote{\url{https://www.middleeasteye.net/news/israel-palestine-war-doctors-call-gaza-hospitals-bombed}, samt den hebreiska originalartikeln: \url{https://www.bhol.co.il/news/1613301}}

    \item \textbf{Ung kvinna i grupp som blockerar hjälpsändningar till Gaza:} \textit{"When there is a war it doesn't matter who your enemy is, you need to destroy their offspring to prevent them from creating more offspring."} — Uttalat i samband med en organiserad blockad där demonstranter hindrade lastbilar med förnödenheter från att nå Gaza. \footnote{\url{https://x.com/s_m_marandi/status/1925722444041502937}}

    \item \textbf{Revital "Tally" Gotliv, Knessetledamot (Likud):} \textit{"Doomsday weapon! [...] Crushing and flattening Gaza… without mercy!"} — Krävde kärnvapenanvändning mot Gaza. \footnote{\url{https://news.antiwar.com/2023/10/10/us-israeli-lawmakers-call-for-genocide-of-palestinians-in-gaza/}}

    \item \textbf{Israelisk kvinna i Rumble-intervju:} \textit{"The only innocent people that are in Gaza now are the 229 hostages... Once they go back to Israel, we will bomb Shifa Hospital... and kill them all."} — Uttalat inför sonens avfärd till Gaza. \footnote{\url{https://rumble.com/v3t0h2t-israeli-women-calls-for-genocide-with-warren-thornton.html}}

\item \textbf{Reshit Guela – ung kvinna i intervju med Sky News:} \textit{"We should kill them, every last one of them."} — Uttalande från en ung israelisk kvinna, dotter till en soldat som tjänstgjort i Gaza. Citatet fälldes under en Sky News-intervju i samband med en konferens om bosättning av Gaza, där hon även hänvisade till Toran och soldaters offer som skäl att ”fullfölja vad de startat”. Intervjun sändes i oktober 2024. \footnote{\url{https://www.informationliberation.com/?id=64705}}


    \item \textbf{Premiärminister Benjamin Netanyahu:} \textit{"You must remember what Amalek has done to you, says our Holy Bible, and we are acting accordingly."} — Referens till 1 Samuelsboken 15:3, där Gud befaller folkmord. \footnote{\url{https://x.com/kthalps/status/1723538810221338973}}

    \item \textbf{Moshe Feiglin, f.d. Knessetledamot:} \textit{"This is not about Hamas. This is a biblical moment – flatten Gaza entirely."} — Krävde fullständig utplåning av Gaza. \footnote{\url{https://x.com/KintsugiMuslim/status/1723144789292371971}}

    \item \textbf{Giora Eiland, f.d. nationell säkerhetsrådgivare:} I en artikel i \textit{Yedioth Ahronoth} föreslår Eiland att civilbefolkningen i Gaza ska tvingas på flykt genom att Gaza görs obeboeligt. Han skriver: \textit{“Gaza should be smaller – the only way to shorten the war is by creating a humanitarian crisis that will make residents flee.”} Han menar att Israel inte har något ansvar för befolkningens välfärd, föreslår att inga matleveranser släpps in, och avslutar: \textit{“The State of Israel must not supply Gaza with one watt of electricity, one drop of water or one liter of fuel.”} Uttalandet, som innebär kollektiv bestraffning och tvångsfördrivning, har citerats av flera människorättsorganisationer som exempel på krigsbrottsligt uppviglande. \footnote{\url{https://www.haaretz.com/israel-news/2023-10-12/ty-article-opinion/.premium/former-israeli-general-says-flatten-gaza-create-a-humanitarian-crisis/0000018b-9c66-d21e-ab8f-bef71a580000}}


    \item \textbf{Demonstrant i New York:} \textit{"We got to wipe them off... flatten them like a parking lot."} — Offentligt uttryckt i intervju. \footnote{\url{https://www.youtube.com/watch?v=qgAkv4SdCJ0}}

    \item \textbf{Lilia Sandler (sjuksköterska, DC):} \textit{"I love it! I don’t think they’re dying enough. Death to Gaza!"} — Filmas när hon river ner fredsbudskap. \footnote{\url{https://www.youtube.com/watch?v=nbkm2v5Hjos}}

    \item \textbf{President Isaac Herzog:} \textit{"It’s not true this rhetoric about civilians not aware..."} — Avvisar idén om civila som oskyldiga. \footnote{\url{https://web.archive.org/web/20231015190209/https://en-volve.com/2023/10/14/watch-israeli-president-claims-there-are-no-innocents-in-gaza-including-civilians/}}

    \item[] \textbf{Systematisk dokumentation – 107 fall:}
    Abu Bakr Hussain har sammanställt 107 fall av uppvigling till folkmord:
    \begin{itemize}
        \item \textbf{Del 1:} \url{https://x.com/KintsugiMuslim/status/1723144789292371971}
        \item \textbf{Del 2:} \url{https://x.com/KintsugiMuslim/status/1723145632594952606}
        \item \textbf{Samlad version:} \url{https://x.com/KintsugiMuslim/status/1723145848895230432}
    \end{itemize}

    \item \textbf{Florida-politiker:} \textit{"All of them."} — Svarade på fråga om hur många fler palestinier som bör dödas. \footnote{\url{https://www.youtube.com/watch?v=qdIoOh3S6PM}}

    \item \textbf{Tzipi Hotovely, Israels ambassadör i Storbritannien:} \textit{"I have zero empathy... they committed the crime of attacking Israel."} — Förnekade oskyldiga civila i Gaza. \footnote{\url{https://www.youtube.com/watch?v=n15yKYHTwSw}}

    \item \textbf{Demonstration i New York:} Citat från flera deltagare:
    \begin{quote}
    \textit{"Kill all Palestinians. Not one left."}\\
    \textit{"Wipe them flat off the map."}\\
    \textit{"I’m not stopping until all Arabs are wiped out."}
    \end{quote}
    \footnote{\url{https://www.youtube.com/watch?v=qgAkv4SdCJ0}}

    \item \textbf{Avi Dichter, jordbruksminister:} \textit{"We are now rolling out the Gaza Nakba."} — Bekräftade ny fördrivningskampanj. \footnote{\url{https://electronicintifada.net/blogs/ali-abunimah/we-are-now-rolling-out-gaza-nakba-israeli-minister-announces}}

    \item \textbf{Demonstration med barn:} Barn och vuxna ropar \textit{"En bra arab är en död arab"} — uttryck för folkmordsnormer i barnuppfostran. \footnote{\url{https://x.com/DrLoupis/status/1723268444236268025}}

    \item \textbf{Advokat Nauri Nilli:} \textit{"There are no innocent civilians in Gaza. Not even the children."} — I israelisk TV. \footnote{\url{https://x.com/StopZionistHate/status/1723474723672068318}}

    \item \textbf{Ezra Yachin, 95 år, IDF-reservist:} \textit{"Erase them, their families, mothers and children."} — Uttalande till soldater inför offensiv. \footnote{\url{https://www.naturalnews.com/2023-10-22-oldest-idf-reservist-wants-to-eradicate-palestinians.html}}

\item \textbf{Kanal 14: Patriotprogram med folkmordshets som underhållning.} I det israeliska tv-programmet ”Patrioten” på Kanal 14 tävlar deltagare i att uttrycka mest extrema åsikter om vad som bör hända med Gazas befolkning. Programmet präglas av ”rörande enighet” om eliminatoriska åtgärder. I ett klipp utropas: \textit{“Skjut civila! Självklart!!”}, vilket mottas med skratt och applåder.\footnote{\url{https://www.youtube.com/shorts/JcQhwT2kJ_4}}

\item \textbf{Kanal 13: Debatt om spädbarns oskuld.} I ett inslag från mars 2025 diskuterade programledare Eyal Berkovic och Moriah Asraf med f.d. försvarsminister Moshe Ya’alon huruvida även nyfödda barn i Gaza bör betraktas som legitima mål. Berkovic argumenterade: \textit{“Det finns inga oskyldiga i Gaza. Alla är terrorister.”} Ya’alon varnade för att denna inställning legitimerar folkmord. Debatten speglar hur eliminatoriskt tänkande institutionaliserats i israelisk massmedia.\footnote{\url{https://www.informationliberation.com/?id=64915}}

\item \textbf{Bezalel Smotrich:} Israels finansminister förklarade i ett uttalande att kriget mot Gaza inte ska avslutas ”förrän Amalek är fullständigt utrotat”, en teologisk kod som i sammanhanget tolkas som en uppmaning till total förintelse av palestinier. Liknande språk har använts av premiärminister Netanyahu och andra regeringsmedlemmar.\footnote{Referenser i tidigare fotnoter, samt \url{https://x.com/infolibnews/status/1725234779090563548?}}

\item \textbf{Knesset-ledamot försvarar tortyr och sexuellt våld:} I en debatt i Knesset svarar Hanoch Milwidsky (Likud) \textit{"Yes! If he is a Nukhba everything is legitimate to do him!"} på en fråga från MK Ahmad Tibi (TA'AL) om det är legitimt att föra in ett föremål i en palestinsk fånges ändtarm. Uttalandet gjordes i kontexten av en revolt vid det ökända fånglägret Sde Teiman, där soldater vägrade låta kollegor gripas för grova övergrepp mot en palestinsk fånge. Händelsen ledde till att högerpolitiker och reservister samlades för att stoppa gripandena, och Milwidsky svarade med att utlysa en "röststrejk" i protest mot rättsprocessen.\footnote{\url{https://x.com/ireallyhateyou/status/1817904053462196523}}

\item \textbf{Hög rabbin välsignar gruppvåldtäkt av palestinsk fånge:} Meir Mazuz, en av Israels mest inflytelserika rabbiner och nära allierad till Netanyahu och hans regering, kommenterade gruppvåldtäkt av en palestinsk fånge med orden: \textit{"You beat the enemy, so what? It's all good… Don't we have the right to do it?… In any other country, they'd get medals… Don't fear the goyim."} Uttalandet följde efter att soldater misstänkta för våldtäkt välsignats offentligt, och sammanföll med våldsamma protester mot gripanden vid lägret Sde Teiman. Uttalandet illustrerar hur även extrema övergrepp försvaras inom vissa religiösa kretsar i Israel.\footnote{\url{https://x.com/davidsheen/status/1832362225052307581}}

\item \textbf{Barnkör i Israel sjunger om total utplåning:} I en video uppladdad av Israels statliga TV-bolag \textit{Kan}, sjunger israeliska barn en så kallad vänskapssång med textrader som: \textit{"Inom ett år kommer vi att utplåna alla, och sedan återvänder vi för att plöja våra fält."} Videon togs senare bort efter massiv kritik, men arkiverades. Sången skrevs av Ofer Rosenbaum, en PR-expert känd för att förespråka etnisk rensning, och används för att stärka stödet för kriget mot Gaza. Barnens "änglalika röster" kombineras här med explicit folkmordsretorik, något som enligt internationell rätt kan utgöra ett brott i sig.\footnote{\url{https://electronicintifada.net/blogs/ali-abunimah/watch-israeli-children-sing-we-will-annihilate-everyone-gaza}}

\item \textbf{Ministern dansar till barnens utplåning:} I en video från en offentlig manifestation ses ungdomar jubla, dansa och sjunga ramsan: \textit{"Gaza is a graveyard, Gaza is a graveyard. There is no school in Gaza, because there are no children left in Gaza."} Israels säkerhetsminister Itamar Ben Gvir deltar\footnote{\url{https://x.com/AngDem29/status/1725262889936671035}}

\item \textbf{“There are no innocent Palestinians”:} Den amerikanske radioprofilen Mark Levine, en neokonservativ opinionsbildare med judisk trosbekännelse, skrev på X (tidigare Twitter): \textit{“There are no ‘innocent Palestinians’.”} Uttalandet gjordes i samband med pågående bombningar i Gaza och tusentals döda barn. Uttalandet väckte starka reaktioner och jämförelser med hur omvärlden skulle reagera om någon uttryckt sig på liknande sätt om judar efter ett massmord. Sådan generaliserande retorik kan utgöra en del av en folkmordsnormaliserande diskurs.\footnote{\url{https://x.com/jakeshieldsajj/status/1732995903966040435}}

\item \textbf{Viceborgmästare: “Begrav dem levande”} — Jerusalems viceborgmästare Aryeh Yitzhak King föreslog i ett inlägg på X (senare raderat) att palestinska civila fångar i Gaza borde begravas levande: \textit{“Om det vore upp till mig, skulle jag ha skickat D-9-bulldozrar och låtit täcka dessa hundratals myror medan de fortfarande levde.”} King kallade palestinierna för “subhumans” och anspelade, likt Netanyahu, på Amalek – en biblisk fiende vars fullständiga utplåning beordras i Första Samuelsboken 15:3. Uttalandet har väckt omfattande kritik som ett explicit folkmordsuttalande från en högt uppsatt politiker med administrativt ansvar över både västra och ockuperade östra Jerusalem. \footnote{\url{https://www.middleeasteye.net/news/israel-palestine-war-politician-calls-civilians-buried-alive}}

\item \textbf{Journalist deltar i beskjutning av Gaza.} — Rotem Achihun, en reporter för israelisk statlig TV, filmades medan hon aktivt deltog i artilleribeskjutning av Gaza. På kameran ses hon skriva “Hälsningar till Gazaborna” på granater som sedan avfyras. Hon kommenterar: \textit{“Jag känner för att göra något ont mot dem.”} Händelsen följer på att en radiovärd i Israel 103 FM öppet erkänt sitt deltagande i dödandet av palestinier samtidigt som han rapporterar om det. \footnote{\url{https://x.com/davidsheen/status/1732765077625827389}}

\item \textbf{F.d. Knessetledamot: Förinta hela Gaza.} — Advokat och tidigare parlamentsledamot Danny Neuman uttryckte i israelisk TV att hela Gaza bör "förintas", att området ska jämnas med marken och dess "damm rensas bort". Han kallade samtliga invånare för terrorister och föreslog att ett nytt, säkert område för Israel borde byggas i deras ställe. Uttalandet har väckt oro över de verkliga syftena med kriget i Gaza. \footnote{\url{https://www.middleeastmonitor.com/20231213-former-knesset-member-advocates-for-exterminating-all-of-gaza/}}

\item \textbf{Rabbin och professor Dov Fischer: Gazas civila är inte oskyldiga.} — I en krönika publicerad på Israel National News förnekar Rabbi Prof. Dov Fischer, juridikprofessor och ledande religiös företrädare, att någon i Gaza kan betraktas som civil eller oskyldig. Han hävdar att befolkningen röstade fram Hamas, aktivt stödjer attacker mot judar, och därmed har förverkat varje skydd. Artikeln uttrycker en total moraliskt-politisk diskvalificering av en hel civilbefolkning, vilket ger ideologiskt stöd till eliminatoriskt våld och utgör ett varnande exempel på hur akademiska och religiösa auktoriteter kan legitimera folkrättsbrott.\footnote{\url{https://www.israelnationalnews.com/news/382016}}

\item \textbf{”America’s genocide” – USA:s medansvar för folkmord.} Enligt en genomgång från World Socialist Web Site bidrar USA inte bara med vapen och finansiering, utan även med diplomatisk täckmantel och ideologiskt stöd till Israels militära offensiv i Gaza. Artikeln citerar bland annat israeliska politiker som uttryckt önskan att Gaza ska ”se ut som Auschwitz” och som öppet förespråkar etnisk rensning. Samtidigt försvarar USA:s utrikesminister Blinken Israels agerande som självförsvar och avfärdar krav på eldupphör. Artikeln menar att folkmordet i Gaza inte bara är Israels krig, utan också ett amerikanskt krig, drivet av imperiella intressen och repressiv inrikespolitik.\footnote{\url{https://www.wsws.org/en/articles/2023/12/22/vvtj-d22.html}}

\item \textbf{”There are no civilians in Gaza” – demonisering av hela befolkningen.} I en opinionsartikel i den judiska nyhetstidningen JNS hävdar den israelisk-amerikanske kolumnisten Daniel Greenfield att Hamas inte är en avgränsad terroristgrupp utan en kulturell och familjebaserad struktur djupt rotad i Gazas samhälle. Han menar att civila hushåll gömmer gisslan och att klanbaserade nätverk fungerar som logistisk och moralisk infrastruktur för Hamas. Artikeln avfärdar begreppet ”civil” i Gaza som irrelevant och föreslår att Israel bör definiera befolkningen utifrån lojalitet – som ”fiender” eller ”neutrala” – snarare än som stridande eller icke-stridande. Detta synsätt förkastar den internationella humanitärrättens grundläggande distinktioner och rättfärdigar därigenom kollektiv bestraffning.\footnote{\url{https://www.jns.org/there-are-no-civilians-in-gaza/}}


\end{itemize}


\subsection{Avslutande psykologisk bedömning}
%\addcontentsline{toc}{subsection}{Avslutande psykologisk bedömning}

Vad som i början av denna analys framstod som politiska och militära övertramp från ett enskilt statsledarskap har i takt med materialets omfattning visat sig vara något mycket mer djuptgående. Det rör sig inte om isolerade uttryck för extremism, utan om en genomgripande samhällelig förskjutning i synen på en skyddad folkgrupp. 

Ett växande antal individer med judisk identitet – inklusive politiker, journalister, akademiker och sjukvårdspersonal – uttrycker nu öppet stöd för eliminatoriska idéer. De säger det med sina riktiga namn, inför kamera, i nationell TV, i parlament, i tidningsspalter och på sociala medier. Detta sker utan skam, utan fördömande från deras egna institutioner och ofta utan några rättsliga eller sociala konsekvenser.

Detta avslöjar att folkmordsretoriken inte längre behöver maskeras. Den är inte längre skamfylld eller socialt sanktionerad. Tvärtom – den framförs som en rimlig, rationell och i vissa fall religiöst föreskriven ståndpunkt.


Det är dessutom tydligt att denna ideologi:
\begin{itemize}
    \item inte är bunden till Israel som geografisk plats,
    \item inte begränsas till militära eller teologiska kretsar,
    \item inte förutsätter låg utbildning eller social marginalisering – men vilar ofta på ideologiska villfarelser, såsom föreställningen om en bibliskt sanktionerad rätt till land eller existensen av en judisk ras i genetisk mening.
\end{itemize}


Tvärtom uttrycks uppmaningar till utrotning av välutbildade, resursstarka individer – med hög samhällsstatus – vilket visar att tankefiguren om palestiniernas obotliga ondska har internaliserats i breda samhällsskikt, oberoende av utbildningsnivå eller politisk gren.

Att sådana yttranden inte möts med institutionell bestraffning – som i fallet med senator Michelle Salzman i Florida – visar att tyst acceptans nu har blivit normen\footnote{Andrew Mitrovica, \textit{Getting away with a call to genocide in Gaza}, Al Jazeera, 20 nov 2023. \url{https://www.aljazeera.com/opinions/2023/11/20/all-of-them}}. Det är inte bara den som hatar som bär ansvar – det är också den som tiger.

I psykologiska termer handlar detta om moralisk desensibilisering och det Bandura kallat \textit{“moral disengagement”} – där förstörelse inte längre ses som ett moraliskt problem utan som ett nödvändigt “försvar”. Begrepp som “no ceasefire” fungerar som kodspråk för fortsatt förintelse utan samvetsansvar.

Det är också ett exempel på hur hatideologier institutionaliseras – när samhällets struktur (politik, religion, media och utbildning) inte längre bara tolererar, utan bekräftar och sprider eliminatoriska normer. När folkmordsideologi blir “förnuftig ståndpunkt” är folkmord inte längre bara en risk. Det är en process.

Det är därför denna promemoria inte bör tolkas som en moralisk appell, utan som ett folkrättsligt underlag för handling. Sveriges skyldigheter enligt folkmordskonventionen omfattar inte bara straff – utan \textbf{förhindrande}. Och det finns ingenting mer preventivt än att känna igen när ett samhälle befinner sig mitt i en psykologisk och retorisk normalisering av det mest förbjudna brottet i mänsklighetens historia.


\subsubsection*{En tweet som sammanfattar den psykologiska strukturen}
%\addcontentsline{toc}{subsubsection}{En tweet som sammanfattar den psykologiska strukturen}

Den israeliske journalisten Gideon Levy har i flera sammanhang, bland annat i föreläsningen \textit{The Zionist Tango}\footnote{\url{https://www.youtube.com/watch?v=JQS-_9K5-Dk&t=1022s}}, identifierat tre djupgående psykologiska föreställningar som bär upp den israeliska självbilden – men som, vilket denna analys visar, även förekommer globalt bland personer med stark judisk identitet oavsett geografisk hemvist eller religiös livsåskådning:

\begin{enumerate}
    \item \textbf{Föreställningen om att vara ett utvalt folk} – vilket ger en upplevd rätt att omdefiniera internationell rätt och etik.
    \item \textbf{Offerrollen som moralisk immunisering} – där även aggressivitet och dödande tolkas som defensiva handlingar.
    \item \textbf{Avhumanisering av palestinier} – som beskrivs som fundamentalt onda, irrationella och ovärdiga skydd.
\end{enumerate}

Ett tydligt exempel på hur dessa tre föreställningar samverkar finner vi i ett antal inlägg från den amerikanske barnläkaren Dr. Darren Klugman, verksam vid Johns Hopkins Hospital i Baltimore\footnote{Andrew Mitrovica, \textit{Getting away with a call to genocide in Gaza}, Al Jazeera, 20 nov 2023. \url{https://www.aljazeera.com/opinions/2023/11/20/all-of-them}}. På sociala medier beskrev han palestinier som:

\begin{quote}
\textit{“barbaric,” “savage,” and “blood thirsty, morally depraved animals who want nothing short of every inch of Israel and all Jews dead.”}
\end{quote}

Han fortsatte med att förorda massfördrivning:
\begin{quote}
\textit{“There is lots of sand for Palestinians in Sinai which Israel gave to Egypt.”}
\end{quote}

Och avslutade med ett religiöst färgat godkännande av folkmord:
\begin{quote}
\textit{“G-d willing.”}
\end{quote}

Här ryms hela Gideon Levys modell:

\begin{itemize}
    \item \textbf{Avhumaniseringen} är explicit i beskrivningen av palestinier som \textit{“morally depraved animals”}, ett klassiskt retoriskt mönster i folkmordspropaganda där den andra parten fråntas mänskliga egenskaper. Det möjliggör handlingar som annars vore otänkbara.
    
    \item \textbf{Utvaldheten} manifesteras i idén att Gaza kan “reclaimas” som en legitim rätt, trots att området är befolkat av andra människor. Att tala om “återtagande” av Gaza är inte en neutral territoriell term – det bygger på en föreställning om gudagiven rätt, historisk oförrätt och kollektiv återlösning. Det är inte bara en politisk anspråksrätt, utan en psykologisk upplevelse av exklusiv moralisk prioritet. Världen är inte bara tyst – den ska helst förstå att detta undantag är berättigat.

    \item \textbf{Den religiösa legitimeringen} förstärks av uttrycket \textit{“G-d willing”}, som inte bara fungerar som förstärkare, utan som moraliskt ankare: om Gud vill det, så kan det inte vara fel. Även om detta uttrycks i angliciserad, till synes lågintensiv form, är det en kodifiering av helgat våld.

    \item \textbf{Offerrollen} återfinns i den underliggande logiken: palestinierna är inte bara farliga – de vill ha “every inch of Israel and all Jews dead”. Det är alltså inte vi som angriper – vi försvarar vår existens. Det psykologiska syftet är att vända på rollerna: förövaren blir beskyddare, och offret blir det verkliga hotet.
\end{itemize}

Det är särskilt den andra pelaren i Levys modell – föreställningen om den exklusiva offerrollen – som förtjänar särskilt fokus. Levy framhåller att Israel är unikt i världshistorien: aldrig tidigare har en ockuperande makt lyckats etablera en självbild där den inte bara ser sig som ett offer bland andra, utan som det enda moraliskt relevanta offret. Inte bara i den pågående konflikten – utan i hela världen, genom hela historien. En sådan självbild lämnar inget utrymme för andras lidande att erkännas, än mindre att bemötas.


Detta synsätt är inte ett retoriskt knep utan en genuint internaliserad övertygelse. Det bygger på en djup historisk sedimentering där Förintelsen utgör inte bara ett trauma, utan ett absolut moraliskt paradigm. Inget lidande i världen har varit större, renare eller mer oförskyllt än det judiska lidandet under andra världskriget – och eftersom detta brott utfördes av omvärlden mot ”Guds egendomsfolk” kan ingen annan grupp i eftervärlden riktigt mäta sig med det judiska folkets sårbarhet. 

Därför blir det, i detta psykologiska raster, möjligt att döda barn i Gaza samtidigt som man ser sig själv som ytterst sårbar. Offerrollen har här inte blivit mindre med statsmakt, kärnvapen eller territoriell kontroll – tvärtom. Den har institutionaliserats som nationell identitet.

I fallet med Dr. Klugman är detta tydligt. Hans beskrivning av palestinier som bestialiska fiender som vill döda alla judar är inte en rationell säkerhetsbedömning – det är en spegling av en existentiell självbild där varje avvikelse från judisk dominans tolkas som hot om förintelse. För att det judiska folket ska existera tryggt, måste andra – i detta fall palestinierna – förnekas samma mänskliga status. Hans språkbruk är inte bara hatfyllt – det är psykologiskt ritualiserat, ett uttryck för att bekräfta vem som har rätt att finnas och vem som inte har det.

Det är därför Gideon Levy har rätt: så länge denna psykologiska struktur förblir intakt – utvaldhet, exklusiv offeridentitet, avhumanisering – kommer inga materiella förändringar eller förhandlingar att kunna leda till verklig fred. Och som denna promemoria visar: strukturen har inte bara förblivit intakt, den har blivit globaliserad.


Det mest alarmerande är inte att en enskild individ uttrycker detta, utan att han:
\begin{itemize}
    \item är högt utbildad och yrkesverksam som barnläkare vid ett av världens mest prestigefyllda sjukhus,
    \item gör uttalandena öppet, i sitt eget namn,
    \item inledningsvis inte möter någon bred offentlig fördömelse.
\end{itemize}

Detta visar att den psykologiska strukturen som Levy beskriver inte är beroende av israelisk nationalitet eller ortodox religiös tro. Den har blivit ett globalt, transnationellt tankesystem där sekulära, moderna, ofta liberala personer kan bära på samma kognitiva byggstenar som religiösa extremister – bara uttryckta i annan tonart.

Därmed har vi inte bara identifierat en ideologi. Vi har identifierat en inre struktur för hur denna ideologi rationaliseras och reproduceras. Det är denna struktur som måste exponeras, ifrågasättas och – i enlighet med folkmordskonventionens förebyggande syfte – desarmeras innan den fullbordar sin destruktiva logik.


\subsubsection*{Prof. Raz Segal: Från avsikt till akademisk legitimitet – ett ekosystem för undantagstillståndet}
%\addcontentsline{toc}{subsubsection}{Prof. Raz Segal: Från avsikt till akademisk legitimitet – ett ekosystem för undantagstillståndet}

I en uppmärksammad intervju med Owen Jones\footnote{\url{https://www.youtube.com/watch?v=VqHoRP9u5Bk}}, publicerad den 3 juni 2024 – åtta månader efter attackerna den 7 oktober – förklarar professor Raz Segal hur Israels språkbruk och agerande inte bara bär tecken på folkmordsavsikt enligt konventionens definition, utan dessutom faller in i ett historiskt mönster av kolonial dominans legitimerad genom rättssystem och forskning. Han noterar bland annat följande:


\begin{itemize}
\item Israels avsikt har uttryckts öppet sedan dag ett. Språket om “\textit{human animals}”, uppmaningar till “\textit{hell}” och kollektiv bestraffning genom belägring utgör inte bara retorik – utan en faktisk, planerad verkställighet.
\item Han pekar på hur internationell humanitär rätt vänds mot sin egen intention, genom att “säkra evakueringsvägar” används som måltavlor. Det är vad Segal kallar en “weaponization” av juridiken – ett brott med folkrättens anda.
\item Den inledande svälten, bombningarna, targeting av sjukhus och 15 000 döda barn utgör, enligt Segal, “förhållanden avsedda att förgöra gruppen i sin helhet eller delvis”, ett nyckelbegrepp i folkmordskonventionen.
\item Han kopplar också avsikten till fantasibilden om ett kolonialt krig: israeliska ledare föreställer sig att de utkämpar ett civilisatoriskt krig mot “barbarer”, “nazister” eller “djur” – vilket skapar kognitiv frihet att avvika från normala rättsliga och moraliska ramar.
\end{itemize}

Segal konstaterar därefter något som direkt återknyter till Gideon Levys psykologiska modell: Denna koloniala världsbild är inte bara ett inhemskt fenomen i Israel utan bekräftas och legitimeras i västvärlden genom två avgörande mekanismer:

\begin{enumerate}
\item Det internationella rättssystemet, inklusive ICC och ICJ, är fortfarande präglat av kolonial maktstruktur. Det har historiskt skyddat västmakters intressen, och är därför långsamt eller oförmöget att ingripa mot en “klientstat” som Israel.
\item Forskningen kring folkmord har strukturerats kring vad Segal tidigare kallat en “helig särställning för Förintelsen”. Namnvalet “\textit{Holocaust and Genocide Studies}” speglar inte en neutral akademisk kronologi, utan en ideologisk hierarki. Den bekräftar och förstärker uppfattningen att det judiska lidandet är historiskt unikt och utan jämförelse, vilket effektivt hindrar en vetenskaplig analys av Israels agerande inom samma kategori.
\end{enumerate}

\noindent
Precis som Gideon Levy beskrev i sin modell, så formas ett psykologiskt fundament där:
\begin{itemize}
\item Man är det utvalda folket,
\item Det enda verkliga offret i historien,
\item Motståndaren (palestinierna) är inte fullt mänsklig,
\item Och hela det västerländska juridiska och akademiska ramverket är konstruerat för att bekräfta denna självbild.
\end{itemize}

\noindent
Detta utgör vad som i vetenskapsteorin kallas en positiv återkopplingsslinga. När det akademiska fältet och den juridiska infrastrukturen bekräftar den egna särställningen, upphör all självkritik – vilket möjliggör systematisk grymhet utan skuldmedvetande. Det är detta som gör situationen så farlig.



\subsubsection*{Den exklusiva offerrollen som moraliskt imperativ}\footnote{Begreppet ”moraliskt imperativ” anspelar på Immanuel Kants kategoriska imperativ: ett moraliskt påbud som inte är beroende av syfte eller konsekvens, utan som gäller ovillkorligt. I denna kontext innebär det att offeridentiteten blir något absolut – inte ett val, utan en skyldighet.}
%\addcontentsline{toc}{subsubsection}{Den exklusiva offerrollen som moraliskt imperativ}

Det är särskilt den andra pelaren i Levys modell – föreställningen om den exklusiva offerrollen – som förtjänar särskilt fokus. Vad Levy säger, och vad Segal bekräftar från ett akademiskt perspektiv, är detta: Israel utgör ett historiskt unikum – aldrig tidigare har en ockupationsmakt lyckats etablera och exportera en självbild där den uppfattas som det enda moraliskt relevanta offret. Inte bara i den aktuella konflikten, utan i världshistorien.

Detta är inte ett retoriskt verktyg utan en psykologisk struktur, förankrad i en teologisk och historisk självförståelse där Förintelsen inte bara var det värsta brottet i mänsklighetens historia – utan det enda verkligt oförskyllt begångna. Det betraktas som utfört av yttervärlden mot 'Guds egendomsfolk' – och ses, enligt detta raster, som utan jämförelse. Därför, menar Levy, erkänns inget annat folks lidande fullt ut – och framför allt inte palestiniernas.

I detta raster är offerstatusen inte ett tillstånd, utan en moralisk rättighet. Den rättigheten måste försvaras, även med våld. Det är inte trots barnadödande som man förblir offer – utan genom det. För i den psykologiska logiken är det inte handlingen som definierar moralen, utan identiteten. Vem du är avgör om du har rätt att döda.

Detta får långtgående konsekvenser: varje jämförelse reduceras till antisemitism, varje palestinsk överlevare blir ett kognitivt hot, och varje försök att applicera folkrätt blir en kränkning. Vad vi står inför är därför inte bara en etisk kris – utan ett ideologiskt vakuum där all moralisk prioritet tilldelas en enda grupp. I en sådan struktur finns ingen fred att förhandla om – bara kapitulation.

\medskip
\paragraph*{Rättvisa som motstånd – ledarskapets ansvar}

Men att erkänna detta ideologiska vakuum innebär inte att ge upp på idén om rättvisa. Tvärtom. Det är inte ett uttryck för humanism att undvika konflikt eller att stryka en part medhårs. Verklig anständighet kräver vuxet ledarskap – ett ledarskap som inte väjer för obekväma sanningar, utan som orubbligt försvarar principen om lika värde och rätt för alla.

Att insistera på rättsstatens principer, även när det smärtar, är inte ett angrepp på judisk identitet – det är ett skydd för mänskligheten. Det är först när rättvisa blir universell som trygghet kan bli möjlig – även för dem som i dag identifierar sig genom den judiska erfarenheten. Det är just därför som folkmordskonventionen antogs – inte för att skydda en moralisk särställning, utan för att förhindra framtida brott, oavsett gärningsman.



