%filnamn: kondenserad_sammanfattning_officiella_positioner.tex
%SYFTE: Att kompakt sammanfatta allt i regeringens_officiella_positioner.tex för maximal överblick

%%Sammanfattning Sammanfattning av regeringens uttalanden såsom rättsliga handlingar
\subsection{Sammanfattning av regeringens uttalanden såsom rättsliga handlingar}
\label{subsec:regeringens_positioner}

\subsubsection{Kritiska uttalanden: maj 2025}
\begin{tabular}{p{0.12\textwidth}p{0.85\textwidth}}
\textbf{Datum} & \textbf{Uttalande \& rättslig kontext} \\
\hline
23 maj & \textit{Debattartikel:} "Israel har rätt att försvara sig – men den rätten måste utövas i enlighet med folkrätten" \\
& \footnotesize\textit{Rättslig kontext:} ICJ rådgivande yttrande 2004 (muren) \& 2024 (ockupationen) fastslår att artikel 51 FN-stadga ej tillämplig \\
\hline
27 maj & \textit{UD-uttalande:} "Terroristorganisationen Hamas ansvar för den aktuella situationen väger tungt" \\
& \footnotesize\textit{Rättslig kontext:} Bröt mot neutralitetsprincipen (Hagakonventionen 1907) och principen om individuellt ansvar (Nürnbergartikel 6) \\
\end{tabular}

\subsubsection{Rättslig desinformation om självförsvarsrätten}
\begin{itemize}
\item \textbf{Faktamässig felaktighet}: Israel är ockupationsmakt (ICJ 2004 \& 2024), inte angripen stat
\item \textbf{Rättslig konsekvens}: Skapar normativ förskjutning från \textit{legalitet} till \textit{proportionalitet}
\item \textbf{Jämförelse}:Analogt med att hävda att en fängelsevakt har rätt att försvara sig mot en intagen — trots att denne har kontroll över situationen, skyldighet att värna liv och tillgång till överlägsna medel.

Självförsvarsrätten existerar, men den är strikt reglerad. En vakt som skjuter en obeväpnad ungdomsbrottsling i huvudet efter ett inbrott, när faran inte längre är överhängande, kommer inte att frias.

\item \textbf{Jämförelse:} På samma sätt kan en ockupationsmakt inte åberopa självförsvarsrätt mot en befolkning som den själv:
  \begin{itemize}
    \item förvägrar en politisk uppgörelse enligt FN:s säkerhetsrådsresolution 242,
    \item håller långvarigt instängd och isolerad,
    \item kontrollerar genom militär dominans,
    \item bär ett juridiskt skyddsansvar gentemot.
  \end{itemize}

\end{itemize}

\subsubsection{Selektiv demonisering som folkrättsbrott}
\textbf{Asymmetrisk tillämpning av 'terrorist'-begreppet:}
\begin{itemize}
\item \textbf{Hamas}: Kategoriskt terroriststämpel trots:
  \begin{itemize}
  \item Folkvald regering (2006)
  \item Accept av FN-resolution 242 (1967-gränser)\footnote{\url{https://www.972mag.com/hamas-fatah-elections-israel-arrogance/}}
  \item Intern repression av extrema och kriminella grupper\footnote{\url{https://www.newsweek.com/hamas-arresting-and-torturing-jihadis-prevent-war-israel-752108}} \footnote{\url{https://www.middleeasteye.net/news/israel-palestine-hamas-arrests-two-rocket-fire}} \footnote{\url{https://www.israelnationalnews.com/news/166785}} \footnote{\url{https://www.israelnationalnews.com/news/259035}}
  \end{itemize}
\item \textbf{Israeliska aktörer}: Ingen kritik av:
  \begin{itemize}
  \item IDF:s "Hannibal-direktiv" 7 oktober 2023\footnote{\url{https://thegrayzone.com/2024/06/21/israeli-army-friendly-fire-october-7/}} \footnote{\url{https://thegrayzone.com/2025/02/25/bibas-israeli-govt-propaganda-hostage-killings/}} \footnote{\url{https://thegrayzone.com/2023/11/21/haaretz-grayzone-conspiracy-israeli-festivalgoers/}} \footnote{\url{https://thegrayzone.com/2023/10/27/israels-military-shelled-burning-tanks-helicopters/}}
  \item Samarbete med kriminella klaner (Grayzone 2025)\footnote{\url{https://thegrayzone.com/2025/06/05/israel-arming-isis-gang-gaza/}}
  \end{itemize}
\end{itemize}
\lagrumsinline{Genèvekonvention IV, artikel 33\quad Förbud mot kollektiv bestraffning}

\subsubsection{Underlåtenhetsansvar enligt svensk rätt}
\textbf{23 kap. 6 § BrB} aktualiseras genom:
\begin{enumerate}
\item Kännedom om systematiska folkrättsbrott (ICJ, ICC, FN-rapporter)
\item Underlåtenhet att vidta juridiskt mandaterade åtgärder
\item Samtidigt språkbruk som döljer brottens karaktär
\end{enumerate}

\subsubsection{Vilseledande legitimering som stämpling}
\begin{itemize}
\item \textbf{Rättslig grund}: Nürnbergprincip VI(c) om medverkan till brott mot mänskligheten
\item \textbf{Sveriges agerande}: Felaktig juridisk karaktärisering med brottsbefrämjande effekt
\item \textbf{Analog}: Att felaktigt hävda att någon har rätt att bruka våld
\end{itemize}

\subsubsection{Koppling till folkmordskonventionen}

\begin{tabular}{p{0.25\textwidth}p{0.7\textwidth}}
\textbf{Handling som krävts} & \textbf{Sveriges underlåtenhet} \\
\hline
Stöd till ICJ-processen & Ej anslutit sig till Sydafrikas talan \\
\hline
Vapenembargo & Fortsatt export av militärteknologi\footnote{\url{https://www.isp.se/internationella-sanktioner/}} \footnote{\url{https://proletaren.se/artikel/vapexporten-till-israel-okar/}} \\
\hline
Diplomatisk isolering & Besökt Israel av UD-tjänstemän \\
\end{tabular}



Enligt ICJ i \textit{Bosnia v. Serbia} (2007, §432) kan medverkan aktualiseras genom:
\begin{itemize}
\item Underlåtenhet att stoppa brottet när möjlighet finns
\item Bidragande handlingar som underlättar brottsligheten
\item Skapande av normativa ramar som legitimerar brott
\end{itemize}
Sveriges agerande uppfyller alla tre kriterier.



\subsubsection{Sammanfattande rättslig bedömning}
Regeringens uttalanden uppfyller rekvisiten för:
\begin{itemize}
\item \textbf{Medhjälp} enligt ILC:s Draft Articles on State Responsibility (art. 16)
\item \textbf{Underlåtenhet att avslöja brott} (BrB 23:6)
\item \textbf{Indirekt stämpling} till folkrättsbrott (Nürnbergartikel 6)
\end{itemize}