%filnamn: kondenserad_sammanfattning_officiella_positioner.tex
%SYFTE: Att kompakt sammanfatta allt i regeringens_officiella_positioner.tex för maximal överblick

%%Sammanfattning Sammanfattning av regeringens uttalanden såsom rättsliga handlingar
\subsection{Sammanfattning av regeringens uttalanden såsom rättsliga handlingar}
\label{subsec:regeringens_positioner}

\subsubsection{Kritiska uttalanden: maj 2025}
\begin{tabular}{p{0.12\textwidth}p{0.85\textwidth}}
\textbf{Datum} & \textbf{Uttalande \& rättslig kontext} \\
\hline
23 maj & \textit{Debattartikel:} "Israel har rätt att försvara sig – men den rätten måste utövas i enlighet med folkrätten" \\
& \footnotesize\textit{Rättslig kontext:} ICJ rådgivande yttrande 2004 (muren) \& 2024 (ockupationen) fastslår att artikel 51 FN-stadga ej tillämplig \\
\hline
27 maj & \textit{UD-uttalande:} "Terroristorganisationen Hamas ansvar för den aktuella situationen väger tungt" \\
& \footnotesize\textit{Rättslig kontext:} Bröt mot neutralitetsprincipen (Hagakonventionen 1907) och principen om individuellt ansvar (Nürnbergartikel 6) \\
\end{tabular}

\subsubsection{Rättslig desinformation om självförsvarsrätten}
\begin{itemize}
\item \textbf{Faktamässig felaktighet}: Israel är ockupationsmakt (ICJ 2004 \& 2024), inte angripen stat
\item \textbf{Rättslig konsekvens}: Skapar normativ förskjutning från \textit{legalitet} till \textit{proportionalitet}
\item \textbf{Jämförelse}:Analogt med att hävda att en fängelsevakt har rätt att försvara sig mot en intagen — trots att denne har kontroll över situationen, skyldighet att värna liv och tillgång till överlägsna medel.

Självförsvarsrätten existerar, men den är strikt reglerad. En vakt som skjuter en obeväpnad ungdomsbrottsling i huvudet efter ett inbrott, när faran inte längre är överhängande, kommer inte att frias.

\item \textbf{Jämförelse:} På samma sätt kan en ockupationsmakt inte åberopa självförsvarsrätt mot en befolkning som den själv:
  \begin{itemize}
    \item förvägrar en politisk uppgörelse enligt FN:s säkerhetsrådsresolution 242,
    \item håller långvarigt instängd och isolerad,
    \item kontrollerar genom militär dominans,
    \item bär ett juridiskt skyddsansvar gentemot.
  \end{itemize}

\end{itemize}

\subsubsection{Selektiv demonisering som folkrättsbrott}
\textbf{Asymmetrisk tillämpning av 'terrorist'-begreppet:}
\begin{itemize}
\item \textbf{Hamas}: Kategoriskt terroriststämpel trots:
  \begin{itemize}
  \item Folkvald regering (2006)
  \item Accept av FN-resolution 242 (1967-gränser)\footnote{\url{https://www.972mag.com/hamas-fatah-elections-israel-arrogance/}}
  \item Intern repression av extrema och kriminella grupper\footnote{\url{https://www.newsweek.com/hamas-arresting-and-torturing-jihadis-prevent-war-israel-752108}} \footnote{\url{https://www.middleeasteye.net/news/israel-palestine-hamas-arrests-two-rocket-fire}} \footnote{\url{https://www.israelnationalnews.com/news/166785}} \footnote{\url{https://www.israelnationalnews.com/news/259035}}
  \end{itemize}
\item \textbf{Israeliska aktörer}: Ingen kritik av:
  \begin{itemize}
  \item IDF:s "Hannibal-direktiv" 7 oktober 2023\footnote{\url{https://thegrayzone.com/2024/06/21/israeli-army-friendly-fire-october-7/}} \footnote{\url{https://thegrayzone.com/2025/02/25/bibas-israeli-govt-propaganda-hostage-killings/}} \footnote{\url{https://thegrayzone.com/2023/11/21/haaretz-grayzone-conspiracy-israeli-festivalgoers/}} \footnote{\url{https://thegrayzone.com/2023/10/27/israels-military-shelled-burning-tanks-helicopters/}}
  \item Samarbete med kriminella klaner (Grayzone 2025)\footnote{\url{https://thegrayzone.com/2025/06/05/israel-arming-isis-gang-gaza/}}
  \end{itemize}
\end{itemize}
\lagrumsinline{Genèvekonvention IV, artikel 33\quad Förbud mot kollektiv bestraffning}

\subsubsection{Underlåtenhetsansvar enligt svensk rätt}
\textbf{23 kap. 6 § BrB} aktualiseras genom:
\begin{enumerate}
\item Kännedom om systematiska folkrättsbrott (ICJ, ICC, FN-rapporter)
\item Underlåtenhet att vidta juridiskt mandaterade åtgärder
\item Samtidigt språkbruk som döljer brottens karaktär
\end{enumerate}

\subsubsection{Vilseledande legitimering som stämpling}
\begin{itemize}
\item \textbf{Rättslig grund}: Nürnbergprincip VI(c) om medverkan till brott mot mänskligheten
\item \textbf{Sveriges agerande}: Felaktig juridisk karaktärisering med brottsbefrämjande effekt
\item \textbf{Analog}: Att felaktigt hävda att någon har rätt att bruka våld
\end{itemize}

\subsubsection{Koppling till folkmordskonventionen}

\begin{tabular}{p{0.25\textwidth}p{0.7\textwidth}}
\textbf{Handling som krävts} & \textbf{Sveriges underlåtenhet} \\
\hline
Stöd till ICJ-processen & Ej anslutit sig till Sydafrikas talan \\
\hline
Vapenembargo & Fortsatt export av militärteknologi\footnote{\url{https://www.isp.se/internationella-sanktioner/}} \footnote{\url{https://proletaren.se/artikel/vapexporten-till-israel-okar/}} \\
\hline
Diplomatisk isolering & Besökt Israel av UD-tjänstemän \\
\end{tabular}



Enligt ICJ i \textit{Bosnia v. Serbia} (2007, §432) kan medverkan aktualiseras genom:
\begin{itemize}
\item Underlåtenhet att stoppa brottet när möjlighet finns
\item Bidragande handlingar som underlättar brottsligheten
\item Skapande av normativa ramar som legitimerar brott
\end{itemize}
Sveriges agerande uppfyller alla tre kriterier.



\subsubsection{Sammanfattande rättslig bedömning}
Regeringens uttalanden uppfyller rekvisiten för:
\begin{itemize}
\item \textbf{Medhjälp} enligt ILC:s Draft Articles on State Responsibility (art. 16)
\item \textbf{Underlåtenhet att avslöja brott} (BrB 23:6)
\item \textbf{Indirekt stämpling} till folkrättsbrott (Nürnbergartikel 6)
\end{itemize}

\section*{Rättslig grund för ansvar vid underlåtenhet att ingripa mot våld}

\subsection*{1. Underlåtenhet att hindra brott (Brottsbalken 23 kap.)}

Svensk straffrätt skiljer mellan:

\begin{itemize}
  \item \textbf{Förbiseende av tjänsteplikt} (tjänstefel, BrB 20 kap. 1 §),
  \item och \textbf{underlåtenhet att hindra brott} (BrB 23:6), om man har särskild rättslig förpliktelse att agera.
\end{itemize}

\textit{Exempel:} En polis som ser A misshandla B, men låter det ske, trots att han har makt och skyldighet att ingripa, kan hållas straffrättsligt ansvarig.  
Om B därefter slår tillbaka och dödar A i affekt – kan polisens underlåtenhet indirekt ha möjliggjort våldsspiralen.

\subsection*{2. Lojalitetsplikt och passivitetsansvar i internationell rätt (ICJ – Bosnia v. Serbia)}

ICJ slog fast att \textbf{underlåtenhet att skydda} (\textit{failure to prevent genocide}) är en självständig folkrättskränkning.  
Det gäller inte bara stater – utan i praktiken även myndighetsrepresentanter.

En regering (eller myndighetsperson) som har kunskap om ett pågående mönster av förtryck och inte agerar, bryter mot internationell rätt, särskilt om de är part till tvingande konventioner (\textit{jus cogens}).

\subsection*{3. Civilrättslig analogi: Culpa in omittendo (oaktsamhet genom underlåtenhet)}

I civilrätten har en part ansvar inte bara när man agerar skadligt, utan även när man inte agerar där man borde.

\textit{Exempel:} En fastighetsägare som ser att ett barn faller i en brunn på gården han ansvarar för, men inte ingriper, kan bli skadeståndsansvarig.

\textbf{Analogt:} En polis som upprepade gånger fått höra att ”B kommer att slå tillbaka om A inte stoppas”, men ändå inte ingriper – kan i moralisk och rättslig mening bära ansvar för hela kedjan av händelser.

\subsection*{4. Europeiska Människorättsdomstolen (EMD): skyldighet att skydda}

Staten (inkl. dess myndigheter och tjänstemän) har positiva skyldigheter enligt Europakonventionen, särskilt:

\begin{itemize}
  \item Artikel 2 – Rätt till liv
  \item Artikel 3 – Förbud mot tortyr och omänsklig behandling
\end{itemize}

\textbf{EMD-dom:} \textit{Opuz v. Turkey (2009)} – Turkiet fälldes för att ha inte skyddat en kvinna från en våldsam make, trots att hon gjort flera anmälningar.  
Hon slog sedan tillbaka. Staten ansågs ansvarig för både hennes och hans lidande.

\subsection*{5. Analogin: Konkludent anstiftan genom språklig legitimering}

När politiker eller statstjänstemän använder språk som:

\begin{itemize}
  \item \textit{”X har rätt att försvara sig”} trots övervåld
  \item \textit{”B får skylla sig själv”} trots att B är förtryckt
\end{itemize}

...kan det enligt folkrättsdoktrin tolkas som \textbf{legitimering av brott}, vilket liknar anstiftan eller uppvigling.

\subsection*{Sammanfattande tabell: Finns det rättslig grund för ansvar?}

\begin{tabular}{|p{5.5cm}|p{9cm}|}
\hline
\textbf{Situation} & \textbf{Rättslig grund för ansvar} \\
\hline
Polis eller myndighet som inte ingriper trots kännedom & Brottsbalken 20 kap. (tjänstefel), BrB 23:6 (underlåtenhet), Europakonventionen (artiklar 2 och 3), EMD-praxis \\
\hline
Staten som inte sätter press på angripare & ICJ-praxis (Bosnia v. Serbia), Folkmordskonventionen (artikel I), ILC:s artiklar om statsansvar \\
\hline
Språklig eller politisk legitimering av övervåld & Kan likställas med uppvigling eller anstiftan (BrB), samt brott mot folkrättsliga grundsatser, särskilt vid uppsåt \\
\hline
\end{tabular}
