% filnamn: summerande_skuldsats.tex
% SYFTE: Ställ hypotesen klart. "Regeringen är skyldig." – detta ska bevisas.
% Introducera grundlagen och folkrätten som rättskällor. Skapa förväntan.

\subsection{Regeringen har brutit mot grundlagen och folkrätten}

Sveriges regering har, i sin hantering av Israels agerande gentemot den palestinska civilbefolkningen, åsidosatt såväl konstitutionella skyldigheter som internationellt bindande rättsregler. Det rör sig inte om enbart politisk passivitet, utan om rättsligt relevanta underlåtelser – och i vissa fall konkludent medgivande till en pågående folkrättsöverträdelse.

\lagrum{10 kap. 1 § Regeringsformen\quad  Överenskommelser med andra stater eller med mellanfolkliga organisationer ingås av regeringen.}

Detta konstitutionella bemyndigande omfattar inte en rätt att godtyckligt efterleva förpliktelser, utan innebär en skyldighet att lojalt implementera och upprätthålla de traktater Sverige anslutit sig till. Principen om \textit{pacta sunt servanda}, grundläggande inom såväl civilrätt som folkrätt, kräver att avtal inte enbart respekteras till formen, utan att deras syfte faktiskt förverkligas.

I svensk rätt återkommer denna norm i lojalitetsprincipen, vars mest grundläggande uttryck är rättsfiguren \textit{culpa in contrahendo} – det vill säga vårdslöshet i samband med avtalsingående. En avtalspart som agerar i strid med avtalets ändamål, eller som medvetet förhåller sig passiv inför dess överträdelse, gör sig skyldig till ett lojalitetsbrott. Detta gäller även för staten i dess offentligrättsliga funktion.

\medskip

Sverige har sedan 1952 varit part i Konventionen om förebyggande och bestraffning av brottet folkmord (1948). Konventionen stadgar inte enbart ett förbud mot att själv begå folkmord, utan ålägger varje fördragsslutande stat en positiv och proaktiv skyldighet att vidta förebyggande åtgärder – även i situationer där brottet begås av tredje stat, och även när det sker utanför det egna territoriet.

\lagrum{Artikel I, Folkmordskonventionen\quad De fördragsslutande parterna bekräftar att folkmord [...] är ett brott enligt internationell rätt som de åtar sig att förebygga och bestraffa.}

Internationella domstolen (ICJ) har i sin dom i målet \textit{Bosnia and Herzegovina v. Serbia and Montenegro} (2007) tydliggjort att denna skyldighet föreligger oberoende av om folkmordet fullbordats, och att underlåtenhet att förebygga kan ge upphov till statsansvar, även i frånvaro av direkt medverkan.

Därtill förpliktar artikel 56 i FN-stadgan medlemsstaterna att vidta gemensamma och enskilda åtgärder i samverkan med Förenta Nationerna, i syfte att säkerställa respekt för mänskliga rättigheter, rättsstatens principer och folkrättens bindande normstruktur.

\lagrum{Artikel 56, FN-stadgan\quad All Members pledge themselves to take joint and separate action in co-operation with the Organization for the achievement of the purposes set forth in Article 55.}

\medskip

Mot bakgrund av dessa normkällor – konstitutionella, traktaträttsliga och sedvanerättsliga – föreligger en rättslig skyldighet för Sverige att agera vid kännedom om ett pågående folkmord. Regeringens underlåtenhet att vidta adekvata åtgärder, trots att flertalet internationella rättsorgan – däribland ICC, ICJ, FN:s särskilda rapportörer och ledande människorättsorganisationer – har dokumenterat folkrättsbrott av systematisk och allvarlig karaktär, utgör ett rättsbrott.

Den svenska regeringens fortsatta diplomatiska, ekonomiska och militärstrategiska samverkan med staten Israel – utan tillkommande rättslig ansvarsutkrävning – konstituerar en folkrättsstridig underlåtenhet. Denna underlåtenhet är rättsligt kvalificerad och har konkludent verkan.

\medskip

Det kan därför konstateras:

\begin{itemize}
  \item att Sveriges regering har brutit mot folkmordskonventionen genom underlåtenhet att agera i förebyggande syfte,
  \item att Sveriges regering har brutit mot FN-stadgan genom att inte uppfylla skyldigheten att upprätthålla internationell rättsordning,
  \item att Sveriges regering har brutit mot 10 kap. 1 § Regeringsformen genom att inte lojalt fullgöra de förpliktelser som följer av ingångna internationella överenskommelser.
\end{itemize}

\medskip

Det föreligger således ett konstitutionellt, folkrättsligt och potentiellt individuellt ansvar.  
Detta dokument är inte en inbjudan till dialog.  
Det är en formell anklagelsehandling.
