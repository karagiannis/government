% filnamn: 3_Sveriges_folkrättsliga_skyldigheter_och_ansvar.tex
%
% SYFTE:
% Tillämpa det folkrättsliga ramverk som beskrivs i föregående avsnitt på Sveriges agerande.
% Visa att Sveriges regering – genom att offentligt förvanska gällande rätt (t.ex. genom formuleringen
% "Israel har rätt att försvara sig, men måste följa folkrätten") – riskerar att konkludent medverka till 
% folkrättsbrott, inklusive folkmord.
%
% Regeringens uttalanden tolkas i folkrätten som uttryck för statens rättsliga ståndpunkt. Sådana uttalanden 
% som felaktigt tillskriver en ockupationsmakt rätt till självförsvar mot den ockuperade befolkningen – trots 
% att detta uttryckligen avvisats av Internationella domstolen – kan utgöra en otillåten omtolkning av 
% internationella avtal.
%
% Att i Sveriges namn påstå att "lagen" innebär något annat än vad de ratificerade konventionerna föreskriver 
% är inte en neutral bedömning. Det är ett rättsligt handlande som avviker från Sveriges bindande åtaganden 
% enligt FN-stadgan och folkmordskonventionen.
%
% Analogt med civilrättens principer om obehörig rättshandling (t.ex. stämpling eller otillbörlig vilseledning) 
% kan detta ses som ett medvetet konkludent avsteg från den internationella rättsordningen.



\subsection{Sveriges folkrättsliga skyldigheter och ansvar}
\label{subsec:svenska_skyldigheter}

\subsubsection{Konstitutionell bindning till folkrätten}
Enligt Regeringsformen 1 kap. 10 § är Sverige bundet av folkrätten:
\begin{quote}
\textit{Den allmänna folkrätten och EU-rätten gäller som svensk lag.}
\end{quote}
Detta innebär att folkrättsliga förpliktelser har samma rättsliga status som svensk lagstiftning.

\subsubsection{Skyldighet att förebygga folkmord}
Sverige som part i folkmordskonventionen (1952) har en aktiv skyldighet:
\begin{itemize}
\item Att förebygga folkmord (artikel I)
\item Att inte bistå eller tolerera folkmordsrelaterade handlingar
\end{itemize}
ICJ klargjorde i \textit{Bosnien mot Serbien} (2007) att denna skyldighet:
\begin{itemize}
\item Aktiveras vid \textit{risk} för folkmord
\item Kräver ”alla medel som står till buds”
\item Gäller även utanför eget territorium
\end{itemize}

\subsubsection{Medverkan till folkrättsbrott}
Enligt ILC:s Draft Articles on State Responsibility (2001) artikel 16:
\begin{quote}
\textit{En stat som hjälper eller bistår en annan stat att begå ett folkrättsstridigt handling är internationellt ansvarig om:}
\begin{enumerate}
\item \textit{staten gör detta med kännedom om omständigheterna kring den folkrättsstridiga handlingen; och}
\item \textit{handlingen skulle vara folkrättsstridig om den begåtts av den bistående staten.}
\end{enumerate}
\end{quote}

\subsubsection{Särskilda skyldigheter mot ockuperade folk}
Sverige har ytterligare skyldigheter som medlem i Genèvekonventionerna:
\begin{itemize}
\item \textbf{Neutralitetsplikt}: Att inte legitimera ockupationsmaktens brott
\item \textbf{Interventionsplikt}: Att agera vid grova konventionsbrott
\item \textbf{Skyddsplikt}: Att säkerställa humanitär tillgång
\end{itemize}

\subsubsection{Sammanfattande juridiska parametrar}
Sveriges agerande bedöms mot följande kriterier:
\begin{enumerate}
\item \textbf{Kunskapsrekvisitet}: Kännedom om risk för folkmord
\item \textbf{Handlingsrekvisitet}: Förebyggande åtgärder vidtas
\item \textbf{Abstinensrekvisitet}: Ingen direkt/indirekt medverkan
\item \textbf{Transparensrekvisitet}: Rättvis rapportering om brott
\end{enumerate}