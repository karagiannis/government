% filnamn: 2b_statligt_ansvar_och_medverkan.tex
%
% SYFTE:
% Etablera folkrättens grundstruktur och staternas skyldigheter att agera.
% Visa att passivitet och affärsrelationer med förövarstater kan konstituera folkrättsbrott.
% Stegra från allmän rättsregel till särskilt ansvar vid folkmordsrisk.
% Skapa bro mellan principnivå och konkret tillämpning i fil 3.
%% OBS: Nästa fil (3_Sveriges_folkrättsliga_skyldigheter_och_ansvar.tex) konkretiserar denna analys med fokus på Sveriges agerande.

\subsection{Staters skyldighet att bidra till upprätthållandet av folkrätten}

I en internationell rättsordning utan central maktstruktur vilar ansvaret för 
lagens upprätthållande på medlemsstaternas samlade agerande. Det är därmed inte bara förbjudet att 
själv begå folkrättsbrott – stater har även en skyldighet att inte bistå andra som gör det, 
samt att vidta rimliga åtgärder för att förhindra sådana brott.

\subsubsection*{1. Förbud mot bistånd – ILC:s artiklar om statsansvar}

Den mest centrala rättskällan i detta sammanhang är Artikel 16 i ILC:s utkast till artiklar om staters ansvar:

\begin{quote}
\textit{A State which aids or assists another State in the commission of an internationally wrongful act by the latter is internationally responsible if:}\\
(a) that State does so with knowledge of the circumstances of the internationally wrongful act; and\\
(b) the act would be internationally wrongful if committed by that State.
\end{quote}

Detta innebär att en stat – exempelvis Sverige – som hjälper en annan stat att begå folkrättsbrott 
(t.ex. genom vapenhandel, diplomatiskt stöd, eller fortsatt handel) är internationellt 
ansvarig om den känner till omständigheterna kring brottet.

\subsubsection*{ 2. Särskilt förbud vid grova brott – Artikel 41}

Vid brott mot \textit{jus cogens} – tvingande folkrättsnormer såsom förbud mot folkmord, 
apartheid eller olaglig annektering – skärps skyldigheten ytterligare:

\begin{quote}
\textit{No State shall recognize as lawful a situation created by a serious breach of a peremptory norm of general international law (jus cogens), nor render aid or assistance in maintaining that situation.}
\end{quote}

Detta innebär att stater varken får erkänna en sådan situation som laglig, eller 
upprätthålla affärs- eller diplomatiska relationer som förlänger den. 
Att inte införa sanktioner eller avbryta samarbeten kan därmed – i vissa fall – utgöra medverkan till brott.

\subsubsection*{ 3. FN-stadgans krav på kollektiv rättsordning}

Att stater ska göra sin beskärda del i att upprätthålla folkrätten följer också direkt av FN-stadgans grundartiklar:

\begin{itemize}
  \item Artikel 1(1): FN:s syfte är att upprätthålla internationell fred och säkerhet.
  \item Artikel 1(3): FN ska främja och uppmuntra respekt för mänskliga rättigheter och folkrätt.
\end{itemize}

Även om FN:s säkerhetsråd har primärt ansvar under Kapitel VII, innebär detta inte att 
andra stater står utan ansvar. Tvärtom följer av sedvanerätt och av principen 
om \textit{due diligence} att stater har ett eget ansvar att inte bidra till 
folkrättsbrott – särskilt vid brott mot tvingande rättsnormer.% filnamn: 2_folkrattens_grundprinciper.tex

\subsubsection*{ 4. Förhöjt ansvar vid risk för folkmord – Folkmordskonventionen}

När en stat har kännedom om risk för folkmord, aktiveras en ytterligare skyldighet enligt 
artikel I i folkmordskonventionen:

\begin{quote}
\textit{The Contracting Parties confirm that genocide [...] is a crime under international law which they undertake to prevent and to punish.}
\end{quote}

Detta innebär att neutralitet inte är rättsligt tillåtet. En stat är skyldig att 
vidta alla rimliga och proportionella åtgärder för att förhindra att folkmord sker – även 
när gärningen begås av en tredje stat, och även utanför det egna territoriet.

\subsubsection*{ 5. ICJ-domen \textit{Bosnia v. Serbia} (2007) – rättspraxis om passivt ansvar}

Internationella domstolen fastslog i domen \textit{Bosnia and Herzegovina v. Serbia and Montenegro} att 
Serbien bröt mot folkmordskonventionen genom att underlåta att förhindra folkmordet i Srebrenica:

\begin{quote}
\textit{Serbia could, and should, have acted to prevent the genocide, but failed to do so.}
\end{quote}

Domen slog fast att en stat inte måste ha deltagit direkt i folkmordet för att kunna hållas 
ansvarig – det räcker att staten haft kännedom om risk och inte gjort tillräckligt för att förebygga brottet.

\subsubsection*{ Slutsats – Folkrättslig förpliktelse att inte möjliggöra brott}% filnamn: 2_folkrattens_grundprinciper.tex
  \item inte erkänna rättsstridiga situationer som lagliga,
  \item inte upprätthålla relationer som möjliggör fortsatt brott,
  \item och – i synnerhet – att agera när risk för folkmord föreligger.
\end{itemize}

Denna skyldighet är inte politisk – den är juridisk, och följer av internationella traktater, sedvanerätt och rättspraxis.

% Övergång
I följande avsnitt tillämpar vi dessa rättsprinciper konkret på Sveriges agerande i relation till Palestina och Israel.



