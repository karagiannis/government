%filnamn: slutpladering_anklagelse.tex
% SYFTE: Sammanfattande bevisning. "Regeringen är skyldig." genom att knyta samman lag och 
%regeringens underlåtenheter och andra felsteg.


%filnamn: slutpladering_anklagelse.tex
% SYFTE: Sammanfattande bevisning. "Regeringen är skyldig." genom att knyta samman lag och 
%regeringens underlåtenheter och andra felsteg.

%filnamn: slutpladering_anklagelse.tex
\subsection{Slutplädering: Regeringens juridiska ansvar är ovedersägligt}


Den samlade bevisningen visar klart och entydigt att Sveriges regering genom handling och underlåtenhet brutit mot bindande folkrättsliga och konstitutionella skyldigheter. Regeringens agerande uppfyller samtliga rekvisit för medverkan till folkrättsbrott enligt etablerad rättspraxis.

\subsection*{Sammanfattning av beviskedjan}
\begin{enumerate}
    \item \textbf{Legitimering av olagligt våld} \\
    Genom konsekvent felaktig tillämpning av artikel 51 FN-stadga har regeringen rättsligt legitimerat Israels agerande som "självförsvar" trots klart fastställd ockupationsstatus (ICJ 2004, 2024).
    
    \item \textbf{Selektiv tillämpning av rättsnormer} \\
    Regeringen har tillämpat "terrorist"-begreppet asymmetriskt genom att:
    \begin{itemize}
        \item Systematiskt demonisera Hamas trots dess folkvalda status och accepterande av FN-resolution 242
        \item Underlåta att granska IDF:s dokumenterade brott (Hannibal-direktivet, samarbete med kriminella klaner)
    \end{itemize}
    
    \item \textbf{Aktiv medverkan genom underlåtenhet} \\
    Tabell nedan sammanfattar centrala underlåtenheter som uppfyller rekvisiten för medhjälp enligt ILC:s artikel 16:
\end{enumerate}

\begin{table}[h]
\centering
\caption{Sveriges underlåtenheter och deras rättsliga konsekvenser}
\label{tab:underlatenhet}
\begin{tabular}{p{0.4\textwidth}p{0.55\textwidth}}
\textbf{Underlåten handling} & \textbf{Rättslig kvalifikation} \\ \midrule
Inget stöd till ICJ-processen & Underlåtenhet att förebygga folkmord (art. I Folkmordskonv.) \\
Fortsatt vapenexport & Medverkan till krigsförbrytelser (ILC art. 16) \\
Avsaknad av sanktioner & Brott mot neutralitetsplikt (Haagkonventionen) \\
\end{tabular}
\end{table}

\subsection*{Juridiska följdslut}
Med stöd i dokumenterad bevisning och etablerad rättspraxis konstateras:

\begin{itemize}
    \item Regeringen har \textbf{aktivt medverkat} till folkmord enligt artikel III(e) i folkmordskonventionen genom:
    \begin{itemize}
        \item Diplomatiskt stöd som underlättat fortsatt brottslig verksamhet
        \item Språkbruk som skapat normativa ramar för acceptans av brott
    \end{itemize}
    
    \item Regeringen har \textbf{brutit mot Regeringsformen} (1 kap. 10 §) genom:
    \begin{itemize}
        \item Underlåtenhet att lojalt fullgöra folkrättsliga förpliktelser
        \item Aktivt bidrag till urholkning av internationell rättsordning
    \end{itemize}
    
    \item Regeringen har \textbf{uppfyllt rekvisiten} för underlåtenhet att avslöja brott (BrB 23:6) genom:
    \begin{itemize}
        \item Kännedom om systematiska folkrättsbrott
        \item Avsaknad av tillräckliga åtgärder trots handlingsmöjligheter
    \end{itemize}
\end{itemize}

\subsection*{Krav på rättslig prövning}
På grundval av ovanstående yrkas:

\begin{itemize}
    \item Att \textbf{Konstitutionsutskottet} omedelbart inleder granskning av regeringens agerande ur statsrättsligt perspektiv
    
    \item Att \textbf{Justitieombudsmannen} granskar eventuella tjänstefel inom berörda departement
    
    \item Att \textbf{Utrikesdepartementet} omedelbart:
    \begin{itemize}
        \item Upphör all militär samverkan med Israel
        \item Ansluter sig till ICJ-processen mot Israel
        \item Inför omfattande sanktioner
    \end{itemize}
    
    \item Att \textbf{Högsta domstolen} prövar frågan om regeringens folkrättsliga ansvar enligt 13 kap. 3 § Regeringsformen
\end{itemize}

\textit{Denna slutplädering baseras på den samlade bevisningen i detta dokument och kräver omedelbar rättslig åtgärd.}