% filnamn: regeringens_officiella_positioner.tex
% SYFTE: Visa exakt vad som sagts – skapa koppling till brottsrekvisit.
% Fungerar som "bevisupptagning" i ett åtal: citat, datum, kontext.



\subsection{Regeringens officiella positioner: innehåll och rättslig betydelse}


\subsubsection*{Felaktig åberopan av självförsvar i ockupationskontext}

Regeringen har åberopat Israels oförbehellna till självförsvar trots över 10 000 dödade\footnote{\url{https://www.aftonbladet.se/nyheter/a/0QR2eE/billstroms-ord-om-vapenvila-i-gaza-far-stark-kritik}} , medan numera \enquote{Israel har rätt att försvara sig men det måste ske enligt folkrätten}

ICJ konstaterade redan 2004 att Israel inte kunde åberopa rätten till självförsvar mot en ockuperad befolkning.
\begin{quote}
\itshape
"Israel could not rely on a right of self-defence or on a state of necessity in order to preclude the wrongfulness of the construction of the wall, and that such construction and its associated régime were accordingly contrary to international law."
\end{quote}

I det rådgivande yttrandet från Internationella domstolen (ICJ) från 2004 fastslogs tydligt att Israel inte har rätt att åberopa \textit{självförsvar} enligt artikel 51 i FN-stadgan inom ramen för sin ockupation av Palestina. Detta eftersom ockupationen utgör en kontroll av ett territorium där ingen främmande stat angriper Israel från utsidan. Självförsvar enligt artikel 51 förutsätter ett väpnat angrepp från en annan suverän stat, vilket inte är fallet här.

Att Sveriges regering — även efter denna rättspraxis och med kunskap om det ockuperade territoriets status — fortsatte uttala att \enquote{Israel har rätt att försvara sig} under hösten 2023, utgör därmed inte bara ett politiskt ställningstagande. Det kan ses som ett folkrättsligt felaktigt uttalande som riskerar att legitimerar övervåld mot en skyddad civilbefolkning.

Att i detta läge dessutom kräva att detta självförsvar ska ske \enquote{inom ramen för folkrätten} antyder felaktigt att Israel överhuvudtaget befinner sig i en rättslig position där ett våldsutövande skulle kunna vara legitimt. Det rättsliga utgångsläget enligt ICJ är att Israel \textit{inte har rätt till självförsvar inom ockuperat område}, och varje form av våld mot den ockuperade civilbefolkningen strider i sig mot internationell rätt.

Uttalanden från svenska regeringsföreträdare som \enquote{det är inte rättvist att kräva vapenvila} eller att \enquote{Israel har rätt att bekämpa Hamas} innebär i detta sammanhang inte en neutral position, utan närmar sig det som i artikel 11 i ARSIWA definieras som att en stat \textit{erkänner och antar en annan aktörs handling som sin egen}. Detta kan i sin tur grunda ansvar enligt internationell rätt.



\subsubsection*{Skyldighet att ta emot barn från Gaza för vård}

Enligt artikel 24 i den fjärde Genèvekonventionen åligger det stater att \textit{underlätta mottagandet av barn i neutrala länder} när dessa barn är föräldralösa eller separerade från sina familjer som följd av väpnad konflikt. Denna skyldighet aktualiseras i hög grad av situationen i Gaza, där sjukvårdssystemet har kollapsat och tillgången till medicin, kirurgi och livsuppehållande behandling är starkt begränsad.\footnote{Statement by Principals of the Inter-Agency Standing Committee, 2024-12-18.}

\lagrum{Article 24, Geneva Convention IV (1949): The Parties to the conflict shall facilitate the reception of such children in a neutral country for the duration of the conflict with the consent of the Protecting Power, if any, and under due safeguards for the observance of the principles stated in the first paragraph.}

Sverige är, som neutral stat och part till Genèvekonventionerna, folkrättsligt förpliktat att bistå när barn löper risk att dö i frånvaro av adekvat vård. Vårdkapacitet finns – bland annat Karolinska universitetssjukhuset har bekräftat att barn från Gaza kan tas emot utan att det påverkar annan vård. Trots detta har Sveriges regering konsekvent vägrat att hörsamma WHO:s och EU:s uppmaningar om evakuering.

Barnkonventionen, som är inkorporerad i svensk lag, understryker varje barns rätt till liv, hälsa och skydd – även genom internationellt samarbete. Att neka evakuering av barn med akuta vårdbehov från ett ockuperat och bombat territorium där humanitär kollaps råder, utgör ett allvarligt folkrättsligt och etiskt svek.

\lagrum{Barnkonventionen, artikel 24.1: States Parties recognize the right of the child to the enjoyment of the highest attainable standard of health [...] and shall strive to ensure that no child is deprived of his or her right of access to such health care services.}

\textbf{Vi uppmanar därför Sveriges regering att omedelbart bistå WHO:s evakueringsinsats och ta emot svårt skadade och svårt sjuka barn från Gaza, i enlighet med internationell humanitär rätt och barns rätt till liv och hälsa.}


\subsubsection*{ILC Artikel 16 – Medhjälp till en annan stats folkrättsbrott}

\begin{quote}
\itshape
\textbf{Article 16} \\
\textit{Aid or assistance in the commission of an internationally wrongful act}

A State which aids or assists another State in the commission of an internationally wrongful act by the latter is internationally responsible for doing so if: \\
(a) that State does so with knowledge of the circumstances of the internationally wrongful act; and \\
(b) the act would be internationally wrongful if committed by that State.
\end{quote}

\begin{quote}
\itshape
\textbf{Artikel 16} \\
\textit{Medhjälp till en annan stats folkrättsbrott}

En stat som bistår eller hjälper en annan stat att begå en internationellt rättsstridig handling är själv internationellt ansvarig, om: \\
(a) staten gör detta med kännedom om omständigheterna kring den rättsstridiga handlingen; och \\
(b) handlingen hade varit internationellt rättsstridig även om den hade begåtts av den bistående staten själv.
\end{quote}

Sveriges officiella hållning efter den 7 oktober 2023 har varit att "Israel har rätt att försvara sig", trots att:

\begin{itemize}
\item Internationella domstolen (ICJ) redan 2004 fastslog att Israel inte får åberopa självförsvar inom ockuperat territorium.
\item Israel efter bara några veckor hade dödat över 13,000 civila och släppt bomber motsvarande flera Hiroshima.
\item Israel inte accepterar en tvåstatslösning enligt FN:s resolutioner, medan Hamas 2017 införde erkännande av gränserna från resolution 242 i sin stadga.
\end{itemize}


Då Sverige samtidigt fortsätter att:

\begin{itemize}
\item förse Israel med militärteknik eller komponenter,
\item ge diplomatiskt skydd i internationella forum,
\item eller ekonomiskt stöd genom bilaterala projekt,
\end{itemize}

uppfyller detta rekvisiten i \textbf{artikel 16} därför att:

\begin{enumerate}
\item Sverige har \textbf{kännedom} om det folkrättsbrott som pågår; och
\item de handlingar Sverige utför hade varit folkrättsbrott om Sverige själv utfört dem under motsvarande förhållanden.
\end{enumerate}

Retorik som "Israel har rätt att försvara sig" skall därför i detta sammanhang tolkas som en \emph{legitimering} av ett pågående folkrättsbrott. När sådan retorik kombineras med konkret bistånd, bör Sverige anses \textbf{medverka till} brottet i folkrättslig mening.






Den 27 maj 2025 offentliggjorde Utrikesdepartementet ett uttalande med anledning av att Israels ambassadör i Stockholm kallats upp:

\begin{quote}
\textit{”Uppkallandet gjordes för att upprepa och inskärpa regeringens krav på Israels regering att omedelbart säkerställa säkert och obehindrat humanitärt tillträde till Gaza. [...] Det sätt som kriget nu bedrivs på är oacceptabelt. [...] Terroristorganisationen Hamas ansvar för den aktuella situationen väger tungt.”}
\end{quote}

Utrikesdepartementet anger även:
\begin{quote}
\textit{”Israel har rätt att försvara sig. Den rätten måste utövas i enlighet med folkrätten.”}\\
(Källa: \url{https://www.regeringen.se/pressmeddelanden/2025/05/uttalande-fran-utrikesdepartementet-med-anledning-av-uppkallande-av-israels-ambassador/})
\end{quote}

Detta följer på en debattartikel den 23 maj 2025 undertecknad av fyra statsråd:

\begin{quote}
\textit{”Att blockera mat och annat bistånd till civila är oförsvarligt. [...] Samtidigt har Israel rätt att försvara sig – men den rätten måste utövas i enlighet med folkrätten.”}\\
(Källa: \url{https://www.regeringen.se/artiklar/2025/05/debattartikel-regeringen-okar-pressen-mot-israel/})
\end{quote}

\medskip

Detta språkbruk är rättsligt problematiskt på flera punkter.

För det första används begreppet \textit{självförsvar} i strid med folkrättens tillämplighet. Israel är en ockupationsmakt, inte en stat i självförsvar. FN-stadgans artikel 51 får inte åberopas mot en civilbefolkning som står under ens kontroll, vilket fastslogs i ICJ:s rådgivande yttrande om \textit{muren i Palestina} (2004). Regeringens formulering är därmed inte neutral – den innebär rättslig desinformation och legitimerar folkrättsbrott.

\medskip

För det andra: språket åtföljs inte av några faktiska rättsliga eller politiska åtgärder. Det saknas:

\begin{itemize}
  \item sanktioner eller restriktioner i handeln,
  \item avbrutna diplomatiska förbindelser,
  \item deltagande i rättsliga processer, t.ex. Sydafrikas ICJ-talande mot Israel.
\end{itemize}

Utrikesdepartementets språkbruk stannar vid en retorisk apparat – en markering utan påföljd. Detta är i strid med den \textit{positiva handlingsskyldighet} som följer av artikel I i folkmordskonventionen, vilken ålägger konventionsstater att förhindra brottet i den mån det står i deras makt.

\medskip

Enligt ICJ:s dom \textit{Bosnia v. Serbia} (2007) är det otillräckligt att ”fördöma” eller ”vädja”. Staten måste agera genom att:

\begin{itemize}
    \item frysa relationer,
    \item isolera den misstänkta gärningsmannen,
    \item delta i rättsliga åtgärder.
\end{itemize}

Sveriges regering har inte uppfyllt någon av dessa skyldigheter.

\medskip

\subsubsection{Regeringens kännedom om Gazas folkrättsliga status}

Det är vidare ostridigt att Gazas de facto-regering – Hamas – sedan 2006 i handling och från 2017 även i skrift, 
uttryckligen godtagit en tvåstatslösning inom 1967 års gränser. Policydokumentet från 2017, samt tidigare initiativ 
såsom \textit{Prisoners’ Document} (2006), \textit{Mecka-avtalet} (2007) och \textit{Försoningsöverenskommelsen} 2020, 
utgör dokumenterade steg mot en politisk lösning inom ramen för folkrätten.\footnote{Se t.ex. \url{https://en.wikipedia.org/wiki/Hamas_Covenant\#2017_document} och \url{https://en.wikipedia.org/wiki/2020_Palestinian_reconciliation_agreement}}

Detta är välkänt inom internationell diplomati och forskning. Professor Neve Gordon och professor Menachem Klein 
har i flera publikationer visat att denna linje var genuin och syftade till att uppnå internationell legitimitet. 
Enligt Klein underminerades dessa initiativ systematiskt av Israel för att bibehålla narrativet om en frånvarande fredspartner.

Det är ett känt faktum att Hamas accepterar FN:s säkerhetsrådsresolution 242, som anger 1967 års gränser, medan Israel inte gör det.

Mot denna bakgrund saknar regeringens retorik om Israels självförsvar förankring i faktisk eller rättslig verklighet. 
Det utgör i stället en politisk konstruktion med allvarliga rättsliga konsekvenser.

Professor Menachem Klein, en av Israels ledande experter på konflikten och tidigare rådgivare i förhandlingar med PLO, 
konstaterar att Hamas i Fatah-Hamas-överenskommelsen från 2021 uttryckligen förband sig till internationell rätt, 
erkände PLO:s auktoritet, accepterade en stat inom 1967 års gränser med Östra Jerusalem som huvudstad, 
och åtog sig att föra en fredlig kamp.%
\footnote{Klein, M. (2023). \textit{Israeli arrogance thwarted a Palestinian political path}. +972 Magazine. \url{https://www.972mag.com/hamas-fatah-elections-israel-arrogance/}} 
Hamas avstod dessutom från att ställa upp med en egen presidentkandidat, i syfte att bana väg för ett demokratiskt mandat 
till en gemensam palestinsk ledning.

Detta är inte en obekräftad tolkning utan en dokumenterad och känd del av den diplomatiska processen – erkänd av 
bland andra EU och USA under Bidenadministrationen, som mottog överenskommelsen i syfte att stödja val.

Att i detta sammanhang fortsätta beskriva Israel som en part som agerar i självförsvar mot Gaza är en förvanskning 
av den faktiska rättspositionen. Hamas har deklarerat vilja att följa internationell rätt. Israel däremot, som ockupationsmakt, 
har avvisat både val, FN-resolutioner och fredsinitiativ – och brutit mot Genèvekonventionerna. 
Regeringens påstående saknar därför såväl faktisk som folkrättslig grund, och innebär i praktiken ett rättsligt vilseledande 
från en konventionsstat, med följd att ansvar enligt BrB 23 kap. eller artikel III i folkmordskonventionen 
kan aktualiseras.

\medskip

\subsubsection{Språklig tyngdpunktsförskjutning: från legalitet till proportionalitet}

Enligt internationell rätt har Israel inte rätt till självförsvar gentemot den ockuperade palestinska befolkningen. 
Detta följer av Internationella domstolens (ICJ) rådgivande yttrande från 2004, där murens konstruktion på palestinskt område 
förklarades strida mot internationell rätt, samt av ICJ:s rådgivande yttrande den 17 juli 2024, där hela den israeliska 
ockupationen av de palestinska territorierna – med hänvisning till FN:s säkerhetsrådsresolution 242 – bedömdes vara olaglig. 

Trots detta fortsätter den svenska regeringen att upprepa formuleringen att ”Israel har rätt att försvara sig”, 
utan att samtidigt erkänna att Israel är en ockupationsmakt vars rättigheter är strikt begränsade enligt folkrätten. 
Ett sådant språkbruk förskjuter tyngdpunkten i den rättsliga diskussionen: 
från att gälla om våld överhuvudtaget är tillåtet – till att handla om huruvida våldet är ”proportionerligt”.

Med andra ord: där noll skott vore rättsligt tillåtet, flyttas samtalet till huruvida tusen eller tiotusen skott är rimligt. 
Regeringens retorik bidrar därmed till en allvarlig normförskjutning med potentiellt rättsligt ansvar för medverkan 
till folkrättsbrott. 

Att i detta sammanhang ändå åberopa självförsvarsrätten utgör inte endast en felaktig rättstillämpning – det är ett 
normativt stöd till folkrättsbrott. 
När detta sker i officiella uttalanden från en konventionsstat, aktualiseras frågan om \textit{konkludent handlande} 
enligt folkmordskonventionen och svensk straffrätt.


\subsubsection{Ytterligare brott: underlåtenhet att avslöja pågående folkrättsbrott}

Med denna retoriska förskjutning som fond blir regeringens underlåtenhet att vidta konkreta åtgärder särskilt allvarlig. 
Svenska strafflagens 23 kap. 6 § BrB föreskriver ansvar för den som underlåter att i tid avslöja ett förestående eller pågående brott:

\lagrum{23 kap. 6 § 1 st. BrB\quad Den som underlåter att i tid anmäla eller annars avslöja ett förestående eller pågående brott ska, i de fall det är särskilt föreskrivet, dömas för underlåtenhet att avslöja brottet enligt vad som är föreskrivet för den som medverkat endast i mindre mån.}

Med kännedom om det pågående mönstret av brottslighet i Gaza – dokumenterat av FN-organ, ICC, ICJ och civilsamhällesaktörer – 
föreligger en rättslig skyldighet att agera. Uteblivna sanktioner, rättsliga åtgärder eller andra former av press utgör inte 
bara en underlåtenhet att agera – utan, i detta sammanhang, en underlåtenhet att avslöja brott.

När detta dessutom åtföljs av ett språkbruk som döljer den verkliga rättsliga kontexten – och felaktigt hänvisar 
till en icke tillämplig rätt till självförsvar – förstärks den rättsliga betydelsen. 
Det utgör en passiv medverkan, eller i vissa tolkningar, en \textit{stämplingsliknande handling} i folkrättslig mening.

\medskip

\subsubsection{Stämpling till folkrättsbrott genom vilseledande legitimering}

När den svenska regeringen i officiella uttalanden påstår att ”Israel har rätt att försvara sig” – trots att detta 
enligt folkrätten inte gäller en ockuperande makt – utgör det inte enbart en felaktig analys. 
Det är ett vilseledande rättsligt påstående med potentiellt brottsbefrämjande verkan.

I svensk rätt är stämpling ett förberedelsebrott där gärningsmannen uppmuntrar eller förstärker någon annans brottsavsikt. 
Analogt gäller här: genom att påstå att ett olagligt agerande är rättfärdigat enligt internationell rätt, 
ger Sverige ett rättsligt stöd till fortsatt våld, vilket i sig kan bidra till brottets fortsättning.

Jämför: den som, felaktigt och med auktoritet, säger till någon att denne ”har rätt att skjuta inkräktare” trots 
att det enligt gällande rätt inte föreligger någon sådan rätt – begår en form av stämpling, 
om detta får till följd att brott begås.

Internationella domstolen (ICJ) har i två rådgivande yttranden – 2004 och 2024 – fastställt att Israel är ockupationsmakt 
och att blockaden av Gaza utgör en folkrättsstridig kollektiv bestraffning. I detta rättsläge är 
självförsvarsrätten enligt FN-stadgans artikel 51 inte tillämplig gentemot Gaza. 

Detta gäller särskilt eftersom Gazas lagligt valda de facto-regering uttryckligen erkänt internationell rätt, inklusive 
Säkerhetsrådets resolutioner såsom 242, medan Israel konsekvent har avvisat både FN-beslut och folkrättens begränsningar.

Att i detta sammanhang ändå tillerkänna Israel en rätt till självförsvar är inte endast ett rättsligt felsteg, 
utan en aktiv vilseledning med potentiellt brottsbefrämjande effekt. Det förskjuter diskussionen från legalitetsfrågan till en proportionalitetsbedömning – ett felgrepp med normförändrande verkan.
Därigenom normaliseras ett våldsutövande som enligt folkrätten aldrig borde ha påbörjats.

När detta sker i ett läge där:\\
- det finns överväldigande dokumentation om pågående folkrättsbrott, och\\
- gärningsmannen (Israel) står i fortsatt militärt interagerande med civilbefolkningen,\\

...kan regeringens uttalande inte enbart förstås som passivitet. 
Det är en aktiv handling med potentiellt uppviglande karaktär – det närmar sig *stämpling* till folkrättsbrott.

\subsubsection{Stämpling till folkrättsbrott genom vilseledande demonisering}

Regeringen skriver: \enquote{Mer än ett år efter Hamas fruktansvärda terroristattacker den 7 oktober 2023}.\footnote{\url{https://www.regeringen.se/regeringens-politik/med-anledning-av-situationen-israelpalestina/vad-regeringen-gor-med-anledning-av-kriget-mellan-israel-och-hamas/}}

Hur har regeringen fastställt att det rör sig om terroristattacker som Hamas som organisation bär skuld för?

\paragraph{}
Hamas politiska ledning – som utgör Gazas folkvalda regering – har inte förnekat att civila israeler dödats den 7 oktober 2023, men har förklarat att sådana handlingar inte beordrats av den centrala ledningen. De ska i stället ha utförts av odisciplinerade element, andra fraktioner eller civila utom kontroll i en kaotisk situation. Denna distinktion har ignorerats av den svenska regeringen, som utan juridisk prövning tillmätt det israeliska narrativet full trovärdighet. Därmed förvägras Gazas representanter en grundläggande rättsstatlig princip: att handlingar ska bedömas individuellt och att skuld inte får kollektiviseras – särskilt inte under ockupation. En sådan asymmetrisk normtillämpning innebär att självförsvarsrätten fråntas den förtryckta parten, samtidigt som den förbehålls ockupationsmakten.

\paragraph{}
Den svenska regeringens språkbruk avslöjar en djup normativ asymmetri: det som för Israel är \enquote{misstag}, är för Gaza \enquote{terrorism}. Denna begreppsanvändning är inte neutral, utan bygger på en rasifierad och kolonial tolkningsstruktur där den ockuperade betraktas som kollektivt ansvarig, medan ockupationsmakten ursäktas med begrepp som \enquote{rätt till självförsvar} eller \enquote{komplex säkerhetssituation}.

\begin{itemize}
\item Den palestinska regeringen hålls ansvarig även för handlingar begångna av okontrollerade grupper eller civila i ett kaosartat läge,
\item medan Israel tillåts skjuta raketer mot sjukhus, avrätta medicinsk personal och begrava kroppar i massgravar – utan rättsliga följder, så länge det finns en narrativ reserv (”operationellt misstag”).
\end{itemize}

Hamas har i internationella intervjuer (t.ex. BBC 2023-11-07) tillstått att civila israeler dödats, men förklarat att detta inte varit målet. Beordringar från den militära ledningen ska uttryckligen ha undantagit kvinnor, barn och gamla från angrepp. Det finns också andra väpnade fraktioner i Gaza som inte lyder under Hamas’ politiska ledning. Men detta juridiska faktum – att ansvar måste individualiseras – nonchaleras systematiskt av västliga regeringar när det gäller palestinska aktörer.

\paragraph{}
Resultatet blir en språklig stämpling där palestinier som kollektiv förlorar rätten till självförsvar, samtidigt som deras våld reduceras till ett kulturellt problem. I praktiken förväntas en förtryckt och instängd befolkning upprätthålla full kontroll och juridisk disciplin, medan en teknologiskt överlägsen ockupationsmakt tillåts begå systematiska övergrepp utan ansvar – eftersom den anses vara ”civiliserad”.

\paragraph{}
Detta är inte bara en moralisk skandal, utan också en rättslig förskjutning där internationell humanitär rätt tillämpas selektivt – i strid med Genèvekonventionernas krav på likabehandling av civilbefolkningar, och ICJ:s påpekande (Bosnia v. Serbia, §430–432) att skuld inte kan härledas kollektivt, särskilt inte i asymmetri mellan ockupant och ockuperad.

\paragraph{}
Samtidigt accepterar den svenska regeringen utan prövning Israels påståenden om att läkare, sjukvårdsarbetare, ambulansförare och journalister varit \enquote{terrorister} – även i fall där massgravar avslöjat avrättningar av civila med bakbundna händer\footnote{\url{https://en.wikipedia.org/wiki/Gaza_Strip_mass_graves}} \footnote{\url{https://www.middleeasteye.net/news/new-video-evidence-disputes-israeli-armys-account-medic-killings}}, där ambulanser förstörts och samtliga sjukhus bombats\footnote{\url{https://en.wikipedia.org/wiki/Attacks_on_health_facilities_during_the_Gaza_war}}. Läkare sitter fortsatt fängslade och flera har torterats till döds.\footnote{\url{https://www.middleeasteye.net/news/war-gaza-prominent-palestinian-doctor-tortured-and-killed-israeli-detention}}

\paragraph{}
Israels försvar – att sådana dödsfall varit \enquote{operationella misstag} – accepteras utan vidare. Men varför godtas inte även Gazas regerings förklaringar? Om samma måttstock tillämpades borde även deras utsagor tillmätas trovärdighet.
Regeringen förhåller sig dessutom tyst till de uppgifter som publicerats i israelisk press, 
enligt vilka IDF bekräftat att ett s.k. \enquote{mass Hannibal event} genomfördes den 7 oktober, 
då israelisk militär öppnade eld mot civila israeler i syfte att förhindra att dessa togs som 
gisslan av Hamas.\footnote{\url{https://electronicintifada.net/blogs/asa-winstanley/we-blew-israeli-houses-7-october-says-israeli-colonel}} Dessa uppgifter behandlas mer ingående längre fram i dokumentet.

\paragraph{}
Hamas har själv uppgett att den 7 oktober utgjorde en legitim militär operation riktad mot israeliska militära mål, 
och detta har också bekräftats av internationellt respekterade militära bedömare såsom 
Scott Ritter.\footnote{\url{https://scottritter.substack.com/p/the-october-7-hamas-assault-on-israel}} 

\paragraph{}
Ett av Hamas deklarerade mål var att befria tusentals palestinier som hålls i israeliskt förvar utan rättegång, 
i strid med folkrätten, genom så kallad ”administrativ internering”.\footnote{\url{https://www.btselem.org/administrative_detention/statistics}}

Genom att kategoriskt och utan åtskillnad beteckna alla väpnade palestinska aktörer som ”terrorister” 
bidrar regeringen till en normativ förskjutning som suddar ut folkrättens tydliga skillnad mellan legitim väpnad kamp och folkrättsbrott. 
Hamas, som utgör Gazas folkvalda regering och säger sig erkänna folkrätten, 
har vid flera tillfällen arresterat medlemmar av andra väpnade grupper – såsom Islamiska Jihad 
och salafistiska fraktioner – för oauktoriserade raketattacker 
mot Israel.\footnote{Se t.ex. Haaretz (2015-05-27): \textit{Hamas Arrests Islamic Jihad Activists for Rocket Fire}.\url{https://www.haaretz.com/2015-05-27/ty-article/.premium/hamas-arrests-islamic-jihad-activists-for-rocket-fire/0000017f-e9cd-df2c-a1ff-ffdd5ef50000}}

Trots folklig förankring buntar regeringen samman Hamas med de grupper Hamas själv betraktar som kriminella – inklusive en ISIS-associerad 
knarksmugglande klan i södra Gaza, som Israels premiärminister Netanyahu 
bekräftat att Israel beväpnar för att bekämpa Hamas.\footnote{Se \textit{The Grayzone} (2025-06-05): \textit{Israel arming ‘ISIS-affiliated’ gang in southern Gaza}, \url{https://thegrayzone.com/2025/06/05/israel-arming-isis-gang-gaza/}.} Var är regeringens fördömande av att Israel samarbetar med sådana aktörer?

Att systematiskt underlåta att erkänna den mångfald av väpnade aktörer i Gaza – samt den 
interna repression Hamas riktar mot gangsters, extremistiska eller odisciplinerade grupper – leder till en orättvis 
kollektiv demonisering av hela det palestinska motståndet. 

Regeringens retorik beskriver alla väpnade palestinska grupper som ”terrorister”, oavsett om de är folkvalda eller 
agerar inom folkrättens ramar. Samtidigt har Hamas – Gazas folkvalda regering – konsekvent fördömts, medan motsvarande israeliska policy inte ens nämns av regeringen.
Denna förvrängning strider mot neutralitetsprincipen i folkrätten, vilken kräver att neutrala stater förhåller sig 
objektiva och faktabaserade i konfliktsituationer.

Denna form av sammangående demonisering, utan grund i rättslig bedömning och utan kravet att erkänna fler 
aktörer eller en mångfasetterad verklighet, representerar ett brott mot Genèvekonventionernas förbud mot kollektiv 
bestraffning. Det aktualiserar även principerna i Nürnbergstadgan, särskilt principerna I och VI, där indirekt stöd 
genom propaganda eller legitimering bör, mot bakgrund av Nürnbergprinciperna och Genèvekonventionerna, 
betraktas som en form av medverkan enligt internationell straffrätt.

På samma sätt beskriver regeringen Hezbollah som en ”terroristorganisation”, trots att Hezbollah utgör en omfattande 
politisk och social rörelse med djupt folkligt stöd i Libanon, och ingår i landets regeringskoalition. 
Under flera perioder har rörelsen haft parlamentarisk majoritet.

Att enskilda medlemmar inom en folkförankrad befrielserörelse begår handlingar som kan klassificeras som 
terrorbrott innebär inte att hela organisationen därmed juridiskt definieras som en terroristisk aktör. 
Om regeringen drar sådana generaliserande slutsatser utan nyansering, aktualiseras frågan om kollektiv skuldbeläggning – något 
som enligt folkrätten är förbjudet. Det skulle i förlängningen innebära att hela etniska eller nationella grupper 
riskerar att demoniseras, något som står i direkt strid med principen om individuellt ansvar i internationell rätt.

Sådana handlingar faller inom ramen för medverkan till aggressionsbrott och brott mot mänskligheten, enligt princip VI (c) i Nürnbergstadgan, 
vilket också omfattar propaganda och offentlig legitimering av handlingar som strider mot folkrätten.

Genom att använda statliga uttryck som vilseleder, rättfärdigar eller anonymiserar brottsliga handlingar, bidrar regeringen till en 
rättslig stämpel – en funktionell form av stämpling – snarare än en oskyldig politisk ton. 
Det kan karakteriseras som en form av stämpling till folkrättsbrott, eftersom det ges politisk legitimitet 
till en ockupationsmäktig makt som bryter mot grundläggande folkrättsliga normer.
Mot denna bakgrund framstår regeringens kategoriska språkbruk som en form av stämpling till 
folkrättsbrott – inte genom direkt deltagande i våldshandlingar, utan genom att ge politisk och retorisk 
täckning för handlingar som i internationell rätt saknar legitimitet.\footnote{Princip VI(c) i Nürnbergstadgan omfattar även medverkan i brott mot mänskligheten, inklusive att genom offentlig kommunikation uppmana till eller rättfärdiga sådana handlingar. Se även: \textit{Charter of the International Military Tribunal – Annex to the Agreement for the prosecution and punishment of the major war criminals of the European Axis} (1945), artikel 6(c).}




\subsubsection{Regeringens vägran att stödja vapenvila – uttryck för konkludent medgivande}

Sveriges agerande inför och efter FN:s generalförsamlings omröstning om vapenvila i december 2023 tydliggör en strategiskt betingad vägran att fullt ut fördöma Israels krigföring, trots då redan dokumenterade civila massdöd\footnote{\url{https://euromedmonitor.org/en/article/5908/Israel-hits-Gaza-Strip-with-the-equivalent-of-two-nuclear-bombs}}.

Utrikesminister Tobias Billström deklarerade att det inte vore “rättvist” att kräva vapenvila eftersom “Israel måste kunna bekämpa Hamas” – en formulering som tillför ett rättsligt undantag till själva principen om eldupphör. Stödet för FN-resolutionen kom sent och var uppenbart taktiskt: Sveriges ja-röst åtföljdes av fortsatta uttalanden om Israels rätt att genomföra militära operationer, vilket urholkar både den rättsliga och moraliska effekten av röstningen.

\begin{quote}
\textit{”Vi kan inte mana till en vapenvila som skulle innebära att Israel inte kan bekämpa Hamas. Det vore inte rättvist.”} \\
– Tobias Billström till TT, 2023-12-11 (Aftonbladet)\footnote{\url{https://www.aftonbladet.se/nyheter/a/0QR2eE/billstroms-ord-om-vapenvila-i-gaza-far-stark-kritik}}
\end{quote}

Detta utgör ett skolboksexempel på \textit{läpparnas bekännelse}: en diplomatisk markering som i sak neutraliseras av reservationer som möjliggör fortsatt våld. Juridiskt innebär det ett underminerande av skyldigheten att agera preventivt enligt folkmordskonventionens artikel I, samt en vilseledande kommunikation med potentiellt brottsbefrämjande effekt.

\textbf{Rättslig följd:} Regeringens uttalanden kombinerar:
\begin{itemize}
  \item en formell eftergift (sen ja-röst),
  \item med ett materiellt medgivande till fortsatt bombkampanj (”Israel måste kunna bekämpa Hamas”),
  \item vilket i folkrättslig mening utgör \textit{konkludent medgivande} till folkrättsbrott.
\end{itemize}

Enligt ICJ i \textit{Bosnia v. Serbia} (2007, §432) kan detta kvalificeras som:
\begin{itemize}
  \item underlåtelse att använda inflytande för att stoppa brottet,
  \item språkligt och diplomatiskt agerande som legitimerar fortsatt våld,
  \item indirekt uppmuntran genom rättslig och politisk vilseledning.
\end{itemize}


\subsection*{Utrikesdepartementets svar: Strategisk passivitet förklädd till neutralitet}

Den svenska regeringen erkänner i sitt svar till en svensk medborgare att "ett stort antal civila dödsoffer inklusive barn" har förekommit i Gaza, och att situationen är "djupt oroande".

Som stöd för vår analys av regeringens retorik och rättsliga undandraganden återges här ett officiellt mejlsvar från Utrikesdepartementet. Skrivelsen utgör ett bevis för hur regeringen – trots kännedom om omfattande civila dödsoffer – medvetet undviker att vidta några rättsliga eller diplomatiska åtgärder mot Israel, och istället reducerar sin hållning till "uppmaningar" och vaga hänvisningar till EU-samverkan.

\begin{quote}
\begin{verbatim}
From: UD MENA Brevsvar <ud.mena.brevsvar@gov.se>
Date: Mon, Apr 14, 2025 at 10:31 AM 
Subject: Sv: [Pchr_english] Israel Intensifies Airstrikes... 
To: lasse.l.karagiannis@gmail.com

Hej,

Tack för din e-post som inkommit till Utrikesdepartementet. 
Jag vill inleda med att be om ursäkt för att du har fått vänta på svar. 
Vi får in många brev till departementet just nu, vilket har fört med sig 
att vi har långa handläggningstider.

Regeringen ser allvarligt på att stridigheterna har återupptagits. 
Uppgifterna om ett stort antal civila dödsoffer, inklusive barn, 
är djupt oroande.

Regeringen kräver en omedelbar vapenvila 
och uppmanar parterna att återgå till förhandlingarna 
så att all kvarvarande gisslan kan släppas, 
det humanitära tillträdet säkerställas 
och ett varaktigt slut på stridigheterna uppnås.

Kraven på respekt för folkrätten, inklusive den internationella 
humanitära rätten, har varit – och fortsätter att vara – 
ett av regeringens nyckelbudskap.

Dessa budskap framförs i våra egna kontakter med Israel 
och vi gör det tillsammans med andra EU-länder och likasinnade. 
Ibland sker det offentligt, och ibland på annat sätt.

Hur regler respekteras och om krigsförbrytelser begåtts 
måste bedömas från fall till fall utifrån 
den internationella humanitära rätten.

Det är inte regeringens roll att göra sådana bedömningar. 
För regeringen är det centralt att eventuella överträdelser 
av den internationella humanitära rätten 
och möjliga krigsförbrytelser utreds 
och att ansvarsutkrävande säkerställs.

Både ICC och ICJ har pågående utredningar 
om situationen i Palestina. 
Hittills har de kommit fram till 
att Israel måste göra mer för att skydda 
den civila befolkningen. 
Det är något regeringen också har framfört till Israel.

Med vänlig hälsning,
Mellanöstern- och Nordafrikaenheten 
Utrikesdepartementet 
103 39 Stockholm 
\end{verbatim}
\end{quote}

\vspace{1em}

Detta dokument visar att regeringen, trots erkännande av civila dödsoffer, medvetet undviker att själv göra 
rättsliga bedömningar, vilket strider mot dess skyldighet enligt artikel I i folkmordskonventionen att 
agera preventivt. Denna passivitet möjliggör fortsatt straffrihet och utgör en form av konkludent 
medverkan genom underlåtenhet.

Trots detta:
\begin{itemize}
\item kräver man inte ansvar från Israel,
\item avstår från att uttala sig om huruvida krigsbrott eller folkmord begåtts,
\item och överlåter all juridisk bedömning till internationella domstolar, utan att själv agera i enlighet med folkmordskonventionens bindande artikel I, som föreskriver att alla stater måste förebygga och förhindra folkmord, inte passivt invänta domstolsbeslut.
\end{itemize}

I mejlsvaret heter det:

\begin{quote}
”Hur regler respekteras och om krigsförbrytelser begåtts måste bedömas från fall till fall... Det är inte regeringens roll att göra sådana bedömningar.”
\end{quote}

Detta är felaktigt i folkrättslig mening. Enligt ICJ:s dom i *Bosnia v. Serbia* (2007, §§430–431) har alla stater ett 
självständigt ansvar att \textbf{förebygga, förhindra och stävja} folkmord så fort de rimligen kunnat förutse att sådana 
brott kan äga rum. Den svenska regeringen avstår här från att fullgöra sin \textbf{positiva förpliktelse enligt artikel I i 
folkmordskonventionen} och gömmer sig istället bakom en retorik om neutralitet.

Samtidigt visar regeringen genom sin vapenviljeuppmaning att man mycket väl har kapacitet att påverka konflikten. 
Men genom att kräva att båda parter ska "återgå till förhandlingar", utan att erkänna den asymmetri som råder 
mellan en instängd, bombad civilbefolkning och en modern militärmakt med flyg och tanks, skapas en \textbf{falsk ekvivalens} 
mellan angripare och offer.




\subsubsection{Hälsningssignaler till blockaden – bordning av \textit{Madleen} och drönarattacken mot \textit{Conscience}}

Den svenska regeringens agerande i dessa två incidenter – bordningen av \textit{Madleen} den 9 juni 2025 och attacken mot \textit{Conscience} den 2 maj 2025 – är juridiskt betydelsefullt på flera punkter:

\begin{enumerate}
  \item \textbf{Bombningen av \textit{Conscience}} bekräftades i riksdagen, men regeringen valde att inte framföra någon protest mot Israels militära agerande – trots att attacken skedde i internationellt vatten mot ett civilt fartyg med humanitär last. Det föreligger således en aktiv tystnad som kan tolkas som \emph{konkludent medgivande} till brott som strider mot sjöfartens frihet och civil skyddad status.
  
  \item \textbf{Bordningen av \textit{Madleen}} följdes av utrikesminister Maria Malmer Stenergärds uttalande om att Israel med hänvisning till folkrätten "har vissa möjligheter att eskortera skeppet". Detta ger militära definitioner rättslig legitimitet och bör betraktas som en del av en normativ förskjutning där civilt sjöfartsutrymme under humanitära aktioner reduceras till säkerhetszon.
\end{enumerate}

\noindent
\textbf{Rättslig betydelse:}
\begin{itemize}
  \item \emph{Konkludent samtycke} till blockadens rättsvidrighet då Sverige inte motsatte sig attackerna, vilket underlättar fortsatt folkrättsvidrig praxis på internationellt vatten.
  \item Språkliga legitimationsreserver för angripande stat – när det uttryckligen anges att ”Israel har möjligheter…”.
  \item Detta mönster följer en trend: sjukhus attackeras (över 660 hälsocenter), läkare torteras eller dödas, massgravar framträder – men svenska regeringsuttalanden blundar. Dessa hamnar alla i samma mönster av tyst acceptans för brott mot Genèvekonventionerna.  
\end{itemize}

\noindent
Utifrån praxis i ICJ (*Bosnia v. Serbia*, §432) innebär:

\begin{itemize}
  \item \emph{Underlåtenhet att protestera} – vilket kan ses som medverkan genom att legitimera fortsatta brott,
  \item \emph{Informellt stöd} via regeringsuttalanden som skapar en ram där våld på humanitärt uppdrag anses acceptabelt,
  \item Förstärkning av narrativ där civila offer antingen ignoreras eller sägs stå i vägen – snarare än att skyddas.
\end{itemize}

\noindent
Resultatet: Sveriges regering har, genom språk och undvikande agerande, medverkat till – eller åtminstone inte hindrat – uppbyggnaden av ett folkrättsstridigt regeringsmönster i internationellt vatten.




\subsubsection{Folkrättslig passivitet som medverkan}

Enligt artikel I i folkmordskonventionen är varje stat förpliktad att inte bara avstå från att begå folkmord, utan även att aktivt förhindra sådana brott inom ramen för sin möjlighet. Detta bekräftades i ICJ:s dom \textit{Bosnia v. Serbia} (2007), där domstolen slog fast att stater har en faktisk och preventiv handlingsskyldighet.

Regeringens underlåtenhet att:

\begin{itemize}
    \item frysa bilaterala relationer, inklusive diplomatiska besök och samarbetsavtal,
    \item införa ekonomiska och militära sanktioner, däribland vapenembargo,
    \item ansluta sig till internationella rättsprocesser såsom ICJ-målet initierat av Sydafrika,
\end{itemize}


...innebär ett kontraktsbrott mot Sveriges traktatförpliktelser enligt folkmordskonventionen och FN-stadgan. Det kan också utgöra medverkan till fortsatt folkrättsbrott, i den mån passiviteten sker med kännedom om risk och utan proportionalt motverkande åtgärder.

\medskip

\textbf{Slutsats:} Regeringens uttalanden den 23 och 27 maj 2025 utgör inte endast politiska markörer. De är rättsligt relevanta dokument som – mot bakgrund av regeringens faktiska handlingsvägran – får betraktas som konkludent medgivande till ett pågående folkrättsbrott, och i förlängningen en möjlig form av medverkan enligt svensk och internationell rätt.


