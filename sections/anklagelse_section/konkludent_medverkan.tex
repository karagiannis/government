%filnamn: konkludent_medverkan.tex

% SYFTE: Sätt regeringens språk i juridisk kontext.
% Visa att språket inte är oskyldigt – det utgör handling i rättslig mening.

\subsection{Konkludent beteende i politiskt hänseende}

Konkludent handlande är en civilrättslig princip som innebär att bindande rättsverkningar kan uppstå även i avsaknad av uttryckligt samtycke, om parternas agerande objektivt sett ger intryck av att ett avtal föreligger. I svensk avtalsrätt – liksom inom den kontinentala civilrättsliga traditionen – kan avtal således uppstå genom faktisk handling, så länge detta handlande motsvarar en tyst accept.

I ett politiskt sammanhang motsvaras detta av principen \textit{substance over form}: det är inte vad som sägs, utan vad som faktiskt görs, som bör ligga till grund för rättslig bedömning.

Ett särskilt talande exempel är Turkiets president Erdoğan. I offentligheten har han gått till hårt angrepp mot Israel – och i juli 2024 till och med hotat med militär intervention, med hänvisning till tidigare turkiska operationer i Karabach och Libyen.\footnote{Reuters/Jerusalem Post, \textit{“Erdogan threatens Israel: 'Like we entered Karabakh and Libya – we will do the same to Israel'”}, 28 juli 2024.} I retoriken framstår han som en ovillkorlig försvarare av det palestinska folket.

Men bakom denna fasad fortgår materiella kontakter. Trots offentliga utspel har Turkiet inte brutit alla relationer med Israel. Tvärtom har handel, logistik och andra tyst legitimerande aktiviteter fortsatt i skuggan av Erdogans bombastiska tal. Detta skapar ett spel för gallerierna, där Erdogan – medvetet eller ej – tillåts fungera som en kontrollerad ventil. Han kritiserar, Israel svarar med fördömanden, och båda parter vinner inrikespolitiska poäng. Resultatet är ett tyst samförstånd där symbolisk konflikt ersätter faktisk konfrontation.

Den svenska regeringen agerar på ett liknande sätt. Den uttrycker ibland "oro" över situationen i Gaza, men fortsätter parallellt att upprätthålla handel, diplomatiska kontakter och säkerhetssamarbete med Israel – även under perioder där landets agerande beskrivs som folkrättsvidrigt av FN:s specialrapportörer. När detta sker utan protester eller motåtgärder, utgör det en form av konkludent medgivande – en tyst accept av den pågående folkrättsbrottsligheten.

Detta får särskild tyngd i ljuset av Regeringsformen:

\lagrum{10 kap. 1 § RF\quad Överenskommelser med andra stater eller med mellanfolkliga organisationer ingås av regeringen. Lag (2010:1408).}

Sådana överenskommelser – formella eller implicita – får inte strida mot Sveriges folkrättsliga förpliktelser. Om regeringens samlade agerande utgör en faktabaserad acceptans av Israels folkrättsstridiga politik, då föreligger enligt internationell rätt ett \textit{konkludent samförstånd} – med rättsverkningar som kan aktivera medansvar enligt bland annat Nürnbergprincip I.

I det följande dokumenteras de handlingar, uttalanden och underlåtelser som sammantaget utgör konkludent medverkan till folkrättsbrott:

\begin{itemize}
    \item Regeringen har vid upprepade tillfällen tillerkänt Israel en \enquote{rätt till självförsvar} gentemot Gaza, trots att Israel enligt ICJ är en ockupationsmakt, och att blockaden av Gaza sedan 2007 är olaglig. Självförsvarsrätten enligt FN-stadgans artikel 51 är inte tillämplig på ockupationer.\footnote{ICJ:s rådgivande yttrande 2004, bekräftat 2024.}
    
    \item Israel vägrar erkänna FN:s säkerhetsrådsresolution 242 (1967), som fastslår att Israel måste dra sig tillbaka från ockuperade områden. Den palestinska sidan, inklusive Hamas och andra partier, har däremot accepterat gränserna från 1967 som grund för en framtida stat.\footnote{\url{https://www.972mag.com/hamas-fatah-elections-israel-arrogance/}}
    
    \item Internationella domstolen (ICJ) har slagit fast att ockupationen är olaglig och karaktäriserat den som ett apartheidsystem – vilket enligt Romstadgan är ett brott mot mänskligheten. Trots detta har Sverige inte infört några sanktionsåtgärder motsvarande dem som riktades mot apartheidregimen i Sydafrika.
    
    \item Israel har fängslat framstående palestinska läkare, inklusive professorer, utan rättssäker process. I april 2024 dog Dr. Adnan al-Bursh, ortoped och professor, efter att ha torterats till döds i israeliskt förvar.\footnote{\url{https://www.middleeasteye.net/news/war-gaza-prominent-palestinian-doctor-tortured-and-killed-israeli-detention}}
    
    \item Regeringen har inte fördömt Israels tillämpning av det så kallade Hannibal-direktivet, vilket enligt rapporter kan ha bidragit till dödandet av hundratals israeliska civila under 7 oktober-attackerna.\footnote{\url{https://www.france24.com/en/live-news/20231215-israel-social-security-data-reveals-true-picture-of-oct-7-deaths}}
    
    \item Under hela 2023 dödades minst 507 palestinier på Västbanken, inklusive minst 81 barn. Av dessa dödades 299 efter den 7 oktober, vilket innebär att över 200 dödades under årets första nio månader – långt innan kriget i Gaza inleddes.\footnote{\url{https://www.amnesty.org/en/latest/news/2024/02/shocking-spike-in-use-of-unlawful-lethal-force-by-israeli-forces-against-palestinians-in-the-occupied-west-bank/}}
    
    \item Regeringen har aktivt underminerat FN-organet UNRWA:s arbete genom att frysa bistånd, trots att inga bevis framkommit som rättfärdigar kollektiva bestraffningar av FN-personal.
    
    \item Israel har systematiskt bombat sjukhus i Gaza utan att Sverige har agerat utöver återhållsamma verbala uttalanden. Detta trots att sjukhus åtnjuter särskilt skydd enligt Genèvekonventionerna.
    
    \item Israeliska styrkor har avrättat ambulanspersonal – bland annat dokumenterat genom video – utan att den svenska regeringen har reagerat proportionerligt. När brottet till sist erkändes av Israel benämndes det som ett \enquote{operationellt misstag}.\footnote{\url{https://www.middleeasteye.net/news/new-video-evidence-disputes-israeli-armys-account-medic-killings}}
    
    \item Öppen och återkommande hets till folkmord i israeliska medier, parlamentet och sociala plattformar har dokumenterats sedan minst 2010. Den svenska regeringen har aldrig offentligt fördömt dessa uttryck – trots att hets till folkmord utgör ett särskilt brott enligt FN:s folkmordskonvention.\footnote{\url{https://www.youtube.com/watch?v=9GbKsAvuBDM}}
\end{itemize}


\subsection{Medverkan genom handling och underlåtenhet}
\label{subsec:konkludent}

\subsubsection{Rättslig grund för medhjälparansvar}
Enligt folkrätten kan staters ansvar aktiveras inte endast genom direkta brott utan även genom medverkan till annan stats brottslighet. I enlighet med ILC:s \textit{Draft Articles on State Responsibility} (2001) artikel 16:

\begin{quote}
\textit{"A State which aids or assists another State in the commission of an internationally wrongful act by the latter is internationally responsible for doing so if:}
\begin{enumerate}
    \item \textit{that State does so with knowledge of the circumstances of the internationally wrongful act; and}
    \item \textit{the act would be internationally wrongful if committed by that State."}
\end{enumerate}
\end{quote}

Denna princip har tillämpats av ICJ i \textit{Bosnien mot Serbien} (2007, §420) och förstärkts i \textit{Belgien mot Senegal} (2012) rörande universell jurisdiktion.

\subsubsection{Konkludent godkännande genom språkbruk}
Regeringens semantiska val konstituerar rättsligt relevanta handlingar:

\begin{itemize}
    \item \textbf{Terminologisk undvikande}: Konsistent vägran att använda termen "folkmord" trots:
    \begin{itemize}
        \item ICJ:s preliminära åtgärdsbeslut 26 jan 2024 (indikation på plausibelt brott)
        \item FN:s specialrapportörs uttalande 26 mars 2024: "reasonable grounds to believe genocide is occurring"
    \end{itemize}
    
    \item \textbf{Legitimeringstekniker}: Användning av fraser som:
    \begin{quote}
        "Israels rätt att försvara sig" (Uttalande UD, 28 okt 2023)\\
        "Komplex situation" (Intervju Tobias Billström, SVT 15 nov 2023)
    \end{quote}
    vilka i kontexten av systematisk förstörelse av civil infrastruktur (enligt FN OCHA-rapport 45/2024) uppfyller rekvisiten för medhjälp genom moraliskt stöd.
\end{itemize}

\subsubsection{Aktiv stödhändelser}
\begin{tabular}{p{0.2\textwidth}p{0.75\textwidth}}
\textbf{Datum} & \textbf{Händelse \& rättslig kvalifikation} \\
\hline
25 okt 2023 & \textbf{Undertecknande av vapenleveransavtal}\footnote{\url{https://www.riksdagen.se/sv/dokument-och-lagar/dokument/skriftlig-fraga/sveriges-avtal-med-israeliska-elbit_hb11396/}} \\
& \footnotesize\textit{Rättskvalifikation}: Ekonomiskt samarbete; underlåtenhet att minska finansiell förmåga hos brottsbenägen stat \\
\hline
15 jan 2024 & \textbf{Underlåtenhet att suspendera vapenexportlicenser} \\
& \footnotesize\textit{Rättskvalifikation}: Ökad vapenexport till brottsbenägen stat\footnote{\url{https://proletaren.se/artikel/vapexporten-till-israel-okar/}} \\
\hline
10 dec 2020 & \textbf{Undertecknande av innovationsavtal med Israel} \\
& \footnotesize\textit{Rättskvalifikation}: Ekonomiskt stöd till brottsbenägna stat enl. EU:s definition av "tredje parts samarbeten" \\
\end{tabular}

\subsubsection{Underlåtelsehandlingar med verkan}
\begin{itemize}
    \item \textbf{Inaktivitetsgradient}: Frånvaron av diplomatiska åtgärder trots eskalering:
    \begin{itemize}
        \item Inga återkallade ambassadörer (jämfört med Sydafrika-fallet)
        \item Ingen tillämpning av Magnitsky-liknande sanktioner
    \end{itemize}
    
    \item \textbf{Procedurell sabotage}: Blockering av EU-försök till gemensam deklaration (enl. läckta dokument från EUMC 12 feb 2024)
\end{itemize}

\subsubsection{Slutlig rättslig bedömning}
Sveriges agerande uppfyller kriterierna för medhjälp enligt artikel 16 genom:
\begin{enumerate}
    \item \textit{Kunskapsrekvisitet}: Överväldigande bevis om brottslighet
    \item \textit{Bidragsrekvisitet}: Faktiskt stöd genom vapen, diplomati och ekonomi
    \item \textit{Gränsöverskridande effekt}: Sverige har jurisdiktion enligt universell princip
\end{enumerate}
