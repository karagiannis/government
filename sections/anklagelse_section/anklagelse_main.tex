
%filnamn: anklagelse_main.tex
% Huvudpåstående: att Sveriges regering är juridiskt ansvarig enligt folkrätt och grundlag

% Denna sektion bör etablera själva grunden för åklagarskriften: att regeringen inte bara varit passiv utan genom konkludent handlande och aktiv samverkan brutit mot rättsligt bindande normer.

% Nedan följer planerade subsections, där varje ingår i kedjan av bevis.
% filnamn: sammantattande_skuldsats.tex
% SYFTE: Ställ hypotesen klart. "Regeringen är skyldig." – detta ska bevisas.
% Introducera grundlagen och folkrätten som rättskällor. Skapa förväntan.

\subsection{Regeringens brott mot folkrätten och grundlagen}

\noindent
Sveriges regering har, i sin hantering av Israels agerande i Palestina, genom både underlåtenhet och aktivt agerande brutit mot såväl internationella rättsförpliktelser som konstitutionella normer. Ansvaret omfattar fyra distinkta, men inbördes sammanhängande rättsöverträdelser:

\begin{enumerate}
    \item \textbf{Underlåtenhet att förebygga folkmord} i strid med artikel I i folkmordskonventionen
    \item \textbf{Brott mot FN-stadgans artikel 56} genom passivitet i främjandet av mänskliga rättigheter
    \item \textbf{Lojalitetsbrott mot RF 10 kap. 1 §} genom kontraktsbrott mot folkrättsliga åtaganden
    \item \textbf{Stämpling till krigsbrott enligt 23 kap. 6 § brottsbalken}, genom felaktiga offentliga uttalanden om att "Israel har rätt att försvara sig" på ockuperat territorium i strid med etablerad folkrätt och ICJ:s utlåtanden
\end{enumerate}

\lagrum{Artikel I, Folkmordskonventionen\quad De fördragsslutande parterna förbinder sig att förebygga och bestraffa folkmord.}

\noindent
ICJ:s praxis i målet \textit{Bosnien mot Serbien} (ICJ Reports 2007, §430) fastslår att denna skyldighet:

\begin{itemize}
    \item är \textit{erga omnes} – gäller gentemot hela det internationella samfundet,
    \item aktiveras vid konstaterad risk för folkmord – inte först vid fullbordat brott,
    \item innefattar en plikt att vidta rimliga åtgärder för att förebygga folkmord.
\end{itemize}

\medskip

\lagrum{Artikel 56, FN-stadgan\quad Medlemsstaterna förbinder sig att vidta gemensamma och enskilda åtgärder [...] för främjande av mänskliga rättigheter.}

\noindent
Sveriges passivitet står i skarp kontrast till de \textit{officiella varningar} som utgått från:

\begin{itemize}
    \item FN:s särskilde rapportör om folkmordsförebyggande (A/HRC/55/73, 25 mars 2024),
    \item ICC:s åklagare, som uttalat att skälig grund för folkmordsutredning föreligger,
    \item ICJ:s beslut om preliminära åtgärder i målet \textit{Sydafrika mot Israel} (januari 2024).
\end{itemize}

\medskip


\lagrum{10 kap. 1 § Regeringsformen (RF)\quad Överenskommelser med andra stater [...] ingås av regeringen.}
\noindent
Detta konstitutionella mandat innefattar en underförstådd \textit{lojalitetsplikt} att säkerställa efterlevnaden av traktatsförpliktelser. Motsvarande folkrättslig princip återfinns i \textit{pacta sunt servanda} (Wienkonventionen om traktaträtten, artikel 26), vilken av Internationella domstolen (ICJ) har bekräftats som sedvanerätt i målet \textit{Gabčíkovo–Nagymaros} (ICJ Reports 1997, s. 38).

\medskip

\lagrum{23 kap. 6 § brottsbalken\quad Den som underlåter att i tid anmäla eller annars avslöja ett förestående eller pågående brott ska [...] dömas för underlåtenhet att avslöja brott [...] Den som har ett bestämmande inflytande i en sammanslutning [...] ska dömas för underlåtenhet att förhindra brott.}

\noindent
Den svenska regeringens felaktiga och vilseledande offentliga uttalanden – där man påstår att ”Israel har rätt att försvara sig” – utgör inte endast en folkrättslig förvanskning, utan kan även kvalificeras som \textit{stämpling till krigsbrott}. Detta gäller särskilt i ljuset av Internationella domstolens (ICJ) förklaring i juli 2024 att ockupationen av Gaza och Västbanken är olaglig, samt upprepade varningar från FN-organ om överhängande folkmordsrisk.

\noindent
Regeringens vägran att vidareförmedla eller agera på tidigare premiärminister Ehud Olmerts uttalanden – där han själv erkänner att Israel bedriver ett utrotningskrig – samt att inte ansluta sig till Sydafrikas folkmordsstämning mot Israel, kan därmed aktualisera ansvar enligt 23 kap. 6 § första stycket brottsbalken, om underlåtenhet att avslöja ett förestående eller pågående brott.


\vspace{0.5cm}

\noindent
\textbf{Slutsats:} Genom att:

\begin{enumerate}
    \item konsekvent vägra erkänna folkmordsrisken trots överväldigande bevisläge,
    \item fortsätta tillhandahålla politiskt, ekonomiskt och militärt stöd till Israel,
    \item aktivt motverka internationella sanktionsåtgärder,
    \item offentligt legitimera ett pågående krigsbrott genom juridiskt felaktiga uttalanden,
\end{enumerate}

har Sveriges regering uppfyllt rekvisiten för \textit{medverkan till folkmord} enligt artikel III(e) i folkmordskonventionen samt brutit mot Regeringsformens krav på lojal och rättstrogen uppfyllelse av internationella överenskommelser.


% SYFTE: Ställ hypotesen klart. "Regeringen är skyldig." – detta ska bevisas.
% Introducera grundlagen och folkrätten som rättskällor. Skapa förväntan.


% filnamn: folkrattens_grundprinciper.tex
% SYFTE: Förklara den rättsliga struktur som Sverige bryter mot.
% Skapa spelplanen: folkrätt, RF, skyldigheter, suveränitetsgränser.

\subsection{Folkrättens bindande ramverk}

Folkrätten utgör det överstatliga regelverk som reglerar staternas inbördes relationer. Den vilar på tre bärande principer: suverän statsmakt, territoriell integritet och internationella förpliktelsers bindande karaktär. Dessa är kodifierade genom FN-stadgan, genom sedvanerättsliga normer och genom multilaterala traktater såsom Genèvekonventionerna och folkmordskonventionen.

Sverige är folkrättsligt bunden dels genom undertecknande och ratificering av sådana instrument, dels genom att vara medlem i Förenta Nationerna, där medlemskapet medför särskilda skyldigheter i enlighet med stadgans kapitel I.

\lagrum{Artikel 2(1), FN-stadgan\quad The Organization is based on the principle of the sovereign equality of all its Members.}

\lagrum{Artikel 2(4), FN-stadgan\quad All Members shall refrain in their international relations from the threat or use of force against the territorial integrity or political independence of any state.}

Dessa artiklar fastställer ett generellt våldsförbud, vilket utgör folkrättens kärna. Endast två undantag erkänns: beslut från säkerhetsrådet i enlighet med kapitel VII, samt rätten till självförsvar enligt artikel 51 – förutsatt att det föreligger ett väpnat angrepp riktat mot en suverän stat.

\lagrum{Artikel 51, FN-stadgan\quad Nothing in the present Charter shall impair the inherent right of individual or collective self-defence if an armed attack occurs against a Member of the United Nations.}

Begreppet ”självförsvar” är således territoriellt och rättssubjektmässigt begränsat. Det kan inte åberopas av en ockupationsmakt gentemot den befolkning som står under dess kontroll. Denna princip är bekräftad av Internationella domstolen (ICJ) i en rad avgöranden, däribland \textit{Legal Consequences of the Construction of a Wall in the Occupied Palestinian Territory} (2004).

\medskip

Ockupation regleras i detalj av 1907 års Haagreglemente och 1949 års fjärde Genèvekonvention. Dessa rättskällor stadgar att en ockupant har ett positivt skyddsansvar gentemot civilbefolkningen inom det ockuperade territoriet. Syftet med dessa regler är att avvärja just det mönster vi i dag bevittnar: systematiskt övervåld, kollektiva bestraffningar och åsidosättande av civila skyddsintressen.

\lagrum{Artikel 43, Haagreglementet\quad The authority of the legitimate power having in fact passed into the hands of the occupant, the latter shall take all the measures in his power to restore, and ensure, as far as possible, public order and safety, while respecting, unless absolutely prevented, the laws in force in the country.}

\lagrum{Artikel 27, Genèvekonvention IV\quad Protected persons are entitled, in all circumstances, to respect for their persons, their honour, their family rights, their religious convictions and practices, and their manners and customs. They shall at all times be humanely treated, and shall be protected especially against all acts of violence or threats thereof.}

\medskip

Vid bedömningen av en stats ansvar aktualiseras även frågan om underlåtenhet att agera. Detta har särskilt behandlats inom ramen för folkmordskonventionen, där konventionsstaternas skyldighet att förebygga är självständig och inte betingad av föregående folkrättslig bedömning från FN:s säkerhetsråd eller Internationella domstolen.

Sverige, såsom konventionsstat, är därför förpliktat att inte endast avstå från att begå folkmord, utan även att aktivt verka för att förhindra det. Detta omfattar förbud mot stöd, legitimering eller tyst acceptans av åtgärder som sannolikt utgör brott mot konventionen.

\medskip

Sammanfattningsvis är följande rättsprinciper centrala för den fortsatta bedömningen:

\begin{itemize}
  \item att folkrättens våldsförbud är absolut och endast medger snävt definierade undantag,
  \item att självförsvar enligt artikel 51 FN-stadgan inte kan åberopas av en ockupationsmakt gentemot civilbefolkningen i det ockuperade territoriet,
  \item att en ockupationsmakt har ett förstärkt ansvar att skydda civilbefolkningen enligt Genèvekonventionen,
  \item att Sverige är folkrättsligt förpliktat att vidta förebyggande åtgärder mot folkmord, även vid brott begångna av annan stat,
  \item att underlåtenhet att agera, under vissa förutsättningar, kan grunda internationellt ansvar.
\end{itemize}

Den fortsatta analysen kommer att visa att Sveriges regering inte har iakttagit dessa rättsprinciper.
 % A. Rättsramen
% - Hierarki mellan folkrätt/svensk rätt
% - Status folkmordskonventionen i svensk rätt

% filnamn: regeringens_officiella_positioner.tex
% SYFTE: Visa exakt vad som sagts – skapa koppling till brottsrekvisit.
% Fungerar som "bevisupptagning" i ett åtal: citat, datum, kontext.



\subsection{Regeringens officiella positioner: innehåll och rättslig betydelse}

Den 27 maj 2025 offentliggjorde Utrikesdepartementet ett uttalande med anledning av att Israels ambassadör i Stockholm kallats upp:

\begin{quote}
\textit{”Uppkallandet gjordes för att upprepa och inskärpa regeringens krav på Israels regering att omedelbart säkerställa säkert och obehindrat humanitärt tillträde till Gaza. [...] Det sätt som kriget nu bedrivs på är oacceptabelt. [...] Terroristorganisationen Hamas ansvar för den aktuella situationen väger tungt.”}
\end{quote}

Utrikesdepartementet anger även:
\begin{quote}
\textit{”Israel har rätt att försvara sig. Den rätten måste utövas i enlighet med folkrätten.”}\\
(Källa: \url{https://www.regeringen.se/pressmeddelanden/2025/05/uttalande-fran-utrikesdepartementet-med-anledning-av-uppkallande-av-israels-ambassador/})
\end{quote}

Detta följer på en debattartikel den 23 maj 2025 undertecknad av fyra statsråd:

\begin{quote}
\textit{”Att blockera mat och annat bistånd till civila är oförsvarligt. [...] Samtidigt har Israel rätt att försvara sig – men den rätten måste utövas i enlighet med folkrätten.”}\\
(Källa: \url{https://www.regeringen.se/artiklar/2025/05/debattartikel-regeringen-okar-pressen-mot-israel/})
\end{quote}

\medskip

Detta språkbruk är rättsligt problematiskt på flera punkter.

För det första används begreppet \textit{självförsvar} i strid med folkrättens tillämplighet. Israel är en ockupationsmakt, inte en stat i självförsvar. FN-stadgans artikel 51 får inte åberopas mot en civilbefolkning som står under ens kontroll, vilket fastslogs i ICJ:s rådgivande yttrande om \textit{muren i Palestina} (2004). Regeringens formulering är därmed inte neutral – den innebär rättslig desinformation och legitimerar folkrättsbrott.

\medskip

För det andra: språket åtföljs inte av några faktiska rättsliga eller politiska åtgärder. Det saknas:

\begin{itemize}
  \item sanktioner eller restriktioner i handeln,
  \item avbrutna diplomatiska förbindelser,
  \item deltagande i rättsliga processer, t.ex. Sydafrikas ICJ-talande mot Israel.
\end{itemize}

Utrikesdepartementets språkbruk stannar vid en retorisk apparat – en markering utan påföljd. Detta är i strid med den \textit{positiva handlingsskyldighet} som följer av artikel I i folkmordskonventionen, vilken ålägger konventionsstater att förhindra brottet i den mån det står i deras makt.

\medskip

Enligt ICJ:s dom \textit{Bosnia v. Serbia} (2007) är det otillräckligt att ”fördöma” eller ”vädja”. Staten måste agera genom att:

\begin{itemize}
    \item frysa relationer,
    \item isolera den misstänkta gärningsmannen,
    \item delta i rättsliga åtgärder.
\end{itemize}

Sveriges regering har inte uppfyllt någon av dessa skyldigheter.

\medskip

\textbf{Regeringens kännedom om Gazas folkrättsliga status}

Det är vidare ostridigt att Gazas de facto-regering – Hamas – sedan 2006 i handling och från 2017 även i skrift, 
uttryckligen godtagit en tvåstatslösning inom 1967 års gränser. Policydokumentet från 2017, samt tidigare initiativ 
såsom \textit{Prisoners’ Document} (2006), \textit{Mecka-avtalet} (2007) och \textit{Försoningsöverenskommelsen} 2020, 
utgör dokumenterade steg mot en politisk lösning inom ramen för folkrätten.\footnote{Se t.ex. \url{https://en.wikipedia.org/wiki/Hamas_Covenant\#2017_document} och \url{https://en.wikipedia.org/wiki/2020_Palestinian_reconciliation_agreement}}

Detta är välkänt inom internationell diplomati och forskning. Professor Neve Gordon och professor Menachem Klein 
har i flera publikationer visat att denna linje var genuin och syftade till att uppnå internationell legitimitet. 
Enligt Klein underminerades dessa initiativ systematiskt av Israel för att bibehålla narrativet om en frånvarande fredspartner.

Det är ett känt faktum att Hamas accepterar FN:s säkerhetsrådsresolution 242, som anger 1967 års gränser, medan Israel inte gör det.

Mot denna bakgrund saknar regeringens retorik om Israels självförsvar förankring i faktisk eller rättslig verklighet. 
Det utgör i stället en politisk konstruktion med allvarliga rättsliga konsekvenser.

Professor Menachem Klein, en av Israels ledande experter på konflikten och tidigare rådgivare i förhandlingar med PLO, 
konstaterar att Hamas i Fatah-Hamas-överenskommelsen från 2021 uttryckligen förband sig till internationell rätt, 
erkände PLO:s auktoritet, accepterade en stat inom 1967 års gränser med Östra Jerusalem som huvudstad, 
och åtog sig att föra en fredlig kamp.%
\footnote{Klein, M. (2023). \textit{Israeli arrogance thwarted a Palestinian political path}. +972 Magazine. \url{https://www.972mag.com/hamas-fatah-elections-israel-arrogance/}} 
Hamas avstod dessutom från att ställa upp med en egen presidentkandidat, i syfte att bana väg för ett demokratiskt mandat 
till en gemensam palestinsk ledning.

Detta är inte en obekräftad tolkning utan en dokumenterad och känd del av den diplomatiska processen – erkänd av 
bland andra EU och USA under Bidenadministrationen, som mottog överenskommelsen i syfte att stödja val.

Att i detta sammanhang fortsätta beskriva Israel som en part som agerar i självförsvar mot Gaza är en förvanskning 
av den faktiska rättspositionen. Hamas har deklarerat vilja att följa internationell rätt. Israel däremot, som ockupationsmakt, 
har avvisat både val, FN-resolutioner och fredsinitiativ – och brutit mot Genèvekonventionerna. 
Regeringens påstående saknar därför såväl faktisk som folkrättslig grund, och innebär i praktiken ett rättsligt vilseledande 
från en konventionsstat, med följd att ansvar enligt BrB 23 kap. eller artikel III i folkmordskonventionen 
kan aktualiseras.

\medskip

\textbf{Språklig tyngdpunktsförskjutning: från legalitet till proportionalitet}

Enligt internationell rätt har Israel inte rätt till självförsvar gentemot den ockuperade palestinska befolkningen. 
Detta följer av Internationella domstolens (ICJ) rådgivande yttrande från 2004, där murens konstruktion på palestinskt område 
förklarades strida mot internationell rätt, samt av ICJ:s rådgivande yttrande den 17 juli 2024, där hela den israeliska 
ockupationen av de palestinska territorierna – med hänvisning till FN:s säkerhetsrådsresolution 242 – bedömdes vara olaglig. 

Trots detta fortsätter den svenska regeringen att upprepa formuleringen att ”Israel har rätt att försvara sig”, 
utan att samtidigt erkänna att Israel är en ockupationsmakt vars rättigheter är strikt begränsade enligt folkrätten. 
Ett sådant språkbruk förskjuter tyngdpunkten i den rättsliga diskussionen: 
från att gälla om våld överhuvudtaget är tillåtet – till att handla om huruvida våldet är ”proportionerligt”.

Med andra ord: där noll skott vore rättsligt tillåtet, flyttas samtalet till huruvida tusen eller tiotusen skott är rimligt. 
Regeringens retorik bidrar därmed till en allvarlig normförskjutning med potentiellt rättsligt ansvar för medverkan 
till folkrättsbrott. 

Att i detta sammanhang ändå åberopa självförsvarsrätten utgör inte endast en felaktig rättstillämpning – det är ett 
normativt stöd till folkrättsbrott. 
När detta sker i officiella uttalanden från en konventionsstat, aktualiseras frågan om \textit{konkludent handlande} 
enligt folkmordskonventionen och svensk straffrätt.


\textbf{Ytterligare brott: underlåtenhet att avslöja pågående folkrättsbrott}

Med denna retoriska förskjutning som fond blir regeringens underlåtenhet att vidta konkreta åtgärder särskilt allvarlig. 
Svenska strafflagens 23 kap. 6 § BrB föreskriver ansvar för den som underlåter att i tid avslöja ett förestående eller pågående brott:

\lagrum{23 kap. 6 § 1 st. BrB\quad Den som underlåter att i tid anmäla eller annars avslöja ett förestående eller pågående brott ska, i de fall det är särskilt föreskrivet, dömas för underlåtenhet att avslöja brottet enligt vad som är föreskrivet för den som medverkat endast i mindre mån.}

Med kännedom om det pågående mönstret av brottslighet i Gaza – dokumenterat av FN-organ, ICC, ICJ och civilsamhällesaktörer – 
föreligger en rättslig skyldighet att agera. Uteblivna sanktioner, rättsliga åtgärder eller andra former av press utgör inte 
bara en underlåtenhet att agera – utan, i detta sammanhang, en underlåtenhet att avslöja brott.

När detta dessutom åtföljs av ett språkbruk som döljer den verkliga rättsliga kontexten – och felaktigt hänvisar 
till en icke tillämplig rätt till självförsvar – förstärks den rättsliga betydelsen. 
Det utgör en passiv medverkan, eller i vissa tolkningar, en \textit{stämplingsliknande handling} i folkrättslig mening.

\medskip
\textbf{Stämpling till folkrättsbrott genom vilseledande legitimering}

När den svenska regeringen i officiella uttalanden påstår att ”Israel har rätt att försvara sig” – trots att detta 
enligt folkrätten inte gäller en ockuperande makt – utgör det inte enbart en felaktig analys. 
Det är ett vilseledande rättsligt påstående med potentiellt brottsbefrämjande verkan.

I svensk rätt är stämpling ett förberedelsebrott där gärningsmannen uppmuntrar eller förstärker någon annans brottsavsikt. 
Analogt gäller här: genom att påstå att ett olagligt agerande är rättfärdigat enligt internationell rätt, 
ger Sverige ett rättsligt stöd till fortsatt våld, vilket i sig kan bidra till brottets fortsättning.

Jämför: den som, felaktigt och med auktoritet, säger till någon att denne ”har rätt att skjuta inkräktare” trots 
att det enligt gällande rätt inte föreligger någon sådan rätt – begår en form av stämpling, 
om detta får till följd att brott begås.

Internationella domstolen (ICJ) har i två rådgivande yttranden – 2004 och 2024 – fastställt att Israel är ockupationsmakt 
och att blockaden av Gaza utgör en folkrättsstridig kollektiv bestraffning. I detta rättsläge är 
självförsvarsrätten enligt FN-stadgans artikel 51 inte tillämplig gentemot Gaza. 

Detta gäller särskilt eftersom Gazas lagligt valda de facto-regering uttryckligen erkänt internationell rätt, inklusive 
Säkerhetsrådets resolutioner såsom 242, medan Israel konsekvent har avvisat både FN-beslut och folkrättens begränsningar.

Att i detta sammanhang ändå tillerkänna Israel en rätt till självförsvar är inte endast ett rättsligt felsteg, 
utan en aktiv vilseledning med potentiellt brottsbefrämjande effekt. Det förskjuter diskursens tyngdpunkt 
från frågan om legalitet till en diskussion om proportionalitet, och därigenom normaliseras ett våldsutövande 
som enligt folkrätten aldrig borde ha påbörjats.

När detta sker i ett läge där:\\
- det finns överväldigande dokumentation om pågående folkrättsbrott, och\\
- gärningsmannen (Israel) står i fortsatt militärt interagerande med civilbefolkningen,\\

...kan regeringens uttalande inte enbart förstås som passivitet. 
Det är en aktiv handling med potentiellt uppviglande karaktär – det närmar sig *stämpling* till folkrättsbrott.

\textbf{Stämpling till folkrättsbrott genom vilseledande demonisering}

Regeringen skriver: \enquote{Mer än ett år efter Hamas fruktansvärda terroristattacker den 7 oktober 2023}.\footnote{\url{https://www.regeringen.se/regeringens-politik/med-anledning-av-situationen-israelpalestina/vad-regeringen-gor-med-anledning-av-kriget-mellan-israel-och-hamas/}}

Hur har regeringen fastställt att det rör sig om terroristattacker? 
Gazas folkvalda regering har konsekvent förnekat att några sådana handlingar har begåtts. 
Samtidigt accepterar regeringen utan prövning Israels påståenden om att läkare, sjukvårdsarbetare,
ambulansförare och journalister varit ”terrorister” – även i de fall då dessa avrättats med bakbundna händer 
eller när ambulanser förstört och sjukhus bombats. 
Israels försvar, att sådana dödsfall varit \enquote{operationella misstag}, godtas. 
Varför godtas då inte även Gazas regerings nekande? Enligt samma måttstock borde även deras utsagor ges förtroende.

Regeringen förhåller sig dessutom tyst till de uppgifter som publicerats i israelisk press, 
enligt vilka IDF bekräftat att ett s.k. \enquote{mass Hannibal event} genomfördes den 7 oktober, 
då israelisk militär öppnade eld mot civila israeler i syfte att förhindra att dessa togs som 
gisslan av Hamas.\footnote{\url{https://electronicintifada.net/blogs/asa-winstanley/we-blew-israeli-houses-7-october-says-israeli-colonel}} Dessa uppgifter behandlas mer ingående längre fram i dokumentet.

Hamas har själv uppgett att den 7 oktober utgjorde en legitim militär operation riktad mot israeliska militära mål, 
och detta har också bekräftats av internationellt respekterade militära bedömare såsom 
Scott Ritter.\footnote{\url{https://scottritter.substack.com/p/the-october-7-hamas-assault-on-israel}} 

Ett av Hamas deklarerade mål var att befria tusentals palestinier som hålls i israeliskt förvar utan rättegång, 
i strid med folkrätten, genom så kallad ”administrativ internering”.\footnote{\url{https://www.btselem.org/administrative_detention/statistics}}

Genom att kategoriskt och utan åtskillnad beteckna alla väpnade palestinska aktörer som ”terrorister” 
bidrar regeringen till en normativ förskjutning som suddar ut folkrättens tydliga skillnad mellan legitim väpnad kamp och folkrättsbrott. 
Hamas, som utgör Gazas folkvalda regering och säger sig erkänna folkrätten, 
har vid flera tillfällen arresterat medlemmar av andra väpnade grupper – såsom Islamiska Jihad 
och salafistiska fraktioner – för oauktoriserade raketattacker 
mot Israel.\footnote{Se t.ex. Haaretz (2015-05-27): \textit{Hamas Arrests Islamic Jihad Activists for Rocket Fire}.}

Trots detta buntar regeringen samman Hamas med de grupper Hamas själv betraktar som kriminella – inklusive en ISIS-associerad 
knarksmugglande klan i södra Gaza, som Israels premiärminister Netanyahu 
bekräftat att Israel beväpnar för att bekämpa Hamas.\footnote{Se \textit{The Grayzone} (2025-06-05): \textit{Israel arming ‘ISIS-affiliated’ gang in southern Gaza}, \url{https://thegrayzone.com/2025/06/05/israel-arming-isis-gang-gaza/}.} Var är regeringens fördömande av att Israel samarbetar med sådana aktörer?

Att systematiskt underlåta att erkänna den mångfald av väpnade aktörer i Gaza – samt den 
interna repression Hamas riktar mot gangsters, extremistiska eller odisciplinerade grupper – leder till en orättvis 
kollektiv demonisering av hela det palestinska motståndet. 

Regeringens retorik beskriver alla väpnade palestinska grupper som ”terrorister”, oavsett om de är folkvalda eller 
agerar inom folkrättens ramar. Samtidigt har Hamas – Gazas folkvalda regering – konsekvent fördömts, medan motsvarande israeliska policy inte ens nämns av regeringen.
Denna förvrängning strider mot neutr alitetsprincipen i folkrätten, vilken kräver att neutrala stater förhåller sig 
objektiva och faktabaserade i konfliktsituationer.

Denna form av sammangående demonisering, utan grund i rättslig bedömning och utan kravet att erkänna fler 
aktörer eller en mångfasetterad verklighet, representerar ett brott mot Genèvekonventionernas förbud mot kollektiv 
bestraffning. Det aktualiserar även principerna i Nürnbergstadgan, särskilt principerna I och VI, där indirekt stöd 
genom propaganda eller legitimering bör, mot bakgrund av Nürnbergprinciperna och Genèvekonventionerna, 
betraktas som en form av medverkan enligt internationell straffrätt.

På samma sätt beskriver regeringen Hezbollah som en ”terroristorganisation”, trots att Hezbollah utgör en omfattande 
politisk och social rörelse med djupt folkligt stöd i Libanon, och ingår i landets regeringskoalition. 
Under flera perioder har rörelsen haft parlamentarisk majoritet.

Att enskilda medlemmar inom en folkförankrad befrielserörelse begår handlingar som kan klassificeras som 
terrorbrott innebär inte att hela organisationen därmed juridiskt definieras som en terroristisk aktör. 
Om regeringen drar sådana generaliserande slutsatser utan nyansering, aktualiseras frågan om kollektiv skuldbeläggning – något 
som enligt folkrätten är förbjudet. Det skulle i förlängningen innebära att hela etniska eller nationella grupper 
riskerar att demoniseras, något som står i direkt strid med principen om individuellt ansvar i internationell rätt.

Sådana handlingar faller inom ramen för medverkan till aggressionsbrott och brott mot mänskligheten, enligt princip VI (c) i Nürnbergstadgan, 
vilket också omfattar propaganda och offentlig legitimering av handlingar som strider mot folkrätten.

Genom att använda statliga uttryck som vilseleder, rättfärdigar eller anonymiserar brottsliga handlingar, bidrar regeringen till en 
rättslig stämpel – en funktionell form av stämpling – snarare än en oskyldig politisk ton. 
Det kan karakteriseras som en form av stämpling till folkrättsbrott, eftersom det ges politisk legitimitet 
till en ockupationsmäktig makt som bryter mot grundläggande folkrättsliga normer.
Mot denna bakgrund framstår regeringens kategoriska språkbruk som en form av stämpling till 
folkrättsbrott – inte genom direkt deltagande i våldshandlingar, utan genom att ge politisk och retorisk 
täckning för handlingar som i internationell rätt saknar legitimitet.\footnote{Princip VI(c) i Nürnbergstadgan omfattar även medverkan i brott mot mänskligheten, inklusive att genom offentlig kommunikation uppmana till eller rättfärdiga sådana handlingar. Se även: \textit{Charter of the International Military Tribunal – Annex to the Agreement for the prosecution and punishment of the major war criminals of the European Axis} (1945), artikel 6(c).}





\textbf{Folkrättslig passivitet som medverkan}

Enligt artikel I i folkmordskonventionen är varje stat förpliktad att inte bara avstå från att begå folkmord, utan även att aktivt förhindra sådana brott inom ramen för sin möjlighet. Detta bekräftades i ICJ:s dom \textit{Bosnia v. Serbia} (2007), där domstolen slog fast att stater har en faktisk och preventiv handlingsskyldighet.

Regeringens underlåtenhet att:

\begin{itemize}
    \item frysa bilaterala relationer,
    \item införa sanktioner eller vapenembargon,
    \item ansluta sig till internationella rättsprocesser,
\end{itemize}

...innebär ett kontraktsbrott mot Sveriges traktatförpliktelser enligt folkmordskonventionen och FN-stadgan. Det kan också utgöra medverkan till fortsatt folkrättsbrott, i den mån passiviteten sker med kännedom om risk och utan proportionalt motverkande åtgärder.

\medskip

\textbf{Slutsats:} Regeringens uttalanden den 23 och 27 maj 2025 utgör inte endast politiska markörer. De är rättsligt relevanta dokument som – mot bakgrund av regeringens faktiska handlingsvägran – får betraktas som konkludent medgivande till ett pågående folkrättsbrott, och i förlängningen en möjlig form av medverkan enligt svensk och internationell rätt.


 % B. Faktabild
% - Kronologisk dokumentation av uttalanden
% - Diplomatisk korrespondens

%filnamn: kondenserad_sammanfattning_officiella_positioner.tex
%SYFTE: Att kompakt sammanfatta allt i regeringens_officiella_positioner.tex för maximal överblick

%%Sammanfattning Sammanfattning av regeringens uttalanden såsom rättsliga handlingar
\subsection{Sammanfattning av regeringens uttalanden såsom rättsliga handlingar}
\label{subsec:regeringens_positioner}

\subsubsection{Kritiska uttalanden: maj 2025}
\begin{tabular}{p{0.12\textwidth}p{0.85\textwidth}}
\textbf{Datum} & \textbf{Uttalande \& rättslig kontext} \\
\hline
23 maj & \textit{Debattartikel:} "Israel har rätt att försvara sig – men den rätten måste utövas i enlighet med folkrätten" \\
& \footnotesize\textit{Rättslig kontext:} ICJ rådgivande yttrande 2004 (muren) \& 2024 (ockupationen) fastslår att artikel 51 FN-stadga ej tillämplig \\
\hline
27 maj & \textit{UD-uttalande:} "Terroristorganisationen Hamas ansvar för den aktuella situationen väger tungt" \\
& \footnotesize\textit{Rättslig kontext:} Bröt mot neutralitetsprincipen (Hagakonventionen 1907) och principen om individuellt ansvar (Nürnbergartikel 6) \\
\end{tabular}

\subsubsection{Rättslig desinformation om självförsvarsrätten}
\begin{itemize}
\item \textbf{Faktamässig felaktighet}: Israel är ockupationsmakt (ICJ 2004 \& 2024), inte angripen stat
\item \textbf{Rättslig konsekvens}: Skapar normativ förskjutning från \textit{legalitet} till \textit{proportionalitet}
\item \textbf{Jämförelse}: Analogt med att hävda att en fängelsevakt har "rätt att försvara sig" mot interner
\end{itemize}

\subsubsection{Selektiv demonisering som folkrättsbrott}
\textbf{Asymmetrisk tillämpning av 'terrorist'-begreppet:}
\begin{itemize}
\item \textbf{Hamas}: Kategoriskt terroriststämpel trots:
  \begin{itemize}
  \item Folkvald regering (2006)
  \item Accept av FN-resolution 242 (1967-gränser)
  \item Intern repression av opålitatliga grupper
  \end{itemize}
\item \textbf{Israeliska aktörer}: Ingen kritik av:
  \begin{itemize}
  \item IDF:s "Hannibal-direktiv" 7 oktober 2023
  \item Samarbete med kriminella klaner (Grayzone 2025)
  \end{itemize}
\end{itemize}
\lagrumsinline{Genèvekonvention IV, artikel 33\quad Förbud mot kollektiv bestraffning}

\subsubsection{Underlåtenhetsansvar enligt svensk rätt}
\textbf{23 kap. 6 § BrB} aktualiseras genom:
\begin{enumerate}
\item Kännedom om systematiska folkrättsbrott (ICJ, ICC, FN-rapporter)
\item Underlåtenhet att vidta juridiskt mandaterade åtgärder
\item Samtidigt språkbruk som döljer brottens karaktär
\end{enumerate}

\subsubsection{Vilseledande legitimering som stämpling}
\begin{itemize}
\item \textbf{Rättslig grund}: Nürnbergprincip VI(c) om medverkan till brott mot mänskligheten
\item \textbf{Sveriges agerande}: Felaktig juridisk karaktärisering med brottsbefrämjande effekt
\item \textbf{Analog}: Att felaktigt hävda att någon har rätt att bruka våld
\end{itemize}

\subsubsection{Koppling till folkmordskonventionen}

\begin{tabular}{p{0.25\textwidth}p{0.7\textwidth}}
\textbf{Handling som krävts} & \textbf{Sveriges underlåtenhet} \\
\hline
Stöd till ICJ-processen & Ej anslutit sig till Sydafrikas talan \\
\hline
Vapenembargo & Fortsatt export av militärteknologi \\
\hline
Diplomatisk isolering & Besökt Israel av UD-tjänstemän \\
\end{tabular}



Enligt ICJ i \textit{Bosnia v. Serbia} (2007, §432) kan medverkan aktualiseras genom:
\begin{itemize}
\item Underlåtenhet att stoppa brottet när möjlighet finns
\item Bidragande handlingar som underlättar brottsligheten
\item Skapande av normativa ramar som legitimerar brott
\end{itemize}
Sveriges agerande uppfyller alla tre kriterier.



\subsubsection{Sammanfattande rättslig bedömning}
Regeringens uttalanden uppfyller rekvisiten för:
\begin{itemize}
\item \textbf{Medhjälp} enligt ILC:s Draft Articles on State Responsibility (art. 16)
\item \textbf{Underlåtenhet att avslöja brott} (BrB 23:6)
\item \textbf{Indirekt stämpling} till folkrättsbrott (Nürnbergartikel 6)
\end{itemize}
%SYFTE: Att kompakt sammanfatta allt i regeringens_officiella_positioner.tex för maximal överblick

%filnamn: konkludent_medverkan.tex

% SYFTE: Sätt regeringens språk i juridisk kontext.
% Visa att språket inte är oskyldigt – det utgör handling i rättslig mening.

\subsection{Konkludent beteende i politiskt hänseende}

Konkludent handlande är en civilrättslig princip som innebär att bindande rättsverkningar kan uppstå även i avsaknad av uttryckligt samtycke, om parternas agerande objektivt sett ger intryck av att ett avtal föreligger. I svensk avtalsrätt – liksom inom den kontinentala civilrättsliga traditionen – kan avtal således uppstå genom faktisk handling, så länge detta handlande motsvarar en tyst accept.

I ett politiskt sammanhang motsvaras detta av principen \textit{substance over form}: det är inte vad som sägs, utan vad som faktiskt görs, som bör ligga till grund för rättslig bedömning.

Ett särskilt talande exempel är Turkiets president Erdoğan. I offentligheten har han gått till hårt angrepp mot Israel – och i juli 2024 till och med hotat med militär intervention, med hänvisning till tidigare turkiska operationer i Karabach och Libyen.\footnote{Reuters/Jerusalem Post, \textit{“Erdogan threatens Israel: 'Like we entered Karabakh and Libya – we will do the same to Israel'”}, 28 juli 2024.} I retoriken framstår han som en ovillkorlig försvarare av det palestinska folket.

Men bakom denna fasad fortgår materiella kontakter. Trots offentliga utspel har Turkiet inte brutit alla relationer med Israel. Tvärtom har handel, logistik och andra tyst legitimerande aktiviteter fortsatt i skuggan av Erdogans bombastiska tal. Detta skapar ett spel för gallerierna, där Erdogan – medvetet eller ej – tillåts fungera som en kontrollerad ventil. Han kritiserar, Israel svarar med fördömanden, och båda parter vinner inrikespolitiska poäng. Resultatet är ett tyst samförstånd där symbolisk konflikt ersätter faktisk konfrontation.

Den svenska regeringen agerar på ett liknande sätt. Den uttrycker ibland "oro" över situationen i Gaza, men fortsätter parallellt att upprätthålla handel, diplomatiska kontakter och säkerhetssamarbete med Israel – även under perioder där landets agerande beskrivs som folkrättsvidrigt av FN:s specialrapportörer. När detta sker utan protester eller motåtgärder, utgör det en form av konkludent medgivande – en tyst accept av den pågående folkrättsbrottsligheten.

Detta får särskild tyngd i ljuset av Regeringsformen:

\lagrum{10 kap. 1 § RF\quad Överenskommelser med andra stater eller med mellanfolkliga organisationer ingås av regeringen. Lag (2010:1408).}

Sådana överenskommelser – formella eller implicita – får inte strida mot Sveriges folkrättsliga förpliktelser. Om regeringens samlade agerande utgör en faktabaserad acceptans av Israels folkrättsstridiga politik, då föreligger enligt internationell rätt ett \textit{konkludent samförstånd} – med rättsverkningar som kan aktivera medansvar enligt bland annat Nürnbergprincip I.

I det följande dokumenteras de handlingar, uttalanden och underlåtelser som sammantaget utgör konkludent medverkan till folkrättsbrott:

\begin{itemize}
    \item Regeringen har vid upprepade tillfällen tillerkänt Israel en \enquote{rätt till självförsvar} gentemot Gaza, trots att Israel enligt ICJ är en ockupationsmakt, och att blockaden av Gaza sedan 2007 är olaglig. Självförsvarsrätten enligt FN-stadgans artikel 51 är inte tillämplig på ockupationer.\footnote{ICJ:s rådgivande yttrande 2004, bekräftat 2024.}
    
    \item Israel vägrar erkänna FN:s säkerhetsrådsresolution 242 (1967), som fastslår att Israel måste dra sig tillbaka från ockuperade områden. Den palestinska sidan, inklusive Hamas och andra partier, har däremot accepterat gränserna från 1967 som grund för en framtida stat.\footnote{\url{https://www.972mag.com/hamas-fatah-elections-israel-arrogance/}}
    
    \item Internationella domstolen (ICJ) har slagit fast att ockupationen är olaglig och karaktäriserat den som ett apartheidsystem – vilket enligt Romstadgan är ett brott mot mänskligheten. Trots detta har Sverige inte infört några sanktionsåtgärder motsvarande dem som riktades mot apartheidregimen i Sydafrika.
    
    \item Israel har fängslat framstående palestinska läkare, inklusive professorer, utan rättssäker process. I april 2024 dog Dr. Adnan al-Bursh, ortoped och professor, efter att ha torterats till döds i israeliskt förvar.\footnote{\url{https://www.middleeasteye.net/news/war-gaza-prominent-palestinian-doctor-tortured-and-killed-israeli-detention}}
    
    \item Regeringen har inte fördömt Israels tillämpning av det så kallade Hannibal-direktivet, vilket enligt rapporter kan ha bidragit till dödandet av hundratals israeliska civila under 7 oktober-attackerna.\footnote{\url{https://www.france24.com/en/live-news/20231215-israel-social-security-data-reveals-true-picture-of-oct-7-deaths}}
    
    \item Under hela 2023 dödades minst 507 palestinier på Västbanken, inklusive minst 81 barn. Av dessa dödades 299 efter den 7 oktober, vilket innebär att över 200 dödades under årets första nio månader – långt innan kriget i Gaza inleddes.\footnote{\url{https://www.amnesty.org/en/latest/news/2024/02/shocking-spike-in-use-of-unlawful-lethal-force-by-israeli-forces-against-palestinians-in-the-occupied-west-bank/}}
    
    \item Regeringen har aktivt underminerat FN-organet UNRWA:s arbete genom att frysa bistånd, trots att inga bevis framkommit som rättfärdigar kollektiva bestraffningar av FN-personal.
    
    \item Israel har systematiskt bombat sjukhus i Gaza utan att Sverige har agerat utöver återhållsamma verbala uttalanden. Detta trots att sjukhus åtnjuter särskilt skydd enligt Genèvekonventionerna.
    
    \item Israeliska styrkor har avrättat ambulanspersonal – bland annat dokumenterat genom video – utan att den svenska regeringen har reagerat proportionerligt. När brottet till sist erkändes av Israel benämndes det som ett \enquote{operationellt misstag}.\footnote{\url{https://www.middleeasteye.net/news/new-video-evidence-disputes-israeli-armys-account-medic-killings}}
    
    \item Öppen och återkommande hets till folkmord i israeliska medier, parlamentet och sociala plattformar har dokumenterats sedan minst 2010. Den svenska regeringen har aldrig offentligt fördömt dessa uttryck – trots att hets till folkmord utgör ett särskilt brott enligt FN:s folkmordskonvention.\footnote{\url{https://www.youtube.com/watch?v=9GbKsAvuBDM}}
\end{itemize}


\subsection{Medverkan genom handling och underlåtenhet}
\label{subsec:konkludent}

\subsubsection{Rättslig grund för medhjälparansvar}
Enligt folkrätten kan staters ansvar aktiveras inte endast genom direkta brott utan även genom medverkan till annan stats brottslighet. I enlighet med ILC:s \textit{Draft Articles on State Responsibility} (2001) artikel 16:

\begin{quote}
\textit{"A State which aids or assists another State in the commission of an internationally wrongful act by the latter is internationally responsible for doing so if:}
\begin{enumerate}
    \item \textit{that State does so with knowledge of the circumstances of the internationally wrongful act; and}
    \item \textit{the act would be internationally wrongful if committed by that State."}
\end{enumerate}
\end{quote}

Denna princip har tillämpats av ICJ i \textit{Bosnien mot Serbien} (2007, §420) och förstärkts i \textit{Belgien mot Senegal} (2012) rörande universell jurisdiktion.

\subsubsection{Konkludent godkännande genom språkbruk}
Regeringens semantiska val konstituerar rättsligt relevanta handlingar:

\begin{itemize}
    \item \textbf{Terminologisk undvikande}: Konsistent vägran att använda termen "folkmord" trots:
    \begin{itemize}
        \item ICJ:s preliminära åtgärdsbeslut 26 jan 2024 (indikation på plausibelt brott)
        \item FN:s specialrapportörs uttalande 26 mars 2024: "reasonable grounds to believe genocide is occurring"
    \end{itemize}
    
    \item \textbf{Legitimeringstekniker}: Användning av fraser som:
    \begin{quote}
        "Israels rätt att försvara sig" (Uttalande UD, 28 okt 2023)\\
        "Komplex situation" (Intervju Tobias Billström, SVT 15 nov 2023)
    \end{quote}
    vilka i kontexten av systematisk förstörelse av civil infrastruktur (enligt FN OCHA-rapport 45/2024) uppfyller rekvisiten för medhjälp genom moraliskt stöd.
\end{itemize}

\subsubsection{Aktiv stödhändelser}
\begin{tabular}{p{0.2\textwidth}p{0.75\textwidth}}
\textbf{Datum} & \textbf{Händelse \& rättslig kvalifikation} \\
\hline
25 okt 2023 & \textbf{Undertecknande av vapenleveransavtal}\footnote{\url{https://www.riksdagen.se/sv/dokument-och-lagar/dokument/skriftlig-fraga/sveriges-avtal-med-israeliska-elbit_hb11396/}} \\
& \footnotesize\textit{Rättskvalifikation}: Ekonomiskt samarbete; underlåtenhet att minska finansiell förmåga hos brottsbenägen stat \\
\hline
15 jan 2024 & \textbf{Underlåtenhet att suspendera vapenexportlicenser} \\
& \footnotesize\textit{Rättskvalifikation}: Ökad vapenexport till brottsbenägen stat\footnote{\url{https://proletaren.se/artikel/vapexporten-till-israel-okar/}} \\
\hline
10 dec 2020 & \textbf{Undertecknande av innovationsavtal med Israel} \\
& \footnotesize\textit{Rättskvalifikation}: Ekonomiskt stöd till brottsbenägna stat enl. EU:s definition av "tredje parts samarbeten" \\
\end{tabular}

\subsubsection{Underlåtelsehandlingar med verkan}
\begin{itemize}
    \item \textbf{Inaktivitetsgradient}: Frånvaron av diplomatiska åtgärder trots eskalering:
    \begin{itemize}
        \item Inga återkallade ambassadörer (jämfört med Sydafrika-fallet)
        \item Ingen tillämpning av Magnitsky-liknande sanktioner
    \end{itemize}
    
    \item \textbf{Procedurell sabotage}: Blockering av EU-försök till gemensam deklaration (enl. läckta dokument från EUMC 12 feb 2024)
\end{itemize}

\subsubsection{Slutlig rättslig bedömning}
Sveriges agerande uppfyller kriterierna för medhjälp enligt artikel 16 genom:
\begin{enumerate}
    \item \textit{Kunskapsrekvisitet}: Överväldigande bevis om brottslighet
    \item \textit{Bidragsrekvisitet}: Faktiskt stöd genom vapen, diplomati och ekonomi
    \item \textit{Gränsöverskridande effekt}: Sverige har jurisdiktion enligt universell princip
\end{enumerate}
 % C. Rättslig kvalifikation
% - Juridisk analys av handlingar/icke-handlingar
% - Jämförelse med medhjälpansvar i brottsbalken

%filnamn: slutpladering_anklagelse.tex
% SYFTE: Sammanfattande bevisning. "Regeringen är skyldig." genom att knyta samman lag och 
%regeringens underlåtenheter och andra felsteg.


%filnamn: slutpladering_anklagelse.tex
% SYFTE: Sammanfattande bevisning. "Regeringen är skyldig." genom att knyta samman lag och 
%regeringens underlåtenheter och andra felsteg.

%filnamn: slutpladering_anklagelse.tex
\subsection{Slutplädering: Regeringens juridiska ansvar är ovedersägligt}


Den samlade bevisningen visar klart och entydigt att Sveriges regering genom handling och underlåtenhet brutit mot bindande folkrättsliga och konstitutionella skyldigheter. Regeringens agerande uppfyller samtliga rekvisit för medverkan till folkrättsbrott enligt etablerad rättspraxis.

\subsection*{Sammanfattning av beviskedjan}
\begin{enumerate}
    \item \textbf{Legitimering av olagligt våld} \\
    Genom konsekvent felaktig tillämpning av artikel 51 FN-stadga har regeringen rättsligt legitimerat Israels agerande som "självförsvar" trots klart fastställd ockupationsstatus (ICJ 2004, 2024).
    
    \item \textbf{Selektiv tillämpning av rättsnormer} \\
    Regeringen har tillämpat "terrorist"-begreppet asymmetriskt genom att:
    \begin{itemize}
        \item Systematiskt demonisera Hamas trots dess folkvalda status och accepterande av FN-resolution 242
        \item Underlåta att granska IDF:s dokumenterade brott (Hannibal-direktivet, samarbete med kriminella klaner)
    \end{itemize}
    
    \item \textbf{Aktiv medverkan genom underlåtenhet} \\
    Tabell \ref{tab:underlatenhet} sammanfattar centrala underlåtenheter som uppfyller rekvisiten för medhjälp enligt ILC:s artikel 16:
\end{enumerate}

\begin{table}[h]
\centering
\caption{Sveriges underlåtenheter och deras rättsliga konsekvenser}
\label{tab:underlatenhet}
\begin{tabular}{p{0.4\textwidth}p{0.55\textwidth}}
\textbf{Underlåten handling} & \textbf{Rättslig kvalifikation} \\ \midrule
Inget stöd till ICJ-processen & Underlåtenhet att förebygga folkmord (art. I Folkmordskonv.) \\
Fortsatt vapenexport & Medverkan till krigsförbrytelser (ILC art. 16) \\
Avsaknad av sanktioner & Brott mot neutralitetsplikt (Haagkonventionen) \\
\end{tabular}
\end{table}

\subsection*{Juridiska följdslut}
Med stöd i dokumenterad bevisning och etablerad rättspraxis konstateras:

\begin{itemize}
    \item Regeringen har \textbf{aktivt medverkat} till folkmord enligt artikel III(e) i folkmordskonventionen genom:
    \begin{itemize}
        \item Diplomatiskt stöd som underlättat fortsatt brottslig verksamhet
        \item Språkbruk som skapat normativa ramar för acceptans av brott
    \end{itemize}
    
    \item Regeringen har \textbf{brutit mot Regeringsformen} (1 kap. 10 §) genom:
    \begin{itemize}
        \item Underlåtenhet att lojalt fullgöra folkrättsliga förpliktelser
        \item Aktivt bidrag till urholkning av internationell rättsordning
    \end{itemize}
    
    \item Regeringen har \textbf{uppfyllt rekvisiten} för underlåtenhet att avslöja brott (BrB 23:6) genom:
    \begin{itemize}
        \item Kännedom om systematiska folkrättsbrott
        \item Avsaknad av tillräckliga åtgärder trots handlingsmöjligheter
    \end{itemize}
\end{itemize}

\subsection*{Krav på rättslig prövning}
På grundval av ovanstående yrkas:

\begin{itemize}
    \item Att \textbf{Konstitutionsutskottet} omedelbart inleder granskning av regeringens agerande ur statsrättsligt perspektiv
    
    \item Att \textbf{Justitieombudsmannen} granskar eventuella tjänstefel inom berörda departement
    
    \item Att \textbf{Utrikesdepartementet} omedelbart:
    \begin{itemize}
        \item Upphör all militär samverkan med Israel
        \item Ansluter sig till ICJ-processen mot Israel
        \item Inför omfattande sanktioner
    \end{itemize}
    
    \item Att \textbf{Högsta domstolen} prövar frågan om regeringens folkrättsliga ansvar enligt 13 kap. 3 § Regeringsformen
\end{itemize}

\textit{Denna slutplädering baseras på den samlade bevisningen i detta dokument och kräver omedelbar rättslig åtgärd.} % D. Slutsats
% - Slutplädering med stöd av etablerade fakta & analys




% SYFTE: Sammanfattande bevisning. "Regeringen är skyldig." genom att knyta samman lag och 
%regeringens brist på ageranden, deras faktiska ageranden och deras uttalanden.

%\include{sections/anklagelse_section/underminering_av_rattsorgan}
% SYFTE: Visa att regeringen aktivt försvårar rättskipning.
% Ex: uttalanden om ICC, vägran att agera på FN-beslut, skydd av förövare.

%\include{sections/anklagelse_section/brott_genom_bistandspolitik}
% SYFTE: Beskriv konkreta handlingar som utgör medverkan.
% Omdirigering av stöd, tystnad om skjutningar, legitimering av militär kontroll.

%\include{sections/anklagelse_section/vapenavtal_som_medverkan}
% SYFTE: Slutlig spik i kistan – regeringen signerade samarbete mitt i folkmord.
% Argumentera att detta är inte bara likgiltighet utan positiv handling.



\begin{comment}
\section{Folkrättens bindande ramverk}

% Sammanfatta relevanta rättskällor: FN-stadgan, Folkmordskonventionen, Genèvekonventionerna, sedvanerätt.
% Förklara varför folkrätten inte kan kringgås av politiska hänsyn.
% Avsluta med Sveriges konstitutionella skyldigheter enligt Regeringsformen 10:1 och 13:3.


Regeringen skriver i sin kommuniké från den 27 maj 2025:\footnote{\url{https://www.government.se/statements/2025/05/statement-from-the-ministry-for-foreign-affairs-on-summoning-of-israeli-ambassador/}}

\begin{quote}
\textit{“When the Ambassador was summoned, it was stressed that Israel has the right to defend itself. However, that right must be exercised in accordance with international law.”}
\end{quote}

Vid en första anblick framstår detta som ett balanserat uttalande. Men den retoriska konstruktionen döljer en djupare förskjutning. Vad som presenteras som ett villkorat erkännande av folkrätten, fungerar i själva verket som en rättslig täckmantel för folkrättsbrott.

FN-stadgan artikel 51\footnote{\url{https://www.un.org/en/about-us/un-charter/full-text}} ger endast rätt till självförsvar vid ett väpnat angrepp – riktat mot en suverän medlemsstat. Rätten gäller alltså inte en ockupationsmakt som utövar kontroll över ett territorium och dess civilbefolkning.

\lagrum{Artikel 51, FN-stadgan\quad Nothing in the present Charter shall impair the inherent right of individual or collective self-defence if an armed attack occurs against a Member of the United Nations...}

Att hävda att Israel har rätt att försvara sig mot Gaza är därför inte enbart en politisk förenkling – det är en rättslig förfalskning av folkrättens grundprinciper. Det ger det strukturella våldet en rättslig fernissa, trots att folkrätten bygger på ansvar – inte på retoriska undantag.

Internationella domstolen (ICJ) fastslog den 19 juli 2024 att Israels ockupation av Gaza, Västbanken och östra Jerusalem är olaglig, och att den utgör apartheid.\footnote{\url{https://mondoweiss.net/2024/07/in-a-historic-ruling-icj-declares-israeli-occupation-unlawful-calls-for-settlements-to-be-evacuated-and-for-palestinian-reparations/}} Detta är inte en tolkning. Det är ett rättsligt bindande konstaterande.

Trots detta fortsätter den svenska regeringen att referera till Israels \enquote{rätt till självförsvar} som om Gaza vore en angripande stat, inte ett ockuperat territorium. Uttalandet neutraliserar den folkrättsliga kontexten och bekräftar därigenom en rättsvidrig världsbild.

Därmed gör sig regeringen skyldig till mer än en vilseledande formulering. Den medverkar – om än indirekt – till att undergräva den internationella rättsordningen. Och detta sker inte av juridisk okunskap, utan som del av en förutsägbar, politiskt motiverad ordkonstruktion:

\begin{quote}
Ett sätt att säga något – men samtidigt inte mena det.  
Detta är inte ansvarsutkrävande. Det är ett medvetet undandragande av ansvar, maskerat som neutral diplomati.
\end{quote}

\medskip

\textbf{Frågan är därför inte om Israel har rätt till självförsvar – utan var den rätten får utövas.}

För att bedöma giltigheten i regeringens påstående krävs en tydlig förståelse av folkrättens territoriella begränsning: självförsvar enligt FN-stadgan artikel 51 får endast utövas inom en stats eget internationellt erkända territorium. Ockupationsmaktens rättigheter är däremot reglerade av helt andra folkrättsliga instrument.


\section{Sveriges folkrättsliga ansvar}
% Redogör för Sveriges positiva skyldigheter enligt internationell rätt.
% Bevisa att tystnad, fortsatt vapenhandel eller selektiv diplomati utgör medansvar.
% Redogör för tidigare exempel där Sverige agerat kraftfullt mot folkrättsbrott (t.ex. Ryssland, Myanmar, Iran).

\subsection*{Israel saknar rätt till försvar utanför israelisk mark}
\addcontentsline{toc}{subsection}{Israel saknar rätt till försvar utanför israelisk mark}

Oavsett legaliteten i själva ockupationen kan en ockupationsmakt aldrig åberopa självförsvar mot den befolkning som står under dess kontroll. Enligt folkrätten får en ockupationsmakt endast vidta åtgärder som är strikt nödvändiga för att upprätthålla allmän ordning och säkerhet, i enlighet med artikel 43 i Haagreglementet:\footnote{\url{https://ihl-databases.icrc.org/en/ihl-treaties/hague-regulations-1899/article-43}}

\lagrum{Article 43\quad The authority of the legitimate power having in fact passed into the hands of the occupant, the latter shall take all the measures in his power to restore, and ensure, as far as possible, public order and safety, while respecting, unless absolutely prevented, the laws in force in the country.}

Samtidigt är ockupationsmakten skyldig att skydda civilbefolkningen från våld och hot enligt artikel 27 i den fjärde Genèvekonventionen:\footnote{\url{https://ihl-databases.icrc.org/en/ihl-treaties/gciv-1949/article-27}}

\lagrum{Article 27\quad Protected persons are entitled, in all circumstances, to respect for their persons, their honour, their family rights, their religious convictions and practices, and their manners and customs. They shall at all times be humanely treated, and shall be protected especially against all acts of violence or threats thereof.}

Därtill får ockupationsmakten inte ersätta det lokala rättssystemet, utan endast temporärt administrera det för upprätthållande av civil ordning, enligt artikel 64:\footnote{\url{https://ihl-databases.icrc.org/en/ihl-treaties/gciv-1949/article-64}}

\lagrum{Article 64\quad The penal laws of the occupied territory shall remain in force, with the exception that they may be repealed or suspended by the Occupying Power in cases where they constitute a threat to its security or an obstacle to the application of the present Convention...}

Att hävda rätt till militärt självförsvar inom ett territorium där man själv är ockupant innebär att man juridiskt förväxlar våld med skydd, och därigenom ger repressionen ett falskt moraliskt anspråk. Det är att kriminalisera motstånd – och samtidigt rättfärdiga fortsatt förtryck.

Självförsvar enligt FN-stadgans artikel 51 är territoriellt begränsat till internationellt erkänd, egen mark:

\lagrum{Article 51\quad Nothing in the present Charter shall impair the inherent right of individual or collective self-defence if an armed attack occurs against a Member of the United Nations...}

Detta innebär att självförsvar inte kan åberopas av en stat som befinner sig utanför sitt eget erkända territorium – särskilt inte mot den civilbefolkning som staten i egenskap av ockupationsmakt har skyldighet att skydda.

Detta är också exakt den folkrättsliga position Sverige intar gentemot Ryssland och som man framhåller som självklart i officiella sammanhang: Ryssland kan inte hävda självförsvar på ukrainsk mark, eftersom Ukraina inte är ryskt territorium.  

Att däremot medge Israel rätt till självförsvar på ockuperad palestinsk mark är ett flagrant avsteg från denna princip. Det är en inkonsekvent rättstillämpning – där likabehandlingsprincipen offras till förmån för politisk opportunism.


Att därtill tillfoga formuleringar som \enquote{men det måste ske i enlighet med internationell rätt} är utan betydelse. Ty det är själva användningen av begreppet \textit{självförsvar} i detta sammanhang som utgör ett folkrättsbrott.

När en regering offentligt erkänner rätten till självförsvar för en stat som agerar utanför sitt eget territorium – i strid med folkrätten – så förskjuts hela den rättsliga tyngdpunkten. Diskussionen handlar då inte längre om \textit{huruvida} våldet är tillåtet, utan \textit{hur mycket} våld som kan anses proportionerligt.

Detta medför två rättsliga och moraliska implikationer:

\begin{enumerate}
  \item \textbf{Konkludent handlande i folkrättslig mening.}  
  Genom att använda termen \enquote{självförsvar} utan förbehåll, agerar regeringen i strid med den folkrättsliga huvudprincipen att självförsvar enligt FN-stadgan endast får utövas på egen, internationellt erkänd mark.

  Det utgör ett tyst – och därmed konkludent – godkännande av ett pågående folkrättsbrott, såsom olaglig ockupation eller övervåld, eftersom formuleringen ger dessa handlingar en legal fernissa.

 \item \textbf{Medverkan eller stämpling i straffrättslig mening.}  
Om en regering genom officiella uttalanden uppmuntrar, normaliserar eller skyddar ett folkrättsbrott, kan detta – i ett internationellt rättsligt sammanhang – jämställas med det som i svensk straffrätt motsvarar stämpling till brott.

\lagrum{23 kap. 2 § 2 st BrB\quad  I de fall det särskilt anges döms för stämpling till brott. Med stämpling förstås att någon i samråd med någon annan beslutar gärningen eller att någon söker anstifta någon annan eller åtar eller erbjuder sig att utföra den.}

Ett exempel: Om Sveriges statsminister skulle säga \enquote{Ryssland har rätt att försvara sig på ukrainsk mark, men det måste ske enligt internationell rätt}, så har man redan legitimerat det första skottet. Diskussionen handlar därefter inte om \textit{att} Ryssland får använda våld – utan \textit{hur mycket} våld som är acceptabelt. På så sätt förskjuts skuldbedömningen från själva olagligheten till våldets omfattning, vilket innebär att man rättsligt och moraliskt gör sig medskyldig till brottet.
\end{enumerate}

\subsubsection*{Sammanfattning}
\noindent En ockupationsmakt har ingen rätt till självförsvar mot den befolkning som står under dess kontroll. Att erkänna sådan rätt är att förskjuta rättens tyngdpunkt från olagligt våld till "godtagbar nivå av våld", och därmed aktivt legitimera förtryck. 

Regeringens uttalanden står därmed i direkt motsättning till både Sveriges officiella folkrättsliga linje i andra konflikter och till grundläggande principer i internationell humanitär rätt.



\subsection*{Apartheid som brott mot mänskligheten enligt internationell rätt}
\addcontentsline{toc}{subsection}{Apartheid som brott mot mänskligheten enligt internationell rätt}

Trots att Internationella domstolen (ICJ) i sitt rådgivande yttrande från juli 2024 slår fast att Israels styre över de ockuperade palestinska områdena utgör apartheid i folkrättslig mening, har den svenska regeringen ännu inte uttalat sig om detta. Det är anmärkningsvärt.

Apartheid är inte en politisk etikett utan ett juridiskt begrepp med entydig definition i internationell rätt. Det klassas som ett brott mot mänskligheten och utgör därmed en av de allvarligaste kränkningarna av den folkrättsliga ordningen.

Hade motsvarande dom fastställt att någon annan stat – exempelvis Ryssland, Iran eller Syrien – tillämpade en regim av apartheid, hade detta sannolikt omedelbart föranlett politiska åtgärder, sanktioner och fördömanden. När det gäller Israel råder däremot en påfallande tystnad.

Detta kan jämföras med hur Sverige – med rätta – fördömde apartheidsystemet i Sydafrika under 1980-talet. Det är därför särskilt bekymmersamt att samma politiska krafter som då motsatte sig sanktioner mot Pretoria i dag återfinns i en regering som tiger om Tel Aviv.

Att upprätthålla en apartheidregim är inte bara en politisk fråga – det är ett folkrättsligt definierat brott mot mänskligheten. Detta har fastslagits i flera tvingande rättskällor, däribland:

\utlandslagrum{Romstadgan för Internationella brottmålsdomstolen (ICC), artikel 7(1)(j):}{https://www.icc-cpi.int/sites/default/files/RS-Eng.pdf}{, s. 10.}
\textit{For the purpose of this Statute, “crime against humanity” means any of the following
acts when committed as part of a widespread or systematic attack directed against any
civilian population, with knowledge of the attack:  [...] (j) The crime of apartheid;}

\vspace{4mm}

\utlandslagrum{Romstadgan, artikel 7(2)(h):}{https://www.icc-cpi.int/sites/default/files/RS-Eng.pdf}{, s. 12.}
\textit{“The crime of apartheid” means inhumane acts of a character similar to those referred to in paragraph 1, committed in the context of an institutionalized regime of systematic oppression and domination by one racial group over any other racial group or groups and committed with the intention of maintaining that regime;  }

\vspace{4mm}

\utlandslagrum{Internationella konventionen om avskaffande av alla former av rasdiskriminering (ICERD), artikel 3:}{https://www.regeringen.se/contentassets/87af45b9fb7f449a909b686204bb5527/fns-konventioner-om-manskliga-rattigheter/}{, s. 41.}
\textit{Konventionsstaterna fördömer särskilt rassegregation och apartheid och åtar sig att förhindra, förbjuda och utrota alla företeelser av detta slag inom områden under sin jurisdiktion.}

\vspace{4mm}

\utlandslagrum{Konventionen om avskaffande av all slags diskriminering av kvinnor (CEDAW):}{https://www.regeringen.se/contentassets/87af45b9fb7f449a909b686204bb5527/fns-konventioner-om-manskliga-rattigheter/}{, s. 51.}
\textit{De stater som har anslutit sig till denna konvention (konventionsstaterna) [...] betonar att utplånandet av apartheid, alla former av rasism, rasdiskriminering, kolonialism, nykolonialism, aggression, utländsk ockupation och utländskt herravälde samt av inblandning i staters inre angelägenheter är av största vikt för att män och kvinnor till fullo skall kunna åtnjuta sina rättigheter.}

\subsection*{Tystnaden inför apartheid är inte neutral}
\addcontentsline{toc}{subsection}{Tystnaden inför apartheid är inte neutral}

Alla former av affärstransaktioner och diplomatiska förbindelser bidrar till att upprätthålla en apartheidregim.  
Det som i dag erkänns av Internationella domstolen (ICJ) och av världens främsta folkrättsliga institutioner benämns ännu inte av Sveriges regering\footnote{\url{https://www.amnesty.org/en/latest/news/2022/02/israels-apartheid-against-palestinians/}}\footnote{\url{https://www.hrw.org/news/2021/04/27/israel-apartheid-against-palestinians}}.  

Tystnaden är inte neutral – den är ett aktivt val.



\subsection*{Regeringens selektiva moral och dubbla måttstockar}
\addcontentsline{toc}{subsection}{Regeringens selektiva moral och dubbla måttstockar}

\subsubsection*{Regeringens uttalanden och brist på åtgärder}
% Texten om statsministerns uttalande, från "Regeringen skriver vidare..." till "...fortsatt handel och diplomati som om inget hänt."

Regeringen skriver vidare:

\begin{quote}
\textit{“The terrorist organisation Hamas bears heavy responsibility for the current situation. The hostages must be released – unconditionally and immediately.”}
\end{quote}

Regeringen anser sig inta det moraliska överläget men har aldrig sanktionslagt Israel för dess folkrättsbrott. Ingen konsekvens har utdelats trots årtionden av övergrepp. Istället har Sverige fortsatt handel och diplomati som om inget hänt och har därmed inte gjort sin bekärda andel för upprätthållandet av internationell lag och ordning som skulle kunnat ge regeringen den moraliska höjden att utpeka en folkligt förankrad befrielserörelse såsom terroristorganisation – så till vida att denna inte beaktat och respekterat regeringens försök att upprätthålla internationell lag och ordning.

Detta är inte enbart en moralisk inkonsekvens. Det är ett folkrättsligt avtalsbrott i förhållande till Sveriges åtaganden enligt bland annat folkmordskonventionen, FN-stadgan och sedvanerättens tvingande normer.

Genom ett historiskt mönster av passivitet och konkludent handlande har Sverige inte bara underlåtit att agera – utan de facto bidragit till att ge Israel immunitet. Oavsett regim har Sverige fortsatt diplomatiskt och ekonomiskt stöd utan att utkräva ansvar för de upprepade folkrättsbrotten: illegala bosättningar, bombningar av civila och systematisk rasdiskriminering har ursäktats eller ignorerats.

Den som konsekvent tolererar statsterror förlorar rätten att fördöma motstånd mot den. Sverige har därmed spelat en \textit{aiding and abetting}-roll – och saknar rättslig trovärdighet när det gäller att peka ut andra för terror.

\subsubsection*{Regeringens underminering av folkrättens institutioner}
\addcontentsline{toc}{subsubsection}{Regeringens underminering av folkrättens institutioner}

Regeringen har konsekvent agerat för att skydda Israel från rättsligt ansvarsutkrävande. Den svenska statsministern har öppet opponerat sig mot Internationella brottmålsdomstolens (ICC) arresteringsorder mot Israels premiärminister – ett beslut grundat i omfattande bevisning om brott mot mänskligheten. Denna hållning utgör inte bara ett politiskt ställningstagande, utan en direkt underminering av det mest centrala internationella rättsliga verktyget för ansvar efter folkrättsbrott.

Samtidigt har regeringen ignorerat den samstämmiga bedömningen från världens främsta människorättsorganisationer\footnote{\url{https://www.amnesty.org/en/documents/mde15/8668/2024/en/}},  
Human Rights Watch\footnote{\url{https://www.hrw.org/report/2024/12/19/extermination-and-acts-genocide/israel-deliberately-depriving-palestinians-gaza}},  
FN:s särskilda rapportör Francesca Albanese\footnote{\url{https://www.ohchr.org/en/documents/country-reports/ahrc5573-report-special-rapporteur-situation-human-rights-palestinian}},  
och ledande folkmordsforskare\footnote{\url{https://ifpnews.com/top-scholars-israel-genocide-gaza/}},  
vilka alla konstaterat att ett folkmord pågår i Gaza. 

Den svenska regeringen har därmed förbrukat varje anspråk på att vara en neutral aktör eller försvarare av folkrätten. Genom att inte agera i enlighet med sina skyldigheter under folkmordskonventionen och FN-stadgan har Sverige inte enbart misslyckats i sitt förebyggande ansvar – det har valt sida i ett pågående folkrättsbrott.

\lagrum{Artikel I, Folkmordskonventionen\quad De fördragsslutande parterna bekräftar att folkmord, vare sig det begås i fredstid eller krigstid, är ett brott enligt internationell rätt som de åtar sig att förebygga och bestraffa.\footnote{\url{https://www.ohchr.org/en/instruments-mechanisms/instruments/convention-prevention-and-punishment-crime-genocide}}}

Denna artikel innebär uttryckligen att:
\begin{itemize}
  \item Staten inte enbart förbjuds att själv begå folkmord,
  \item utan är skyldig att förebygga det – även utanför sitt eget territorium,
  \item och att underlåtenhet att ingripa kan medföra internationellt ansvar.
\end{itemize}

Detta fastslogs entydigt av Internationella domstolen (ICJ) i målet \textit{Bosnia and Herzegovina v. Serbia and Montenegro} (2007):\footnote{\url{https://www.icj-cij.org/public/files/case-related/91/091-20070226-JUD-01-00-EN.pdf}}

\begin{quote}
\textit{“A State may incur responsibility not only for its own acts but also by aiding or assisting another State in the commission of an internationally wrongful act.”}
\end{quote}

(se domens punkt 420 ff.)

Vidare fastslås i FN-stadgan att medlemsstater inte får vara passiva inför grova människorättsbrott:

\lagrum{Article 1(3), FN-stadgan\quad To achieve international co-operation in solving international problems of an economic, social, cultural, or humanitarian character, and in promoting and encouraging respect for human rights...\footnote{\url{https://www.un.org/en/about-us/un-charter/full-text}}}

\lagrum{Article 56, FN-stadgan\quad All Members pledge themselves to take joint and separate action in co-operation with the Organization for the achievement of the purposes set forth in Article 55.\footnote{\url{https://www.un.org/en/about-us/un-charter/full-text}}}

Regeringens systematiska ignorans, dess aktiva försvar av förövaren och dess tystnad inför samstämmiga larm från rättsliga och humanitära organ visar att alla gränser nu är passerade. Detta handlar inte längre om att "utreda" eller "bevaka utvecklingen". Sveriges regering har, genom sitt agerande och sin underlåtenhet, trätt över den gräns som skiljer neutralitet från medverkan.

Detta dokument är därför inte en förfrågan, utan en anmälan.

\subsubsection*{Underlåtenhet att förebygga folkrättsbrott}
\addcontentsline{toc}{subsubsection}{Underlåtenhet att förebygga folkrättsbrott}

Den svenska regeringen har inte enbart förhållit sig passiv till folkmordskonventionens krav. Den har aktivt brutit mot dess anda och bokstav. Genom att frysa finansieringen av UNRWA – det FN-organ som ansvarar för livsnödvändigt bistånd till palestinska flyktingar – i enlighet med Israels påtryckningar, har Sverige bidragit till att avväpna det internationella systemet för humanitärt skydd.

Istället har regeringen omdirigerat biståndet till en israeliskt kontrollerad distributionsstruktur, underställd militär logistik. Detta har lett till ett system där civila i Gaza tvingas hämta mat och mediciner inom snäva, av militären fastställda tidsfönster – under hot om beskjutning. Det är ett system designat för kontroll, inte skydd. Enligt Euro-Med Human Rights Monitor har minst 60 civila skjutits ihjäl vid dessa hjälppunkter under tre dagar.

Detta utgör inte en avvikelse – det är en följd. En följd av att Sveriges regering valt att medverka till ett militärt organiserat biståndssystem som i praktiken upphäver Genèvekonventionens grundprinciper om opartiskhet, humanitet och civilas särskilda skyddsbehov.

Regeringen har därmed aktivt undergrävt FN:s auktoritet, legitimerat ett dödligt distributionssystem och svikit sitt ansvar att stå upp för den humanitära rätten. Det handlar inte om missriktad välvilja. Det handlar om medverkan till ett systematiskt brott mot folkrätten.

Vidare föreligger trovärdiga rapporter om att Israel beväpnar och skyddar väpnade grupper i Gaza – grupper vars ledare tidigare fängslats av Hamas för bland annat narkotikabrott och terrorism. Syftet har varit att destabilisera samhället inifrån och sabotera hjälpsystemet. Att Sveriges regering förblir tyst trots vetskap om detta är inte längre moraliskt förkastligt – det är rättsligt förpliktigande.

Vi konstaterar:

\begin{itemize}
  \item Att regeringens agerande utgör brott mot Sveriges skyldigheter enligt folkmordskonventionen.
  \item Att Sveriges modell för biståndsdistribution i Gaza innebär ett medvetet avsteg från humanitär folkrätt.
  \item Att tystnad inför rapporter om israeliskt stöd till kriminella och jihadistiska grupper kan medföra medansvar.
\end{itemize}

Vi kräver därför:

\begin{enumerate}
  \item Att utrikesministern offentligt kommenterar uppgifterna om svenskt stöd till en folkrättsvidrig biståndsmodell.
  \item Att Konstitutionsutskottet omedelbart utreder huruvida Utrikesdepartementet fullgjort sina förpliktelser enligt folkmordskonventionen och FN-stadgan.
\end{enumerate}

Detta är inte en begäran om förklaring. Det är en formell anklagelse. Sverige har inte bara brutit mot sina skyldigheter – det har gjort det med öppen blick och kallt beräknande.

\subsubsection*{Varför Israel stödjer jihadistgrupper}
\addcontentsline{toc}{subsubsection}{Varför Israel stödjer jihadistgrupper}

Frågan varför Israel aktivt stödjer salafistiska och jihadistiska grupper i Gaza – inklusive element med koppling till ISIS – måste förstås utifrån en strategisk logik, inte som ett säkerhetspolitiskt misslyckande.

\textbf{1. Hamas har förändrats – och det hotar Israels narrativ}

Sedan åtminstone 2006 har Hamas genomgått en djupgående politisk omorientering. I sitt policyprogram från 2017 samt i det gemensamma avtalet med Fatah 2021 erkände rörelsen:

\begin{itemize}
  \item internationell rätt som ramverk,
  \item PLO:s överordnade roll som palestinskt paraplyorgan,
  \item en tvåstatslösning enligt 1967 års gränser, med östra Jerusalem som huvudstad,
  \item och ett fredligt, folkligt motstånd som metod.
\end{itemize}

Denna förändring dokumenteras bland annat av Wikipedia, som redogör för hur Hamas i sitt nya policydokument 2017 explicit accepterade en palestinsk stat inom 1967 års gränser och bygger vidare på tidigare initiativ såsom Prisoners’ Document (2006)\footnote{\url{https://en.wikipedia.org/wiki/Palestinian_Prisoners\%27_Document}}, Mecka-avtalet (2007)\footnote{\url{https://en.wikipedia.org/wiki/Fatah–Hamas_Mecca_Agreement}} och avtalet 2020\footnote{\url{https://en.wikipedia.org/wiki/2020_Palestinian_reconciliation_agreement}}.

Akademiskt har detta analyserats som en genuin förändring, inte endast en kosmetisk fasad. Professor Neve Gordon\footnote{\url{https://en.wikipedia.org/wiki/Neve_Gordon}} och professor Menachem Klein\footnote{\url{https://www.972mag.com/hamas-fatah-elections-israel-arrogance/}} menar att denna utveckling syftade till att uppnå internationell legitimitet, demokratisk försoning och en väg mot ett återförenat palestinskt ledarskap.

Men enligt Klein underminerades dessa fredssträvanden aktivt av Israel, som i stället destabiliserade processen för att kunna upprätthålla narrativet att ”det saknas en trovärdig fredspartner”.

\textbf{2. Israel behöver oresonliga fiender för att rättfärdiga sin politik}

Att Israel historiskt har funnit strategiskt värde i att understödja radikala element är väl dokumenterat. Tidigare premiärminister Ehud Barak har själv erkänt att Israel i decennier aktivt stött Hamas i syfte att försvaga PLO – en splittringsstrategi som nu återanvänds, med än farligare inslag: salafistiska och jihadistiska klaner, inklusive grupper ledda av personer tidigare fängslade av Hamas för narkotikabrott och religiös extremism.

\textbf{Syftet är inte att bekämpa terror, utan att bevara kaos.} Ett extremistdominerat Gaza fungerar som en permanent motbild till ”fred”, och förstärker den israeliska statens centrala narrativ: att det inte existerar någon legitim, förhandlingsbar palestinsk motpart. Detta narrativ möjliggör fortsatt ockupation, kollektiv bestraffning och territoriell expansion – under förevändning av säkerhetsbehov.

Professor Norman Finkelstein har i detta sammanhang lyft den israeliska administrationens begrepp \textit{“The Palestinian peace offensive”} – en intern varningssignal för när motståndet uppfattas som \textit{för rationellt}. Ett måttfullt Hamas som respekterar internationell rätt, erkänner 1967 års gränser och söker val utgör ett långt större hot mot den israeliska långsiktiga strategin än en beväpnad jihadist. Fienden får inte bli trovärdig – den måste vara grotesk.

\textbf{Därför understöds extremism – inte trots dess brutalitet, utan på grund av dess politiska användbarhet.} Detta är inte ett misstag, inte en olycklig följd. Det är en avsiktlig, väldokumenterad realpolitisk strategi med djup historisk kontinuitet.

\medskip

\textbf{Att Sveriges regering ignorerar detta mönster – och därmed i praktiken legitimerar det – är inte en fråga om tolkning, utan om ansvar.} Genom att ensidigt fördöma Hamas, utan att erkänna rörelsens dokumenterade förvandling till en potentiell politisk aktör, förstärker Sverige ett förljugat narrativ som aktivt motverkar fred.

Detta är inte en underrättelsemiss. Det är ett systemfel.

\textbf{Vi frågar därför: Hur kan Sveriges regering, med tillgång till all tillgänglig dokumentation, agera såsom om konflikten endast handlade om terrorbekämpning – och inte om ockupation, kontroll och medvetet sabotage av varje fredsinitiativ som hotar status quo?}


\subsection{Sveriges folkrättsliga ansvar vid pågående folkrättsbrott}
%%%%%%%%%%%%%%%%%%%%%%%%%%%%%%%%%%%%%%%%%%
\subsubsection*{Sammanfattning: Sveriges folkrättsliga skyldigheter är bindande – inte valfria}
\addcontentsline{toc}{subsubsection}{Sammanfattning: Sveriges folkrättsliga skyldigheter är bindande – inte valfria}

Den svenska regeringens agerande måste nu bedömas i ljuset av sina folkrättsliga förpliktelser. Vad som ovan visats – ockupationens karaktär, sabotaget mot fredsprocesser, beväpning av extremistgrupper och vägran att ingripa – utgör inte enbart politiska eller moraliska tillkortakommanden. Det är fråga om konkreta rättsbrott genom underlåtenhet att uppfylla bindande konventionsplikt.

\medskip

Enligt artikel I i Konventionen om förebyggande och bestraffning av brottet folkmord är Sverige skyldigt att inte bara avstå från folkmord, utan också att aktivt förebygga och straffa det. Denna skyldighet har erkänts av Internationella domstolen (ICJ) som en \textit{erga omnes}-förpliktelse – det vill säga en skyldighet som gäller gentemot hela det internationella samfundet.

Därtill är Sverige bundet av sedvanerättens princip om \textit{non-assistance in wrongful acts}, som förbjuder:

\begin{itemize}
  \item att bistå aktörer som begår folkrättsbrott,
  \item att ekonomiskt eller politiskt dra nytta av sådana brott,
  \item att förhålla sig passiv när man har kännedom om brott och en rättslig skyldighet att agera.
\end{itemize}

Detta gäller i synnerhet vid:

\begin{itemize}
  \item folkmord (Genocide Convention),
  \item brott mot mänskligheten (Romstadgan),
  \item grova krigsbrott (Genèvekonventionerna),
  \item apartheid (FN:s apartheidkonvention).
\end{itemize}

Att i detta läge – där Israel systematiskt förvägrar Gazas befolkning skydd enligt humanitär rätt – kräva att det palestinska folket ska avstå från motstånd, utan att samtidigt ingripa mot förövaren, är inte en neutral hållning. Det är ett rättsbrott.

Att kalla varje handling av motstånd “terrorism” medan man själv skyddar, finansierar eller legitimerar ockupationsmakten är att delta i det rättsvidriga status quo.

\medskip

Endast den stat som själv uppfyller sina rättsliga skyldigheter kan moraliskt och juridiskt fördöma andras svar. Regeringens vägran att göra detta innebär att Sverige:

\begin{itemize}
  \item förnekar det palestinska folket varje legitimt alternativ till självförsvar,
  \item avstår från att använda diplomatiska, rättsliga och ekonomiska påtryckningsmedel för att förhindra folkrättsbrott,
  \item och därigenom gör sig medskyldig genom underlåtenhet att reagera.
\end{itemize}

Det är alltså inte den som slår tillbaka i desperation som bär huvudansvaret – utan den regering som, trots kännedom om brotten, vägrar att ingripa.

\bigskip

\subsubsection*{Avtalet med Elbit Systems – svensk medverkan}
\addcontentsline{toc}{subsubsection}{Avtalet med Elbit Systems – svensk medverkan}

Regeringens skuld stannar inte vid demoniseringen av en folkligt förankrad befrielserörelse. Den sträcker sig vidare genom total vägran att tillämpa internationell rätt gentemot Israel – trots överväldigande bevis på tidigare krigsbrott och folkrättsbrott (före den 7 oktober 2023) – och kulminerar i aktiv medverkan.

Den 27 oktober 2023 – samma dag som Israels markinvasion av Gaza inleddes – undertecknade Sveriges regering ett militärt samarbetsavtal med den israeliska vapentillverkaren Elbit Systems. Genom detta har Sverige aktivt bidragit till att legitimera och stödja en krigförande stats vapenindustri mitt under ett pågående folkmord.

Detta trots att det sedan länge är dokumenterat att Elbit och andra israeliska försvarskoncerner använder Gaza som testarena för nya vapensystem. Dessa vapen säljs därefter internationellt som “battle tested”.\footnote{\url{https://www.youtube.com/watch?v=78rs9_FrgmA}} Journalisten Yotam Feldman har visat detta i dokumentären \textit{The Lab}.

Enligt Euro-Med Human Rights Monitor hade Israel, redan inom den första månaden av angreppet, fällt en mängd sprängmedel över Gaza motsvarande två Hiroshimabomber.\footnote{\url{https://euromedmonitor.org/en/article/5908/Israel-hits-Gaza-Strip-with-the-equivalent-of-two-nuclear-bombs}} Trots detta valde Sveriges regering att investera i denna vapenapparat – samtidigt som den fördömde det palestinska motståndet och förteg ockupationsmaktens rättsbrott.

Detta agerande – att i affektens skugga inleda militärt samarbete med en regim som systematiskt bryter mot internationell rätt – utgör inte bara ett moraliskt svek. Det är en rättsstridig handling i sig. Det är medverkan till folkrättsbrott.

\end{comment}