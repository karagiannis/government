% filnamn: 2a_folkrattens_grundprinciper.tex
%
% SYFTE:
% - Etablera det rättsliga ramverk som styr staternas agerande under väpnad konflikt och ockupation.
% - Identifiera de folkrättsliga skyldigheter som följer av detta ramverk.
% - Klargöra vad som utgör en laglig respektive olaglig ockupation, och dess rättsverkningar.
% - Förklara vilka folkrättsliga normer som är absoluta (jus cogens): t.ex. våldsförbud, folkmordsförbud, skyddsansvar.
% - Filen nämner inga stater vid namn – analysen sker uteslutande på normativ nivå.
%


\subsection{Folkrättens bindande ramverk}
\label{subsec:folkrattens_bindande_ramverk}

\subsubsection*{1. Folkrättens struktur och rättskällor}

Folkrätten utgör det överstatliga regelverk som styr staternas inbördes relationer. 
Den vilar på tre grundprinciper: staters suveränitet, territoriell integritet och den 
bindande karaktären hos internationella åtaganden. Dessa principer är kodifierade genom FN-stadgan, 
sedvanerätt samt multilaterala traktater såsom Genèvekonventionerna och folkmordskonventionen.

Sverige är folkrättsligt bundet dels genom ratificering av sådana traktater, dels genom sitt 
medlemskap i Förenta nationerna, vilket innebär särskilda skyldigheter enligt FN-stadgans kapitel I\footnote{\url{https://www.un.org/en/about-us/un-charter/chapter-1}}.

\subsubsection*{2. Våldsförbudet som absolut rättsnorm}

\lagrum{FN-stadgan, Artikel 2(1)\quad The Organization is based on the principle of the sovereign equality of all its Members.}
\lagrum{FN-stadgan, Artikel 2(4)\quad All Members shall refrain in their international relations from the threat or use of force against the territorial integrity or political independence of any state.}

Dessa artiklar fastslår ett generellt våldsförbud – en grundpelare i modern folkrätt. 
Endast två undantag erkänns: åtgärder godkända av säkerhetsrådet enligt kapitel VII, samt 
rätten till självförsvar enligt artikel 51 – vilket förutsätter ett väpnat angrepp mot en suverän stat.

\lagrum{FN-stadgan, Artikel 51 \quad Nothing in the present Charter shall impair the inherent right of individual or collective self-defence if an armed attack occurs against a Member of the United Nations, until the Security Council has taken measures necessary to maintain international peace and security. Measures taken by Members in the exercise of this right of self-defence shall be immediately reported to the Security Council and shall not in any way affect the authority and responsibility of the Security Council under the present Charter to take at any time such action as it deems necessary in order to maintain or restore international peace and security.}


\subsubsection*{3. Självförsvar kan inte åberopas av ockupationsmakt}

Begreppet \enquote{självförsvar} är strikt begränsat i både territoriellt och subjektivt hänseende. 
Det kan endast åberopas av en stat som blivit föremål för ett väpnat angrepp från en annan suverän stat – 
inte mot en befolkning eller ett område som den redan ockuperar.

En ockupationsmakt kan därmed inte hävda självförsvar enligt artikel 51 i FN-stadgan mot invånare 
i det ockuperade territoriet. I sådana fall gäller istället särskilda folkrättsliga skyldigheter enligt 
Haagreglementet och Genèvekonventionen IV.

Denna princip har bekräftats av Internationella domstolen (ICJ) i det rådgivande yttrandet 
\textit{Legal Consequences of the Construction of a Wall in the Occupied Palestinian Territory} (2004).\footnote{\url{https://www.icj-cij.org/case/131}}

\lagrum{\ldots{} Lastly, the Court concluded that Israel could not rely on a right of self-defence or on a state of necessity in order to preclude the wrongfulness of the construction of the wall, and that such construction and its associated régime were accordingly contrary to international law.}

\subsubsection*{4. Ockupationens rättsliga regler – skyddsansvar}

Ockupation regleras av 1907 års Haagreglemente och 1949 års fjärde Genèvekonvention. 
Enligt dessa rättskällor bär ockupationsmakten ett positivt skyddsansvar gentemot civilbefolkningen i 
det ockuperade territoriet.

\lagrum{Haagreglementet, artikel 43:\quad The authority of the legitimate power having in fact passed into the hands of the occupant, the latter shall take all the measures in his power to restore, and ensure, as far as possible, public order and safety, while respecting, unless absolutely prevented, the laws in force in the country.\footnote{\url{https://ihl-databases.icrc.org/en/ihl-treaties/hague-conv-iv-1907/regulations-art-43?activeTab=}}}

\begin{quote}
Genevekonvention IV, artikel 27\footnote{\url{https://ihl-databases.icrc.org/en/ihl-treaties/gciv-1949/article-27?activeTab=}}:

Protected persons are entitled, in all circumstances, to respect for their persons, their honour, their family rights, their religious convictions and practices, and their manners and customs. They shall at all times be humanely treated, and shall be protected especially against all acts of violence or threats thereof and against insults and public curiosity.

Women shall be especially protected against any attack on their honour, in particular against rape, enforced prostitution, or any form of indecent assault.

Without prejudice to the provisions relating to their state of health, age and sex, all protected persons shall be treated with the same consideration by the Party to the conflict in whose power they are, without any adverse distinction based, in particular, on race, religion or political opinion.

However, the Parties to the conflict may take such measures of control and security in regard to protected persons as may be necessary as a result of the war.
\end{quote}


\medskip

% Övergång
\medskip

\noindent
I nästa avsnitt fördjupar vi analysen av det ansvar som vilar på varje stat att upprätthålla folkrättens normer — inte bara genom att avstå från egna överträdelser, utan genom att inte bidra till eller tolerera andras. 

Där klargörs vilka rättskällor — inklusive ILC:s artiklar om statsansvar, folkmordskonventionen och FN-stadgan — som ålägger stater ett positivt ansvar att förebygga, motverka och inte medverka till internationella rättsbrott.






















