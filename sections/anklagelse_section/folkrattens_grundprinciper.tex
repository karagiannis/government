% filnamn: folkrattens_grundprinciper.tex
% SYFTE: Förklara den rättsliga struktur som Sverige bryter mot.
% Skapa spelplanen: folkrätt, RF, skyldigheter, suveränitetsgränser.

\subsection{Folkrättens bindande ramverk}

Folkrätten utgör det överstatliga regelverk som reglerar staternas inbördes relationer. Den vilar på tre bärande principer: suverän statsmakt, territoriell integritet och internationella förpliktelsers bindande karaktär. Dessa är kodifierade genom FN-stadgan, genom sedvanerättsliga normer och genom multilaterala traktater såsom Genèvekonventionerna och folkmordskonventionen.

Sverige är folkrättsligt bunden dels genom undertecknande och ratificering av sådana instrument, dels genom att vara medlem i Förenta Nationerna, där medlemskapet medför särskilda skyldigheter i enlighet med stadgans kapitel I.

\lagrum{Artikel 2(1), FN-stadgan\quad The Organization is based on the principle of the sovereign equality of all its Members.}

\lagrum{Artikel 2(4), FN-stadgan\quad All Members shall refrain in their international relations from the threat or use of force against the territorial integrity or political independence of any state.}

Dessa artiklar fastställer ett generellt våldsförbud, vilket utgör folkrättens kärna. Endast två undantag erkänns: beslut från säkerhetsrådet i enlighet med kapitel VII, samt rätten till självförsvar enligt artikel 51 – förutsatt att det föreligger ett väpnat angrepp riktat mot en suverän stat.

\lagrum{Artikel 51, FN-stadgan\quad Nothing in the present Charter shall impair the inherent right of individual or collective self-defence if an armed attack occurs against a Member of the United Nations.}

Begreppet ”självförsvar” är således territoriellt och rättssubjektmässigt begränsat. Det kan inte åberopas av en ockupationsmakt gentemot den befolkning som står under dess kontroll. Denna princip är bekräftad av Internationella domstolen (ICJ) i en rad avgöranden, däribland \textit{Legal Consequences of the Construction of a Wall in the Occupied Palestinian Territory} (2004).

\medskip

Ockupation regleras i detalj av 1907 års Haagreglemente och 1949 års fjärde Genèvekonvention. Dessa rättskällor stadgar att en ockupant har ett positivt skyddsansvar gentemot civilbefolkningen inom det ockuperade territoriet. Syftet med dessa regler är att avvärja just det mönster vi i dag bevittnar: systematiskt övervåld, kollektiva bestraffningar och åsidosättande av civila skyddsintressen.

\lagrum{Artikel 43, Haagreglementet\quad The authority of the legitimate power having in fact passed into the hands of the occupant, the latter shall take all the measures in his power to restore, and ensure, as far as possible, public order and safety, while respecting, unless absolutely prevented, the laws in force in the country.}

\lagrum{Artikel 27, Genèvekonvention IV\quad Protected persons are entitled, in all circumstances, to respect for their persons, their honour, their family rights, their religious convictions and practices, and their manners and customs. They shall at all times be humanely treated, and shall be protected especially against all acts of violence or threats thereof.}

\medskip

Vid bedömningen av en stats ansvar aktualiseras även frågan om underlåtenhet att agera. Detta har särskilt behandlats inom ramen för folkmordskonventionen, där konventionsstaternas skyldighet att förebygga är självständig och inte betingad av föregående folkrättslig bedömning från FN:s säkerhetsråd eller Internationella domstolen.

Sverige, såsom konventionsstat, är därför förpliktat att inte endast avstå från att begå folkmord, utan även att aktivt verka för att förhindra det. Detta omfattar förbud mot stöd, legitimering eller tyst acceptans av åtgärder som sannolikt utgör brott mot konventionen.

\medskip

Sammanfattningsvis är följande rättsprinciper centrala för den fortsatta bedömningen:

\begin{itemize}
  \item att folkrättens våldsförbud är absolut och endast medger snävt definierade undantag,
  \item att självförsvar enligt artikel 51 FN-stadgan inte kan åberopas av en ockupationsmakt gentemot civilbefolkningen i det ockuperade territoriet,
  \item att en ockupationsmakt har ett förstärkt ansvar att skydda civilbefolkningen enligt Genèvekonventionen,
  \item att Sverige är folkrättsligt förpliktat att vidta förebyggande åtgärder mot folkmord, även vid brott begångna av annan stat,
  \item att underlåtenhet att agera, under vissa förutsättningar, kan grunda internationellt ansvar.
\end{itemize}

Den fortsatta analysen kommer att visa att Sveriges regering inte har iakttagit dessa rättsprinciper.
