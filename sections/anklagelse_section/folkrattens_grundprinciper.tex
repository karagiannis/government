% filnamn: folkrattens_grundprinciper.tex
% SYFTE: Förklara den rättsliga struktur som Sverige bryter mot.
% Skapa spelplanen: folkrätt, RF, skyldigheter, suveränitetsgränser.

\subsection{Folkrättens bindande ramverk}

Folkrätten utgör det överstatliga regelverk som styr staternas inbördes relationer. Den vilar på tre grundprinciper: staters suveränitet, territoriell integritet och den bindande karaktären hos internationella åtaganden. Dessa principer är kodifierade genom FN-stadgan, sedvanerätt samt multilaterala traktater såsom Genèvekonventionerna och folkmordskonventionen.

Sverige är folkrättsligt bundet dels genom ratificering av sådana traktater, dels genom sitt medlemskap i Förenta nationerna, vilket innebär särskilda skyldigheter enligt FN-stadgans kapitel I.

\lagrum{Artikel 2(1), FN-stadgan\quad The Organization is based on the principle of the sovereign equality of all its Members.}

\lagrum{Artikel 2(4), FN-stadgan\quad All Members shall refrain in their international relations from the threat or use of force against the territorial integrity or political independence of any state.}

Dessa artiklar fastslår ett generellt våldsförbud – en grundpelare i modern folkrätt. Endast två undantag erkänns: åtgärder godkända av säkerhetsrådet enligt kapitel VII, samt rätten till självförsvar enligt artikel 51 – vilket förutsätter ett väpnat angrepp mot en suverän stat.

\lagrum{Artikel 51, FN-stadgan\quad Nothing in the present Charter shall impair the inherent right of individual or collective self-defence if an armed attack occurs against a Member of the United Nations.}

Begreppet ”självförsvar” är strikt begränsat i både territoriellt och subjektivt hänseende. Det kan inte åberopas av en ockupationsmakt mot den befolkning den själv kontrollerar. Denna princip har bekräftats av Internationella domstolen (ICJ) i bland annat det rådgivande yttrandet \textit{Legal Consequences of the Construction of a Wall in the Occupied Palestinian Territory} (2004).

\medskip

Ockupation regleras av 1907 års Haagreglemente och 1949 års fjärde Genèvekonvention. Enligt dessa rättskällor bär ockupationsmakten ett positivt skyddsansvar gentemot civilbefolkningen i det ockuperade territoriet. Reglerna syftar till att förhindra det mönster vi i dag bevittnar: systematiskt övervåld, kollektiva bestraffningar och åsidosättande av civila skyddsintressen.

\lagrum{Artikel 43, Haagreglementet\quad The authority of the legitimate power having in fact passed into the hands of the occupant, the latter shall take all the measures in his power to restore, and ensure, as far as possible, public order and safety, while respecting, unless absolutely prevented, the laws in force in the country.}

\lagrum{Artikel 27, Genèvekonvention IV\quad Protected persons are entitled, in all circumstances, to respect for their persons, their honour, their family rights, their religious convictions and practices, and their manners and customs. They shall at all times be humanely treated, and shall be protected especially against all acts of violence or threats thereof.}

\medskip

Frågan om statligt ansvar aktualiseras även vid underlåtenhet att agera. Folkmordskonventionen föreskriver att konventionsstaterna har en självständig skyldighet att förebygga folkmord – en skyldighet som gäller oberoende av om säkerhetsrådet eller ICJ har gjort en formell bedömning.

Sverige är som konventionsstat därmed förpliktat inte bara att avstå från att själv begå folkmord, utan också att aktivt verka för att förebygga sådana brott. Denna skyldighet innefattar förbud mot stöd, legitimering eller tyst acceptans av handlingar som sannolikt strider mot konventionen.

\medskip

Sammanfattningsvis är följande rättsprinciper avgörande för den fortsatta analysen:

\begin{itemize}
  \item folkrättens våldsförbud är absolut och medger endast snävt definierade undantag,
  \item rätten till självförsvar enligt artikel 51 kan inte åberopas av en ockupationsmakt mot civilbefolkningen i det ockuperade territoriet,
  \item en ockupationsmakt har ett förstärkt skyddsansvar enligt Genèvekonventionen,
  \item konventionsstater enligt folkmordskonventionen har en aktiv skyldighet att förebygga folkmord, även när brott begås av tredje stat,
  \item underlåtenhet att agera kan, under vissa omständigheter, grunda internationellt rättsligt ansvar.
\end{itemize}

Följande avsnitt visar att Sveriges regering har åsidosatt dessa principer i strid med såväl internationell som nationell rätt.
