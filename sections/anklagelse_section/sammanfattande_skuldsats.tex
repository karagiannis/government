% filnamn: sammantattande_skuldsats.tex
% SYFTE: Ställ hypotesen klart. "Regeringen är skyldig." – detta ska bevisas.
% Introducera grundlagen och folkrätten som rättskällor. Skapa förväntan.

% filnamn: summerande_skuldsats.tex
\subsection{Regeringens brott mot folkrätten och grundlagen}

Sveriges regering har, i sin hantering av Israels agerande i Palestina, genom underlåtenhet och aktivt handlande brutit mot både internationella förpliktelser och nationella konstitutionella normer. Ansvaret omfattar tre distinkta men sammanhängande rättsöverträdelser:

\begin{enumerate}
    \item \textbf{Underlåtenhet att förebygga folkmord} enligt artikel I i folkmordskonventionen
    \item \textbf{Brutt mot FN-stadgans artikel 56} genom underlåtenhet att vidta nödvändiga åtgärder
    \item \textbf{Lojalitetsbrott mot RF 10 kap. 1 §} genom svek av traktatförpliktelser
\end{enumerate}

\lagrum{10 kap. 1 § Regeringsformen (RF)\quad Överenskommelser med andra stater [...] ingås av regeringen.}

Detta konstitutionella mandat innehåller en underförstådd \textit{lojalitetsplikt} som kräver aktivt arbete för att säkerställa traktaters efterlevnad. I folkrätten återfinns parallellen i \textit{pacta sunt servanda}-principen (Wienkonventionen 1969, artikel 26), vilken ICJ bekräftat som sedvanerätt i \textit{Gabčíkovo-Nagymaros}-målet (1997).

\lagrum{Artikel I, Folkmordskonventionen\quad Fördragsslutande parter förbinder sig att förebygga och bestraffa folkmord}

ICJ:s tolkning i \textit{Bosnien mot Serbien} (ICJ Reports 2007, §430) klargör att denna skyldighet:
\begin{itemize}
    \item Är \textit{erga omnes} (gäller gentemot alla stater)
    \item Aktiveras vid "risk för folkmord", inte först vid fullbordat brott
    \item Inkluderar skyldighet att inte underlåta rimliga åtgärder
\end{itemize}

\lagrum{Artikel 56, FN-stadgan\quad Medlemsstater förbinder sig att vidta gemensamma och enskilda åtgärder [...] för främjande av mänskliga rättigheter}

Sveriges passivitet trots \textit{officiella varningar} från:
\begin{itemize}
    \item FN:s folkmordsexpert (rapport A/HRC/55/73, 25 mars 2024)
    \item ICC åklagarens uttalande om skälig grund för folkmordsutredning
    \item ICJ:s preliminära åtgärder i \textit{Sydafrika mot Israel} (januari 2024)
\end{itemize}

\vspace{0.5cm}
\textbf{Slutsats:} Regeringen har genom:
\begin{enumerate}
    \item Upprepad vägran att erkänna folkmordsrisken trots överväldigande bevis
    \item Fortsatt militärt och diplomatiskt stöd till Israel
    \item Aktiv motverkan av internationella sanktionsförsök
\end{enumerate}
uppfyllt rekvisiten för medhjälp till folkmord enligt artikel III(e) i folkmordskonventionen, samt brutit mot RF:s krav på lojal avtalsuppfyllnad.