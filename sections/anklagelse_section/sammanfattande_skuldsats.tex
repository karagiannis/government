% filnamn: sammantattande_skuldsats.tex
% SYFTE: Ställ hypotesen klart. "Regeringen är skyldig." – detta ska bevisas.
% Introducera grundlagen och folkrätten som rättskällor. Skapa förväntan.

\subsection{Regeringens brott mot folkrätten och grundlagen}

Sveriges regering har, i sin hantering av Israels agerande i Palestina, genom både underlåtenhet och aktivt agerande brutit mot såväl internationella rättsförpliktelser som konstitutionella normer. Ansvaret omfattar tre distinkta, men inbördes sammanhängande rättsöverträdelser:

\begin{enumerate}
    \item \textbf{Underlåtenhet att förebygga folkmord} i strid med artikel I i folkmordskonventionen
    \item \textbf{Brott mot FN-stadgans artikel 56} genom passivitet i främjandet av mänskliga rättigheter
    \item \textbf{Lojalitetsbrott mot RF 10 kap. 1 §} genom kontraktsbrott mot folkrättsliga åtaganden
\end{enumerate}

\lagrum{10 kap. 1 § Regeringsformen (RF)\quad Överenskommelser med andra stater [...] ingås av regeringen.}

Detta konstitutionella mandat innefattar en underförstådd \textit{lojalitetsplikt} att säkerställa efterlevnaden av traktatsförpliktelser. Motsvarande folkrättslig princip återfinns i \textit{pacta sunt servanda} (Wienkonventionen om traktaträtten, artikel 26), vilken av Internationella domstolen (ICJ) har bekräftats som sedvanerätt i målet \textit{Gabčíkovo–Nagymaros} (ICJ Reports 1997, s. 38).

\lagrum{Artikel I, Folkmordskonventionen\quad De fördragsslutande parterna förbinder sig att förebygga och bestraffa folkmord.}

ICJ:s praxis i målet \textit{Bosnien mot Serbien} (ICJ Reports 2007, §430) fastslår att denna skyldighet:

\begin{itemize}
    \item är \textit{erga omnes} – gäller gentemot hela det internationella samfundet,
    \item aktiveras vid konstaterad risk för folkmord – inte först vid fullbordat brott,
    \item innefattar en plikt att vidta rimliga åtgärder för att förebygga folkmord.
\end{itemize}

\lagrum{Artikel 56, FN-stadgan\quad Medlemsstaterna förbinder sig att vidta gemensamma och enskilda åtgärder [...] för främjande av mänskliga rättigheter.}

Sveriges passivitet står i skarp kontrast till de \textit{officiella varningar} som utgått från:

\begin{itemize}
    \item FN:s särskilde rapportör om folkmordsförebyggande (A/HRC/55/73, 25 mars 2024),
    \item ICC:s åklagare, som uttalat att skälig grund för folkmordsutredning föreligger,
    \item ICJ:s beslut om preliminära åtgärder i målet \textit{Sydafrika mot Israel} (januari 2024).
\end{itemize}

\vspace{0.5cm}
\textbf{Slutsats:} Genom att:

\begin{enumerate}
    \item konsekvent vägra erkänna folkmordsrisken trots överväldigande bevisläge,
    \item fortsätta tillhandahålla politiskt, ekonomiskt och militärt stöd till Israel,
    \item aktivt motverka internationella sanktionsåtgärder,
\end{enumerate}

har Sveriges regering uppfyllt rekvisiten för \textit{medverkan till folkmord} enligt artikel III(e) i folkmordskonventionen samt brutit mot Regeringsformens krav på lojal och rättstrogen uppfyllelse av internationella överenskommelser.
