


% Pröva om Hamas juridiskt kan klassificeras som terroristorganisation givet folkrättsbrottens art från Israel.
% Använd historisk analogi (ex: Nat Turner, slavuppror).
% Argumentera att terrorbegreppet förlorar legitimitet om det tillämpas selektivt.

\section{Rättslig bedömning av Hamas status}
Regeringen proklamerar:
\textit{The terrorist organisation Hamas bears heavy responsibility for the current situation. }\\

För resonemangets skull – och enbart som hypotetisk premiss – antar vi att samtliga israeliska uppgifter om övergrepp från Hamas är korrekta, trots att flera av dem senare visat sig vara obestyrkta eller direkt falska.
Låt oss dessutom anta att Hamas inte är en folkvald regering och dessutom inte är en folkföränkrad befrielserörelse.


Då regeringen inte uppfyller sina förpliktelser enligt ingångna folkrättsliga avtal – såsom att verka för respekt av FN:s resolutioner och folkrättens regler – utan istället tillerkänner Israel praktisk immunitet trots fortlöpande brott mot internationell rätt, uppstår ett ansvar enligt principerna om staters medverkan till folkrättsbrott.

Detta ansvar förstärks när Sverige inte enbart underlåter att ingripa, utan även aktivt karaktäriserar det motstånd som utövas av den ockuperade parten som ``terrorism'' – utan att samtidigt erkänna att rätten till beskydd enligt internationell humanitär rätt nekats den befolkningen.

I en sådan situation sker inte bara ett brott mot neutralitetsprincipen och ett passivt medansvar – utan även en retorisk och diplomatisk handling som riskerar att inflammera konflikten ytterligare. Denna eskalation medför ett förstärkt medansvar även för de illdåd som följer, om det kan visas att staten bidragit till att skapa förtvivlan eller rättslöshet hos den utsatta parten.

\lagrum{Artikel 16, ILC Articles on State Responsibility\quad 
A State which aids or assists another State in the commission of an internationally wrongful act by the latter is internationally responsible\footnote{\url{https://legal.un.org/ilc/texts/instruments/english/draft_articles/9_6_2001.pdf}}...}

\lagrum{Artikel 1, Genèvekonventionerna\quad 
The High Contracting Parties undertake to respect and to ensure respect for the present Convention in all circumstances\footnote{\url{https://ihl-databases.icrc.org/en/ihl-treaties/gciv-1949/article-1}}.}


Det är denna konstruktion som förklarar varför en betydande del av det internationella samfundet inte erkänner Hamas som en terroristorganisation, utan istället betraktar dem som en aktör i en asymmetrisk konflikt där de rättsliga definitionerna måste ses i ljuset av förvägrad rättsordning och ockupation.\footnote{\url{https://en.wikipedia.org/wiki/Hamas\#/media/File:International_views_on_Hamas.svg}}

Ur ett realpolitiskt perspektiv kan en regering i extrema undantagsfall avvika från ingångna förpliktelser under hot om nationell undergång. Men detta får aldrig ske i det tysta eller som medvetet hållen, långvarig politik. Principen om \textit{pacta sunt servanda} är inte förhandlingsbar. Att aktivt bryta mot egna åtaganden och samtidigt bidra till en situation där våldet förvärras, kan aldrig rättfärdigas inom ramen för en rättsstat.

\begin{tcolorbox}[title=Rättsfigurer som aktualiseras vid terroristutpekning, colback=gray!10, colframe=black, sharp corners=south]
\textbf{Följande rättsliga principer och figurer aktualiseras i samband med att Sverige benämner Hamas som terroristorganisation, utan att samtidigt erkänna Israels rättsbrott:}

\begin{itemize}
    \item \textbf{Statsansvar vid medverkan} – enligt \textit{ILC Articles on State Responsibility} (särskilt art. 16 och 41)\footnote{\url{https://legal.un.org/ilc/texts/instruments/english/draft_articles/9_6_2001.pdf}}, riskerar Sverige att bli folkrättsligt medansvarigt om det bistår eller möjliggör folkrättsbrott genom partiskhet.
    
    \item \textbf{Neutralitetsplikt} – enligt sedvanerätt och internationell humanitär rätt\footnote{\url{https://ihl-databases.icrc.org/en}}, får en icke-stridande stat inte selektivt stödja en konfliktpart som själv bryter mot Genèvekonventionerna\footnote{\url{https://ihl-databases.icrc.org/en/ihl-treaties/gciv-1949}}.
    
    \item \textbf{Förnekande av skydd} – att neka en befolkning, eller dess representanter, skydd under IHL utgör rättsstridigt undandragande av erkända rättigheter och kan betraktas som kollektiv bestraffning.
    
    \item \textbf{Culpa in contrahendo} – staten agerar i strid med lojalitetsplikten när den undertecknar avtal om humanitär rätt, men medvetet vägrar tillämpa dess principer konsekvent.
    
    \item \textbf{Pacta sunt servanda} – folkrättens kärnprincip: ingångna avtal ska hållas enligt Wienkonventionen om traktaträtten, artikel 26\footnote{\url{https://legal.un.org/ilc/texts/instruments/english/conventions/1_1_1969.pdf}}. Retorik och politik som strider mot denna princip försvagar hela det traktatbaserade folkrättssystemet.
    
    \item \textbf{Förstärkt medansvar} – genom att retoriskt utpeka en ockuperad befolkning som "terrorister", utan erkännande av rättslöshet och förtryck, bidrar staten aktivt till konfliktens eskalation och bär därmed del i det moraliska och folkrättsliga ansvaret.
\end{itemize}
\end{tcolorbox}

Det är i denna kontext som även språkbruket i sig får rättsverkan.

När en stat benämner en aktör i en ockuperad befolkning som “terroristorganisation”, utan att samtidigt erkänna det rättsliga sammanhanget – ockupationen, rättslösheten och vägran att ge skydd enligt IHL – så sker ett folkrättsligt konkludent handlande.

Formuleringen blir inte neutral, utan ger det sken av att det endast finns en skyldig part. Detta utgör i sig en tyst sanktionering av ockupationen och bidrar till att normalisera den rättsvidriga ordningen.

Att därefter tillfoga “men Israel måste följa internationell rätt” saknar betydelse om man redan i språkbruket legitimerat själva asymmetrin. Retoriken skapar en ny rättslig utgångspunkt: inte om förtrycket är lagligt, utan hur hårt den förtryckta får slå tillbaka.

Detta är vad som i folkrätten betraktas som ett konkludent godkännande – en handling med juridisk effekt, trots att den sker i form av språk och diplomatiskt ställningstagande.

Ytterligare en aspekt aktualiseras här – nämligen den folkrättsliga motsvarigheten till \textit{stämpling till brott}.

Att benämna ett motstånd inom en ockuperad befolkning som "terrorism", utan att samtidigt tillerkänna befolkningen det skydd som Genèvekonventionerna förutsätter, är inte bara en retorisk gärning. Det riskerar att uppfattas som att staten – i samråd eller i lojal sympati – legitimerar ockupationsmaktens rättsbrott. Detta kan liknas vid en form av diplomatisk eller språklig \textit{anstiftan}.

På motsvarande sätt aktualiseras även principen om \textit{underlåtenhet att avslöja eller förhindra brott}, så som den uttrycks i 23 kap. 6 § BrB. En stat som är bunden av folkrätten men underlåter att protestera mot grova rättsbrott – trots kännedom – uppfyller de moraliska och juridiska kriterierna för passivt medverkansansvar, särskilt om staten samtidigt förfogar över diplomatisk eller institutionell makt att påverka utvecklingen.

I ett internationellt sammanhang motsvaras dessa handlingar av de former för ansvar som regleras i \textit{ILC Articles on State Responsibility}, särskilt artiklarna 16 (aid or assistance), 41 (duty not to recognize the situation as lawful) och 40 (responsibility for serious breaches of peremptory norms).

Sammantaget uppstår ett mönster: staten utövar ett aktivt språkbruk som konkludent legitimerar en rättsvidrig ordning, underlåter att ingripa trots skyldighet, och bidrar därmed – med eller utan uppsåt – till att möjliggöra fortsatt rättsbrott. Detta skapar vad som i straffrättsliga termer motsvarar ett samverkansansvar.


\subsection{Från Nat Turner till Gaza: Vem bär ansvaret för våldets födelse?}
%\addcontentsline{toc}{subsection}{Från Nat Turner till Gaza: Vem bär ansvaret för våldets födelse?}

En slav utan tillgång till lag, som gör uppror, är inte en terrorist.

Nat Turner\footnote{\url{https://www.normanfinkelstein.com/nat-turner-in-gaza/}}, slavupprorsledaren som 1831 dödade oskyldiga vita i den största slavrevolten i USA:s historia, betraktades som ”terrorist” av dem som försvarade slaveriet. Abolitionisterna erkände våldets fasor – men vägrade fördöma upprorsmännen. De riktade sin vrede mot det system som födde våldet. I dag erkänns Turner som hjälte.

Här finns paralleller. Palestina har aldrig erbjudits rättvisa. Att fördöma dess motstånd, utan att erkänna den rättslösa omgivning det fötts i, är moraliskt absurt och juridiskt ohållbart.

I rättslig mening vilar det tyngsta ansvaret inte på den som slår tillbaka – utan på den som vägrar ingripa. Det är de som åtagit sig att vara systemets väktare, men som låter rättsordningen falla, som bär huvudansvaret.

En människa utan tillgång till rättsligt skydd, vars vädjan om beskydd ignoreras av världens självutnämnda ”guardians”, kan inte utan vidare stämplas som terrorist. Hon är, i sin desperation, inte ett hot mot rättsstaten – utan dess spegelbild när skyddet uteblir.

Nat Turner dödade inte i brist på moral, utan i brist på erkännande. Han förvägrades sin människovärdighet av ett samhälle som inte såg honom som människa. Hans uppror utmålades som religiös fanatism – men historien visade att det var systemet som var vansinnigt, inte han.

På samma sätt föddes våldet i Gaza inte ur hat – utan ur ett konsekvent nekande av lagligt skydd. De unga män som i oktober 2023 trängde ut genom murarna var födda i ett rättslöst fängelse – utan flyktvägar, utan framtid. Deras död var väntad. Deras motstånd förbjudet.

När Sverige då inte bara vägrar att ingripa, utan dessutom benämner deras handlingar som ”terrorism”, är det inte rättvisa man försvarar – utan status quo. Det är inte rättsstat. Det är moraliskt sammanbrott.

I detta rättsläge upplöses de juridiska strukturerna – och nya, desperata logiker tar vid:

\begin{quote}
Om skydd enligt internationell humanitär rätt konsekvent nekas oss, återstår endast den arkaiska rättsformen: \textit{öga för öga, tand för tand} – som en sista metod för avskräckning. Då vilar de oskyldigas blod inte på oss som slog tillbaka, utan på de ansvariga för systemets sammanbrott.
\end{quote}

\begin{quote}
Om lagsystemets väktare – dess guardians – förvägrar oss människostatus, om vi behandlas som djur, som restposter utan rättslig existens – varför skulle vi då låtsas följa mänsklighetens kodex? När vår mänsklighet förnekas, återstår bara djungelns lag – och den som behandlas som ett djur kommer till slut att slå som ett djur. Ingen kan kräva att vi följer lagar som ingen upprätthåller för oss.
\end{quote}

Detta är inte ett försvar av dödande. Det är ett åtal mot den värld som tvingar fram det.




\subsection{Pedagogisk liknelse – när uppstår skuld hos systemets väktare?}
\%addcontentsline{toc}{subsection}{Pedagogisk liknelse – när uppstår skuld hos systemets väktare?}

\begin{quote}
En polis iakttar hur grupp A, med hänvisning till en gudagiven rätt, fördriver grupp B från ett land där B bott i generationer. Grupp A hävdar att B är illegitima inkräktare. Grupp B vädjar om beskydd och påstår sig utsättas för etnisk rensning.
\end{quote}

\textbf{1. Den första vädjan:}
\begin{quote}
En representant för grupp B ber polismannen om hjälp. Polisen lyssnar – men ingriper inte. Han säger att konflikten är ”komplex” och att det inte är hans uppgift att ta ställning.
\end{quote}

\textit{Här uppstår den första formen av ansvar: underlåtenhet att skydda den som söker rättsskydd. Att inte ingripa trots skyldighet är ett brott mot skyddsprincipen – en grundpelare i såväl humanitär rätt som polisärt uppdrag.}
\lagrum{Jfr artikel 1 i Genèvekonventionen\footnote{\url{https://ihl-databases.icrc.org/en/ihl-treaties/gciv-1949/article-1}} och artikel 41.1 i ILC Articles\footnote{\url{https://legal.un.org/ilc/texts/instruments/english/draft_articles/9_6_2001.pdf}}}

\textbf{2. Det aktiva ställningstagandet:}
\begin{quote}
Snart slutar polisen inte bara att ignorera grupp B:s situation, utan inleder dessutom avtal och samarbete med grupp A – som fortsatt fördriver och dödar medlemmar ur grupp B.
\end{quote}

\textit{Här förstärks ansvaret: från passivitet till aktivt medansvar. Det är att bidra till rättsbrottet – både i folkrättslig mening (ILC:s artikel 16) och enligt Genèvekonventionernas skyldighet att ”respektera och säkerställa” att konventionerna efterlevs.}
\lagrum{Se ILC Articles, artikel 16\footnote{\url{https://legal.un.org/ilc/texts/instruments/english/draft_articles/9_6_2001.pdf}} och sedvanerättsregel 139 i ICRC:s sammanställning\footnote{\url{https://ihl-databases.icrc.org/en/customary-ihl/v1/rule139}}}

\textbf{3. Den institutionaliserade tystnaden:}
\begin{quote}
Under åren fortsätter övergreppen. Grupp B förlorar mark, försörjning och tillgång till rättsmedel. Trots återkommande larm hänvisar polisen till ”neutralitet” – men säger inget, gör inget.
\end{quote}

\textit{Här blir tystnaden ett strukturellt svek. Väktaren som inte skyddar offret utan upprätthåller gärningsmannens straffrihet sviker hela rättsordningen. Det är inte neutralitet – det är medverkan genom likgiltighet.}
\lagrum{Se artikel 40 och 41.2 i ILC Articles\footnote{\url{https://legal.un.org/ilc/texts/instruments/english/draft_articles/9_6_2001.pdf}}}

\textbf{4. Det desperata svaret:}
\begin{quote}
Till sist slår vissa ur grupp B tillbaka. De angriper civila från grupp A – i ett försök att spegla det lidande de själva utsatts för och tvinga fram internationell uppmärksamhet.
\end{quote}

\textit{Detta är inte lagligt. Men det är begripligt. Det är desperation i frånvaro av rättvisa. Det är rättsstatens sammanbrott i realtid.}
\lagrum{Se proportionalitetsprincipen i IHL och artikel 1(4) i Tilläggsprotokoll I\footnote{\url{https://ihl-databases.icrc.org/en/ihl-treaties/api-1977/article-1}}}

\textbf{5. Den slutgiltiga domen:}
\begin{quote}
Då först reagerar polisen. Han fördömer våldet från grupp B, kallar dem ”terrorister” och kräver att de ställs inför rätta. Han nämner aldrig sin egen roll.
\end{quote}

\textit{Här uppstår dubbel skuld:}
\begin{itemize}
    \item \textbf{Skuld gentemot grupp B:} för att han förvägrade dem skydd, tillät deras fördrivning, och förnekade dem status som skyddsberättigade.
    \lagrum{Brott mot skyldigheten att förhindra folkrättsbrott enligt Genèvekonventionerna\footnote{\url{https://ihl-databases.icrc.org/en/ihl-treaties/gciv-1949/article-1}} och artikel 16 i ILC Articles\footnote{\url{https://legal.un.org/ilc/texts/instruments/english/draft_articles/9_6_2001.pdf}}}

    \item \textbf{Skuld gentemot grupp A:} eftersom hans vägran att upprätthålla lag och rätt bidrog till att våld föddes ur desperation – och riktades mot oskyldiga även i grupp A.
    \lagrum{Indirekt ansvar enligt artikel 16 och 41 i ILC Articles – medverkansansvar för följdskador\footnote{\url{https://legal.un.org/ilc/texts/instruments/english/draft_articles/9_6_2001.pdf}}}
\end{itemize}

\textbf{Detta är Sveriges roll.}\\
Inte som neutral observatör. Inte som fredsbevarare.\\
Utan som den polis som vägrade skydda offret, slöt avtal med förövaren – och till sist fördömde det motstånd han själv möjliggjorde.

\vspace{1em}
\noindent\textit{Den rättsliga analysen är därmed fullbordad. Vad följer är en etisk och retorisk kommentar – inte som ersättning, utan som illustration av den rättsliga nödvändigheten.}



\subsection{Om gisslan och administrativt förvar}
%\addcontentsline{toc}{subsection}{Om gisslan och administrativt förvar}

Regeringen:\textit{“The terrorist organisation Hamas bears heavy responsibility for the current situation. The hostages must be released – unconditionally and immediately.”}\\

Regeringen kräver att Hamas ovillkorligen friger gisslan – ett legitimt krav enligt humanitär rätt. Men varför riktas inga motsvarande krav mot Israel, som systematiskt och i strid med folkrätten berövar tusentals palestinier friheten utan vare sig åtal, rättegång eller fastställd tidsgräns?

\textit{Så kallat “administrativt förvar” innebär att en individ frihetsberövas enbart på grundval av säkerhetstjänstens påståenden – utan insyn, utan beviskrav, utan rättslig prövning, utan möjlighet till försvar. Det är ett rättsligt undantagstillstånd som kan förlängas i månader, år – i praktiken utan slut.}

Barn hämtas nattetid av beväpnade soldater. Föräldrar får inga besked. Minderåriga grips, isoleras och förhörs – ibland under tortyrliknande former. Tusentals palestinier har hållits frihetsberövade under dessa förhållanden – utan att någonsin delges misstanke om brott.

\textit{Den israeliska människorättsorganisationen B’Tselem har i åratal dokumenterat denna praxis, som bryter mot både fjärde Genèvekonventionen och FN:s konvention om medborgerliga och politiska rättigheter.}\footnote{\url{https://www.btselem.org/administrative_detention}}

Att regeringen väljer att offentligt fördöma Hamas gisslantagning – men tiger om Israels institutionella massinternering av civila, inklusive barn – utgör en folkrättslig medverkan genom:

\begin{itemize}
  \item \textbf{Konkludent handlande:} Staten Sverige erkänner indirekt det ena brottet (gisslantagning) som avvikelse från rättsordningen, men det andra (administrativt förvar) som normalitet. Det är ett rättsligt ställningstagande genom underlåtenhet, i strid med \lagrum{artikel 1 i fjärde Genèvekonventionen}\footnote{\url{https://ihl-databases.icrc.org/en/ihl-treaties/gciv-1949/article-1}}.

  \item \textbf{Underlåtenhet att förhindra eller avslöja brott:} En stat som har insyn och möjlighet att agera, men inte gör det, bryter mot skyldigheten att ”respektera och säkerställa” konventionens efterlevnad (\lagrum{GC IV art. 1}) och \lagrum{ICCPR artikel 2(1)}\footnote{\url{https://www.ohchr.org/en/instruments-mechanisms/instruments/international-covenant-civil-and-political-rights}}, särskilt i relation till \lagrum{ICCPR artiklarna 9 och 14}\footnote{\url{https://www.ohchr.org/en/instruments-mechanisms/instruments/international-covenant-civil-and-political-rights}}.

  \item \textbf{Stämplingsliknande ansvar:} Genom att den ockuperande maktens brott systematiskt ursäktas, tonas ner eller förtigs, och endast motståndaren fördöms, legitimeras brotten. Detta bryter mot \lagrum{artikel 16 i ILC:s utkast till statsansvar}\footnote{\url{https://legal.un.org/ilc/texts/instruments/english/draft_articles/9_6_2001.pdf}} samt folkrättens tvingande normer (\textit{jus cogens}), där förbudet mot godtyckligt frihetsberövande och tortyr är absoluta och icke-derogabla.
\end{itemize}

\textit{Det är inte brottsligt att kräva frigivning av gisslan.}\\
\textit{Men det är rättsvidrigt att endast kräva det från den ena parten.}\\
\textit{Att tyst legitimera den andres systematiska och olagliga massfångenskap – det är medverkan.}


\subsection{Slutsats om Hamas juridiska status}

Hamas utgör sedan 2006 den folkvalda regeringen i Gaza och åtnjuter, enligt flera internationella
källor, fortsatt bred folklig förankring bland palestinier. Denna faktiska kontroll och lokala legitimitet
innebär att Hamas – oavsett västerländska klassificeringar – i folkrättslig mening är en icke-statlig väpnad
aktör med territoriell kontroll och därmed också folkrättsliga rättigheter och skyldigheter.

Det är därför juridiskt problematiskt att entydigt tillskriva Hamas fullt ansvar för alla civila dödsoffer
i Israel den 7 oktober 2023. Hamas politiska ledarskap, genom Moussa Abu Marzouk, har förnekat att kvinnor,
barn och civila utgjort mål för attacken och hänvisat till militära order från Qassam-brigadernas ledare
Mohamed el-Deif om att civila skulle skonas.\footnote{\url{https://www.the-star.co.ke/counties/nyanza/2023-11-07-hamas-leader-denies-killing-of-civilians-in-israel}} Marzouk hävdar att endast soldater och
reservister attackerats, samt att det bevismaterial som lagts fram i västliga medier inte nödvändigtvis
kan tillskrivas Hamas direkt, då flera fraktioner varit verksamma i Gaza vid tillfället.

Det bör även framhållas att väpnade grupper såsom Palestinska Islamiska Jihad verkar parallellt med Hamas,
och att vissa hämndaktioner kan ha genomförts av odisciplinerade enskilda individer, lokala miliser eller kriminella
nätverk i kaosets inledningsskede. Till detta kommer att Israel sedan tidigare erkänt sin tillämpning av det
s.k. Hannibal-direktivet – en militär doktrin som tillåter beskjutning av områden där egna soldater tagits som
gisslan, även om det riskerar deras död – vilket i sig kan ha lett till israeliska civila dödsoffer.

Det är inte folkrättsligt rimligt att kräva av en belägrad och ockuperad civilbefolkning – instängd bakom en
olaglig mur enligt ICJ:s rådgivande yttrande 2004 – att förutse, kontrollera eller isolera alla fraktioner
som agerar i ett plötsligt maktvakuum efter att israeliska befästningar slagits ut. Ett sådant krav skulle
etablera en doktrin där endast väpnat motstånd med perfektion i intern disciplin kan erkännas, vilket i praktiken
utesluter samtliga befrielserörelser ur den juridiska rättighetssfären. Att hävda detta i fallet Gaza, under
årtionden av blockad och övergrepp, skulle vara att tillämpa ett mått av ansvar som aldrig riktas mot de stater
som utövar ursprungligt våld.


Flera internationellt profilerade Palestinaaktivister som tidigare varit kritiska mot Hamas har sedan
oktober 2023 reviderat sina uppfattningar. Den kanadensiske advokaten Dimitri Lascaris, som driver bloggen
\textit{Reason2Resist}, är ett exempel på detta skifte.\footnote{\url{https://www.youtube.com/@reason2resist}}


I den amerikanska kontexten är det i praktiken omöjligt att offentligt ifrågasätta terroristklassificeringen
av Hamas utan att riskera social eller professionell utstötning. Istället uttrycks en växande förståelse för
hur israeliskt våld och blockadpolitik framkallar väpnat motstånd – inte som rättfärdigande, men som förklaring.
Bland dem som formulerat denna linje finns inte bara analytiker som Scott Ritter, utan också inflytelserika
högerprofiler som Elon Musk, Tucker Carlson och Candace Owens, vilka samstämmigt betonat att det är Israels
agerande som “skapar terrorister”.

Denna hållning – att se våldsutbrottet som en konsekvens av långvarig förnedring, instängning och statligt våld –
innebär inte nödvändigtvis stöd för Hamas metoder, men utmanar narrativet om ensidig ondska. Det faktum att denna
tolkning nu förs fram av personer inom USA:s konservativa mainstream visar att Israels agerande inte längre bara
kritiseras från vänster, utan även från ideologiskt oväntade håll.

Den amerikanske militäre analytikern och tidigare FN-vapeninspektören Scott Ritter intar en mer
oförsonlig hållning än många andra västliga Palestina-sympatisörer. I sina analyser har han riktat
skarp kritik mot vad han uppfattar som moraliskt hyckleri: individer som säger sig stödja det
palestinska folkets rättigheter, men inte vågar uttala stöd för Hamas väpnade kamp.

Ritter betraktar Hamas som en legitim motståndsrörelse mot ockupation och kolonial dominans,
och menar att det är inkonsekvent att erkänna palestiniernas rätt till självförsvar men samtidigt
förkasta de medel som i praktiken står dem till buds. Enligt Ritter är det inte Israel som
behöver förstås, utan det västerländska behovet av att kategorisera palestinskt våld som "terrorism"
– ett begrepp han menar används strategiskt för att avväpna all motståndsrätt.
För Ritter är det inte nog att förstå Hamas – man måste erkänna rörelsens väpnade
motstånd som legitimt, annars är man enligt honom delaktig i att förneka palestiniernas rätt till kamp.

Den brittiske politikern George Galloway har i detta sammanhang påpekat att det faktum att vissa handlingar
inom en befrielserörelse utgör terror inte innebär att hela rörelsen saknar folkrättslig relevans. Samma
principer som tillämpades på ANC under apartheidregimen måste, i konsekvensens namn, även kunna tillämpas
på Hamas.

Som vi visat ovan, är det folkrättsligt oacceptabelt för en regering att utan granskning beteckna väpnade
motståndsgrupper som "terrorister" samtidigt som man själv undandrar Israel varje form av ansvarsutkrävande
för krigsförbrytelser. Att beteckna samtliga motståndshandlingar från en ockuperad befolkning som terrorism,
utan att erkänna den grundläggande asymmetrin i våldsanvändningen, utgör i sig ett brott mot neutralitetsprincipen
och urholkar den internationella rättsordningen.

\subsection*{Slutlig juridisk bedömning: statens ansvar för båda sidors brott}

Även om enskilda palestinier den 7 oktober begått illdåd mot civila, även om Hamas skulle visa sig skyldiga till övergrepp, förändras inte det centrala folkrättsliga ansvaret: den svenska regeringen – liksom andra västliga regeringar – har under flera decennier vägrat att påföra Israel några som helst rättsliga, ekonomiska eller diplomatiska konsekvenser för landets systematiska brott mot internationell rätt.

Genom att konsekvent undandra Israel ansvar, har man erbjudit en **politisk och moralisk immunitet** som gjort det möjligt för våldet att fortgå. Genom att samtidigt benämna palestinskt motstånd som ”terrorism”, har man fråntagit det ockuperade folket både sin rättsliga status och sina legitima uttrycksformer.

Detta är inte passivitet. Det är **konkludent medverkan** i en rättsordnings kollaps – där staten ger stöd åt den ena parten, vägrar utreda dess brott, och fördömer det motstånd som uppstår.

I rättslig mening innebär detta att den svenska staten bär ansvar **för döda israeler** – eftersom den möjliggjort ett system som provocerat fram våld – och **för döda palestinier**, eftersom den hjälpt till att upprätthålla den straffrihet som gjort det möjligt att fortsätta döda dem.

Detta ansvar är inte sekundärt. Det är inte indirekt. Det är **djupare** – eftersom staten vet vad som sker, men väljer att upprätthålla tystnaden. Det är därför staten – inte Hamas, inte ens Israel – måste hållas till svars för att ha satt sig själv utanför den rättsordning den påstår sig försvara.


