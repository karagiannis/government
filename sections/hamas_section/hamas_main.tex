


\section{Om Hamas och 7 oktober}
% Ge en nykter redogörelse för händelserna den 7 oktober.
% Skilj på vad Hamas faktiskt gjort vs. vad som är propaganda.
% Visa att många dödsfall orsakades av IDF enligt israeliska källor.


\subsection{Vad har motbevisats av FN, rättsmedicin och åklagare?}
%\addcontentsline{toc}{subsection}{Vad har motbevisats av FN, rättsmedicin och åklagare?}
Exempelvis har de påstådda massvåldtäkterna inte kunnat styrkas av vare sig israeliska åklagare, FN:s kommissioner eller oberoende rättsorgan. Den enda FN-rapport som hittills behandlat frågan – ett 17-sidigt dokument från Pramila Patten, generalsekreterarens särskilda sändebud för sexuellt våld i konflikter – klargör att uppdraget varken var \enquote{utredande till sin natur} eller byggde på direkt bevisning. Rapporten vilar nästan uteslutande på material tillhandahållet av israeliska myndigheter och medger uttryckligen att inga konkreta digitala eller forensiska bevis på våldtäkt kunde identifieras bland de över 5 000 granskade bilderna och 50 timmarna av videomaterial. Trots detta drar rapporten rättsligt laddade slutsatser om \enquote{rimlig grund att anta} att våldtäkt förekommit – utan att specificera hur många fall som avses. Professor Norman Finkelstein har i en detaljerad analys visat hur rapporten, trots sin icke-utredande karaktär och brist på substantiell bevisning, ändå bidragit till att legitimera omfattande demonisering av det palestinska motståndet.\footnote{Finkelstein, N. (2024). \textit{Pramila Patten’s Rape Fantasies: A Critical Analysis of the UN Report on Sexual Violence during the 7 October Attack}. In Gaza, 11 mars 2024.}

I en intervju med \textit{Yedioth Ahronoth} i januari 2025 medgav den israeliska chefsåklagaren Moran Gez att det – 15 månader efter händelserna – fortfarande inte finns en enda målsägande i något av fallen.\footnote{\url{https://electronicintifada.net/content/israel-still-cant-find-any-7-october-rape-victims-prosecutor-admits/39601}} FN har i två separata rapporter bekräftat att man, trots tillgång till omfattande bild- och videomaterial, inte funnit några konkreta tecken på våldtäkt eller sexuella övergrepp.\footnote{UN Human Rights Council, A/HRC/55/73 och A/HRC/55/75 (mars 2024).}
Detta innebär att ingen person har trätt fram som målsägande och därmed heller inte gjort anspråk på brottsofferersättning – något som är ytterst ovanligt vid grova våldsbrott om dessa faktiskt har ägt rum.

Samtidigt är det väldokumenterat att israeliska styrkor regelbundet utsätter palestinska fångar – inklusive kvinnor – för sexuellt våld, förnedring och tortyr, bland annat i det ökända Sde Teiman-lägret. Den israeliska journalisten David Sheen har redogjort för dessa övergrepp i flera rapporter, däribland vittnesmål om gruppvåldtäkter på palestinska män.\footnote{\url{https://mondoweiss.net/2024/02/new-reports-confirm-months-of-israeli-torture-abuse-and-sexual-violence-against-palestinian-prisoners/}}

Mot denna bakgrund framstår den svenska regeringens ställningstagande som särskilt anmärkningsvärt. Istället för att agera med försiktighet och efterfråga oberoende utredningar valde regeringen att kraftfullt bekräfta obestyrkta påståenden och fördöma palestinskt motstånd – samtidigt som man förteg Israels redan då väldokumenterade övergrepp.

Det är inte uteslutet att dessa atrocity-berättelser utnyttjats strategiskt – för att motivera fortsatt stöd till Israel, neutralisera kritik och ge opinionen en förevändning att bortse från folkrättsbrott. I detta klimat kunde den svenska regeringen den 27 oktober 2023 – alltså dagen för Israels markinvasion – underteckna ett militärt samarbetsavtal med det israeliska försvarsföretaget Elbit Systems.

Genom att i detta läge köpa in krigsmateriel från Elbit, med svenska skattemedel, har regeringen aktivt medverkat till att legitimera folkrättsbrott och ge ekonomiskt stöd till en ockupationsmakt under pågående folkmord.

\subsection*{Vem hittade på det? Militära talespersoner, ZAKA, United Hatzalah och David Ben Zion}
\addcontentsline{toc}{subsection}{Vem hittade på det? ZAKA, United Hatzalah och David Ben Zion}
\subsubsection*{Upphovsmannen till påståendet om halshuggna spädbarn}
\addcontentsline{toc}{subsubsection}{Upphovsmannen till påståendet om halshuggna spädbarn}

Ett av de mest spridda påståendena efter 7 oktober var att Hamas “halshögg 40 spädbarn”. Detta återgavs av bland andra USA:s president Joe Biden, Israels premiärminister Benjamin Netanyahu och den israeliska utrikesministern. Påståendet visade sig dock sakna bevis och har senare dementerats av såväl israeliska som internationella medier.

The Grayzone kunde identifiera den ursprungliga källan till detta påstående: David Ben Zion, en fanatisk bosättarledare och biträdande befälhavare i Israels armé. I en intervju den 10 oktober med den israeliska tv-kanalen i24 sade han: \textit{“De skar huvuden av barn. De skar huvuden av kvinnor.”} Ben Zion beskrev palestinierna som “djur” utan hjärta och publicerade timmar senare en video på sig själv där han log brett i byn Kfar Aza – platsen för den påstådda massakern. 

Ben Zion har tidigare uppmanat till att “utplåna” palestinska byar såsom Huwara, uttryckt stöd för deportation av palestinier och spridit rasistiska uttalanden om deras “barbariska DNA”. Han är kopplad till den apokalyptiska Tempelrörelsen som vill riva al-Aqsa-moskén och bygga ett tredje tempel. Under tidigare militära operationer mot Gaza har han öppet uttryckt stöd för total förintelse av området.

Det faktum att denna individ – med dokumenterad extremism och rasideologisk agenda – utgör ursprungskällan till ett av de mest spridda påståendena om Hamas övergrepp, underminerar ytterligare trovärdigheten i Israels atrocitynarrativ. Ändå har svenska och västerländska medier inte nämnt detta faktum, trots att Blumenthals granskning varit tillgänglig sedan oktober 2023.\footnote{\url{https://thegrayzone.com/2023/10/11/beheaded-israeli-babies-settler-wipe-out-palestinian/}}

\subsubsection*{Upphovsmännen till de värsta atrocityhistorierna}
%\addcontentsline{toc}{subsubsection}{Upphovsmännen till de värsta atrocityhistorierna}

En av de mest genomgripande granskningarna av atrocitynarrativet efter 7 oktober har genomförts av journalisten Max Blumenthal i The Grayzone.\footnote{\url{https://thegrayzone.com/2023/12/06/scandal-israeli-october-7-fabrications/}} Artikeln visar hur den israeliska organisationen ZAKA – utan medicinsk legitimation – stått bakom flera centrala påståenden, bland annat om halshuggna spädbarn, gravida kvinnor med utskurna foster, och barn som bränts i ugnar. Samtliga berättelser saknar belägg i form av kroppar, dokumentation eller vittnesmål.  

De mest groteska historierna har inte bara spridits av ZAKA:s ledarfigurer Yossi Landau och Simcha Dizingoff, utan även vidareförmedlats av amerikanska och israeliska toppolitiker. Bland annat återgav president Joe Biden och utrikesminister Antony Blinken en berättelse om en familj med två barn (6 och 8 år gamla) som enligt ZAKA skulle ha blivit torterade och mördade inför ögonen på varandra. Ingen död med dessa åldrar är registrerad i det aktuella kibbutzet (Beeri), och inga kroppar har återfunnits i det tillstånd som beskrivs.

Samtidigt har rivaliserande organisationer som United Hatzalah hittat på ännu mer extrema påståenden, såsom att ett spädbarn bakats i en ugn. Dessa berättelser har visat sig vara falska.


\subsection{Vad har bekräftats av Israeliska medborgare?}
\addcontentsline{toc}{subsection}erade på gissningar, missuppfattningar eller i vissa fall – enligt israeliska medier – direkt ekonomiskt motiverad bluff för att driva in donationer från utländska sponsorer.

Avslöjandena visar hur en mycket liten grupp individer har lyckats styra det internationella narrativet, trots att bevisen saknas eller motsägs av dödsregister, vittnesmål, rättsmedicinska uppgifter och israeler på plats.

\subsubsection*{Falska vittnesmål som militärstrategi: fallet med de påstått hängda bebisarna}
\addcontentsline{toc}{subsection}{Falska vittnesmål som militärstrategi: fallet med de påstått hängda bebisarna}

Ett av de mest groteska och spridda propagandapåståendena efter den 7 oktober var att Hamas skulle ha “hängt bebisar på en tvättlina”. Den israeliske journalisten Ishay Cohen publicerade ett videoklipp där han intervjuar en IDF-officer som hävdar detta. Kort därefter raderade Cohen videon efter massiv kritik – men först efter att den fått hundratusentals visningar via kontot “Mossad Commentary”.

Cohen erkände senare att intervjun förmedlats av israeliska arméns talesperson, att talespersonens representant var närvarande under hela inspelningen, och att syftet var att främja Israels “hasbara”, det vill säga propagandainsatser mot omvärlden.\footnote{\url{https://www.uncaptured.media/p/israeli-journalist-retracts-babies} Dan Cohen, “Israeli Journalist Retracts ‘Babies Hung on a Clothesline’”, *Uncaptured Media*, 30 november 2023.}

Officeren i fråga var Yaron Buskila, en reservöverste som tidigare varit verksam i IDF:s informationsavdelning. Buskila uppgav olika versioner i olika medier: i en intervju med Epoch Times hävdade han att han hört historien från en rabbin – medan han i intervjun med Cohen påstod sig ha sett det själv. Den senare versionen förnekades sedermera av Cohen, som även medgav att hans publicering skedde utan kontroll av uppgifternas riktighet: 

\begin{quote}
    “Jag medger att jag inte trodde det var nödvändigt att kontrollera sanningshalten i en berättelse från en överstelöjtnant, operationschef i Gaza-divisionen, och dessutom ackompanjerad av en talesperson från IDF.”
\end{quote}

Den israeliska militären har aldrig dementerat påståendet offentligt, och vissa högerextrema konton fortsätter att sprida det. Det är ett exempel på ett återkommande mönster: makaber propaganda med obestyrkta eller fabricerade påståenden som sprids snabbt, får global spridning, och därefter – när de faller – lämnar ett bestående intryck hos mottagaren trots att de tillbakavisats.

Detta påstående ansluter sig till samma propagandaform som “40 halshuggna spädbarn” – ett påstående som förnekats av både IDF och Vita huset, men som ändå återges i västvärldens ledande nyhetsförmedling.\footnote{Se t.ex. Max Blumenthal och Alexander Rubinstein, “Source of dubious ‘beheaded babies’ claim is Israeli settler leader who incited riots to ‘wipe out’ Palestinian village”, *The Grayzone*, 11 oktober 2023.}

\textbf{Sammanfattningsvis:} De falska påståendena om “hängda” eller “halshuggna” spädbarn är inte slumpmässiga misstag – de är delar av ett mönster. I detta mönster spelar den israeliska arméns talesperson en aktiv roll som innehållsproducent, inte enbart som verifierare. Propagandan bygger på chock, avhumanisering och en vägran att erkänna att informationen kan vara manipulerad, även av egna myndigheter.


\subsection*{Vad har tillbakavisats av civila israeliska medborgare?}
\addcontentsline{toc}{subsection}{Vad har tillbakavisats av israeliska medborgare?}

\subsubsection*{Yasmin Porat: Vi dödades av våra egna}
\addcontentsline{toc}{subsubsection}{Yasmin Porat: Vi dödades av våra egna}
Ett centralt vittnesmål kommer från Yasmin Porat\footnote{\url{https://electronicintifada.net/content/israeli-forces-shot-their-own-civilians-kibbutz-survivor-says/38861}} Ali Abunimah \& David Sheen, “Israeli forces shot their own civilians, kibbutz survivor says”, *The Electronic Intifada*, 16 oktober 2023.
, en israelisk kvinna som överlevde attacken på Kibbutz Be’eri. I israelisk statsradio berättar hon att hon och andra civila hölls som gisslan av Hamas i flera timmar, men att dessa behandlade dem “mycket humant”, gav dem vatten, försökte lugna dem, och klargjorde att de inte ämnade döda dem utan ta dem till Gaza.

Porat beskriver hur israeliska säkerhetsstyrkor anlände efter flera timmar, öppnade eld “med tusentals kulor och två tankskal” och dödade “alla” – inklusive de israeliska civila som hölls som gisslan. Hon säger uttryckligen att “de dödade alla, även gisslan, i mycket, mycket tung korseld.”

Porat intervjuades i flera medier – inklusive Kan, Maariv och Channel 12 – men hennes mest explosiva uttalanden klipptes bort i senare versioner och fanns inte kvar i publicerad arkivversion. De tillgängliggjordes först efter påtryckningar. Hon säger: “Jag är arg på staten, jag är arg på armén. Kibbutzen var övergiven i tio timmar.” 

Detta stödjer teorin att delar av förlustsiffrorna den 7 oktober orsakades av israelisk eldgivning, vilket ytterligare underminerar bilden av en enbart ensidig massaker. Vittnesmålet pekar även mot att Hannibal-doktrinen tillämpats mot civila.

\subsubsection*{Erez Tidhar: “En IDF-helikopter sköt in i kibbutzen”}
\addcontentsline{toc}{subsubsection}{Erez Tidhar: “En IDF-helikopter sköt in i kibbutzen”}

Erez Tidhar, en israelisk militärveteran och frivillig i Eitam-enheten under evakueringsinsatserna den 7 oktober, har i en intervju med den israeliska public service-kanalen Kann berättat att han såg en israelisk Apachehelikopter skjuta rakt in i Kibbutz Be’eri:

\begin{quote}
“Varje minut kommer en missil ner över dig, varje minut. Och plötsligt ser du en missil från en helikopter som skjuter in i kibbutzen. Du säger till dig själv, ‘Jag fattar inte. En IDF-helikopter skjuter in i en israelisk kibbutz.’ Och sen ser du en stridsvagn köra genom gatorna i kibbutzen, den flankerar kanonen och avfyrar en granat in i ett hus. Sånt kan man inte begripa.”\footnote{\url{https://www.uncaptured.media/p/israeli-volunteer-apache-helicopter}}
\end{quote}

Uttalandet är det första dokumenterade ögonvittnesmålet om raketbeskjutning från israeliskt stridsflyg in i ett israeliskt samhälle. Tidigare har Ha’aretz rapporterat om att en helikopter “uppenbarligen också träffade festivaldeltagare”, och Yedioth Ahronoth har avslöjat att 28 militärhelikoptrar öppnade eld mot mål i Israel under de första fyra timmarna av attacken. Piloterna uppges ha haft “stora svårigheter att skilja mellan terrorister, soldater och civila”.

Israels militär erkänner nu att omfattande vänskapseld förekom, men vägrar utreda dessa händelser med hänvisning till att omfattningen är för komplex och att en utredning skulle vara “moraliskt olämplig”.

\subsubsection*{Danielle Aloni och andra frigivna gisslan: “Vi behandlades humant”}
\addcontentsline{toc}{subsubsection}{Danielle Aloni och andra frigivna gisslan: “Vi behandlades humant”}

Flera frigivna israeler som hållits som gisslan av Hamas har beskrivit en oväntat human behandling under sin fångenskap. Mest uppmärksammat är uttalandet från Danielle Aloni, som i ett brev till sina fångvaktare tackade för deras “onaturliga mänsklighet” och “vänlighet trots förlusterna ni själva lidit”.\footnote{\url{https://thegrayzone.com/2023/11/27/israeli-tank-orders-fire-kibbutz/}}

Alonis vittnesmål har inte fått bred medial spridning i Israel, och regeringen har enligt flera rapporter hindrat återvändande gisslan från att tala fritt med medier. Israelen Gadi Peretz har offentligt vittnat om hur han efter hemkomsten blivit instruerad att inte berätta detaljerat om sin tid i fångenskap, särskilt inte om eventuella humanitära aspekter. Liknande rapporter har framkommit från andra frigivna.

Detta mönster tyder på att staten aktivt försökt kontrollera narrativet och tysta de röster som motsäger bilden av Hamas som obetingat brutal och sadistisk. I vissa fall har detta inkluderat direkt censur eller psykologiska utvärderingar innan de frigivna tillåtits återförenas med sin familj eller intervjuas offentligt.

\begin{itemize}
\item Video:Agam Goldstein-Almog\footnote{\url{https://x.com/LasseKaragiann5/status/1930654442912723269}}
\item Audio: Yasmin Porat\footnote{\url{https://x.com/LasseKaragiann5/status/1930656123540939100}}
\item Video:Fru Bibas med barn\footnote{\url{https://x.com/LasseKaragiann5/status/1930648270105432510}}
\item Video:Israelisk media:\enquote{Gisslan behandlades väl}\footnote{\url{https://x.com/LasseKaragiann5/status/1930629355568337170}}
\item Video:Mia Shem hävdade först att hon behandlats väl, därefter att hon inte bahandlats väl, att hon vaknat upp från narkos efter reparation, men senare att hon opererats utan narkos av veterinär\footnote{\url{https://x.com/LasseKaragiann5/status/1930628167342952537}}
\item Video: Avital Aldajem\footnote{\url{https://x.com/LasseKaragiann5/status/1930625119585534222}}
\item En video på tillfångatagandet av den kvinnliga soldaten Naama Levy spreds med påståenden från propagandister om att smuts eller fläckar på hennes byxor utgjorde absolut bevis för att hon blivit våldtagen\footnote{\url{https://x.com/LasseKaragiann5/status/1931025103992594943}}. Detta visade sig vara ännu en falsk anklagelse bland många. I en senare video syns hur hon och hennes kollegor släpps fria, och hur de inför Röda Korset gensvarar på Gazabornas stöd med leenden och vinkningar.\footnote{\url{https://x.com/LasseKaragiann5/status/1931015171381629296}}

\item Video: Gisslan i gott skick – inga tecken på misshandel eller rädsla.\footnote{\url{https://x.com/LasseKaragiann5/status/1930639636600131836}}
\item Video: Flickan Mia Leimberg med valpen från videon ovan\footnote{\url{https://x.com/MiddleEastMnt/status/1732547191204626473}}
\item Video: Gisslan och Hamas soldater vinkar varann avsked.\footnote{\url{https://x.com/LasseKaragiann5/status/1930638815787376920}} \footnote{\url{https://x.com/LasseKaragiann5/status/1930639247473619452}}\footnote{\url{https://x.com/LasseKaragiann5/status/1930637597597978909}}

\item Video: Gisslan high-five med Hamas soldater vid hemfärd\footnote{\url{https://x.com/LasseKaragiann5/status/1930631411238666463}}
\item Video: Yocheved Lifshitz \footnote{\url{https://x.com/LasseKaragiann5/status/1930596740261917095}}
\item Video: Maya Regev \footnote{\url{https://x.com/LasseKaragiann5/status/1930582894772158744}}
\item Video: En kompilering av kvinnliga vittnesbörd\footnote{\url{https://x.com/LasseKaragiann5/status/1930597653861007793}}
\end{itemize}

\textbf{Slutsats:} Vittnesmål från överlevare och frigivna gisslan motsäger i flera avseenden den officiella israeliska berättelsen. Tvärtom beskriver flera en förhållandevis human behandling, vilket direkt underminerar centrala element i det atrocity-narrativ som etablerades omedelbart efter den 7 oktober. I en global mediemiljö där råvideor snabbt kan spridas, verifieras och analyseras i detalj, har Israels informationsstrategi inte bara förlorat sin genomslagskraft – den har i många fall resulterat i ett fullständigt sammanbrott i trovärdighet.

Före detta sympatisörer uttrycker nu öppet antisionistiska ståndpunkter, och det offentliga samtalet domineras i allt högre grad av satir, memes och virala uttryck som anklagar israeliska talespersoner för att medvetet ljuga. Deras systematiska demonisering av palestinier upplevs av många som alltför grov för att vara trovärdig, och beskrivs ofta i termer av "gaslighting" – uppenbart falska påståenden som inte bara undergräver avsändarens trovärdighet, utan upplevs som en direkt förolämpning mot mottagarens intelligens.



\subsection*{Vad har bekräftats av Israeliska medborgare?}
\addcontentsline{toc}{subsection}{Vad har bekräftats av Israeliska medborgare?}

\subsubsection*{Israels egna styrkor orsakade civila dödsfall den 7 oktober}
\addcontentsline{toc}{subsubsection}{Israels egna styrkor orsakade civila dödsfall den 7 oktober}
Flera israeliska källor – inklusive överstelöjtnanten Nof Erez i intervju med Haaretz\footnote{\url{https://electronicintifada.net/blogs/asa-winstanley/we-blew-israeli-houses-7-october-says-israeli-colonel} Asa Winstanley, “We blew up Israeli houses on 7 October, says Israeli colonel”, *The Electronic Intifada*, 5 december 2023.}
 – har bekräftat att israeliska attackhelikoptrar och drönare under morgonen den 7 oktober öppnade eld mot mål inne i israeliska bosättningar, trots att man visste att det kunde innebära dödligt våld mot egna civila. Syftet var att förhindra kidnappningar enligt den så kallade Hannibal-doktrinen, ett militärt protokoll som godtar att israeliska gisslan dödas för att undvika fångutväxling.

Erez beskriver i detalj hur arméns kommandostruktur kollapsade och hur lokala miliser gav direktiv till helikopterpiloter via mobiltelefon om att bomba specifika bostäder där misstänkta palestinier befann sig – även om israeliska civila hölls gisslan där. Detta bekräftas av både piloter och drönaroperatörer, samt i israeliska militära utredningar. Ynet rapporterade redan i oktober att israeliska helikoptrar “sköt mot allt längs gränsstängslet” och att 300 mål attackerades inom de första fyra timmarna, “flertalet på israeliskt territorium.”

Denna systematiska användning av dödligt våld inne i egna bostadsområden är en djupt komprometterande omständighet – särskilt som den bidrog till flera av de dödsfall som sedermera tillskrevs Hamas i internationell propaganda. Ändå har ingen internationell granskning genomförts, och varken FN, EU eller svenska regeringen har efterfrågat ansvar för dessa händelser.

\subsubsection*{Israeler beordrades skjuta mot egna civila}
\addcontentsline{toc}{subsubsection}{Israeler beordrades skjuta mot egna civila}

Fler vittnesmål har framkommit som visar att israeliska soldater den 7 oktober beordrades skjuta in i egna bostadsområden. En granskning av Max Blumenthal i \textit{The Grayzone}\footnote{\url{https://thegrayzone.com/2023/11/27/israeli-tank-orders-fire-kibbutz/} Max Blumenthal, “Israeli tank gunner reveals orders to fire indiscriminately into kibbutz”, \textit{The Grayzone}, 27 november 2023.} redovisar hur unga kvinnliga soldater i en israelisk stridsvagnsenhet vittnar om att de beordrades öppna eld mot hus i Kibbutz Holit – oavsett om civila fanns där eller inte. 

En av soldaterna, endast identifierad som “Karni”, beskriver hur en panikslagen kollega pekade ut ett hus och skrek “bara skjut”. Hon frågade: “Men är det civila där?” och fick svaret “jag vet inte – skjut bara.” Karni valde att avstå från att skjuta med kanonen, men öppnade istället eld med kulspruta mot huset.

Flera av de dödade israeliska civila i Holit kan ha fallit offer för denna eld, enligt artikeln. Liknande händelser ska ha inträffat i Kibbutz Be’eri, där en israelisk stridsvagn öppnade eld mot ett hus och dödade tolv civila, inklusive Liel Hetzroni – ett av de barn som senare användes i propaganda mot Hamas.

Artikeln beskriver också hur israeliska säkerhetstjänster försökt tysta frigivna gisslan som inte bekräftar den officiella berättelsen. Flera frigivna israeler, bland andra Danielle Aloni, vittnar om human behandling i Hamas fångenskap. Aloni skrev ett brev till sina fångvaktare och tackade för deras “onaturliga mänsklighet” och “vänlighet trots förlusterna ni själva lidit”. Hennes offentliga uttalanden har dock förhindrats.

Flera andra gisslan har beskrivit liknande behandling, och israeliska medier har avslöjat att återvändande fångar förbjudits tala fritt. Detta tyder på att regeringen aktivt kontrollerar narrativet, även när det sker på bekostnad av de frigivnas röster.

\subsubsection*{Frigivna gisslan berättar om övergrepp – vad vet vi egentligen?}
\addcontentsline{toc}{subsubsection}{Frigivna gisslan berättar om övergrepp – vad vet vi egentligen?}

En artikel från den israeliska nyhetssajten \textit{All Israel News} återger vittnesmål från nyligen frigivna israeliska gisslan som beskriver sexuella övergrepp utförda av palestinska gärningsmän.\footnote{\url{https://allisraelnews.com/freed-hostages-reveal-the-scope-of-atrocities-and-abuse-against-women-by-hamas-terrorists}} Artikeln saknar dock identifierbara målsägande, konkreta uppgifter om tid och plats, samt oberoende bekräftelse från rättsmedicinsk eller juridisk instans. Enligt uppgift bygger vittnesmålen på samtal med psykologer och socialarbetare, men ingen rättslig förundersökning åberopas och inga anmälningar omnämns.

De berättelser som återges – exempelvis om våldtäkter som ska ha filmats – är i linje med det atrocity-narrativ som etablerades tidigt av israeliska talespersoner. Det bör påpekas att flera liknande påståenden, inklusive från militära källor, sedermera visat sig sakna stöd eller ha motbevisats (se ovan). I detta fall redovisas varken omständigheter, ansvariga grupper eller om några målsägande trätt fram offentligt.

Detta innebär att även denna källa måste bedömas kritiskt. Anonyma utsagor via sekundärkällor uppfyller inte grundläggande rättssäkerhetskrav. Samtidigt utesluts inte att övergrepp kan ha förekommit – särskilt i ett kaotiskt händelseförlopp med flera väpnade fraktioner och även civila inblandade. 

Även om en majoritet av gisslan i intervjuer vittnat om respektfull behandling av sina fångvaktare, vore det sannolikt orealistiskt att utgå från att inga övergrepp förekommit. Det skulle strida mot normalfördelningens förväntade spridning i en situation med låg kontroll, stark stresspåverkan och väpnade aktörer från olika grupper med varierande disciplinära strukturer.


\subsection*{Vad har avslöjats av oberoende röster i sociala medier?}
\addcontentsline{toc}{subsection}{Vad har avslöjats av oberoende röster i sociala medier?}

\subsection*{Propaganda som militärstrategi – atrocitynarrativ i krigföring}
\addcontentsline{toc}{subsection}{Propaganda som militärstrategi – atrocitynarrativ i krigföring}

Mänskliga sköldar, läkare är Hamas, skjuter innefårn sjukhuset, kommandocentraler inne i sjukhuset.



\section{Rättslig bedömning av Hamasstatus}
% Pröva om Hamas juridiskt kan klassificeras som terroristorganisation givet folkrättsbrottens art från Israel.
% Använd historisk analogi (ex: Nat Turner, slavuppror).
% Argumentera att terrorbegreppet förlorar legitimitet om det tillämpas selektivt.

Regeringen proklamerar:
\textit{The terrorist organisation Hamas bears heavy responsibility for the current situation. }\\

För resonemangets skull – och enbart som hypotetisk premiss – antar vi att samtliga israeliska uppgifter om övergrepp från Hamas är korrekta, trots att flera av dem senare visat sig vara obestyrkta eller direkt falska.


Då regeringen inte uppfyller sina förpliktelser enligt ingångna folkrättsliga avtal – såsom att verka för respekt av FN:s resolutioner och folkrättens regler – utan istället tillerkänner Israel praktisk immunitet trots fortlöpande brott mot internationell rätt, uppstår ett ansvar enligt principerna om staters medverkan till folkrättsbrott.

Detta ansvar förstärks när Sverige inte enbart underlåter att ingripa, utan även aktivt karaktäriserar det motstånd som utövas av den ockuperade parten som ``terrorism'' – utan att samtidigt erkänna att rätten till beskydd enligt internationell humanitär rätt nekats den befolkningen.

I en sådan situation sker inte bara ett brott mot neutralitetsprincipen och ett passivt medansvar – utan även en retorisk och diplomatisk handling som riskerar att inflammera konflikten ytterligare. Denna eskalation medför ett förstärkt medansvar även för de illdåd som följer, om det kan visas att staten bidragit till att skapa förtvivlan eller rättslöshet hos den utsatta parten.

\lagrum{Artikel 16, ILC Articles on State Responsibility\quad 
A State which aids or assists another State in the commission of an internationally wrongful act by the latter is internationally responsible\footnote{\url{https://legal.un.org/ilc/texts/instruments/english/draft_articles/9_6_2001.pdf}}...}

\lagrum{Artikel 1, Genèvekonventionerna\quad 
The High Contracting Parties undertake to respect and to ensure respect for the present Convention in all circumstances\footnote{\url{https://ihl-databases.icrc.org/en/ihl-treaties/gciv-1949/article-1}}.}


Det är denna konstruktion som förklarar varför en betydande del av det internationella samfundet inte erkänner Hamas som en terroristorganisation, utan istället betraktar dem som en aktör i en asymmetrisk konflikt där de rättsliga definitionerna måste ses i ljuset av förvägrad rättsordning och ockupation.\footnote{\url{https://en.wikipedia.org/wiki/Hamas\#/media/File:International_views_on_Hamas.svg}}

Ur ett realpolitiskt perspektiv kan en regering i extrema undantagsfall avvika från ingångna förpliktelser under hot om nationell undergång. Men detta får aldrig ske i det tysta eller som medvetet hållen, långvarig politik. Principen om \textit{pacta sunt servanda} är inte förhandlingsbar. Att aktivt bryta mot egna åtaganden och samtidigt bidra till en situation där våldet förvärras, kan aldrig rättfärdigas inom ramen för en rättsstat.

\begin{tcolorbox}[title=Rättsfigurer som aktualiseras vid terroristutpekning, colback=gray!10, colframe=black, sharp corners=south]
\textbf{Följande rättsliga principer och figurer aktualiseras i samband med att Sverige benämner Hamas som terroristorganisation, utan att samtidigt erkänna Israels rättsbrott:}

\begin{itemize}
    \item \textbf{Statsansvar vid medverkan} – enligt \textit{ILC Articles on State Responsibility} (särskilt art. 16 och 41)\footnote{\url{https://legal.un.org/ilc/texts/instruments/english/draft_articles/9_6_2001.pdf}}, riskerar Sverige att bli folkrättsligt medansvarigt om det bistår eller möjliggör folkrättsbrott genom partiskhet.
    
    \item \textbf{Neutralitetsplikt} – enligt sedvanerätt och internationell humanitär rätt\footnote{\url{https://ihl-databases.icrc.org/en}}, får en icke-stridande stat inte selektivt stödja en konfliktpart som själv bryter mot Genèvekonventionerna\footnote{\url{https://ihl-databases.icrc.org/en/ihl-treaties/gciv-1949}}.
    
    \item \textbf{Förnekande av skydd} – att neka en befolkning, eller dess representanter, skydd under IHL utgör rättsstridigt undandragande av erkända rättigheter och kan betraktas som kollektiv bestraffning.
    
    \item \textbf{Culpa in contrahendo} – staten agerar i strid med lojalitetsplikten när den undertecknar avtal om humanitär rätt, men medvetet vägrar tillämpa dess principer konsekvent.
    
    \item \textbf{Pacta sunt servanda} – folkrättens kärnprincip: ingångna avtal ska hållas enligt Wienkonventionen om traktaträtten, artikel 26\footnote{\url{https://legal.un.org/ilc/texts/instruments/english/conventions/1_1_1969.pdf}}. Retorik och politik som strider mot denna princip försvagar hela det traktatbaserade folkrättssystemet.
    
    \item \textbf{Förstärkt medansvar} – genom att retoriskt utpeka en ockuperad befolkning som "terrorister", utan erkännande av rättslöshet och förtryck, bidrar staten aktivt till konfliktens eskalation och bär därmed del i det moraliska och folkrättsliga ansvaret.
\end{itemize}
\end{tcolorbox}

Det är i denna kontext som även språkbruket i sig får rättsverkan.

När en stat benämner en aktör i en ockuperad befolkning som “terroristorganisation”, utan att samtidigt erkänna det rättsliga sammanhanget – ockupationen, rättslösheten och vägran att ge skydd enligt IHL – så sker ett folkrättsligt konkludent handlande.

Formuleringen blir inte neutral, utan ger det sken av att det endast finns en skyldig part. Detta utgör i sig en tyst sanktionering av ockupationen och bidrar till att normalisera den rättsvidriga ordningen.

Att därefter tillfoga “men Israel måste följa internationell rätt” saknar betydelse om man redan i språkbruket legitimerat själva asymmetrin. Retoriken skapar en ny rättslig utgångspunkt: inte om förtrycket är lagligt, utan hur hårt den förtryckta får slå tillbaka.

Detta är vad som i folkrätten betraktas som ett konkludent godkännande – en handling med juridisk effekt, trots att den sker i form av språk och diplomatiskt ställningstagande.

Ytterligare en aspekt aktualiseras här – nämligen den folkrättsliga motsvarigheten till \textit{stämpling till brott}.

Att benämna ett motstånd inom en ockuperad befolkning som "terrorism", utan att samtidigt tillerkänna befolkningen det skydd som Genèvekonventionerna förutsätter, är inte bara en retorisk gärning. Det riskerar att uppfattas som att staten – i samråd eller i lojal sympati – legitimerar ockupationsmaktens rättsbrott. Detta kan liknas vid en form av diplomatisk eller språklig \textit{anstiftan}.

På motsvarande sätt aktualiseras även principen om \textit{underlåtenhet att avslöja eller förhindra brott}, så som den uttrycks i 23 kap. 6 § BrB. En stat som är bunden av folkrätten men underlåter att protestera mot grova rättsbrott – trots kännedom – uppfyller de moraliska och juridiska kriterierna för passivt medverkansansvar, särskilt om staten samtidigt förfogar över diplomatisk eller institutionell makt att påverka utvecklingen.

I ett internationellt sammanhang motsvaras dessa handlingar av de former för ansvar som regleras i \textit{ILC Articles on State Responsibility}, särskilt artiklarna 16 (aid or assistance), 41 (duty not to recognize the situation as lawful) och 40 (responsibility for serious breaches of peremptory norms).

Sammantaget uppstår ett mönster: staten utövar ett aktivt språkbruk som konkludent legitimerar en rättsvidrig ordning, underlåter att ingripa trots skyldighet, och bidrar därmed – med eller utan uppsåt – till att möjliggöra fortsatt rättsbrott. Detta skapar vad som i straffrättsliga termer motsvarar ett samverkansansvar.


\subsection*{Från Nat Turner till Gaza: Vem bär ansvaret för våldets födelse?}
\addcontentsline{toc}{subsection}{Från Nat Turner till Gaza: Vem bär ansvaret för våldets födelse?}

En slav utan tillgång till lag, som gör uppror, är inte en terrorist.

Nat Turner\footnote{\url{https://www.normanfinkelstein.com/nat-turner-in-gaza/}}, slavupprorsledaren som 1831 dödade oskyldiga vita i den största slavrevolten i USA:s historia, betraktades som ”terrorist” av dem som försvarade slaveriet. Abolitionisterna erkände våldets fasor – men vägrade fördöma upprorsmännen. De riktade sin vrede mot det system som födde våldet. I dag erkänns Turner som hjälte.

Här finns paralleller. Palestina har aldrig erbjudits rättvisa. Att fördöma dess motstånd, utan att erkänna den rättslösa omgivning det fötts i, är moraliskt absurt och juridiskt ohållbart.

I rättslig mening vilar det tyngsta ansvaret inte på den som slår tillbaka – utan på den som vägrar ingripa. Det är de som åtagit sig att vara systemets väktare, men som låter rättsordningen falla, som bär huvudansvaret.

En människa utan tillgång till rättsligt skydd, vars vädjan om beskydd ignoreras av världens självutnämnda ”guardians”, kan inte utan vidare stämplas som terrorist. Hon är, i sin desperation, inte ett hot mot rättsstaten – utan dess spegelbild när skyddet uteblir.

Nat Turner dödade inte i brist på moral, utan i brist på erkännande. Han förvägrades sin människovärdighet av ett samhälle som inte såg honom som människa. Hans uppror utmålades som religiös fanatism – men historien visade att det var systemet som var vansinnigt, inte han.

På samma sätt föddes våldet i Gaza inte ur hat – utan ur ett konsekvent nekande av lagligt skydd. De unga män som i oktober 2023 trängde ut genom murarna var födda i ett rättslöst fängelse – utan flyktvägar, utan framtid. Deras död var väntad. Deras motstånd förbjudet.

När Sverige då inte bara vägrar att ingripa, utan dessutom benämner deras handlingar som ”terrorism”, är det inte rättvisa man försvarar – utan status quo. Det är inte rättsstat. Det är moraliskt sammanbrott.

I detta rättsläge upplöses de juridiska strukturerna – och nya, desperata logiker tar vid:

\begin{quote}
Om skydd enligt internationell humanitär rätt konsekvent nekas oss, återstår endast den arkaiska rättsformen: \textit{öga för öga, tand för tand} – som en sista metod för avskräckning. Då vilar de oskyldigas blod inte på oss som slog tillbaka, utan på de ansvariga för systemets sammanbrott.
\end{quote}

\begin{quote}
Om lagsystemets väktare – dess guardians – förvägrar oss människostatus, om vi behandlas som djur, som restposter utan rättslig existens – varför skulle vi då låtsas följa mänsklighetens kodex? När vår mänsklighet förnekas, återstår bara djungelns lag – och den som behandlas som ett djur kommer till slut att slå som ett djur. Ingen kan kräva att vi följer lagar som ingen upprätthåller för oss.
\end{quote}

Detta är inte ett försvar av dödande. Det är ett åtal mot den värld som tvingar fram det.




\subsection*{Pedagogisk liknelse – när uppstår skuld hos systemets väktare?}
\addcontentsline{toc}{subsection}{Pedagogisk liknelse – när uppstår skuld hos systemets väktare?}

\begin{quote}
En polis iakttar hur grupp A, med hänvisning till en gudagiven rätt, fördriver grupp B från ett land där B bott i generationer. Grupp A hävdar att B är illegitima inkräktare. Grupp B vädjar om beskydd och påstår sig utsättas för etnisk rensning.
\end{quote}

\textbf{1. Den första vädjan:}
\begin{quote}
En representant för grupp B ber polismannen om hjälp. Polisen lyssnar – men ingriper inte. Han säger att konflikten är ”komplex” och att det inte är hans uppgift att ta ställning.
\end{quote}

\textit{Här uppstår den första formen av ansvar: underlåtenhet att skydda den som söker rättsskydd. Att inte ingripa trots skyldighet är ett brott mot skyddsprincipen – en grundpelare i såväl humanitär rätt som polisärt uppdrag.}
\lagrum{Jfr artikel 1 i Genèvekonventionen\footnote{\url{https://ihl-databases.icrc.org/en/ihl-treaties/gciv-1949/article-1}} och artikel 41.1 i ILC Articles\footnote{\url{https://legal.un.org/ilc/texts/instruments/english/draft_articles/9_6_2001.pdf}}}

\textbf{2. Det aktiva ställningstagandet:}
\begin{quote}
Snart slutar polisen inte bara att ignorera grupp B:s situation, utan inleder dessutom avtal och samarbete med grupp A – som fortsatt fördriver och dödar medlemmar ur grupp B.
\end{quote}

\textit{Här förstärks ansvaret: från passivitet till aktivt medansvar. Det är att bidra till rättsbrottet – både i folkrättslig mening (ILC:s artikel 16) och enligt Genèvekonventionernas skyldighet att ”respektera och säkerställa” att konventionerna efterlevs.}
\lagrum{Se ILC Articles, artikel 16\footnote{\url{https://legal.un.org/ilc/texts/instruments/english/draft_articles/9_6_2001.pdf}} och sedvanerättsregel 139 i ICRC:s sammanställning\footnote{\url{https://ihl-databases.icrc.org/en/customary-ihl/v1/rule139}}}

\textbf{3. Den institutionaliserade tystnaden:}
\begin{quote}
Under åren fortsätter övergreppen. Grupp B förlorar mark, försörjning och tillgång till rättsmedel. Trots återkommande larm hänvisar polisen till ”neutralitet” – men säger inget, gör inget.
\end{quote}

\textit{Här blir tystnaden ett strukturellt svek. Väktaren som inte skyddar offret utan upprätthåller gärningsmannens straffrihet sviker hela rättsordningen. Det är inte neutralitet – det är medverkan genom likgiltighet.}
\lagrum{Se artikel 40 och 41.2 i ILC Articles\footnote{\url{https://legal.un.org/ilc/texts/instruments/english/draft_articles/9_6_2001.pdf}}}

\textbf{4. Det desperata svaret:}
\begin{quote}
Till sist slår vissa ur grupp B tillbaka. De angriper civila från grupp A – i ett försök att spegla det lidande de själva utsatts för och tvinga fram internationell uppmärksamhet.
\end{quote}

\textit{Detta är inte lagligt. Men det är begripligt. Det är desperation i frånvaro av rättvisa. Det är rättsstatens sammanbrott i realtid.}
\lagrum{Se proportionalitetsprincipen i IHL och artikel 1(4) i Tilläggsprotokoll I\footnote{\url{https://ihl-databases.icrc.org/en/ihl-treaties/api-1977/article-1}}}

\textbf{5. Den slutgiltiga domen:}
\begin{quote}
Då först reagerar polisen. Han fördömer våldet från grupp B, kallar dem ”terrorister” och kräver att de ställs inför rätta. Han nämner aldrig sin egen roll.
\end{quote}

\textit{Här uppstår dubbel skuld:}
\begin{itemize}
    \item \textbf{Skuld gentemot grupp B:} för att han förvägrade dem skydd, tillät deras fördrivning, och förnekade dem status som skyddsberättigade.
    \lagrum{Brott mot skyldigheten att förhindra folkrättsbrott enligt Genèvekonventionerna\footnote{\url{https://ihl-databases.icrc.org/en/ihl-treaties/gciv-1949/article-1}} och artikel 16 i ILC Articles\footnote{\url{https://legal.un.org/ilc/texts/instruments/english/draft_articles/9_6_2001.pdf}}}

    \item \textbf{Skuld gentemot grupp A:} eftersom hans vägran att upprätthålla lag och rätt bidrog till att våld föddes ur desperation – och riktades mot oskyldiga även i grupp A.
    \lagrum{Indirekt ansvar enligt artikel 16 och 41 i ILC Articles – medverkansansvar för följdskador\footnote{\url{https://legal.un.org/ilc/texts/instruments/english/draft_articles/9_6_2001.pdf}}}
\end{itemize}

\textbf{Detta är Sveriges roll.}\\
Inte som neutral observatör. Inte som fredsbevarare.\\
Utan som den polis som vägrade skydda offret, slöt avtal med förövaren – och till sist fördömde det motstånd han själv möjliggjorde.

\vspace{1em}
\noindent\textit{Den rättsliga analysen är därmed fullbordad. Vad följer är en etisk och retorisk kommentar – inte som ersättning, utan som illustration av den rättsliga nödvändigheten.}



\subsection*{Etisk och retorisk kommentar: När regeringen dömer de förtvivlade}
\addcontentsline{toc}{subsection}{Etisk och retorisk kommentar: När regeringen dömer de förtvivlade}
% Här börjar din känslomässiga del.
Regeringen: \textit{The terrorist organisation Hamas bears heavy responsibility for the current situation. }\\
Vad har föregått regeringens senaste uttalande om Hamas såsom terrorister och ytterst ansvariga?

Uttalandet bekräftar inte bara Israels känsla av straffrihet – det görs dessutom:
\begin{enumerate}
\item endast 33 år efter att en internationellt erkänd terrorist tilläts avgå som Israels premiärminister – samma man som beordrade mordet på Hans Majestät konungens gudfar, FN-medlaren Folke Bernadotte,
\item efter att ockupationen av Västbanken, Östra Jerusalem och Gaza pågått oavbrutet sedan 1967,
\item efter en 18 år lång blockad av Gaza – bedömd som olaglig av Internationella domstolen (ICJ) redan 2004 i yttrandet om separationsbarriären,
\item efter återkommande storskaliga bombanfall mot Gaza: 2009, 2012, 2014, 2021 – samtliga med omfattande civila förluster,
\item efter att Israel 2018 skjutit ihjäl över 250 obeväpnade civila under \textit{The Great March of Return}, inklusive barn, kvinnor, äldre, funktionshindrade, medicinsk personal och journalister,
\item efter att Internationella brottmålsdomstolen (ICC) utfärdat arresteringsorder mot Israels premiärminister och försvarsminister,
\item efter att ICJ fastslagit att Israels permanenta närvaro i de ockuperade områdena är folkrättsstridig,
\item efter att samma domstol konstaterat att Israels politik på Västbanken konstituerar ett brott mot apartheidförbudet enligt \textit{Internationella konventionen om avskaffande av rasdiskriminering} (CERD),
\item efter att folkmordshets dokumenterats - i israeliska medier, i parlamentet och bland ledande befälhavare åtminstone sedan år 2010\footnote{\url{https://www.youtube.com/watch?v=9GbKsAvuBDM}}– utan påföljd\footnote{\url{https://law4palestine.org/law-for-palestine-releases-database-with-500-instances-of-israeli-incitement-to-genocide-continuously-updated/}},
\item snart ett år efter att en artikel i \textit{The Lancet} estimerade att uppemot 186 000 palestinier kan ha dödats av Israel sedan oktober 2023\footnote{\url{https://www.thelancet.com/journals/lancet/article/PIIS0140-6736(24)01169-3/fulltext}},
\item efter att samtliga sjukhus i Gaza bombats\footnote{\url{https://news.un.org/en/story/2024/01/1145317}}\footnote{\url{https://en.wikipedia.org/wiki/Attacks_on_health_facilities_during_the_Gaza_war}},
\item efter att massgravar påträffats vid sjukhus med bakbundna patienter, läkare och barn\footnote{\url{https://en.wikipedia.org/wiki/Gaza_Strip_mass_graves}},
\item efter att över 200 journalister dödats – det högsta antalet i en enskild konflikt på flera decennier\footnote{\url{https://en.wikipedia.org/wiki/List_of_journalists_killed_in_the_Gaza_war}},
\item efter att Israel avslöjats med att ha ljugit om massavrättningen av 15 ambulansarbetare – vilka påträffades med bakbundna händer i massgrav tillsammans med sina förstörda ambulanser. Händelsen beskrevs i efterhand som ett "operationellt misstag" där ansvarig befälhavare uppges ha avskedats.
\item efter att chefsläkare och professor kidnappats och torterats till döds\footnote{\url{https://www.middleeasteye.net/news/war-gaza-prominent-palestinian-doctor-tortured-and-killed-israeli-detention}}
\item efter att chefsläkaren för barnsjukhuset Kamal Adwan alltjämnt sitter fängslad och har torterats\footnote{\url{https://www.democracynow.org/2025/4/17/headlines/dr_abu_safiyas_lawyer_warns_kamal_adwan_director_is_being_tortured_in_prison}}
\end{enumerate}


Att i detta rättsläge offentligt utpeka den ockuperade parten som terrorist, utan att samtidigt erkänna den folkrättsliga grunden för dess motstånd, är inte en rättsstatlig handling – det är medverkan till rättens sammanbrott.

Den föregående uppräkningen demonstrerar med obestridlig tydlighet att det i praktiken inte finns något brott – vare sig mot mänskligheten eller mot Sverige – som Israel skulle kunna begå utan att bemötas med undfallenhet av den svenska regeringen. Det saknas varje spår av proportionalitet i reaktionerna: den ockuperade parten fördöms, medan ockupationsmakten undgår varje form av sanktion.

Att Hamas betecknas som terrorist är därför inte ett resultat av objektiv juridisk prövning – utan en logisk konsekvens av regeringens förutbestämda asymmetri. Det är i själva verket listan av israeliska brott som förutsätter att Hamas måste kallas terrorist, för att logiken i regeringens politik ska kunna upprätthållas. Och omvänt: terroristbeteckningen på Hamas är det som möjliggör att listan över israeliska övergrepp kan växa ohämmat, utan rättslig eller diplomatisk reaktion.

Med andra ord: man kan inte förstå det ena utan det andra. Det är två sidor av samma logiska och moraliska konstruktion. Att inte kalla Hamas för terrorist, i ljuset av denna obestridda brottskatalog från Israels sida, skulle framstå som orimligt – just därför att hela konstruktionen bygger på att endast en part kan klandras, medan den andra ges immunitet.

En naiv betraktare skulle kanske invända:
\textit{"Men om exempelvis Greta Thunberg hade dödats i ett israeliskt bombanfall mot Ship to Gaza – på internationellt vatten utanför Malta – skulle inte den svenska regeringen då ha reagerat och vidtagit sanktioner?"}

Det är en begriplig fråga – men svaret är: nej. Händelsen hade sannolikt rapporterats som att Greta “omkommit i samband med en brand ombord”, där det endast av föregående mening indirekt kunnat utläsas att branden orsakats av ett israeliskt angrepp. Hade det däremot rört sig om ett ryskt flyganfall, hade rubriken sannolikt varit: \textit{“Rysk bomb dödade Greta Thunberg”} – utan omskrivningar, utan omsvep.

\textit{"Men om Israel skulle mörda en svensk medborgare på svensk mark – skulle inte regeringen då ha reagerat?"}

Motsvarande har redan inträffat – i Norge. På öppen gata i Lillehammer mördade israeliska agenter en oskyldig man.\footnote{Se t.ex. \url{https://en.wikipedia.org/wiki/Lillehammer_affair}} Ingen diplomatisk kris följde. Israelerna vet att till och med en flimsig ursäkt räcker.

Ingenting i Sveriges diplomatiska historik tyder på att det existerar någon faktisk gräns. Reaktionen hade sannolikt inskränkt sig till en formell beklagan, möjligen ett kraftigt fördömande – men därefter fortsatt normalisering, som om ingenting hänt.

Detta vittnar inte bara om moralisk apati – utan om ett systematiskt undandragande från folkrättens kärna: \textit{ansvar}.



\subsubsection*{Språket som verktyg för förskjutet ansvar}
\addcontentsline{toc}{subsubsection}{Språket som verktyg för förskjutet ansvar}

Svensk nyhetsrapportering präglas i ökande grad av vad som närmast kan beskrivas som språklig psykologisk prissättning – en metod där verkligheten paketeras för att kännas mindre brutal, trots att våldets karaktär är densamma.
När en palestinsk familj dödas av ett israeliskt flyganfall, heter det att de \textit{”omkommit i samband med en explosion”}, \textit{”dött i en brand”} eller att deras kroppar \textit{”hittats under rasmassorna”}. Den kausala länken – vem som orsakade döden – försvinner ur bild.

När motsvarande våld riktas mot israeler är beskrivningen sällan passiv. Då handlar det om att de \textit{”mördades”}, \textit{”slaktades”} eller \textit{”massakrerades”} – med tydlig gärningsman. Skillnaden mellan att \textit{”dö”} och att \textit{”bli dödad”} är mer än språklig. Den är politisk. Den är moralisk. Den är rättslig.

I ett upprop publicerat i augusti 2024, undertecknat av över 70 svenska journalister, författare och forskare, formuleras detta fatala språkliga glidande som ett journalistiskt haveri\footnote{\url{https://www.journalisten.se/debatt/rapporteringen-om-gaza-ar-ett-fatalt-misslyckande/}}. Där varnas för att svensk rapportering riskerar att bli en medskyldig kraft – genom underlåtenhet att benämna det som sker, genom frånvaro av kontext och genom att okritiskt återge krigförande staters propaganda.

Detta handlar inte om åsikter – utan om ansvar. Den tredje statsmakten kan inte frånträda sin granskande funktion utan att hela den demokratiska balansen rubbas.

Särskilt allvarligt blir detta när det sker inom statligt finansierad nyhetsförmedling. Att en regering inte reagerar när public service urholkar skyldigheten att benämna våldets källa är inte passivitet – det är ett strukturellt ansvarsfel. När en folkrättsvidrig ockupation inte benämns som sådan, och dess brott förskjuts från gärningsman till neutral naturhändelse, sker medverkan i rättens sammanbrott – genom språket.



\subsection*{Om gisslan och administrativt förvar}
\addcontentsline{toc}{subsection}{Om gisslan och administrativt förvar}

Regeringen:\textit{“The terrorist organisation Hamas bears heavy responsibility for the current situation. The hostages must be released – unconditionally and immediately.”}\\

Regeringen kräver att Hamas ovillkorligen friger gisslan – ett legitimt krav enligt humanitär rätt. Men varför riktas inga motsvarande krav mot Israel, som systematiskt och i strid med folkrätten berövar tusentals palestinier friheten utan vare sig åtal, rättegång eller fastställd tidsgräns?

\textit{Så kallat “administrativt förvar” innebär att en individ frihetsberövas enbart på grundval av säkerhetstjänstens påståenden – utan insyn, utan beviskrav, utan rättslig prövning, utan möjlighet till försvar. Det är ett rättsligt undantagstillstånd som kan förlängas i månader, år – i praktiken utan slut.}

Barn hämtas nattetid av beväpnade soldater. Föräldrar får inga besked. Minderåriga grips, isoleras och förhörs – ibland under tortyrliknande former. Tusentals palestinier har hållits frihetsberövade under dessa förhållanden – utan att någonsin delges misstanke om brott.

\textit{Den israeliska människorättsorganisationen B’Tselem har i åratal dokumenterat denna praxis, som bryter mot både fjärde Genèvekonventionen och FN:s konvention om medborgerliga och politiska rättigheter.}\footnote{\url{https://www.btselem.org/administrative_detention}}

Att regeringen väljer att offentligt fördöma Hamas gisslantagning – men tiger om Israels institutionella massinternering av civila, inklusive barn – utgör en folkrättslig medverkan genom:

\begin{itemize}
  \item \textbf{Konkludent handlande:} Staten Sverige erkänner indirekt det ena brottet (gisslantagning) som avvikelse från rättsordningen, men det andra (administrativt förvar) som normalitet. Det är ett rättsligt ställningstagande genom underlåtenhet, i strid med \lagrum{artikel 1 i fjärde Genèvekonventionen}\footnote{\url{https://ihl-databases.icrc.org/en/ihl-treaties/gciv-1949/article-1}}.

  \item \textbf{Underlåtenhet att förhindra eller avslöja brott:} En stat som har insyn och möjlighet att agera, men inte gör det, bryter mot skyldigheten att ”respektera och säkerställa” konventionens efterlevnad (\lagrum{GC IV art. 1}) och \lagrum{ICCPR artikel 2(1)}\footnote{\url{https://www.ohchr.org/en/instruments-mechanisms/instruments/international-covenant-civil-and-political-rights}}, särskilt i relation till \lagrum{ICCPR artiklarna 9 och 14}\footnote{\url{https://www.ohchr.org/en/instruments-mechanisms/instruments/international-covenant-civil-and-political-rights}}.

  \item \textbf{Stämplingsliknande ansvar:} Genom att den ockuperande maktens brott systematiskt ursäktas, tonas ner eller förtigs, och endast motståndaren fördöms, legitimeras brotten. Detta bryter mot \lagrum{artikel 16 i ILC:s utkast till statsansvar}\footnote{\url{https://legal.un.org/ilc/texts/instruments/english/draft_articles/9_6_2001.pdf}} samt folkrättens tvingande normer (\textit{jus cogens}), där förbudet mot godtyckligt frihetsberövande och tortyr är absoluta och icke-derogabla.
\end{itemize}

\textit{Det är inte brottsligt att kräva frigivning av gisslan.}\\
\textit{Men det är rättsvidrigt att endast kräva det från den ena parten.}\\
\textit{Att tyst legitimera den andres systematiska och olagliga massfångenskap – det är medverkan.}



