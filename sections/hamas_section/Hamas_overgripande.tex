%filnmanmn. Hamas_overgripande.tex

\subsection{Hamas manifest 2017 – acceptans av en lösning i enlighet med folkrätten}

I sin politiska deklaration från maj 2017, \textit{A Document of General Principles and Policies}, uttrycker Hamas en fortsatt ståndpunkt att hela det historiska Palestina, från Jordanfloden till Medelhavet, utgör ockuperat territorium. Samtidigt slås det fast att:

\begin{itemize}
  \item Hamas inte erkänner staten Israel i formell mening (punkt 19–20),
  \item men att man \textbf{accepterar etablerandet av en fullständigt suverän palestinsk stat inom 1967 års gränser, med Östra Jerusalem som huvudstad}, som en formel för nationell konsensus.
\end{itemize}

\begin{quote}
"Without compromising its rejection of the Zionist entity and without relinquishing any Palestinian rights, Hamas considers the establishment of a fully sovereign and independent Palestinian state, with Jerusalem as its capital along the lines of the 4th of June 1967 [...] to be a formula of national consensus." (punkt 20)
\end{quote}

Detta ska inte förstås som ett undantag från folkrätten, utan som ett politiskt uttryck för folkrättslig efterlevnad – särskilt vad gäller FN:s säkerhetsrådsresolution 242 (1967), som fastslår principen om \textit{land mot fred}. Denna resolution förpliktar Israel att dra sig tillbaka från ockuperade områden och erkänna alla staters rätt till fredliga gränser – inklusive en palestinsk stat inom 1967 års gränser.

\vspace{0.5em}
\noindent
\textbf{Rättslig analys:}  
Den internationella rättsordningen kräver inte ett ömsesidigt erkännande för att etablera rättigheter enligt självbestämmanderätten. Hamas’ uttryckta acceptans av 1967 års gränser innebär därmed ett praktiskt ställningstagande i enlighet med folkrätten – medan Israels fortsatta vägran att erkänna en palestinsk stat utgör ett folkrättsbrott i den mening att det förnekar ett ockuperat folk dess självbestämmanderätt.





\subsection{Om erkännandekravet och karaktären av Israel som etnostat}

Hamas efterfrågar inte bilaterala förhandlingar med Israel, och erkänner varken staten Israel eller dess rätt att utöva kontroll över något territorium i historiska Palestina. Istället kräver Hamas att internationell rätt ska tillämpas fullt ut – i synnerhet FN:s säkerhetsrådsresolution 242, Genèvekonventionerna och principen om folkens rätt till självbestämmande. 

Israels vägran att erkänna denna ståndpunkt måste förstås i ljuset av landets återkommande krav på att bli erkänd som en \textit{judisk stat}. Detta krav innebär inte enbart ett erkännande av Israels existens som suverän stat, utan ett erkännande av dess \textbf{etniska karaktär} – som en stat där politisk suveränitet och territoriell legitimitet är reserverad för en specifik folkgrupp.

Samtidigt har Israel själv konsekvent vägrat att erkänna en palestinsk stat inom 1967 års gränser, trots att dessa gränser utgör utgångspunkt i FN:s resolutioner. I israelisk doktrin betraktas gränserna som preliminära och förhandlingsbara – ett synsätt som undergräver hela FN-systemets legitimitet. Medan Israel kräver erkännande av sin ideologiska självdefinition, förvägrar det palestinierna deras folkrättsligt erkända rätt till självbestämmande och statssuveränitet.


\subsubsection*{Definition av etnostat i folkrättsligt sammanhang}

Med en \textit{etnostat} avses i detta sammanhang en statsbildning där grundlag och författningsstruktur konstituerar politisk suveränitet, kollektiv identitet och rättighetstilldelning med utgångspunkt i etnisk tillhörighet. Detta skiljer sig principiellt från stater med en dominerande statsreligion, där medborgerlig likabehandling – åtminstone formellt – upprätthålls.

\begin{itemize}
  \item Hamas har konsekvent gjort åtskillnad mellan ett \textit{de facto}-accepterande av en tvåstatslösning baserad på FN:s säkerhetsrådsresolution 242, och ett \textit{de jure} erkännande av Israel som etnisk stat. 
  \item Ett formellt erkännande av Israel som ”judisk stat” skulle enligt Hamas och andra aktörer innebära att man accepterar en rättsordning där andra folkgrupper per definition är sekundära.
\end{itemize}

Det är samtidigt av vikt att understryka att Israel själv aldrig har erkänt en palestinsk stat inom 1967 års gränser, trots omfattande internationellt stöd för detta. Israel motsätter sig även fullt medlemskap för Palestina i FN, vilket ytterligare förstärker asymmetrin i erkännandeprocessen.

\subsubsection*{Konstitutionell lagstiftning och diskriminerande struktur}

År 2018 antogs den israeliska grundlagen \textbf{Basic Law: Israel as the Nation-State of the Jewish People}, vilken har betydande konsekvenser i rättslig mening:

\begin{itemize}
  \item Lagen slår fast att endast det judiska folket har nationell självbestämmanderätt i Israel, vilket utesluter andra befolkningsgrupper – inklusive över 20\% palestinska medborgare – från kollektiv suveränitet.
  \item Arabiska språket, tidigare ett av två officiella språk, nedgraderas till en ”speciell status”.
  \item Staten åläggs att främja judisk bosättning som ett nationellt intresse, utan motsvarande skyldigheter gentemot andra grupper.
\end{itemize}

Kritik mot lagen har framförts av såväl israeliska som internationella rättsorganisationer, däribland \textit{Adalah}, \textit{B’Tselem} och \textit{Human Rights Watch}, vilka samstämmigt pekat ut lagstiftningen som systematiskt diskriminerande. Den utgör enligt dessa bedömningar ett konstitutionellt uttryck för apartheid enligt internationell rätt, särskilt i ljuset av FN:s apartheiddefinition (International Convention on the Suppression and Punishment of the Crime of Apartheid, 1973).

\subsubsection*{Konsekvens i rättsligt perspektiv}

Symbolisk representation – såsom en arabisk domare i Högsta domstolen – förändrar inte den strukturella exkluderingen i den rättsordning som grundlagen etablerar. Sådana exempel representerar snarare ett minoritetsundantag inom en överordnad juridisk struktur som bekräftar etnisk prioritet. Folkrätten kräver icke-diskriminering, jämlik representation och rätt till självbestämmande för alla folk – villkor som i dagsläget inte uppfylls inom den israeliska rättsordningen.

