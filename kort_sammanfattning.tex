\section{Sammanfattning}
\subsection*{Citat från Ehud Olmert fastställer att regeringen är i ond tro – \textit{dolus in pacta}}


Se citat från f.d. premiärminister Olmert: \url{https://zionism.observer/}

\begin{quote}
\small
“What we are doing in Gaza is a war of extermination: the indiscriminate, unrestrained, brutal, and criminal killing of civilians.

We are doing this not because of an accidental loss of control in a particular sector, not because of a disproportionate outburst of fighters in some unit — but as a result of a policy dictated by the government, knowingly, intentionally, viciously, maliciously, recklessly. Yes, we are committing war crimes.”\\
\hfill – Ehud Olmert, f.d. premiärminister (Haaretz, 22 maj 2025)
\end{quote}

\noindent
Detta citat från en av Israels tidigare regeringschefer gör det omöjligt för konventionsstater att fortsätta hävda ovetskap, rättslig osäkerhet eller diplomatiskt tolkningsutrymme.

\noindent
Den svenska regeringens fortsatta passivitet och användning av begrepp som “vi inväntar domstolarnas bedömning” utgör mot denna bakgrund \textbf{ond tro vid fullgörande av internationellt avtal – \textit{dolus in pacta}.}

\noindent
Detta är \textbf{inte längre en fråga om vårdslöshet i traktattolkning}, utan om \textbf{avsiktlig borttolkning av konventionens syfte} i syfte att undandra sig det ansvar som följer av artikel I i FN:s folkmordskonvention.

\bigskip
\subsection*{ Statligt uppsåt att orsaka lidande för att tvinga befolkningsflykt}

Israels strategi i Gaza är inte ett svar på ett militärt hot – den är ett medvetet försök att tvinga en civilbefolkning att lämna sin mark genom \textbf{avsiktligt tillfogat massivt lidande}.

\begin{quote}
“The goal is to make life so miserable that they will leave. To do that, they need to kill a significant number of people.”\\
\hfill – Prof. John Mearsheimer \url{https://www.youtube.com/shorts/ygN3lXNWf6w}
\end{quote}

\noindent
Israels sittande premiärminister (efterlyst av ICC) utryckte liknande för 20 år sedan
\url{https://www.sott.net/article/283603-Netanyahu-caught-on-tape-in-2001-America-is-a-thing-you-can-move-very-easily-in-the-right-direction}

\noindent
Detta mål kräver en ständig demonisering av palestinier, för att rättfärdiga uttalanden som i andra fall skulle lagföras som uppvigling till folkmord:

\begin{itemize}
  \item En israelisk läkare har liknat dödandet av palestinier vid att “eliminerar kackerlackor”: \url{https://www.middleeasteye.net/news/israeli-doctor-compared-killing-palestinians-gaza-eliminating-cockroaches}
  \item 100 israeliska läkare har undertecknat en petition som uppmanat till bombningar av Gazas sjukhus: \url{https://www.middleeasteye.net/news/israel-palestine-war-doctors-call-gaza-hospitals-bombed}
\end{itemize}

\noindent
Trots att ständig uppvigling pågår från parlamentariker och höga befattningshavare så har ingen har åtalats.\\

Anledningen är enkel: \textbf{Straff skulle \enquote{sända fel signaler}}, och detta riskerar att underminera moral och eldkraft i det pågående uttunningskriget. Samtidigt krävs det att Israel tillhandahåller tillräckligt med \textbf{\enquote{fikonlöv}} för att västvärldens regeringar inte ska tvingas införa sanktioner.\\

Detta är en balansakt Israel befinner sig i:
\begin{itemize}
\item \textbf{Minskar man våldet för mycket minskar \enquote{motivationen} att fly Gaza.}
\item \textbf{Går man för långt blir det svårt för t.ex. Sverige att rättfärdiga sin tystnad.}
\end{itemize}

\noindent
Ett exempel på misslyckad balans är avrättningen av 15 ambulansarbetare som återfanns bakbundna i massgrav. Deras ambulanser hade krossats. De var själva ute för att söka kollegor som tidigare dödats.\\
\url{https://www.middleeasteye.net/news/new-video-evidence-disputes-israeli-armys-account-medic-killings}\\

\noindent
När dessa avrättningar blev svåra att dölja – då en smartphoneinspelning återfanns i massgraven – valde Israel att \textbf{inte lagföra, utan enbart avskeda} ansvarig officer. Det var ett \enquote{klumpigt misstag}, eftersom det lämnade Sveriges regering \textbf{utan något att peka på} för att motivera fortsatt stöd. Inte ens en symbolisk påföljd utdelades – \textit{inte ens en vristsmäll.}\\

\noindent
Detta följs av ett sällsynt tydligt erkännande från Israels f.d. premiärminister \textbf{Ehud Olmert} i Haaretz (22 maj 2025):\\

\begin{quote}
\bfseries
What we are doing in Gaza is a war of extermination: the indiscriminate, unrestrained, brutal, and criminal killing of civilians.\\

We are doing this not because of an accidental loss of control in a particular sector, not because of a disproportionate outburst of fighters in some unit — but as a result of a policy dictated by the government, knowingly, intentionally, viciously, maliciously, recklessly. Yes, we are committing war crimes.
\normalfont
\end{quote}

\noindent
Ett sådant erkännande från en tidigare premiärminister gör att Sveriges regering \textbf{inte längre kan gömma sig bakom begrepp som \enquote{proportionalitet}, \enquote{intention}, eller \enquote{krigets dimma}}. Utrymmet för \enquote{diplomatiskt tvivel} är eliminerat.

\newpage
\subsection*{Regeringens vägran att agera – ett brott mot folkmordskonventionen}

Regeringen har i skriftliga svar till svenska medborgare framhållit följande:
\begin{quote}
\noindent
\textit{“Hur regler respekteras och om krigsförbrytelser begåtts måste bedömas från fall till fall utifrån den internationella humanitära rätten.}\\
\textit{Det är inte regeringens roll att göra sådana bedömningar. För regeringen är det centralt att eventuella överträdelser utreds och att ansvarsutkrävande säkerställs.}\\
\textit{Både ICC och ICJ har pågående utredningar om situationen i Palestina.}
\textit{Hittills har de kommit fram till att Israel måste göra mer för att skydda den civila befolkningen.}\\
\textit{Det är något regeringen också har framfört till Israel.”}\\
\textit{– Utrikesdepartementets brevsvar (maj 2025)}
\end{quote}

\noindent
Denna hållning är oförenlig med Sveriges skyldigheter enligt \textbf{FN:s folkmordskonvention (1948, artikel I)}, som uttryckligen ålägger konventionsstater en \textbf{positiv skyldighet att förebygga och förhindra folkmord} – inte en passiv roll att “avvakta rättsligt avgörande”.

\noindent
Att vänta med åtgärder tills ICC eller ICJ “eventuellt” fastslår brott utgör därför ett självständigt konventionsbrott i sig. Det är ett \textbf{de facto-medgivande av fortsatt straffrihet för folkmordsavsikt.}\\

\noindent
Detta är särskilt anmärkningsvärt med tanke på att:
\begin{itemize}
\item Internationella domstolen (ICJ) i juli 2024 \textbf{förklarade Israels ockupation av Gaza och Västbanken som olaglig och konstituerande apartheid},

\item Hamas redan 2017, i sin officiella plattform, \textbf{erkände FN:s säkerhetsrådsresolution 242} och därmed uttryckligen accepterade en tvåstatslösning inom 1967 års gränser,

\item medan Israel å sin sida \textbf{vägrar erkänna resolution 242 som bindande} och \textbf{villkorar varje framtida tillämpning av den med att palestinierna först erkänner Israel som etnisk statsbildning – ett krav som saknar juridisk grund}.
\end{itemize}

\noindent
Det är viktigt att tydliggöra att \textbf{UNSC 242 är internationell rätt – inte ett förhandlingsförslag. Den gäller oavsett om Israel erkänner den eller ej, och är inte beroende av känslomässiga eller ideologiska anspråk}.\\

 \noindent
Att i detta läge fortsätta upprepa frasen \textit{“Israel har rätt till självförsvar, men måste följa folkrätten”} är \textbf{inte ett balanserat uttalande}, utan en \textbf{aktiv förvanskning av rättsläget} – som riskerar att tolkas som \textbf{stämpling till krigsförbrytelser}.

\bigskip
\subsection*{Underlåtenhet som självständigt brott enligt ICJ}

Internationella domstolen slog i fallet Bosnia and Herzegovina v. Serbia and Montenegro (ICJ, 2007) fast:\\

\begin{quote}
\noindent
\textit{“A State’s failure to act when it is in a position to prevent genocide, and is aware, or should normally have been aware, of the serious risk of genocide, constitutes a breach of its obligations under the Genocide Convention.”}
\end{quote}

\noindent
Det är alltså ett folkrättsbrott att \textbf{inte agera}, när risken för folkmord är allmänt känd.\\

\noindent
Sveriges regering vet. Den har sett:

\begin{itemize}
 \item Bezalel Smotrich tala om att “utplåna Gaza”
 
 \item Moshe Feiglin kalla varje baby i Gaza “en fiende”
 
\item Ehud Olmert skriva i Haaretz att Israel “för ett utrotningskrig” och “begår krigsbrott”
\end{itemize}

\noindent
Detta är inte spekulationer – det är \textbf{offentliga erkännanden från högsta nivå.}

\subsection*{Inget självförsvar på ockuperat territorium}
Enligt gällande folkrätt – bekräftad av ICJ – har en ockupationsmakt \textbf{ingen rätt till självförsvar mot den befolkning den själv kontrollerar}. FN-stadgans artikel 51 kan inte åberopas mot en civil population inom ockuperat område. Ockupanten har endast \textbf{ett skyddsansvar} – inte en rätt att föra krig.

\subsection*{Ingen rätt till självförsvar på olagligt ockuperat territorium}
När ockupationen dessutom har förklarats olaglig – såsom ICJ gjorde i sitt rådgivande yttrande 2024 – innebär varje försök att åberopa självförsvar inom detta område en \textbf{rättsvidrig förlängning av ett folkrättsbrott}.\\

\noindent
Den svenska regeringens återkommande hänvisning till att “Israel har rätt att försvara sig” inom Gaza, trots att Gaza är ett olagligt ockuperat territorium, är därför \textbf{inte bara juridiskt felaktig} – utan ett \textbf{uttryck för stämpling till brott}.\\

\noindent
Detta är \textbf{inte en neutral formulering}. Det är \textbf{en diplomatisk täckmantel för en förbjuden maktutövning}, som ytterligare legitimerar fortsatt folkrättsbrott i strid med folkmordskonventionen.


\bigskip
\subsection*{23 kap. 6 § BrB – Stämpling till krigsförbrytelser}
Enligt \textbf{23 kap. 6 § brottsbalken} är det straffbart att \textit{“i samråd med annan besluta att ett visst brott ska utföras”}, även om brottet aldrig verkställs. Det gäller även \textbf{råd, stöd, samförstånd och uppmaningar}.

\begin{quote}
När Sveriges regering – trots ICJ:s dom – offentligt uttalar att Israel har rätt till självförsvar inom Gaza, i strid med gällande folkrätt, trots kännedom om pågående krigsförbrytelser och folkrättsbrott, kan detta juridiskt betraktas som \textbf{stämpling till krigsförbrytelser}.
\end{quote}

\noindent
Det är exakt jämförbart med:
\begin{quote}
\textit{Om jag säger till dig: “Du har rätt att skjuta grannen om han genar över din gräsmatta”, och du därefter gör det – då har jag stämplat till brott genom att aktivt förvränga rättsläget och uppmuntra till våld.}
\end{quote}

\noindent
Ett reellt exempel är \textbf{utrikesministerns uttalande juni 2025} om att Israel hade “rätt att borda fartyget \textit{Madleen} på internationellt vatten”. Det signalerade till blockadmakten: \textit{“Vi kommer inte att ingripa.”}\\

\noindent
Det är denna \textbf{fikonlövsdiplomati} som möjliggör fortsatt massdödande, med svensk tystnad som indirekt garant.

\bigskip
\subsection*{Regeringens språkbruk möjliggör fortsatt folkmord}
När Sveriges regering hellre skyddar sig från risken att bli kallad antisemitisk än att skydda en ockuperad civilbefolkning från våld – då är det inte längre fråga om diplomati. Det är \textbf{etiskt, juridiskt och konstitutionellt svek}.

\noindent
Att “vänta och se” om domstolar senare kommer kalla detta för folkmord är \textbf{inte neutralitet} – det är \textbf{underlåtenhet att förhindra}, vilket i sig är ett brott enligt konventionen.
