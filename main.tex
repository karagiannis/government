\documentclass[12pt]{article}
\usepackage[utf8]{inputenc}
\usepackage[T1]{fontenc}
\usepackage[swedish]{babel}
\usepackage{csquotes}
\usepackage[a4paper,margin=2.5cm]{geometry}
\usepackage{setspace}
\usepackage{lmodern}

% Radbrytbara URL:er i fotnoter och text
\PassOptionsToPackage{hyphens}{url}
\usepackage{hyperref}
\hypersetup{
  colorlinks=true,
  linkcolor=blue,
  citecolor=blue,
  urlcolor=blue,
  pdfborder={0 0 0}
}

\usepackage[most]{tcolorbox}

% För att göra URL:er mindre och radbrytbara i fotnoter
\usepackage{footmisc}
\renewcommand*{\UrlFont}{\ttfamily\footnotesize}
\makeatletter
\g@addto@macro{\UrlBreaks}{
  \do\.\do\-\do\_\do\/\do\?\do\&\do\=\do\%
}
\makeatother

% Egna kommandon för juridiska lagrum
\newcommand{\lagrum}[1]{\par\vspace{3mm}\textit{#1}\par\vspace{5mm}}
\newcommand{\lagrumsinline}[1]{\textit{#1}}

\newcommand{\utlandslagrum}[3]{%
  \vspace{3mm}%
  \noindent\textit{#1}\footnote{\url{#2}#3}\\[0.5ex]
}

\title{\textbf{Konstitutionell anklagelse mot Sveriges regering:\\
Brott mot folkrätt och grundlag i hanteringen av Israels krigsförbrytelser}}

% Författare kan läggas till här vid behov
% \author{Lasse Karagiannis}
\date{\today}

\begin{document}
\maketitle
\thispagestyle{empty}
\vspace{1cm}

\section{Syfte och adressater}
\label{sec:syfte}

% Detta inledande avsnitt introducerar adressaterna och det konstitutionella syftet.

Detta dokument är en formell framställan riktad till:

\begin{itemize}
  \item Justitiedepartementet och justitieministern,
  \item Utrikesdepartementet och utrikesministern,
  \item Statsrådsberedningen, såsom samordnande instans för regeringens politik,
  \item Justitieombudsmannen (JO),
  \item Konstitutionsutskottet (KU),
  \item Högsta domstolen (HD).
\end{itemize}

\begin{spacing}{1.2}
Syftet är att formellt uppmärksamma och rättsligt analysera den svenska regeringens agerande i relation till dess skyldigheter enligt:

\begin{itemize}
  \item \textit{Regeringsformen (RF) 1 kap. 10 § – om respekt för internationella åtaganden,}
  \item \textit{FN-stadgan – särskilt artikel 2 om våldsförbud och artikel 1 om fredliga syften,}
  \item \textit{Konventionen om förebyggande och bestraffning av brottet folkmord (1948).}
\end{itemize}

\vspace{0.5cm}
\textit{
Framställan aktualiserar frågan om statligt ansvar för passivitet och medverkan i folkrättsbrott – särskilt i ljuset av pågående brott mot mänskligheten i Gaza och regeringens val att avstå från tydlig rättslig och moralisk positionering.
}

\vspace{0.5cm}
\textit{
Eftersom Konstitutionsutskottets uppgift är att pröva huruvida regeringen eller enskilda statsråd har brutit mot grundlagen, och eftersom en sådan prövning ytterst kan kräva rättslig prövning av Högsta domstolen enligt 13 kap. 3 § regeringsformen, tillställs denna skrivelse även Högsta domstolen.
}

\vspace{0.5cm}
\textit{
Denna framställan bör därför behandlas som en konstitutionell angelägenhet av särskild vikt.
}
\end{spacing}

\newpage
\tableofcontents
\newpage

% Huvudsektionerna inkluderas här i önskad pedagogisk ordning


%filnamn: anklagelse_main.tex
% Huvudpåstående: att Sveriges regering är juridiskt ansvarig enligt folkrätt och grundlag

% Denna sektion bör etablera själva grunden för åklagarskriften: att regeringen inte bara varit passiv utan genom konkludent handlande och aktiv samverkan brutit mot rättsligt bindande normer.

% Förenklad och optimerad struktur

\section{Regeringens ansvar – åtalspunkt enligt folkrätt och grundlag}

% filnamn: 1_sammantattande_skuldsats.tex
% SYFTE: Ställ hypotesen klart. "Regeringen är skyldig." – detta ska bevisas.
% Introducera grundlagen och folkrätten som rättskällor. Skapa förväntan.

\subsection{Regeringens brott mot folkrätten och grundlagen}

\noindent
Sveriges regering har, i sin hantering av Israels agerande i Palestina, genom både underlåtenhet och aktivt agerande brutit mot såväl internationella rättsförpliktelser som konstitutionella normer. Ansvaret omfattar fyra distinkta, men inbördes sammanhängande rättsöverträdelser:

\begin{enumerate}
    \item \textbf{Underlåtenhet att förebygga folkmord} i strid med artikel I i folkmordskonventionen
    \item \textbf{Brott mot FN-stadgans artikel 56} genom passivitet i främjandet av mänskliga rättigheter
    \item \textbf{Lojalitetsbrott mot RF 10 kap. 1 §} genom kontraktsbrott mot folkrättsliga åtaganden
    \item \textbf{Stämpling till krigsbrott enligt 23 kap. 6 § brottsbalken}, genom felaktiga offentliga uttalanden om att "Israel har rätt att försvara sig" på ockuperat territorium i strid med etablerad folkrätt och ICJ:s utlåtanden
\end{enumerate}

\lagrum{Artikel I, Folkmordskonventionen\quad De fördragsslutande parterna förbinder sig att förebygga och bestraffa folkmord.}

\noindent
ICJ:s praxis i målet \textit{Bosnien mot Serbien} (ICJ Reports 2007, §430) fastslår att denna skyldighet:

\begin{itemize}
    \item är \textit{erga omnes} – gäller gentemot hela det internationella samfundet,
    \item aktiveras vid konstaterad risk för folkmord – inte först vid fullbordat brott,
    \item innefattar en plikt att vidta rimliga åtgärder för att förebygga folkmord.
\end{itemize}

\medskip

\lagrum{Artikel 56, FN-stadgan\quad Medlemsstaterna förbinder sig att vidta gemensamma och enskilda åtgärder [...] för främjande av mänskliga rättigheter.}

\noindent
Sveriges passivitet står i skarp kontrast till de \textit{officiella varningar} som utgått från:

\begin{itemize}
    \item FN:s särskilde rapportör om folkmordsförebyggande (A/HRC/55/73, 25 mars 2024),
    \item ICC:s åklagare, som uttalat att skälig grund för folkmordsutredning föreligger,
    \item ICJ:s beslut om preliminära åtgärder i målet \textit{Sydafrika mot Israel} (januari 2024).
\end{itemize}

\medskip


\lagrum{10 kap. 1 § Regeringsformen (RF)\quad Överenskommelser med andra stater [...] ingås av regeringen.}
\noindent
Detta konstitutionella mandat innefattar en underförstådd \textit{lojalitetsplikt} att säkerställa efterlevnaden av traktatsförpliktelser. Motsvarande folkrättslig princip återfinns i \textit{pacta sunt servanda} (Wienkonventionen om traktaträtten, artikel 26), vilken av Internationella domstolen (ICJ) har bekräftats som sedvanerätt i målet \textit{Gabčíkovo–Nagymaros} (ICJ Reports 1997, s. 38).

\medskip

\lagrum{23 kap. 6 § brottsbalken\quad Den som underlåter att i tid anmäla eller annars avslöja ett förestående eller pågående brott ska [...] dömas för underlåtenhet att avslöja brott [...] Den som har ett bestämmande inflytande i en sammanslutning [...] ska dömas för underlåtenhet att förhindra brott.}

\noindent
Den svenska regeringens felaktiga och vilseledande offentliga uttalanden – där man påstår att ”Israel har rätt att försvara sig” – utgör inte endast en folkrättslig förvanskning, utan kan även kvalificeras som \textit{stämpling till krigsbrott}. Detta gäller särskilt i ljuset av Internationella domstolens (ICJ) förklaring i juli 2024 att ockupationen av Gaza och Västbanken är olaglig, samt upprepade varningar från FN-organ om överhängande folkmordsrisk.

\noindent
Regeringens vägran att vidareförmedla eller agera på tidigare premiärminister Ehud Olmerts uttalanden – där han själv erkänner att Israel bedriver ett utrotningskrig – samt att inte ansluta sig till Sydafrikas folkmordsstämning mot Israel, kan därmed aktualisera ansvar enligt 23 kap. 6 § första stycket brottsbalken, om underlåtenhet att avslöja ett förestående eller pågående brott.


\vspace{0.5cm}

\noindent
\textbf{Slutsats:} Genom att:

\begin{enumerate}
    \item konsekvent vägra erkänna folkmordsrisken trots överväldigande bevisläge,
    \item fortsätta tillhandahålla politiskt, ekonomiskt och militärt stöd till Israel,
    \item aktivt motverka internationella sanktionsåtgärder,
    \item offentligt legitimera ett pågående krigsbrott genom juridiskt felaktiga uttalanden,
\end{enumerate}

har Sveriges regering uppfyllt rekvisiten för \textit{medverkan till folkmord} enligt artikel III(e) i folkmordskonventionen samt brutit mot Regeringsformens krav på lojal och rättstrogen uppfyllelse av internationella överenskommelser.

 % Hypotes

% filnamn: 2_folkrattens_grundprinciper.tex
%
% SYFTE:
% Etablera det rättsliga ramverk som styr staternas agerande under väpnad konflikt och ockupation.
% Presentera folkrättens grundprinciper och identifiera de skyldigheter som följer av dessa.
% Klargöra vad som utgör en laglig respektive olaglig ockupation, och vad detta medför för rättigheter och skyldigheter.
% Förklara varför vissa folkrättsliga normer är absoluta (t.ex. våldsförbudet, skyddsansvaret, folkmordsförbudet).
% Denna fil nämner inte stater vid namn – analysen sker på principnivå och är tillämplig oavsett aktörer.
%
% Nästa avsnitt (fil 3) tillämpar dessa principer konkret på Sveriges agerande i relation till Palestina och Israel.


\subsection{Folkrättens bindande ramverk}

Folkrätten utgör det överstatliga regelverk som styr staternas inbördes relationer. 
Den vilar på tre grundprinciper: staters suveränitet, territoriell integritet och den 
bindande karaktären hos internationella åtaganden. Dessa principer är kodifierade genom FN-stadgan, 
sedvanerätt samt multilaterala traktater såsom Genèvekonventionerna och folkmordskonventionen.

Sverige är folkrättsligt bundet dels genom ratificering av sådana traktater, dels genom sitt 
medlemskap i Förenta nationerna, vilket innebär särskilda skyldigheter enligt FN-stadgans kapitel I.

\lagrum{Artikel 2(1), FN-stadgan\quad The Organization is based on the principle of the sovereign equality of all its Members.}

\lagrum{Artikel 2(4), FN-stadgan\quad All Members shall refrain in their international relations from the threat or use of force against the territorial integrity or political independence of any state.}

Dessa artiklar fastslår ett generellt våldsförbud – en grundpelare i modern folkrätt. Endast två undantag erkänns: åtgärder godkända av säkerhetsrådet enligt kapitel VII, samt rätten till självförsvar enligt artikel 51 – vilket förutsätter ett väpnat angrepp mot en suverän stat.

\lagrum{Artikel 51, FN-stadgan\quad Nothing in the present Charter shall impair the inherent right of individual or collective self-defence if an armed attack occurs against a Member of the United Nations.}

Begreppet ”självförsvar” är strikt begränsat i både territoriellt och subjektivt hänseende. Det kan inte åberopas av en ockupationsmakt mot den befolkning den själv kontrollerar. Denna princip har bekräftats av Internationella domstolen (ICJ) i bland annat det rådgivande yttrandet \textit{Legal Consequences of the Construction of a Wall in the Occupied Palestinian Territory} (2004).

\medskip

Ockupation regleras av 1907 års Haagreglemente och 1949 års fjärde Genèvekonvention. 
Enligt dessa rättskällor bär ockupationsmakten ett positivt skyddsansvar gentemot civilbefolkningen i 
det ockuperade territoriet. Reglerna syftar till att förhindra det mönster vi i dag bevittnar: 
systematiskt övervåld, kollektiva bestraffningar och åsidosättande av civila skyddsintressen.

\lagrum{Artikel 43, Haagreglementet\quad The authority of the legitimate power having in fact passed into the hands of the occupant, the latter shall take all the measures in his power to restore, and ensure, as far as possible, public order and safety, while respecting, unless absolutely prevented, the laws in force in the country.}

\lagrum{Artikel 27, Genèvekonvention IV\quad Protected persons are entitled, in all circumstances, to respect for their persons, their honour, their family rights, their religious convictions and practices, and their manners and customs. They shall at all times be humanely treated, and shall be protected especially against all acts of violence or threats thereof.}

\medskip

Frågan om statligt ansvar aktualiseras även vid underlåtenhet att agera. 
Folkmordskonventionen föreskriver att konventionsstaterna har en självständig skyldighet 
att förebygga folkmord – en skyldighet som gäller oberoende av om säkerhetsrådet 
eller ICJ har gjort en formell bedömning.

Sverige är som konventionsstat därmed förpliktat inte bara att avstå från att själv begå folkmord, 
utan också att aktivt verka för att förebygga sådana brott. Denna skyldighet innefattar förbud mot stöd, 
legitimering eller tyst acceptans av handlingar som sannolikt strider mot konventionen.

\medskip
to reaffirm faith in fundamental human rights, in the dignity and worth of the human person, in the equal rights of men and women and of nations large and small,

Sammanfattningsvis är följande rättsprinciper avgörande för den fortsatta analysen:

\begin{itemize}
  \item folkrättens våldsförbud är absolut och medger endast snävt definierade undantag,
  \item rätten till självförsvar enligt artikel 51 kan inte åberopas av en ockupationsmakt 
  mot civilbefolkningen i det ockuperade territoriet,
  \item en ockupationsmakt har ett förstärkt skyddsansvar enligt Genèvekonventionen,
  \item konventionsstater enligt folkmordskonventionen har en aktiv skyldighet att förebygga 
  folkmord, även när brott begås av tredje stat,
  \item underlåtenhet att agera kan, under vissa omständigheter, grunda internationellt rättsligt ansvar.
\end{itemize}

Följande avsnitt visar att Sveriges regering har åsidosatt dessa principer i strid med 
såväl internationell som nationell rätt.
 % Allmän rättsram

\input{sections/anklagelse_section/3_Sveriges_folkrättsliga_skyldigheter_och_ansvar} % Sveriges specifika skyldigheter

% filnamn: regeringens_officiella_positioner.tex
% SYFTE: Visa exakt vad som sagts – skapa koppling till brottsrekvisit.
% Fungerar som "bevisupptagning" i ett åtal: citat, datum, kontext.



\subsection{Regeringens officiella positioner: innehåll och rättslig betydelse}

Den 27 maj 2025 offentliggjorde Utrikesdepartementet ett uttalande med anledning av att Israels ambassadör i Stockholm kallats upp:

\begin{quote}
\textit{”Uppkallandet gjordes för att upprepa och inskärpa regeringens krav på Israels regering att omedelbart säkerställa säkert och obehindrat humanitärt tillträde till Gaza. [...] Det sätt som kriget nu bedrivs på är oacceptabelt. [...] Terroristorganisationen Hamas ansvar för den aktuella situationen väger tungt.”}
\end{quote}

Utrikesdepartementet anger även:
\begin{quote}
\textit{”Israel har rätt att försvara sig. Den rätten måste utövas i enlighet med folkrätten.”}\\
(Källa: \url{https://www.regeringen.se/pressmeddelanden/2025/05/uttalande-fran-utrikesdepartementet-med-anledning-av-uppkallande-av-israels-ambassador/})
\end{quote}

Detta följer på en debattartikel den 23 maj 2025 undertecknad av fyra statsråd:

\begin{quote}
\textit{”Att blockera mat och annat bistånd till civila är oförsvarligt. [...] Samtidigt har Israel rätt att försvara sig – men den rätten måste utövas i enlighet med folkrätten.”}\\
(Källa: \url{https://www.regeringen.se/artiklar/2025/05/debattartikel-regeringen-okar-pressen-mot-israel/})
\end{quote}

\medskip

Detta språkbruk är rättsligt problematiskt på flera punkter.

För det första används begreppet \textit{självförsvar} i strid med folkrättens tillämplighet. Israel är en ockupationsmakt, inte en stat i självförsvar. FN-stadgans artikel 51 får inte åberopas mot en civilbefolkning som står under ens kontroll, vilket fastslogs i ICJ:s rådgivande yttrande om \textit{muren i Palestina} (2004). Regeringens formulering är därmed inte neutral – den innebär rättslig desinformation och legitimerar folkrättsbrott.

\medskip

För det andra: språket åtföljs inte av några faktiska rättsliga eller politiska åtgärder. Det saknas:

\begin{itemize}
  \item sanktioner eller restriktioner i handeln,
  \item avbrutna diplomatiska förbindelser,
  \item deltagande i rättsliga processer, t.ex. Sydafrikas ICJ-talande mot Israel.
\end{itemize}

Utrikesdepartementets språkbruk stannar vid en retorisk apparat – en markering utan påföljd. Detta är i strid med den \textit{positiva handlingsskyldighet} som följer av artikel I i folkmordskonventionen, vilken ålägger konventionsstater att förhindra brottet i den mån det står i deras makt.

\medskip

Enligt ICJ:s dom \textit{Bosnia v. Serbia} (2007) är det otillräckligt att ”fördöma” eller ”vädja”. Staten måste agera genom att:

\begin{itemize}
    \item frysa relationer,
    \item isolera den misstänkta gärningsmannen,
    \item delta i rättsliga åtgärder.
\end{itemize}

Sveriges regering har inte uppfyllt någon av dessa skyldigheter.

\medskip

\subsubsection{Regeringens kännedom om Gazas folkrättsliga status}

Det är vidare ostridigt att Gazas de facto-regering – Hamas – sedan 2006 i handling och från 2017 även i skrift, 
uttryckligen godtagit en tvåstatslösning inom 1967 års gränser. Policydokumentet från 2017, samt tidigare initiativ 
såsom \textit{Prisoners’ Document} (2006), \textit{Mecka-avtalet} (2007) och \textit{Försoningsöverenskommelsen} 2020, 
utgör dokumenterade steg mot en politisk lösning inom ramen för folkrätten.\footnote{Se t.ex. \url{https://en.wikipedia.org/wiki/Hamas_Covenant\#2017_document} och \url{https://en.wikipedia.org/wiki/2020_Palestinian_reconciliation_agreement}}

Detta är välkänt inom internationell diplomati och forskning. Professor Neve Gordon och professor Menachem Klein 
har i flera publikationer visat att denna linje var genuin och syftade till att uppnå internationell legitimitet. 
Enligt Klein underminerades dessa initiativ systematiskt av Israel för att bibehålla narrativet om en frånvarande fredspartner.

Det är ett känt faktum att Hamas accepterar FN:s säkerhetsrådsresolution 242, som anger 1967 års gränser, medan Israel inte gör det.

Mot denna bakgrund saknar regeringens retorik om Israels självförsvar förankring i faktisk eller rättslig verklighet. 
Det utgör i stället en politisk konstruktion med allvarliga rättsliga konsekvenser.

Professor Menachem Klein, en av Israels ledande experter på konflikten och tidigare rådgivare i förhandlingar med PLO, 
konstaterar att Hamas i Fatah-Hamas-överenskommelsen från 2021 uttryckligen förband sig till internationell rätt, 
erkände PLO:s auktoritet, accepterade en stat inom 1967 års gränser med Östra Jerusalem som huvudstad, 
och åtog sig att föra en fredlig kamp.%
\footnote{Klein, M. (2023). \textit{Israeli arrogance thwarted a Palestinian political path}. +972 Magazine. \url{https://www.972mag.com/hamas-fatah-elections-israel-arrogance/}} 
Hamas avstod dessutom från att ställa upp med en egen presidentkandidat, i syfte att bana väg för ett demokratiskt mandat 
till en gemensam palestinsk ledning.

Detta är inte en obekräftad tolkning utan en dokumenterad och känd del av den diplomatiska processen – erkänd av 
bland andra EU och USA under Bidenadministrationen, som mottog överenskommelsen i syfte att stödja val.

Att i detta sammanhang fortsätta beskriva Israel som en part som agerar i självförsvar mot Gaza är en förvanskning 
av den faktiska rättspositionen. Hamas har deklarerat vilja att följa internationell rätt. Israel däremot, som ockupationsmakt, 
har avvisat både val, FN-resolutioner och fredsinitiativ – och brutit mot Genèvekonventionerna. 
Regeringens påstående saknar därför såväl faktisk som folkrättslig grund, och innebär i praktiken ett rättsligt vilseledande 
från en konventionsstat, med följd att ansvar enligt BrB 23 kap. eller artikel III i folkmordskonventionen 
kan aktualiseras.

\medskip

\subsubsection{Språklig tyngdpunktsförskjutning: från legalitet till proportionalitet}

Enligt internationell rätt har Israel inte rätt till självförsvar gentemot den ockuperade palestinska befolkningen. 
Detta följer av Internationella domstolens (ICJ) rådgivande yttrande från 2004, där murens konstruktion på palestinskt område 
förklarades strida mot internationell rätt, samt av ICJ:s rådgivande yttrande den 17 juli 2024, där hela den israeliska 
ockupationen av de palestinska territorierna – med hänvisning till FN:s säkerhetsrådsresolution 242 – bedömdes vara olaglig. 

Trots detta fortsätter den svenska regeringen att upprepa formuleringen att ”Israel har rätt att försvara sig”, 
utan att samtidigt erkänna att Israel är en ockupationsmakt vars rättigheter är strikt begränsade enligt folkrätten. 
Ett sådant språkbruk förskjuter tyngdpunkten i den rättsliga diskussionen: 
från att gälla om våld överhuvudtaget är tillåtet – till att handla om huruvida våldet är ”proportionerligt”.

Med andra ord: där noll skott vore rättsligt tillåtet, flyttas samtalet till huruvida tusen eller tiotusen skott är rimligt. 
Regeringens retorik bidrar därmed till en allvarlig normförskjutning med potentiellt rättsligt ansvar för medverkan 
till folkrättsbrott. 

Att i detta sammanhang ändå åberopa självförsvarsrätten utgör inte endast en felaktig rättstillämpning – det är ett 
normativt stöd till folkrättsbrott. 
När detta sker i officiella uttalanden från en konventionsstat, aktualiseras frågan om \textit{konkludent handlande} 
enligt folkmordskonventionen och svensk straffrätt.


\subsubsection{Ytterligare brott: underlåtenhet att avslöja pågående folkrättsbrott}

Med denna retoriska förskjutning som fond blir regeringens underlåtenhet att vidta konkreta åtgärder särskilt allvarlig. 
Svenska strafflagens 23 kap. 6 § BrB föreskriver ansvar för den som underlåter att i tid avslöja ett förestående eller pågående brott:

\lagrum{23 kap. 6 § 1 st. BrB\quad Den som underlåter att i tid anmäla eller annars avslöja ett förestående eller pågående brott ska, i de fall det är särskilt föreskrivet, dömas för underlåtenhet att avslöja brottet enligt vad som är föreskrivet för den som medverkat endast i mindre mån.}

Med kännedom om det pågående mönstret av brottslighet i Gaza – dokumenterat av FN-organ, ICC, ICJ och civilsamhällesaktörer – 
föreligger en rättslig skyldighet att agera. Uteblivna sanktioner, rättsliga åtgärder eller andra former av press utgör inte 
bara en underlåtenhet att agera – utan, i detta sammanhang, en underlåtenhet att avslöja brott.

När detta dessutom åtföljs av ett språkbruk som döljer den verkliga rättsliga kontexten – och felaktigt hänvisar 
till en icke tillämplig rätt till självförsvar – förstärks den rättsliga betydelsen. 
Det utgör en passiv medverkan, eller i vissa tolkningar, en \textit{stämplingsliknande handling} i folkrättslig mening.

\medskip

\subsubsection{Stämpling till folkrättsbrott genom vilseledande legitimering}

När den svenska regeringen i officiella uttalanden påstår att ”Israel har rätt att försvara sig” – trots att detta 
enligt folkrätten inte gäller en ockuperande makt – utgör det inte enbart en felaktig analys. 
Det är ett vilseledande rättsligt påstående med potentiellt brottsbefrämjande verkan.

I svensk rätt är stämpling ett förberedelsebrott där gärningsmannen uppmuntrar eller förstärker någon annans brottsavsikt. 
Analogt gäller här: genom att påstå att ett olagligt agerande är rättfärdigat enligt internationell rätt, 
ger Sverige ett rättsligt stöd till fortsatt våld, vilket i sig kan bidra till brottets fortsättning.

Jämför: den som, felaktigt och med auktoritet, säger till någon att denne ”har rätt att skjuta inkräktare” trots 
att det enligt gällande rätt inte föreligger någon sådan rätt – begår en form av stämpling, 
om detta får till följd att brott begås.

Internationella domstolen (ICJ) har i två rådgivande yttranden – 2004 och 2024 – fastställt att Israel är ockupationsmakt 
och att blockaden av Gaza utgör en folkrättsstridig kollektiv bestraffning. I detta rättsläge är 
självförsvarsrätten enligt FN-stadgans artikel 51 inte tillämplig gentemot Gaza. 

Detta gäller särskilt eftersom Gazas lagligt valda de facto-regering uttryckligen erkänt internationell rätt, inklusive 
Säkerhetsrådets resolutioner såsom 242, medan Israel konsekvent har avvisat både FN-beslut och folkrättens begränsningar.

Att i detta sammanhang ändå tillerkänna Israel en rätt till självförsvar är inte endast ett rättsligt felsteg, 
utan en aktiv vilseledning med potentiellt brottsbefrämjande effekt. Det förskjuter diskussionen från legalitetsfrågan till en proportionalitetsbedömning – ett felgrepp med normförändrande verkan.
Därigenom normaliseras ett våldsutövande som enligt folkrätten aldrig borde ha påbörjats.

När detta sker i ett läge där:\\
- det finns överväldigande dokumentation om pågående folkrättsbrott, och\\
- gärningsmannen (Israel) står i fortsatt militärt interagerande med civilbefolkningen,\\

...kan regeringens uttalande inte enbart förstås som passivitet. 
Det är en aktiv handling med potentiellt uppviglande karaktär – det närmar sig *stämpling* till folkrättsbrott.

\subsubsection{Stämpling till folkrättsbrott genom vilseledande demonisering}

Regeringen skriver: \enquote{Mer än ett år efter Hamas fruktansvärda terroristattacker den 7 oktober 2023}.\footnote{\url{https://www.regeringen.se/regeringens-politik/med-anledning-av-situationen-israelpalestina/vad-regeringen-gor-med-anledning-av-kriget-mellan-israel-och-hamas/}}

Hur har regeringen fastställt att det rör sig om terroristattacker som Hamas som organisation bär skuld för?

\paragraph{}
Hamas politiska ledning – som utgör Gazas folkvalda regering – har inte förnekat att civila israeler dödats den 7 oktober 2023, men har förklarat att sådana handlingar inte beordrats av den centrala ledningen. De ska i stället ha utförts av odisciplinerade element, andra fraktioner eller civila utom kontroll i en kaotisk situation. Denna distinktion har ignorerats av den svenska regeringen, som utan juridisk prövning tillmätt det israeliska narrativet full trovärdighet. Därmed förvägras Gazas representanter en grundläggande rättsstatlig princip: att handlingar ska bedömas individuellt och att skuld inte får kollektiviseras – särskilt inte under ockupation. En sådan asymmetrisk normtillämpning innebär att självförsvarsrätten fråntas den förtryckta parten, samtidigt som den förbehålls ockupationsmakten.

\paragraph{}
Den svenska regeringens språkbruk avslöjar en djup normativ asymmetri: det som för Israel är \enquote{misstag}, är för Gaza \enquote{terrorism}. Denna begreppsanvändning är inte neutral, utan bygger på en rasifierad och kolonial tolkningsstruktur där den ockuperade betraktas som kollektivt ansvarig, medan ockupationsmakten ursäktas med begrepp som \enquote{rätt till självförsvar} eller \enquote{komplex säkerhetssituation}.

\begin{itemize}
\item Den palestinska regeringen hålls ansvarig även för handlingar begångna av okontrollerade grupper eller civila i ett kaosartat läge,
\item medan Israel tillåts skjuta raketer mot sjukhus, avrätta medicinsk personal och begrava kroppar i massgravar – utan rättsliga följder, så länge det finns en narrativ reserv (”operationellt misstag”).
\end{itemize}

Hamas har i internationella intervjuer (t.ex. BBC 2023-11-07) tillstått att civila israeler dödats, men förklarat att detta inte varit målet. Beordringar från den militära ledningen ska uttryckligen ha undantagit kvinnor, barn och gamla från angrepp. Det finns också andra väpnade fraktioner i Gaza som inte lyder under Hamas’ politiska ledning. Men detta juridiska faktum – att ansvar måste individualiseras – nonchaleras systematiskt av västliga regeringar när det gäller palestinska aktörer.

\paragraph{}
Resultatet blir en språklig stämpling där palestinier som kollektiv förlorar rätten till självförsvar, samtidigt som deras våld reduceras till ett kulturellt problem. I praktiken förväntas en förtryckt och instängd befolkning upprätthålla full kontroll och juridisk disciplin, medan en teknologiskt överlägsen ockupationsmakt tillåts begå systematiska övergrepp utan ansvar – eftersom den anses vara ”civiliserad”.

\paragraph{}
Detta är inte bara en moralisk skandal, utan också en rättslig förskjutning där internationell humanitär rätt tillämpas selektivt – i strid med Genèvekonventionernas krav på likabehandling av civilbefolkningar, och ICJ:s påpekande (Bosnia v. Serbia, §430–432) att skuld inte kan härledas kollektivt, särskilt inte i asymmetri mellan ockupant och ockuperad.

\paragraph{}
Samtidigt accepterar den svenska regeringen utan prövning Israels påståenden om att läkare, sjukvårdsarbetare, ambulansförare och journalister varit \enquote{terrorister} – även i fall där massgravar avslöjat avrättningar av civila med bakbundna händer\footnote{\url{https://en.wikipedia.org/wiki/Gaza_Strip_mass_graves}} \footnote{\url{https://www.middleeasteye.net/news/new-video-evidence-disputes-israeli-armys-account-medic-killings}}, där ambulanser förstörts och samtliga sjukhus bombats\footnote{\url{https://en.wikipedia.org/wiki/Attacks_on_health_facilities_during_the_Gaza_war}}. Läkare sitter fortsatt fängslade och flera har torterats till döds.\footnote{\url{https://www.middleeasteye.net/news/war-gaza-prominent-palestinian-doctor-tortured-and-killed-israeli-detention}}

\paragraph{}
Israels försvar – att sådana dödsfall varit \enquote{operationella misstag} – accepteras utan vidare. Men varför godtas inte även Gazas regerings förklaringar? Om samma måttstock tillämpades borde även deras utsagor tillmätas trovärdighet.
Regeringen förhåller sig dessutom tyst till de uppgifter som publicerats i israelisk press, 
enligt vilka IDF bekräftat att ett s.k. \enquote{mass Hannibal event} genomfördes den 7 oktober, 
då israelisk militär öppnade eld mot civila israeler i syfte att förhindra att dessa togs som 
gisslan av Hamas.\footnote{\url{https://electronicintifada.net/blogs/asa-winstanley/we-blew-israeli-houses-7-october-says-israeli-colonel}} Dessa uppgifter behandlas mer ingående längre fram i dokumentet.

\paragraph{}
Hamas har själv uppgett att den 7 oktober utgjorde en legitim militär operation riktad mot israeliska militära mål, 
och detta har också bekräftats av internationellt respekterade militära bedömare såsom 
Scott Ritter.\footnote{\url{https://scottritter.substack.com/p/the-october-7-hamas-assault-on-israel}} 

\paragraph{}
Ett av Hamas deklarerade mål var att befria tusentals palestinier som hålls i israeliskt förvar utan rättegång, 
i strid med folkrätten, genom så kallad ”administrativ internering”.\footnote{\url{https://www.btselem.org/administrative_detention/statistics}}

Genom att kategoriskt och utan åtskillnad beteckna alla väpnade palestinska aktörer som ”terrorister” 
bidrar regeringen till en normativ förskjutning som suddar ut folkrättens tydliga skillnad mellan legitim väpnad kamp och folkrättsbrott. 
Hamas, som utgör Gazas folkvalda regering och säger sig erkänna folkrätten, 
har vid flera tillfällen arresterat medlemmar av andra väpnade grupper – såsom Islamiska Jihad 
och salafistiska fraktioner – för oauktoriserade raketattacker 
mot Israel.\footnote{Se t.ex. Haaretz (2015-05-27): \textit{Hamas Arrests Islamic Jihad Activists for Rocket Fire}.\url{https://www.haaretz.com/2015-05-27/ty-article/.premium/hamas-arrests-islamic-jihad-activists-for-rocket-fire/0000017f-e9cd-df2c-a1ff-ffdd5ef50000}}

Trots folklig förankring buntar regeringen samman Hamas med de grupper Hamas själv betraktar som kriminella – inklusive en ISIS-associerad 
knarksmugglande klan i södra Gaza, som Israels premiärminister Netanyahu 
bekräftat att Israel beväpnar för att bekämpa Hamas.\footnote{Se \textit{The Grayzone} (2025-06-05): \textit{Israel arming ‘ISIS-affiliated’ gang in southern Gaza}, \url{https://thegrayzone.com/2025/06/05/israel-arming-isis-gang-gaza/}.} Var är regeringens fördömande av att Israel samarbetar med sådana aktörer?

Att systematiskt underlåta att erkänna den mångfald av väpnade aktörer i Gaza – samt den 
interna repression Hamas riktar mot gangsters, extremistiska eller odisciplinerade grupper – leder till en orättvis 
kollektiv demonisering av hela det palestinska motståndet. 

Regeringens retorik beskriver alla väpnade palestinska grupper som ”terrorister”, oavsett om de är folkvalda eller 
agerar inom folkrättens ramar. Samtidigt har Hamas – Gazas folkvalda regering – konsekvent fördömts, medan motsvarande israeliska policy inte ens nämns av regeringen.
Denna förvrängning strider mot neutralitetsprincipen i folkrätten, vilken kräver att neutrala stater förhåller sig 
objektiva och faktabaserade i konfliktsituationer.

Denna form av sammangående demonisering, utan grund i rättslig bedömning och utan kravet att erkänna fler 
aktörer eller en mångfasetterad verklighet, representerar ett brott mot Genèvekonventionernas förbud mot kollektiv 
bestraffning. Det aktualiserar även principerna i Nürnbergstadgan, särskilt principerna I och VI, där indirekt stöd 
genom propaganda eller legitimering bör, mot bakgrund av Nürnbergprinciperna och Genèvekonventionerna, 
betraktas som en form av medverkan enligt internationell straffrätt.

På samma sätt beskriver regeringen Hezbollah som en ”terroristorganisation”, trots att Hezbollah utgör en omfattande 
politisk och social rörelse med djupt folkligt stöd i Libanon, och ingår i landets regeringskoalition. 
Under flera perioder har rörelsen haft parlamentarisk majoritet.

Att enskilda medlemmar inom en folkförankrad befrielserörelse begår handlingar som kan klassificeras som 
terrorbrott innebär inte att hela organisationen därmed juridiskt definieras som en terroristisk aktör. 
Om regeringen drar sådana generaliserande slutsatser utan nyansering, aktualiseras frågan om kollektiv skuldbeläggning – något 
som enligt folkrätten är förbjudet. Det skulle i förlängningen innebära att hela etniska eller nationella grupper 
riskerar att demoniseras, något som står i direkt strid med principen om individuellt ansvar i internationell rätt.

Sådana handlingar faller inom ramen för medverkan till aggressionsbrott och brott mot mänskligheten, enligt princip VI (c) i Nürnbergstadgan, 
vilket också omfattar propaganda och offentlig legitimering av handlingar som strider mot folkrätten.

Genom att använda statliga uttryck som vilseleder, rättfärdigar eller anonymiserar brottsliga handlingar, bidrar regeringen till en 
rättslig stämpel – en funktionell form av stämpling – snarare än en oskyldig politisk ton. 
Det kan karakteriseras som en form av stämpling till folkrättsbrott, eftersom det ges politisk legitimitet 
till en ockupationsmäktig makt som bryter mot grundläggande folkrättsliga normer.
Mot denna bakgrund framstår regeringens kategoriska språkbruk som en form av stämpling till 
folkrättsbrott – inte genom direkt deltagande i våldshandlingar, utan genom att ge politisk och retorisk 
täckning för handlingar som i internationell rätt saknar legitimitet.\footnote{Princip VI(c) i Nürnbergstadgan omfattar även medverkan i brott mot mänskligheten, inklusive att genom offentlig kommunikation uppmana till eller rättfärdiga sådana handlingar. Se även: \textit{Charter of the International Military Tribunal – Annex to the Agreement for the prosecution and punishment of the major war criminals of the European Axis} (1945), artikel 6(c).}




\subsubsection{Regeringens vägran att stödja vapenvila – uttryck för konkludent medgivande}

Sveriges agerande inför och efter FN:s generalförsamlings omröstning om vapenvila i december 2023 tydliggör en strategiskt betingad vägran att fullt ut fördöma Israels krigföring, trots då redan dokumenterade civila massdöd\footnote{\url{https://euromedmonitor.org/en/article/5908/Israel-hits-Gaza-Strip-with-the-equivalent-of-two-nuclear-bombs}}.

Utrikesminister Tobias Billström deklarerade att det inte vore “rättvist” att kräva vapenvila eftersom “Israel måste kunna bekämpa Hamas” – en formulering som tillför ett rättsligt undantag till själva principen om eldupphör. Stödet för FN-resolutionen kom sent och var uppenbart taktiskt: Sveriges ja-röst åtföljdes av fortsatta uttalanden om Israels rätt att genomföra militära operationer, vilket urholkar både den rättsliga och moraliska effekten av röstningen.

\begin{quote}
\textit{”Vi kan inte mana till en vapenvila som skulle innebära att Israel inte kan bekämpa Hamas. Det vore inte rättvist.”} \\
– Tobias Billström till TT, 2023-12-11 (Aftonbladet)\footnote{\url{https://www.aftonbladet.se/nyheter/a/0QR2eE/billstroms-ord-om-vapenvila-i-gaza-far-stark-kritik}}
\end{quote}

Detta utgör ett skolboksexempel på \textit{läpparnas bekännelse}: en diplomatisk markering som i sak neutraliseras av reservationer som möjliggör fortsatt våld. Juridiskt innebär det ett underminerande av skyldigheten att agera preventivt enligt folkmordskonventionens artikel I, samt en vilseledande kommunikation med potentiellt brottsbefrämjande effekt.

\textbf{Rättslig följd:} Regeringens uttalanden kombinerar:
\begin{itemize}
  \item en formell eftergift (sen ja-röst),
  \item med ett materiellt medgivande till fortsatt bombkampanj (”Israel måste kunna bekämpa Hamas”),
  \item vilket i folkrättslig mening utgör \textit{konkludent medgivande} till folkrättsbrott.
\end{itemize}

Enligt ICJ i \textit{Bosnia v. Serbia} (2007, §432) kan detta kvalificeras som:
\begin{itemize}
  \item underlåtelse att använda inflytande för att stoppa brottet,
  \item språkligt och diplomatiskt agerande som legitimerar fortsatt våld,
  \item indirekt uppmuntran genom rättslig och politisk vilseledning.
\end{itemize}


\subsection*{Utrikesdepartementets svar: Strategisk passivitet förklädd till neutralitet}

Den svenska regeringen erkänner i sitt svar till en svensk medborgare att "ett stort antal civila dödsoffer inklusive barn" har förekommit i Gaza, och att situationen är "djupt oroande".

Som stöd för vår analys av regeringens retorik och rättsliga undandraganden återges här ett officiellt mejlsvar från Utrikesdepartementet. Skrivelsen utgör ett bevis för hur regeringen – trots kännedom om omfattande civila dödsoffer – medvetet undviker att vidta några rättsliga eller diplomatiska åtgärder mot Israel, och istället reducerar sin hållning till "uppmaningar" och vaga hänvisningar till EU-samverkan.

\begin{quote}
\begin{verbatim}
From: UD MENA Brevsvar <ud.mena.brevsvar@gov.se>
Date: Mon, Apr 14, 2025 at 10:31 AM 
Subject: Sv: [Pchr_english] Israel Intensifies Airstrikes... 
To: lasse.l.karagiannis@gmail.com

Hej,

Tack för din e-post som inkommit till Utrikesdepartementet. 
Jag vill inleda med att be om ursäkt för att du har fått vänta på svar. 
Vi får in många brev till departementet just nu, vilket har fört med sig 
att vi har långa handläggningstider.

Regeringen ser allvarligt på att stridigheterna har återupptagits. 
Uppgifterna om ett stort antal civila dödsoffer, inklusive barn, 
är djupt oroande.

Regeringen kräver en omedelbar vapenvila 
och uppmanar parterna att återgå till förhandlingarna 
så att all kvarvarande gisslan kan släppas, 
det humanitära tillträdet säkerställas 
och ett varaktigt slut på stridigheterna uppnås.

Kraven på respekt för folkrätten, inklusive den internationella 
humanitära rätten, har varit – och fortsätter att vara – 
ett av regeringens nyckelbudskap.

Dessa budskap framförs i våra egna kontakter med Israel 
och vi gör det tillsammans med andra EU-länder och likasinnade. 
Ibland sker det offentligt, och ibland på annat sätt.

Hur regler respekteras och om krigsförbrytelser begåtts 
måste bedömas från fall till fall utifrån 
den internationella humanitära rätten.

Det är inte regeringens roll att göra sådana bedömningar. 
För regeringen är det centralt att eventuella överträdelser 
av den internationella humanitära rätten 
och möjliga krigsförbrytelser utreds 
och att ansvarsutkrävande säkerställs.

Både ICC och ICJ har pågående utredningar 
om situationen i Palestina. 
Hittills har de kommit fram till 
att Israel måste göra mer för att skydda 
den civila befolkningen. 
Det är något regeringen också har framfört till Israel.

Med vänlig hälsning,
Mellanöstern- och Nordafrikaenheten 
Utrikesdepartementet 
103 39 Stockholm 
\end{verbatim}
\end{quote}

\vspace{1em}

Detta dokument visar att regeringen, trots erkännande av civila dödsoffer, medvetet undviker att själv göra 
rättsliga bedömningar, vilket strider mot dess skyldighet enligt artikel I i folkmordskonventionen att 
agera preventivt. Denna passivitet möjliggör fortsatt straffrihet och utgör en form av konkludent 
medverkan genom underlåtenhet.

Trots detta:
\begin{itemize}
\item kräver man inte ansvar från Israel,
\item avstår från att uttala sig om huruvida krigsbrott eller folkmord begåtts,
\item och överlåter all juridisk bedömning till internationella domstolar, utan att själv agera i enlighet med folkmordskonventionens bindande artikel I, som föreskriver att alla stater måste förebygga och förhindra folkmord, inte passivt invänta domstolsbeslut.
\end{itemize}

I mejlsvaret heter det:

\begin{quote}
”Hur regler respekteras och om krigsförbrytelser begåtts måste bedömas från fall till fall... Det är inte regeringens roll att göra sådana bedömningar.”
\end{quote}

Detta är felaktigt i folkrättslig mening. Enligt ICJ:s dom i *Bosnia v. Serbia* (2007, §§430–431) har alla stater ett 
självständigt ansvar att \textbf{förebygga, förhindra och stävja} folkmord så fort de rimligen kunnat förutse att sådana 
brott kan äga rum. Den svenska regeringen avstår här från att fullgöra sin \textbf{positiva förpliktelse enligt artikel I i 
folkmordskonventionen} och gömmer sig istället bakom en retorik om neutralitet.

Samtidigt visar regeringen genom sin vapenviljeuppmaning att man mycket väl har kapacitet att påverka konflikten. 
Men genom att kräva att båda parter ska "återgå till förhandlingar", utan att erkänna den asymmetri som råder 
mellan en instängd, bombad civilbefolkning och en modern militärmakt med flyg och tanks, skapas en \textbf{falsk ekvivalens} 
mellan angripare och offer.




\subsubsection{Hälsningssignaler till blockaden – bordning av \textit{Madleen} och drönarattacken mot \textit{Conscience}}

Den svenska regeringens agerande i dessa två incidenter – bordningen av \textit{Madleen} den 9 juni 2025 och attacken mot \textit{Conscience} den 2 maj 2025 – är juridiskt betydelsefullt på flera punkter:

\begin{enumerate}
  \item \textbf{Bombningen av \textit{Conscience}} bekräftades i riksdagen, men regeringen valde att inte framföra någon protest mot Israels militära agerande – trots att attacken skedde i internationellt vatten mot ett civilt fartyg med humanitär last. Det föreligger således en aktiv tystnad som kan tolkas som \emph{konkludent medgivande} till brott som strider mot sjöfartens frihet och civil skyddad status.
  
  \item \textbf{Bordningen av \textit{Madleen}} följdes av utrikesminister Maria Malmer Stenergärds uttalande om att Israel med hänvisning till folkrätten "har vissa möjligheter att eskortera skeppet". Detta ger militära definitioner rättslig legitimitet och bör betraktas som en del av en normativ förskjutning där civilt sjöfartsutrymme under humanitära aktioner reduceras till säkerhetszon.
\end{enumerate}

\noindent
\textbf{Rättslig betydelse:}
\begin{itemize}
  \item \emph{Konkludent samtycke} till blockadens rättsvidrighet då Sverige inte motsatte sig attackerna, vilket underlättar fortsatt folkrättsvidrig praxis på internationellt vatten.
  \item Språkliga legitimationsreserver för angripande stat – när det uttryckligen anges att ”Israel har möjligheter…”.
  \item Detta mönster följer en trend: sjukhus attackeras (över 660 hälsocenter), läkare torteras eller dödas, massgravar framträder – men svenska regeringsuttalanden blundar. Dessa hamnar alla i samma mönster av tyst acceptans för brott mot Genèvekonventionerna.  
\end{itemize}

\noindent
Utifrån praxis i ICJ (*Bosnia v. Serbia*, §432) innebär:

\begin{itemize}
  \item \emph{Underlåtenhet att protestera} – vilket kan ses som medverkan genom att legitimera fortsatta brott,
  \item \emph{Informellt stöd} via regeringsuttalanden som skapar en ram där våld på humanitärt uppdrag anses acceptabelt,
  \item Förstärkning av narrativ där civila offer antingen ignoreras eller sägs stå i vägen – snarare än att skyddas.
\end{itemize}

\noindent
Resultatet: Sveriges regering har, genom språk och undvikande agerande, medverkat till – eller åtminstone inte hindrat – uppbyggnaden av ett folkrättsstridigt regeringsmönster i internationellt vatten.




\subsubsection{Folkrättslig passivitet som medverkan}

Enligt artikel I i folkmordskonventionen är varje stat förpliktad att inte bara avstå från att begå folkmord, utan även att aktivt förhindra sådana brott inom ramen för sin möjlighet. Detta bekräftades i ICJ:s dom \textit{Bosnia v. Serbia} (2007), där domstolen slog fast att stater har en faktisk och preventiv handlingsskyldighet.

Regeringens underlåtenhet att:

\begin{itemize}
    \item frysa bilaterala relationer, inklusive diplomatiska besök och samarbetsavtal,
    \item införa ekonomiska och militära sanktioner, däribland vapenembargo,
    \item ansluta sig till internationella rättsprocesser såsom ICJ-målet initierat av Sydafrika,
\end{itemize}


...innebär ett kontraktsbrott mot Sveriges traktatförpliktelser enligt folkmordskonventionen och FN-stadgan. Det kan också utgöra medverkan till fortsatt folkrättsbrott, i den mån passiviteten sker med kännedom om risk och utan proportionalt motverkande åtgärder.

\medskip

\textbf{Slutsats:} Regeringens uttalanden den 23 och 27 maj 2025 utgör inte endast politiska markörer. De är rättsligt relevanta dokument som – mot bakgrund av regeringens faktiska handlingsvägran – får betraktas som konkludent medgivande till ett pågående folkrättsbrott, och i förlängningen en möjlig form av medverkan enligt svensk och internationell rätt.


 % Fullständig dokumentation (valfri)

%filnamn: kondenserad_sammanfattning_officiella_positioner.tex
%SYFTE: Att kompakt sammanfatta allt i regeringens_officiella_positioner.tex för maximal överblick

%%Sammanfattning Sammanfattning av regeringens uttalanden såsom rättsliga handlingar
\subsection{Sammanfattning av regeringens uttalanden såsom rättsliga handlingar}
\label{subsec:regeringens_positioner}

\subsubsection{Kritiska uttalanden: maj 2025}
\begin{tabular}{p{0.12\textwidth}p{0.85\textwidth}}
\textbf{Datum} & \textbf{Uttalande \& rättslig kontext} \\
\hline
23 maj & \textit{Debattartikel:} "Israel har rätt att försvara sig – men den rätten måste utövas i enlighet med folkrätten" \\
& \footnotesize\textit{Rättslig kontext:} ICJ rådgivande yttrande 2004 (muren) \& 2024 (ockupationen) fastslår att artikel 51 FN-stadga ej tillämplig \\
\hline
27 maj & \textit{UD-uttalande:} "Terroristorganisationen Hamas ansvar för den aktuella situationen väger tungt" \\
& \footnotesize\textit{Rättslig kontext:} Bröt mot neutralitetsprincipen (Hagakonventionen 1907) och principen om individuellt ansvar (Nürnbergartikel 6) \\
\end{tabular}

\subsubsection{Rättslig desinformation om självförsvarsrätten}
\begin{itemize}
\item \textbf{Faktamässig felaktighet}: Israel är ockupationsmakt (ICJ 2004 \& 2024), inte angripen stat
\item \textbf{Rättslig konsekvens}: Skapar normativ förskjutning från \textit{legalitet} till \textit{proportionalitet}
\item \textbf{Jämförelse}:Analogt med att hävda att en fängelsevakt har rätt att försvara sig mot en intagen — trots att denne har kontroll över situationen, skyldighet att värna liv och tillgång till överlägsna medel.

Självförsvarsrätten existerar, men den är strikt reglerad. En vakt som skjuter en obeväpnad ungdomsbrottsling i huvudet efter ett inbrott, när faran inte längre är överhängande, kommer inte att frias.

\item \textbf{Jämförelse:} På samma sätt kan en ockupationsmakt inte åberopa självförsvarsrätt mot en befolkning som den själv:
  \begin{itemize}
    \item förvägrar en politisk uppgörelse enligt FN:s säkerhetsrådsresolution 242,
    \item håller långvarigt instängd och isolerad,
    \item kontrollerar genom militär dominans,
    \item bär ett juridiskt skyddsansvar gentemot.
  \end{itemize}

\end{itemize}

\subsubsection{Selektiv demonisering som folkrättsbrott}
\textbf{Asymmetrisk tillämpning av 'terrorist'-begreppet:}
\begin{itemize}
\item \textbf{Hamas}: Kategoriskt terroriststämpel trots:
  \begin{itemize}
  \item Folkvald regering (2006)
  \item Accept av FN-resolution 242 (1967-gränser)\footnote{\url{https://www.972mag.com/hamas-fatah-elections-israel-arrogance/}}
  \item Intern repression av extrema och kriminella grupper\footnote{\url{https://www.newsweek.com/hamas-arresting-and-torturing-jihadis-prevent-war-israel-752108}} \footnote{\url{https://www.middleeasteye.net/news/israel-palestine-hamas-arrests-two-rocket-fire}} \footnote{\url{https://www.israelnationalnews.com/news/166785}} \footnote{\url{https://www.israelnationalnews.com/news/259035}}
  \end{itemize}
\item \textbf{Israeliska aktörer}: Ingen kritik av:
  \begin{itemize}
  \item IDF:s "Hannibal-direktiv" 7 oktober 2023\footnote{\url{https://thegrayzone.com/2024/06/21/israeli-army-friendly-fire-october-7/}} \footnote{\url{https://thegrayzone.com/2025/02/25/bibas-israeli-govt-propaganda-hostage-killings/}} \footnote{\url{https://thegrayzone.com/2023/11/21/haaretz-grayzone-conspiracy-israeli-festivalgoers/}} \footnote{\url{https://thegrayzone.com/2023/10/27/israels-military-shelled-burning-tanks-helicopters/}}
  \item Samarbete med kriminella klaner (Grayzone 2025)\footnote{\url{https://thegrayzone.com/2025/06/05/israel-arming-isis-gang-gaza/}}
  \end{itemize}
\end{itemize}
\lagrumsinline{Genèvekonvention IV, artikel 33\quad Förbud mot kollektiv bestraffning}

\subsubsection{Underlåtenhetsansvar enligt svensk rätt}
\textbf{23 kap. 6 § BrB} aktualiseras genom:
\begin{enumerate}
\item Kännedom om systematiska folkrättsbrott (ICJ, ICC, FN-rapporter)
\item Underlåtenhet att vidta juridiskt mandaterade åtgärder
\item Samtidigt språkbruk som döljer brottens karaktär
\end{enumerate}

\subsubsection{Vilseledande legitimering som stämpling}
\begin{itemize}
\item \textbf{Rättslig grund}: Nürnbergprincip VI(c) om medverkan till brott mot mänskligheten
\item \textbf{Sveriges agerande}: Felaktig juridisk karaktärisering med brottsbefrämjande effekt
\item \textbf{Analog}: Att felaktigt hävda att någon har rätt att bruka våld
\end{itemize}

\subsubsection{Koppling till folkmordskonventionen}

\begin{tabular}{p{0.25\textwidth}p{0.7\textwidth}}
\textbf{Handling som krävts} & \textbf{Sveriges underlåtenhet} \\
\hline
Stöd till ICJ-processen & Ej anslutit sig till Sydafrikas talan \\
\hline
Vapenembargo & Fortsatt export av militärteknologi\footnote{\url{https://www.isp.se/internationella-sanktioner/}} \footnote{\url{https://proletaren.se/artikel/vapexporten-till-israel-okar/}} \\
\hline
Diplomatisk isolering & Besökt Israel av UD-tjänstemän \\
\end{tabular}



Enligt ICJ i \textit{Bosnia v. Serbia} (2007, §432) kan medverkan aktualiseras genom:
\begin{itemize}
\item Underlåtenhet att stoppa brottet när möjlighet finns
\item Bidragande handlingar som underlättar brottsligheten
\item Skapande av normativa ramar som legitimerar brott
\end{itemize}
Sveriges agerande uppfyller alla tre kriterier.



\subsubsection{Sammanfattande rättslig bedömning}
Regeringens uttalanden uppfyller rekvisiten för:
\begin{itemize}
\item \textbf{Medhjälp} enligt ILC:s Draft Articles on State Responsibility (art. 16)
\item \textbf{Underlåtenhet att avslöja brott} (BrB 23:6)
\item \textbf{Indirekt stämpling} till folkrättsbrott (Nürnbergartikel 6)
\end{itemize}

\section*{Rättslig grund för ansvar vid underlåtenhet att ingripa mot våld}

\subsection*{1. Underlåtenhet att hindra brott (Brottsbalken 23 kap.)}

Svensk straffrätt skiljer mellan:

\begin{itemize}
  \item \textbf{Förbiseende av tjänsteplikt} (tjänstefel, BrB 20 kap. 1 §),
  \item och \textbf{underlåtenhet att hindra brott} (BrB 23:6), om man har särskild rättslig förpliktelse att agera.
\end{itemize}

\textit{Exempel:} En polis som ser A misshandla B, men låter det ske, trots att han har makt och skyldighet att ingripa, kan hållas straffrättsligt ansvarig.  
Om B därefter slår tillbaka och dödar A i affekt – kan polisens underlåtenhet indirekt ha möjliggjort våldsspiralen.

\subsection*{2. Lojalitetsplikt och passivitetsansvar i internationell rätt (ICJ – Bosnia v. Serbia)}

ICJ slog fast att \textbf{underlåtenhet att skydda} (\textit{failure to prevent genocide}) är en självständig folkrättskränkning.  
Det gäller inte bara stater – utan i praktiken även myndighetsrepresentanter.

En regering (eller myndighetsperson) som har kunskap om ett pågående mönster av förtryck och inte agerar, bryter mot internationell rätt, särskilt om de är part till tvingande konventioner (\textit{jus cogens}).

\subsection*{3. Civilrättslig analogi: Culpa in omittendo (oaktsamhet genom underlåtenhet)}

I civilrätten har en part ansvar inte bara när man agerar skadligt, utan även när man inte agerar där man borde.

\textit{Exempel:} En fastighetsägare som ser att ett barn faller i en brunn på gården han ansvarar för, men inte ingriper, kan bli skadeståndsansvarig.

\textbf{Analogt:} En polis som upprepade gånger fått höra att ”B kommer att slå tillbaka om A inte stoppas”, men ändå inte ingriper – kan i moralisk och rättslig mening bära ansvar för hela kedjan av händelser.

\subsection*{4. Europeiska Människorättsdomstolen (EMD): skyldighet att skydda}

Staten (inkl. dess myndigheter och tjänstemän) har positiva skyldigheter enligt Europakonventionen, särskilt:

\begin{itemize}
  \item Artikel 2 – Rätt till liv
  \item Artikel 3 – Förbud mot tortyr och omänsklig behandling
\end{itemize}

\textbf{EMD-dom:} \textit{Opuz v. Turkey (2009)} – Turkiet fälldes för att ha inte skyddat en kvinna från en våldsam make, trots att hon gjort flera anmälningar.  
Hon slog sedan tillbaka. Staten ansågs ansvarig för både hennes och hans lidande.

\subsection*{5. Analogin: Konkludent anstiftan genom språklig legitimering}

När politiker eller statstjänstemän använder språk som:

\begin{itemize}
  \item \textit{”X har rätt att försvara sig”} trots övervåld
  \item \textit{”B får skylla sig själv”} trots att B är förtryckt
\end{itemize}

...kan det enligt folkrättsdoktrin tolkas som \textbf{legitimering av brott}, vilket liknar anstiftan eller uppvigling.

\subsection*{Sammanfattande tabell: Finns det rättslig grund för ansvar?}

\begin{tabular}{|p{5.5cm}|p{9cm}|}
\hline
\textbf{Situation} & \textbf{Rättslig grund för ansvar} \\
\hline
Polis eller myndighet som inte ingriper trots kännedom & Brottsbalken 20 kap. (tjänstefel), BrB 23:6 (underlåtenhet), Europakonventionen (artiklar 2 och 3), EMD-praxis \\
\hline
Staten som inte sätter press på angripare & ICJ-praxis (Bosnia v. Serbia), Folkmordskonventionen (artikel I), ILC:s artiklar om statsansvar \\
\hline
Språklig eller politisk legitimering av övervåld & Kan likställas med uppvigling eller anstiftan (BrB), samt brott mot folkrättsliga grundsatser, särskilt vid uppsåt \\
\hline
\end{tabular}
 % Juridisk analys + sammanfattning

%filnamn: slutpladering_anklagelse.tex
% SYFTE: Sammanfattande bevisning. "Regeringen är skyldig." genom att knyta samman lag och 
%regeringens underlåtenheter och andra felsteg.


%filnamn: slutpladering_anklagelse.tex
% SYFTE: Sammanfattande bevisning. "Regeringen är skyldig." genom att knyta samman lag och 
%regeringens underlåtenheter och andra felsteg.

%filnamn: slutpladering_anklagelse.tex
\subsection{Slutplädering: Regeringens juridiska ansvar är ovedersägligt}


Den samlade bevisningen visar klart och entydigt att Sveriges regering genom handling och underlåtenhet brutit mot bindande folkrättsliga och konstitutionella skyldigheter. Regeringens agerande uppfyller samtliga rekvisit för medverkan till folkrättsbrott enligt etablerad rättspraxis.

\subsection*{Sammanfattning av beviskedjan}
\begin{enumerate}
    \item \textbf{Legitimering av olagligt våld} \\
    Genom konsekvent felaktig tillämpning av artikel 51 FN-stadga har regeringen rättsligt legitimerat Israels agerande som "självförsvar" trots klart fastställd ockupationsstatus (ICJ 2004, 2024).
    
    \item \textbf{Selektiv tillämpning av rättsnormer} \\
    Regeringen har tillämpat "terrorist"-begreppet asymmetriskt genom att:
    \begin{itemize}
        \item Systematiskt demonisera Hamas trots dess folkvalda status och accepterande av FN-resolution 242
        \item Underlåta att granska IDF:s dokumenterade brott (Hannibal-direktivet, samarbete med kriminella klaner)
    \end{itemize}
    
    \item \textbf{Aktiv medverkan genom underlåtenhet} \\
    Tabell nedan sammanfattar centrala underlåtenheter som uppfyller rekvisiten för medhjälp enligt ILC:s artikel 16:
\end{enumerate}

\begin{table}[h]
\centering
\caption{Sveriges underlåtenheter och deras rättsliga konsekvenser}
\label{tab:underlatenhet}
\begin{tabular}{p{0.4\textwidth}p{0.55\textwidth}}
\textbf{Underlåten handling} & \textbf{Rättslig kvalifikation} \\ \midrule
Inget stöd till ICJ-processen & Underlåtenhet att förebygga folkmord (art. I Folkmordskonv.) \\
Fortsatt vapenexport & Medverkan till krigsförbrytelser (ILC art. 16) \\
Avsaknad av sanktioner & Brott mot neutralitetsplikt (Haagkonventionen) \\
\end{tabular}
\end{table}

\subsection*{Juridiska följdslut}
Med stöd i dokumenterad bevisning och etablerad rättspraxis konstateras:

\begin{itemize}
    \item Regeringen har \textbf{aktivt medverkat} till folkmord enligt artikel III(e) i folkmordskonventionen genom:
    \begin{itemize}
        \item Diplomatiskt stöd som underlättat fortsatt brottslig verksamhet
        \item Språkbruk som skapat normativa ramar för acceptans av brott
    \end{itemize}
    
    \item Regeringen har \textbf{brutit mot Regeringsformen} (1 kap. 10 §) genom:
    \begin{itemize}
        \item Underlåtenhet att lojalt fullgöra folkrättsliga förpliktelser
        \item Aktivt bidrag till urholkning av internationell rättsordning
    \end{itemize}
    
    \item Regeringen har \textbf{uppfyllt rekvisiten} för underlåtenhet att avslöja brott (BrB 23:6) genom:
    \begin{itemize}
        \item Kännedom om systematiska folkrättsbrott
        \item Avsaknad av tillräckliga åtgärder trots handlingsmöjligheter
    \end{itemize}
\end{itemize}

\subsection*{Krav på rättslig prövning}
På grundval av ovanstående yrkas:

\begin{itemize}
    \item Att \textbf{Konstitutionsutskottet} omedelbart inleder granskning av regeringens agerande ur statsrättsligt perspektiv
    
    \item Att \textbf{Justitieombudsmannen} granskar eventuella tjänstefel inom berörda departement
    
    \item Att \textbf{Utrikesdepartementet} omedelbart:
    \begin{itemize}
        \item Upphör all militär samverkan med Israel
        \item Ansluter sig till ICJ-processen mot Israel
        \item Inför omfattande sanktioner
    \end{itemize}
    
    \item Att \textbf{Högsta domstolen} prövar frågan om regeringens folkrättsliga ansvar enligt 13 kap. 3 § Regeringsformen
\end{itemize}

\textit{Denna slutplädering baseras på den samlade bevisningen i detta dokument och kräver omedelbar rättslig åtgärd.} % Avslutande krav




\begin{comment}
\section{Folkrättens bindande ramverk}

% Sammanfatta relevanta rättskällor: FN-stadgan, Folkmordskonventionen, Genèvekonventionerna, sedvanerätt.
% Förklara varför folkrätten inte kan kringgås av politiska hänsyn.
% Avsluta med Sveriges konstitutionella skyldigheter enligt Regeringsformen 10:1 och 13:3.


Regeringen skriver i sin kommuniké från den 27 maj 2025:\footnote{\url{https://www.government.se/statements/2025/05/statement-from-the-ministry-for-foreign-affairs-on-summoning-of-israeli-ambassador/}}

\begin{quote}
\textit{“When the Ambassador was summoned, it was stressed that Israel has the right to defend itself. However, that right must be exercised in accordance with international law.”}
\end{quote}

Vid en första anblick framstår detta som ett balanserat uttalande. Men den retoriska konstruktionen döljer en djupare förskjutning. Vad som presenteras som ett villkorat erkännande av folkrätten, fungerar i själva verket som en rättslig täckmantel för folkrättsbrott.

FN-stadgan artikel 51\footnote{\url{https://www.un.org/en/about-us/un-charter/full-text}} ger endast rätt till självförsvar vid ett väpnat angrepp – riktat mot en suverän medlemsstat. Rätten gäller alltså inte en ockupationsmakt som utövar kontroll över ett territorium och dess civilbefolkning.

\lagrum{Artikel 51, FN-stadgan\quad Nothing in the present Charter shall impair the inherent right of individual or collective self-defence if an armed attack occurs against a Member of the United Nations...}

Att hävda att Israel har rätt att försvara sig mot Gaza är därför inte enbart en politisk förenkling – det är en rättslig förfalskning av folkrättens grundprinciper. Det ger det strukturella våldet en rättslig fernissa, trots att folkrätten bygger på ansvar – inte på retoriska undantag.

Internationella domstolen (ICJ) fastslog den 19 juli 2024 att Israels ockupation av Gaza, Västbanken och östra Jerusalem är olaglig, och att den utgör apartheid.\footnote{\url{https://mondoweiss.net/2024/07/in-a-historic-ruling-icj-declares-israeli-occupation-unlawful-calls-for-settlements-to-be-evacuated-and-for-palestinian-reparations/}} Detta är inte en tolkning. Det är ett rättsligt bindande konstaterande.

Trots detta fortsätter den svenska regeringen att referera till Israels \enquote{rätt till självförsvar} som om Gaza vore en angripande stat, inte ett ockuperat territorium. Uttalandet neutraliserar den folkrättsliga kontexten och bekräftar därigenom en rättsvidrig världsbild.

Därmed gör sig regeringen skyldig till mer än en vilseledande formulering. Den medverkar – om än indirekt – till att undergräva den internationella rättsordningen. Och detta sker inte av juridisk okunskap, utan som del av en förutsägbar, politiskt motiverad ordkonstruktion:

\begin{quote}
Ett sätt att säga något – men samtidigt inte mena det.  
Detta är inte ansvarsutkrävande. Det är ett medvetet undandragande av ansvar, maskerat som neutral diplomati.
\end{quote}

\medskip

\textbf{Frågan är därför inte om Israel har rätt till självförsvar – utan var den rätten får utövas.}

För att bedöma giltigheten i regeringens påstående krävs en tydlig förståelse av folkrättens territoriella begränsning: självförsvar enligt FN-stadgan artikel 51 får endast utövas inom en stats eget internationellt erkända territorium. Ockupationsmaktens rättigheter är däremot reglerade av helt andra folkrättsliga instrument.


\section{Sveriges folkrättsliga ansvar}
% Redogör för Sveriges positiva skyldigheter enligt internationell rätt.
% Bevisa att tystnad, fortsatt vapenhandel eller selektiv diplomati utgör medansvar.
% Redogör för tidigare exempel där Sverige agerat kraftfullt mot folkrättsbrott (t.ex. Ryssland, Myanmar, Iran).

\subsection*{Israel saknar rätt till försvar utanför israelisk mark}
\addcontentsline{toc}{subsection}{Israel saknar rätt till försvar utanför israelisk mark}

Oavsett legaliteten i själva ockupationen kan en ockupationsmakt aldrig åberopa självförsvar mot den befolkning som står under dess kontroll. Enligt folkrätten får en ockupationsmakt endast vidta åtgärder som är strikt nödvändiga för att upprätthålla allmän ordning och säkerhet, i enlighet med artikel 43 i Haagreglementet:\footnote{\url{https://ihl-databases.icrc.org/en/ihl-treaties/hague-regulations-1899/article-43}}

\lagrum{Article 43\quad The authority of the legitimate power having in fact passed into the hands of the occupant, the latter shall take all the measures in his power to restore, and ensure, as far as possible, public order and safety, while respecting, unless absolutely prevented, the laws in force in the country.}

Samtidigt är ockupationsmakten skyldig att skydda civilbefolkningen från våld och hot enligt artikel 27 i den fjärde Genèvekonventionen:\footnote{\url{https://ihl-databases.icrc.org/en/ihl-treaties/gciv-1949/article-27}}

\lagrum{Article 27\quad Protected persons are entitled, in all circumstances, to respect for their persons, their honour, their family rights, their religious convictions and practices, and their manners and customs. They shall at all times be humanely treated, and shall be protected especially against all acts of violence or threats thereof.}

Därtill får ockupationsmakten inte ersätta det lokala rättssystemet, utan endast temporärt administrera det för upprätthållande av civil ordning, enligt artikel 64:\footnote{\url{https://ihl-databases.icrc.org/en/ihl-treaties/gciv-1949/article-64}}

\lagrum{Article 64\quad The penal laws of the occupied territory shall remain in force, with the exception that they may be repealed or suspended by the Occupying Power in cases where they constitute a threat to its security or an obstacle to the application of the present Convention...}

Att hävda rätt till militärt självförsvar inom ett territorium där man själv är ockupant innebär att man juridiskt förväxlar våld med skydd, och därigenom ger repressionen ett falskt moraliskt anspråk. Det är att kriminalisera motstånd – och samtidigt rättfärdiga fortsatt förtryck.

Självförsvar enligt FN-stadgans artikel 51 är territoriellt begränsat till internationellt erkänd, egen mark:

\lagrum{Article 51\quad Nothing in the present Charter shall impair the inherent right of individual or collective self-defence if an armed attack occurs against a Member of the United Nations...}

Detta innebär att självförsvar inte kan åberopas av en stat som befinner sig utanför sitt eget erkända territorium – särskilt inte mot den civilbefolkning som staten i egenskap av ockupationsmakt har skyldighet att skydda.

Detta är också exakt den folkrättsliga position Sverige intar gentemot Ryssland och som man framhåller som självklart i officiella sammanhang: Ryssland kan inte hävda självförsvar på ukrainsk mark, eftersom Ukraina inte är ryskt territorium.  

Att däremot medge Israel rätt till självförsvar på ockuperad palestinsk mark är ett flagrant avsteg från denna princip. Det är en inkonsekvent rättstillämpning – där likabehandlingsprincipen offras till förmån för politisk opportunism.


Att därtill tillfoga formuleringar som \enquote{men det måste ske i enlighet med internationell rätt} är utan betydelse. Ty det är själva användningen av begreppet \textit{självförsvar} i detta sammanhang som utgör ett folkrättsbrott.

När en regering offentligt erkänner rätten till självförsvar för en stat som agerar utanför sitt eget territorium – i strid med folkrätten – så förskjuts hela den rättsliga tyngdpunkten. Diskussionen handlar då inte längre om \textit{huruvida} våldet är tillåtet, utan \textit{hur mycket} våld som kan anses proportionerligt.

Detta medför två rättsliga och moraliska implikationer:

\begin{enumerate}
  \item \textbf{Konkludent handlande i folkrättslig mening.}  
  Genom att använda termen \enquote{självförsvar} utan förbehåll, agerar regeringen i strid med den folkrättsliga huvudprincipen att självförsvar enligt FN-stadgan endast får utövas på egen, internationellt erkänd mark.

  Det utgör ett tyst – och därmed konkludent – godkännande av ett pågående folkrättsbrott, såsom olaglig ockupation eller övervåld, eftersom formuleringen ger dessa handlingar en legal fernissa.

 \item \textbf{Medverkan eller stämpling i straffrättslig mening.}  
Om en regering genom officiella uttalanden uppmuntrar, normaliserar eller skyddar ett folkrättsbrott, kan detta – i ett internationellt rättsligt sammanhang – jämställas med det som i svensk straffrätt motsvarar stämpling till brott.

\lagrum{23 kap. 2 § 2 st BrB\quad  I de fall det särskilt anges döms för stämpling till brott. Med stämpling förstås att någon i samråd med någon annan beslutar gärningen eller att någon söker anstifta någon annan eller åtar eller erbjuder sig att utföra den.}

Ett exempel: Om Sveriges statsminister skulle säga \enquote{Ryssland har rätt att försvara sig på ukrainsk mark, men det måste ske enligt internationell rätt}, så har man redan legitimerat det första skottet. Diskussionen handlar därefter inte om \textit{att} Ryssland får använda våld – utan \textit{hur mycket} våld som är acceptabelt. På så sätt förskjuts skuldbedömningen från själva olagligheten till våldets omfattning, vilket innebär att man rättsligt och moraliskt gör sig medskyldig till brottet.
\end{enumerate}

\subsubsection*{Sammanfattning}
\noindent En ockupationsmakt har ingen rätt till självförsvar mot den befolkning som står under dess kontroll. Att erkänna sådan rätt är att förskjuta rättens tyngdpunkt från olagligt våld till "godtagbar nivå av våld", och därmed aktivt legitimera förtryck. 

Regeringens uttalanden står därmed i direkt motsättning till både Sveriges officiella folkrättsliga linje i andra konflikter och till grundläggande principer i internationell humanitär rätt.



\subsection*{Apartheid som brott mot mänskligheten enligt internationell rätt}
\addcontentsline{toc}{subsection}{Apartheid som brott mot mänskligheten enligt internationell rätt}

Trots att Internationella domstolen (ICJ) i sitt rådgivande yttrande från juli 2024 slår fast att Israels styre över de ockuperade palestinska områdena utgör apartheid i folkrättslig mening, har den svenska regeringen ännu inte uttalat sig om detta. Det är anmärkningsvärt.

Apartheid är inte en politisk etikett utan ett juridiskt begrepp med entydig definition i internationell rätt. Det klassas som ett brott mot mänskligheten och utgör därmed en av de allvarligaste kränkningarna av den folkrättsliga ordningen.

Hade motsvarande dom fastställt att någon annan stat – exempelvis Ryssland, Iran eller Syrien – tillämpade en regim av apartheid, hade detta sannolikt omedelbart föranlett politiska åtgärder, sanktioner och fördömanden. När det gäller Israel råder däremot en påfallande tystnad.

Detta kan jämföras med hur Sverige – med rätta – fördömde apartheidsystemet i Sydafrika under 1980-talet. Det är därför särskilt bekymmersamt att samma politiska krafter som då motsatte sig sanktioner mot Pretoria i dag återfinns i en regering som tiger om Tel Aviv.

Att upprätthålla en apartheidregim är inte bara en politisk fråga – det är ett folkrättsligt definierat brott mot mänskligheten. Detta har fastslagits i flera tvingande rättskällor, däribland:

\utlandslagrum{Romstadgan för Internationella brottmålsdomstolen (ICC), artikel 7(1)(j):}{https://www.icc-cpi.int/sites/default/files/RS-Eng.pdf}{, s. 10.}
\textit{For the purpose of this Statute, “crime against humanity” means any of the following
acts when committed as part of a widespread or systematic attack directed against any
civilian population, with knowledge of the attack:  [...] (j) The crime of apartheid;}

\vspace{4mm}

\utlandslagrum{Romstadgan, artikel 7(2)(h):}{https://www.icc-cpi.int/sites/default/files/RS-Eng.pdf}{, s. 12.}
\textit{“The crime of apartheid” means inhumane acts of a character similar to those referred to in paragraph 1, committed in the context of an institutionalized regime of systematic oppression and domination by one racial group over any other racial group or groups and committed with the intention of maintaining that regime;  }

\vspace{4mm}

\utlandslagrum{Internationella konventionen om avskaffande av alla former av rasdiskriminering (ICERD), artikel 3:}{https://www.regeringen.se/contentassets/87af45b9fb7f449a909b686204bb5527/fns-konventioner-om-manskliga-rattigheter/}{, s. 41.}
\textit{Konventionsstaterna fördömer särskilt rassegregation och apartheid och åtar sig att förhindra, förbjuda och utrota alla företeelser av detta slag inom områden under sin jurisdiktion.}

\vspace{4mm}

\utlandslagrum{Konventionen om avskaffande av all slags diskriminering av kvinnor (CEDAW):}{https://www.regeringen.se/contentassets/87af45b9fb7f449a909b686204bb5527/fns-konventioner-om-manskliga-rattigheter/}{, s. 51.}
\textit{De stater som har anslutit sig till denna konvention (konventionsstaterna) [...] betonar att utplånandet av apartheid, alla former av rasism, rasdiskriminering, kolonialism, nykolonialism, aggression, utländsk ockupation och utländskt herravälde samt av inblandning i staters inre angelägenheter är av största vikt för att män och kvinnor till fullo skall kunna åtnjuta sina rättigheter.}

\subsection*{Tystnaden inför apartheid är inte neutral}
\addcontentsline{toc}{subsection}{Tystnaden inför apartheid är inte neutral}

Alla former av affärstransaktioner och diplomatiska förbindelser bidrar till att upprätthålla en apartheidregim.  
Det som i dag erkänns av Internationella domstolen (ICJ) och av världens främsta folkrättsliga institutioner benämns ännu inte av Sveriges regering\footnote{\url{https://www.amnesty.org/en/latest/news/2022/02/israels-apartheid-against-palestinians/}}\footnote{\url{https://www.hrw.org/news/2021/04/27/israel-apartheid-against-palestinians}}.  

Tystnaden är inte neutral – den är ett aktivt val.



\subsection*{Regeringens selektiva moral och dubbla måttstockar}
\addcontentsline{toc}{subsection}{Regeringens selektiva moral och dubbla måttstockar}

\subsubsection*{Regeringens uttalanden och brist på åtgärder}
% Texten om statsministerns uttalande, från "Regeringen skriver vidare..." till "...fortsatt handel och diplomati som om inget hänt."

Regeringen skriver vidare:

\begin{quote}
\textit{“The terrorist organisation Hamas bears heavy responsibility for the current situation. The hostages must be released – unconditionally and immediately.”}
\end{quote}

Regeringen anser sig inta det moraliska överläget men har aldrig sanktionslagt Israel för dess folkrättsbrott. Ingen konsekvens har utdelats trots årtionden av övergrepp. Istället har Sverige fortsatt handel och diplomati som om inget hänt och har därmed inte gjort sin bekärda andel för upprätthållandet av internationell lag och ordning som skulle kunnat ge regeringen den moraliska höjden att utpeka en folkligt förankrad befrielserörelse såsom terroristorganisation – så till vida att denna inte beaktat och respekterat regeringens försök att upprätthålla internationell lag och ordning.

Detta är inte enbart en moralisk inkonsekvens. Det är ett folkrättsligt avtalsbrott i förhållande till Sveriges åtaganden enligt bland annat folkmordskonventionen, FN-stadgan och sedvanerättens tvingande normer.

Genom ett historiskt mönster av passivitet och konkludent handlande har Sverige inte bara underlåtit att agera – utan de facto bidragit till att ge Israel immunitet. Oavsett regim har Sverige fortsatt diplomatiskt och ekonomiskt stöd utan att utkräva ansvar för de upprepade folkrättsbrotten: illegala bosättningar, bombningar av civila och systematisk rasdiskriminering har ursäktats eller ignorerats.

Den som konsekvent tolererar statsterror förlorar rätten att fördöma motstånd mot den. Sverige har därmed spelat en \textit{aiding and abetting}-roll – och saknar rättslig trovärdighet när det gäller att peka ut andra för terror.

\subsubsection*{Regeringens underminering av folkrättens institutioner}
\addcontentsline{toc}{subsubsection}{Regeringens underminering av folkrättens institutioner}

Regeringen har konsekvent agerat för att skydda Israel från rättsligt ansvarsutkrävande. Den svenska statsministern har öppet opponerat sig mot Internationella brottmålsdomstolens (ICC) arresteringsorder mot Israels premiärminister – ett beslut grundat i omfattande bevisning om brott mot mänskligheten. Denna hållning utgör inte bara ett politiskt ställningstagande, utan en direkt underminering av det mest centrala internationella rättsliga verktyget för ansvar efter folkrättsbrott.

Samtidigt har regeringen ignorerat den samstämmiga bedömningen från världens främsta människorättsorganisationer\footnote{\url{https://www.amnesty.org/en/documents/mde15/8668/2024/en/}},  
Human Rights Watch\footnote{\url{https://www.hrw.org/report/2024/12/19/extermination-and-acts-genocide/israel-deliberately-depriving-palestinians-gaza}},  
FN:s särskilda rapportör Francesca Albanese\footnote{\url{https://www.ohchr.org/en/documents/country-reports/ahrc5573-report-special-rapporteur-situation-human-rights-palestinian}},  
och ledande folkmordsforskare\footnote{\url{https://ifpnews.com/top-scholars-israel-genocide-gaza/}},  
vilka alla konstaterat att ett folkmord pågår i Gaza. 

Den svenska regeringen har därmed förbrukat varje anspråk på att vara en neutral aktör eller försvarare av folkrätten. Genom att inte agera i enlighet med sina skyldigheter under folkmordskonventionen och FN-stadgan har Sverige inte enbart misslyckats i sitt förebyggande ansvar – det har valt sida i ett pågående folkrättsbrott.

\lagrum{Artikel I, Folkmordskonventionen\quad De fördragsslutande parterna bekräftar att folkmord, vare sig det begås i fredstid eller krigstid, är ett brott enligt internationell rätt som de åtar sig att förebygga och bestraffa.\footnote{\url{https://www.ohchr.org/en/instruments-mechanisms/instruments/convention-prevention-and-punishment-crime-genocide}}}

Denna artikel innebär uttryckligen att:
\begin{itemize}
  \item Staten inte enbart förbjuds att själv begå folkmord,
  \item utan är skyldig att förebygga det – även utanför sitt eget territorium,
  \item och att underlåtenhet att ingripa kan medföra internationellt ansvar.
\end{itemize}

Detta fastslogs entydigt av Internationella domstolen (ICJ) i målet \textit{Bosnia and Herzegovina v. Serbia and Montenegro} (2007):\footnote{\url{https://www.icj-cij.org/public/files/case-related/91/091-20070226-JUD-01-00-EN.pdf}}

\begin{quote}
\textit{“A State may incur responsibility not only for its own acts but also by aiding or assisting another State in the commission of an internationally wrongful act.”}
\end{quote}

(se domens punkt 420 ff.)

Vidare fastslås i FN-stadgan att medlemsstater inte får vara passiva inför grova människorättsbrott:

\lagrum{Article 1(3), FN-stadgan\quad To achieve international co-operation in solving international problems of an economic, social, cultural, or humanitarian character, and in promoting and encouraging respect for human rights...\footnote{\url{https://www.un.org/en/about-us/un-charter/full-text}}}

\lagrum{Article 56, FN-stadgan\quad All Members pledge themselves to take joint and separate action in co-operation with the Organization for the achievement of the purposes set forth in Article 55.\footnote{\url{https://www.un.org/en/about-us/un-charter/full-text}}}

Regeringens systematiska ignorans, dess aktiva försvar av förövaren och dess tystnad inför samstämmiga larm från rättsliga och humanitära organ visar att alla gränser nu är passerade. Detta handlar inte längre om att "utreda" eller "bevaka utvecklingen". Sveriges regering har, genom sitt agerande och sin underlåtenhet, trätt över den gräns som skiljer neutralitet från medverkan.

Detta dokument är därför inte en förfrågan, utan en anmälan.

\subsubsection*{Underlåtenhet att förebygga folkrättsbrott}
\addcontentsline{toc}{subsubsection}{Underlåtenhet att förebygga folkrättsbrott}

Den svenska regeringen har inte enbart förhållit sig passiv till folkmordskonventionens krav. Den har aktivt brutit mot dess anda och bokstav. Genom att frysa finansieringen av UNRWA – det FN-organ som ansvarar för livsnödvändigt bistånd till palestinska flyktingar – i enlighet med Israels påtryckningar, har Sverige bidragit till att avväpna det internationella systemet för humanitärt skydd.

Istället har regeringen omdirigerat biståndet till en israeliskt kontrollerad distributionsstruktur, underställd militär logistik. Detta har lett till ett system där civila i Gaza tvingas hämta mat och mediciner inom snäva, av militären fastställda tidsfönster – under hot om beskjutning. Det är ett system designat för kontroll, inte skydd. Enligt Euro-Med Human Rights Monitor har minst 60 civila skjutits ihjäl vid dessa hjälppunkter under tre dagar.

Detta utgör inte en avvikelse – det är en följd. En följd av att Sveriges regering valt att medverka till ett militärt organiserat biståndssystem som i praktiken upphäver Genèvekonventionens grundprinciper om opartiskhet, humanitet och civilas särskilda skyddsbehov.

Regeringen har därmed aktivt undergrävt FN:s auktoritet, legitimerat ett dödligt distributionssystem och svikit sitt ansvar att stå upp för den humanitära rätten. Det handlar inte om missriktad välvilja. Det handlar om medverkan till ett systematiskt brott mot folkrätten.

Vidare föreligger trovärdiga rapporter om att Israel beväpnar och skyddar väpnade grupper i Gaza – grupper vars ledare tidigare fängslats av Hamas för bland annat narkotikabrott och terrorism. Syftet har varit att destabilisera samhället inifrån och sabotera hjälpsystemet. Att Sveriges regering förblir tyst trots vetskap om detta är inte längre moraliskt förkastligt – det är rättsligt förpliktigande.

Vi konstaterar:

\begin{itemize}
  \item Att regeringens agerande utgör brott mot Sveriges skyldigheter enligt folkmordskonventionen.
  \item Att Sveriges modell för biståndsdistribution i Gaza innebär ett medvetet avsteg från humanitär folkrätt.
  \item Att tystnad inför rapporter om israeliskt stöd till kriminella och jihadistiska grupper kan medföra medansvar.
\end{itemize}

Vi kräver därför:

\begin{enumerate}
  \item Att utrikesministern offentligt kommenterar uppgifterna om svenskt stöd till en folkrättsvidrig biståndsmodell.
  \item Att Konstitutionsutskottet omedelbart utreder huruvida Utrikesdepartementet fullgjort sina förpliktelser enligt folkmordskonventionen och FN-stadgan.
\end{enumerate}

Detta är inte en begäran om förklaring. Det är en formell anklagelse. Sverige har inte bara brutit mot sina skyldigheter – det har gjort det med öppen blick och kallt beräknande.

\subsubsection*{Varför Israel stödjer jihadistgrupper}
\addcontentsline{toc}{subsubsection}{Varför Israel stödjer jihadistgrupper}

Frågan varför Israel aktivt stödjer salafistiska och jihadistiska grupper i Gaza – inklusive element med koppling till ISIS – måste förstås utifrån en strategisk logik, inte som ett säkerhetspolitiskt misslyckande.

\textbf{1. Hamas har förändrats – och det hotar Israels narrativ}

Sedan åtminstone 2006 har Hamas genomgått en djupgående politisk omorientering. I sitt policyprogram från 2017 samt i det gemensamma avtalet med Fatah 2021 erkände rörelsen:

\begin{itemize}
  \item internationell rätt som ramverk,
  \item PLO:s överordnade roll som palestinskt paraplyorgan,
  \item en tvåstatslösning enligt 1967 års gränser, med östra Jerusalem som huvudstad,
  \item och ett fredligt, folkligt motstånd som metod.
\end{itemize}

Denna förändring dokumenteras bland annat av Wikipedia, som redogör för hur Hamas i sitt nya policydokument 2017 explicit accepterade en palestinsk stat inom 1967 års gränser och bygger vidare på tidigare initiativ såsom Prisoners’ Document (2006)\footnote{\url{https://en.wikipedia.org/wiki/Palestinian_Prisoners\%27_Document}}, Mecka-avtalet (2007)\footnote{\url{https://en.wikipedia.org/wiki/Fatah–Hamas_Mecca_Agreement}} och avtalet 2020\footnote{\url{https://en.wikipedia.org/wiki/2020_Palestinian_reconciliation_agreement}}.

Akademiskt har detta analyserats som en genuin förändring, inte endast en kosmetisk fasad. Professor Neve Gordon\footnote{\url{https://en.wikipedia.org/wiki/Neve_Gordon}} och professor Menachem Klein\footnote{\url{https://www.972mag.com/hamas-fatah-elections-israel-arrogance/}} menar att denna utveckling syftade till att uppnå internationell legitimitet, demokratisk försoning och en väg mot ett återförenat palestinskt ledarskap.

Men enligt Klein underminerades dessa fredssträvanden aktivt av Israel, som i stället destabiliserade processen för att kunna upprätthålla narrativet att ”det saknas en trovärdig fredspartner”.

\textbf{2. Israel behöver oresonliga fiender för att rättfärdiga sin politik}

Att Israel historiskt har funnit strategiskt värde i att understödja radikala element är väl dokumenterat. Tidigare premiärminister Ehud Barak har själv erkänt att Israel i decennier aktivt stött Hamas i syfte att försvaga PLO – en splittringsstrategi som nu återanvänds, med än farligare inslag: salafistiska och jihadistiska klaner, inklusive grupper ledda av personer tidigare fängslade av Hamas för narkotikabrott och religiös extremism.

\textbf{Syftet är inte att bekämpa terror, utan att bevara kaos.} Ett extremistdominerat Gaza fungerar som en permanent motbild till ”fred”, och förstärker den israeliska statens centrala narrativ: att det inte existerar någon legitim, förhandlingsbar palestinsk motpart. Detta narrativ möjliggör fortsatt ockupation, kollektiv bestraffning och territoriell expansion – under förevändning av säkerhetsbehov.

Professor Norman Finkelstein har i detta sammanhang lyft den israeliska administrationens begrepp \textit{“The Palestinian peace offensive”} – en intern varningssignal för när motståndet uppfattas som \textit{för rationellt}. Ett måttfullt Hamas som respekterar internationell rätt, erkänner 1967 års gränser och söker val utgör ett långt större hot mot den israeliska långsiktiga strategin än en beväpnad jihadist. Fienden får inte bli trovärdig – den måste vara grotesk.

\textbf{Därför understöds extremism – inte trots dess brutalitet, utan på grund av dess politiska användbarhet.} Detta är inte ett misstag, inte en olycklig följd. Det är en avsiktlig, väldokumenterad realpolitisk strategi med djup historisk kontinuitet.

\medskip

\textbf{Att Sveriges regering ignorerar detta mönster – och därmed i praktiken legitimerar det – är inte en fråga om tolkning, utan om ansvar.} Genom att ensidigt fördöma Hamas, utan att erkänna rörelsens dokumenterade förvandling till en potentiell politisk aktör, förstärker Sverige ett förljugat narrativ som aktivt motverkar fred.

Detta är inte en underrättelsemiss. Det är ett systemfel.

\textbf{Vi frågar därför: Hur kan Sveriges regering, med tillgång till all tillgänglig dokumentation, agera såsom om konflikten endast handlade om terrorbekämpning – och inte om ockupation, kontroll och medvetet sabotage av varje fredsinitiativ som hotar status quo?}


\subsection{Sveriges folkrättsliga ansvar vid pågående folkrättsbrott}
%%%%%%%%%%%%%%%%%%%%%%%%%%%%%%%%%%%%%%%%%%
\subsubsection*{Sammanfattning: Sveriges folkrättsliga skyldigheter är bindande – inte valfria}
\addcontentsline{toc}{subsubsection}{Sammanfattning: Sveriges folkrättsliga skyldigheter är bindande – inte valfria}

Den svenska regeringens agerande måste nu bedömas i ljuset av sina folkrättsliga förpliktelser. Vad som ovan visats – ockupationens karaktär, sabotaget mot fredsprocesser, beväpning av extremistgrupper och vägran att ingripa – utgör inte enbart politiska eller moraliska tillkortakommanden. Det är fråga om konkreta rättsbrott genom underlåtenhet att uppfylla bindande konventionsplikt.

\medskip

Enligt artikel I i Konventionen om förebyggande och bestraffning av brottet folkmord är Sverige skyldigt att inte bara avstå från folkmord, utan också att aktivt förebygga och straffa det. Denna skyldighet har erkänts av Internationella domstolen (ICJ) som en \textit{erga omnes}-förpliktelse – det vill säga en skyldighet som gäller gentemot hela det internationella samfundet.

Därtill är Sverige bundet av sedvanerättens princip om \textit{non-assistance in wrongful acts}, som förbjuder:

\begin{itemize}
  \item att bistå aktörer som begår folkrättsbrott,
  \item att ekonomiskt eller politiskt dra nytta av sådana brott,
  \item att förhålla sig passiv när man har kännedom om brott och en rättslig skyldighet att agera.
\end{itemize}

Detta gäller i synnerhet vid:

\begin{itemize}
  \item folkmord (Genocide Convention),
  \item brott mot mänskligheten (Romstadgan),
  \item grova krigsbrott (Genèvekonventionerna),
  \item apartheid (FN:s apartheidkonvention).
\end{itemize}

Att i detta läge – där Israel systematiskt förvägrar Gazas befolkning skydd enligt humanitär rätt – kräva att det palestinska folket ska avstå från motstånd, utan att samtidigt ingripa mot förövaren, är inte en neutral hållning. Det är ett rättsbrott.

Att kalla varje handling av motstånd “terrorism” medan man själv skyddar, finansierar eller legitimerar ockupationsmakten är att delta i det rättsvidriga status quo.

\medskip

Endast den stat som själv uppfyller sina rättsliga skyldigheter kan moraliskt och juridiskt fördöma andras svar. Regeringens vägran att göra detta innebär att Sverige:

\begin{itemize}
  \item förnekar det palestinska folket varje legitimt alternativ till självförsvar,
  \item avstår från att använda diplomatiska, rättsliga och ekonomiska påtryckningsmedel för att förhindra folkrättsbrott,
  \item och därigenom gör sig medskyldig genom underlåtenhet att reagera.
\end{itemize}

Det är alltså inte den som slår tillbaka i desperation som bär huvudansvaret – utan den regering som, trots kännedom om brotten, vägrar att ingripa.

\bigskip

\subsubsection*{Avtalet med Elbit Systems – svensk medverkan}
\addcontentsline{toc}{subsubsection}{Avtalet med Elbit Systems – svensk medverkan}

Regeringens skuld stannar inte vid demoniseringen av en folkligt förankrad befrielserörelse. Den sträcker sig vidare genom total vägran att tillämpa internationell rätt gentemot Israel – trots överväldigande bevis på tidigare krigsbrott och folkrättsbrott (före den 7 oktober 2023) – och kulminerar i aktiv medverkan.

Den 27 oktober 2023 – samma dag som Israels markinvasion av Gaza inleddes – undertecknade Sveriges regering ett militärt samarbetsavtal med den israeliska vapentillverkaren Elbit Systems. Genom detta har Sverige aktivt bidragit till att legitimera och stödja en krigförande stats vapenindustri mitt under ett pågående folkmord.

Detta trots att det sedan länge är dokumenterat att Elbit och andra israeliska försvarskoncerner använder Gaza som testarena för nya vapensystem. Dessa vapen säljs därefter internationellt som “battle tested”.\footnote{\url{https://www.youtube.com/watch?v=78rs9_FrgmA}} Journalisten Yotam Feldman har visat detta i dokumentären \textit{The Lab}.

Enligt Euro-Med Human Rights Monitor hade Israel, redan inom den första månaden av angreppet, fällt en mängd sprängmedel över Gaza motsvarande två Hiroshimabomber.\footnote{\url{https://euromedmonitor.org/en/article/5908/Israel-hits-Gaza-Strip-with-the-equivalent-of-two-nuclear-bombs}} Trots detta valde Sveriges regering att investera i denna vapenapparat – samtidigt som den fördömde det palestinska motståndet och förteg ockupationsmaktens rättsbrott.

Detta agerande – att i affektens skugga inleda militärt samarbete med en regim som systematiskt bryter mot internationell rätt – utgör inte bara ett moraliskt svek. Det är en rättsstridig handling i sig. Det är medverkan till folkrättsbrott.

\end{comment} % Huvudpåstående: att Sveriges regering är juridiskt ansvarig enligt folkrätt och grundlag

%filnamn:israels_brott_main.tex

% Bevisning om Israels folkrättsbrott och krigsförbrytelser

\newpage

\section{Israels brott mot mänskligheten}

\subsection{Informationskrig och atrocity propaganda – förutsättning för statsterror}



Under de första dygnen efter den 7 oktober 2023 spreds brutala och visuellt laddade berättelser – om halshuggna spädbarn\footnote{\url{https://thegrayzone.com/2023/10/11/beheaded-israeli-babies-settler-wipe-out-palestinian/}}, 
\footnote{\url{https://electronicintifada.net/content/how-israeli-colonel-invented-burned-babies-lie-justify-genocide/}}, 
\footnote{\url{https://x.com/LasseKaragiann5/status/1930685630238920853}}, 
gravida kvinnor med utskurna foster\footnote{\url{https://www.youtube.com/watch?v=nlHay-a10j8}}, 
spädbarn som bränts i ugn\footnote{\url{https://thegrayzone.com/2023/12/06/scandal-israeli-october-7-fabrications/}}, 
avrättade barn\footnote{\url{https://x.com/LasseKaragiann5/status/1930659094383255799}} samt 
systematiska massvåldtäkter\footnote{\url{https://x.com/LasseKaragiann5/status/1930739628841320746}}, 
\footnote{\url{https://x.com/LasseKaragiann5/status/1931104316116328887}}.



Påståenden om Hamas kommandocentraler under och inne i sjukhus har inte övertygat BBC\footnote{\url{https://x.com/LasseKaragiann5/status/1930750542130839651}}. En israelisk skådespelerska användes dessutom i ett iscensatt inslag för att ge sken av att hon befann sig inne på al-Shifa-sjukhuset och hotades av Hamas\footnote{\url{https://x.com/LasseKaragiann5/status/1930740855079403552}}.

Kvinnor och barn har påståtts samarbeta med Hamas \footnote{\url{https://x.com/LasseKaragiann5/status/1930760255228813436}}, liksom ambulanspersonal\footnote{\url{https://thehill.com/policy/international/4292765-israel-strike-ambulance-convoy-gaza/}}, sjkukvårdare\footnote{\url{https://thegrayzone.com/2025/04/07/trump-white-house-hamas-execution-palestinian-medics/}}, läkare\footnote{\url{https://www.middleeasteye.net/news/war-gaza-prominent-palestinian-doctor-tortured-and-killed-israeli-detention}} och journalister.
\footnote{\url{https://rsf.org/en/israeli-politicians-call-journalists-gaza-be-killed}}\footnote{\url{https://cpj.org/2023/12/father-of-al-jazeeras-anas-al-sharif-killed-in-gaza-after-journalist-receives-threats/}}\footnote{\url{https://caityjohnstone.medium.com/israel-admits-it-bombed-a-hospital-to-kill-a-journalist-for-doing-journalism-e359203595e3}}. 





\subsubsection{Systematisk påverkan – Israels digitala propagandakrig}
%\addcontentsline{toc}{subsection}{Systematisk påverkan – Israels digitala propagandakrig}

I kölvattnet av den 7 oktober mobiliserade Israel inte bara sin militära arsenal, utan även en sofistikerad global propagandainfrastruktur. Israels regering spenderade, enligt uppgift, 7,1 miljoner dollar på riktad Youtube-reklam redan under de första veckorna efter attacken.\footnote{\url{https://mronline.org/2023/12/04/hugs-smiles-were-enough-to-take-israeli-propaganda-down/}} Staten har sedan länge använt begreppet \textit{hasbara} – ett eufemistiskt uttryck som betyder ”att förklara” – som täcknamn för statsunderstödd påverkan riktad mot internationell opinion.

Enligt Israels utrikesdepartement är kampen om narrativet på sociala medier en integrerad del av krigsinsatsen. Tre dagar efter attacken initierade Israel ett program för att rekrytera amerikanska influencers att delta i informationskriget. Dessa rekryterades med löften om ersättning upp till 5 000 USD per inlägg samt exklusiva förmåner. Influencer-resor, sociala mediekampanjer och PR-byråer har sedan dess använts i massiv skala för att marknadsföra ett narrativ där Israel framställs som ”ljusets försvarare” i kamp mot ”barbariska terrorister”.

Inrikespolitiskt utbildas israeliska ungdomar i att försvara staten på nätet. Externa aktörer som \textit{Vibe Israel}, \textit{Yola Israel}, \textit{Hasbara Fellowships} och \textit{Sharaka} riktar sig till specifika målgrupper såsom arabisktalande, kristna evangelikaler eller inflytelserika amerikanska kommentatorer. Under dessa program överöses deltagarna med lyxupplevelser, samtidigt som de instrueras i hur Israel bör framställas och hur motkritik ska bemötas.\footnote{\url{https://www.youtube.com/watch?v=W8JUcEWmwVI}}

Ett stort antal bot-konton har dessutom använts för att i masskala sprida pro-israeliska kommentarer och desinformation på nyckelplattformar. Enligt undersökningar från Al-Azhar Media Institute samverkar dessa konton för att föra dialoger med varandra och skapa illusionen av folklig legitimitet.

Alison Weir beskriver hur israeliska soldater, ungdomar, och amerikanska volontärer organiseras för att infiltrera plattformar som YouTube och Wikipedia i syfte att censurera innehåll som dokumenterar israeliska övergrepp.\footnote{Se Alison Weir, \textit{How Israel and its partisans work to censor the Internet}, Israel Palestine News, 2018-03-08: \url{https://israelpalestinenews.org/israel-partisans-work-censor-internet/}}


Allt detta sker samtidigt som groteska berättelser – om halshuggna spädbarn, gravida kvinnor med utskurna foster, brända barn och massvåldtäkter – sprids okritiskt av världsledare och etablerad media, trots att flera av dessa påståenden sedermera har dementerats av israeliska myndigheter och oberoende journalister.

Dessa påståenden återgavs okritiskt av internationella ledare, däribland USA:s president Joe Biden och Tysklands förbundskansler Olaf Scholz,\footnote{\url{https://x.com/propandco/status/1739077157807104059}} men har sedermera tillbakavisats – även av israeliska medier och myndigheter.\footnote{\url{https://thegrayzone.com/2023/11/21/haaretz-grayzone-conspiracy-israeli-festivalgoers/}}

Den svenska regeringen har okritiskt upprepat det internationella mantrat om \enquote{Hamas fruktansvärda attack den 7:e oktober}, men samtidigt accepterat explicit hets till folkmord, förekomsten av massgravar utanför sjukhus med patienter och läkare,\footnote{\url{https://en.wikipedia.org/wiki/Gaza_Strip_mass_graves}} bombningar av samtliga sjukhus i Gaza,\footnote{\url{https://en.wikipedia.org/wiki/Attacks_on_health_facilities_during_the_Gaza_war}} samt attacker mot flyktingläger.\footnote{\url{https://en.wikipedia.org/wiki/Attacks_on_refugee_camps_in_the_Gaza_war}}

Trots detta fortsätter Sveriges regering att okritiskt upprepa den israeliska berättelsen, samtidigt som den ignorerar väldokumenterade brott mot folkrätten – inte bara från enskilda soldater, utan från själva staten Israel.



\subsection{Folkmordets syfte}
%\addcontentsline{toc}{subsection}{Folkmordets syfte}

Israels mål är att tömma Gaza på dess befolkning. Detta klargörs av professor John Mearsheimer\footnote{\url{https://www.youtube.com/watch?v=kAfIYtpcBxo}}, som förklarar att eftersom Gazaborna vägrar lämna sitt land, måste de – enligt detta synsätt – tillfogas outhärdlig smärta för att förmås fly. Därför riktas attacker mot bostäder, sjukhus, ambulanser och flyktingläger. Därför förklaras även kvinnor och barn som legitima mål – under förevändningen att det inte längre finns några civila.

\textit{“There are no uninvolved civilians”} är inte bara en slogan – det är en strategi. Genom att upplösa distinktionen mellan civil och kombattant skapas ett rättsligt tomrum där inga skyddade grupper längre existerar.

Internationella domstolen (ICJ) har i sin interimsdom beordrat Israel att vidta omedelbara åtgärder för att förhindra folkmord. Trots detta ignoreras domen, och hets till folkmord bestraffas inte. Förklaringen är att folkviljan i Israel måste hållas intakt – så att soldater fortsätter att agera i enlighet med opinionens förväntningar.

Denna dubbelmoral är tydlig: inrikes sker hets till folkmord öppet – i Knesset, i TV-kanaler och genom offentliga uttalanden. Samtidigt försäkras det i internationella medier att Israel följer folkrätten.


\subsection{Strategisk immunitet – nödvändig för folkmordets genomförande}

När IDF-soldater dokumenteras med att bakbinda och avrätta ambulanspersonal, avfärdas detta som \enquote{operational error}. Inga rättsliga påföljder följer. Detta är inte ett resultat av administrativa tillkortakommanden – det är en medveten statlig strategi. Israel har inte råd att sända signalen att sådana handlingar är förbjudna, eftersom detta skulle hota sammanhållningen inom det militära maskineriet och urholka opinionens stöd för den pågående fördrivningen.

Om soldater lagförs för att ha skjutit kvinnor, barn eller civila vårdarbetare, rubbas den psykologiska drivkraften som krävs för att fortsätta delta i ett program som i praktiken utgör en etnisk rensning. Därför upprätthålls en rättslig immunitet – en strategisk sådan – som möjliggör folkmordets implementation utan interna sammanbrott.

Det spelar ingen roll hur många gånger IDF ändrar sin officiella berättelse, eller om man i efterhand erkänner ett \enquote{operationellt misstag}. När ambulanser bokstavligen krossas eller vårdpersonal avrättas på plats, utkrävs inget ansvar.\footnote{\url{https://www.middleeasteye.net/news/new-video-evidence-disputes-israeli-armys-account-medic-killings}} Denna tystnad, denna immunitet, är inte perifer – den är kärnan i det brottsliga systemet.

Det finns en väletablerad konsensus inom den israeliska ledningen om att palestinierna i Gaza ska fördrivas. Professor John Mearsheimer har belyst denna dynamik: staten behöver tillfoga outhärdlig smärta för att framtvinga exodus, eftersom tvåstatslösningen är politiskt död och apartheidstatusen på sikt ohållbar.

Uttalanden från högt uppsatta israeler, såsom finansminister Bezalel Smotrichs öppna krav på tömning av Gaza,\footnote{\url{https://x.com/ireallyhateyou/status/1927494637490516081}} understryker att detta inte är ett påstående – det är ett uttalat mål.

\subsection{The world’s first livestreamed genocide}
Mot denna bakgrund har människorättsorganisationer och folkmordsforskare börjat beskriva angreppet som \textit{“The world’s first livestreamed genocide”} – ett uttryck som fångar både brutaliteten och brottets ovedersägliga offentlighet. Trots detta fortsätter den svenska regeringen att okritiskt acceptera varje israelisk bortförklaring om att man endast riktar sig mot \enquote{Hamas}.


\medskip

Detta är inte en juridisk gråzon. Det är ett folkrättsbrott i realtid.
Världen ser detta ske – i realtid, i full medvetenhet. Varje svepskäl som accepteras förlänger brottet.

\medskip

För att se helheten bakom brottet krävs en syntes. Nedan sammanfattas folkmordets struktur på tre nivåer:


\subsubsection*{Sammanfattande struktur – tre nivåer av brottet}

För att förstå folkmordet i Gaza krävs att vi håller isär tre samtidiga nivåer:

\begin{itemize}
  \item \textbf{Övergripande mål:} \textit{Folkmordets syfte} slår fast det politiska målet – att tömma Gaza på dess befolkning genom outhärdlig smärta.
  
  \item \textbf{Genomförandevillkor:} \textit{Strategisk immunitet} förklarar hur detta mål möjliggörs inifrån – genom att soldater tillåts agera utan rättsligt ansvar.

  \item \textbf{Global offentlighet:} \textit{The world’s first livestreamed genocide} sätter brottet i sin mediala och moraliska kontext – ett brott som sker öppet, men ändå utan följd.
\end{itemize}

Tillsammans utgör dessa nivåer folkmordets struktur: en målsättning, ett genomförande och en global tystnad.

%%%%%%%%%%%%%%%%


\subsection{Folkmordets implementeringsbegränsningar}

Att genomföra ett folkmord i Gaza – i syfte att förmå den kvarvarande befolkningen att frivilligt lämna området – kräver inte bara militär kapacitet, utan också en noggrant kalibrerad strategi som beaktar det västliga narrativet. Israels strategi präglas därför av flera implementeringsbegränsningar som tjänar till att upprätthålla det internationella stödet samtidigt som det egentliga syftet – att tömma Gaza på dess invånare – fullföljs.

\begin{itemize}
  \item \textbf{Möjlig illusion:} Israel måste vid varje tidpunkt kunna hävda att det primära syftet är att bekämpa Hamas. Detta kräver ett minimum av “bortförklaringar” per massaker, exempelvis genom hänvisning till “operationella misstag”, “enskilda rötägg” eller påstådd närvaro av stridande i civila byggnader.
  \item \textbf{Intern impunitet:} Den systematiska straffriheten – där uppenbara krigsbrott möts med administrativa reprimander eller ingen åtgärd alls – fungerar som ett signalvärde till soldaterna: fortsätt. Detta skapar en “strategisk immunitet” i tjänst för det övergripande målet.
  \item \textbf{Extern plausibilitet:} Genom att fragmentera ansvaret, outsourca våld till klientgrupper och ge dubbla budskap till västliga observatörer, skapas en illusion av komplexitet som avleder kravet på sanktioner.
  \item \textbf{Mediekontroll:} Folkmordet måste genomföras med så få oberoende journalister som möjligt på plats. Avsaknaden av verifierbar bilddokumentation sänker tröskeln för västliga regeringar att “tveka”, även i mötet med stora dödstal.
  \item \textbf{Inget Hiroshima:} Den strategiska begränsningen är just att Israel inte kan “göra processen kort”. Att använda exempelvis kärnvapen eller döda alla civila i ett enda svep skulle eliminera den diplomatiska täckmanteln och tvinga allierade till sanktioner även mot sin vilja.
\end{itemize}

Därför faller argumentet \textit{“Om Israel verkligen ville begå folkmord hade de redan gjort det”}. Tvärtom: just för att Israel ämnar begå folkmord, måste processen fragmenteras, förklaras, och illusioneras – inte avslöjas. Att döda långsamt, gradvis, och med en ständigt föränderlig berättelse om “misstag”, “kamp mot terrorism” och “rötägg” är nödvändigt för att inte väcka sanktioner från omvärlden.

Detta gör folkmordet mer effektivt – inte mindre. Det är inte det öppna våldet som kännetecknar modern eliminering, utan den \textit{optimerade kombinationen av död och ursäkt}, militär brutalitet och diplomatisk taktik, för att möjliggöra en demografisk utrensning utan att bryta de västliga band som krävs för genomförandet.


\subsubsection*{Begränsad dokumentation och hot mot journalister}
Folkmordshandlingar måste implementeras med viss grad av \textit{deniability}, särskilt i relation till den internationella opinionen. Därför har folkmordet i Gaza genomförts under en retorisk fernissa av \enquote{självförsvar}, \enquote{kirurgiska attacker} och \enquote{varningar till civila}, samtidigt som Israel aktivt motverkar oberoende dokumentation.

Internationella nyhetskanaler tillåts inte verka inne i Gaza. Lokala palestinska journalister utsätts systematiskt för hot, förföljelse och direkta attacker. Enligt Committee to Protect Journalists (CPJ):

\begin{quote}
\textit{”The killing of the family members of journalists in Gaza is making it almost impossible for the journalists to continue reporting, as the risk now extends beyond them also to include their beloved ones.”} \\
\textup{– Sherif Mansour, CPJ (11 december 2023)}
\end{quote}

Minst 218 journalister har dödats sedan oktober 2023,\footnote{\url{https://en.wikipedia.org/wiki/List_of_journalists_killed_in_the_Gaza_war}} ofta efter direkta hot. Al-Jazeera-journalisten Anas Al-Sharif och många andra har sett sina familjehem bombade efter att ha vägrat tystna. En flygbombning av ett "journalisthus" innebär regelmässigt en massaker på tre generationer.


\subsection{Outsourcad massaker – lokala gäng och utländska legosoldater som verktyg för folkmord}

Fältdata från Euro-Med Human Rights Monitor den 11 juni 2025 visar att Israel i Rafah har anlitat en lokal väpnad gruppering, kallad \textit{Abu Shabab-gänget}, samt utländska legosoldater från ett amerikanskt säkerhetsföretag, för att bevaka och kontrollera biståndsdistribution – men i praktiken genomföra summariska avrättningar av svältande civila.\footnote{\url{https://countercurrents.org/2025/06/israel-recruits-local-gangs-and-foreign-mercenaries-turning-aid-distribution-centres-into-mass-slaughterhouse/}}

Flera ögonvittnen har beskrivit hur beväpnade män, klädda i uniformer märkta med texten \textit{“Palestinian Counter-Terrorism Service”}, lät civila ställa sig i kö vid ett biståndscenter för att sedan beordra dem att skingras. När människor i desperation närmade sig ändå, öppnade styrkorna eld. Minst 14 civila dödades omedelbart, inklusive en man som sköts på nära håll då han protesterade mot att hans bror blivit skjuten.

När gänget förlorade kontrollen över folkmassan anslöt israeliska styrkor med pansarfordon, drönare och Apachehelikoptrar i vad som beskrivs som ett samordnat anfall mot en hungrig civilbefolkning. Samtidigt rapporteras att en legosoldat från det amerikanska företaget sköt en civil med dödlig utgång. Tårgas användes också upprepade gånger.

Den israeliska premiärministern Benjamin Netanyahu har offentligt bekräftat att dessa \textit{“klaner”} beväpnats för att, enligt honom, spara israeliska soldaters liv. Inrikesoppositionen i Israel, däribland f.d. försvarsministern Avigdor Lieberman och generalmajor Yair Golan, har liknat detta vid att beväpna grupper med koppling till ISIS.\footnote{\url{https://thegrayzone.com/2025/06/05/israel-arming-isis-gang-gaza/}}

Enligt \textit{The Times of Israel} har Israel tillhandahållit Kalasjnikovs till Abu Shabab-gänget, inklusive beslagtagna vapen från Hamas. Gänget har i flera rapporter kopplats till organiserad kriminalitet, tidigare mordutredningar och systematiskt plundrande av FN-konvojer. Enligt en FN-memo från 2024 utgör gänget den “mest inflytelserika aktören bakom den systematiska och massiva plundringen” av humanitärt bistånd.

Enligt den internationella konventionen mot användning av legosoldater (1989) kan dessa individer – i egenskap av utländska aktörer i organiserad våldsutövning på uppdrag av en främmande makt – klassificeras som legosoldater. Israel bär i så fall juridiskt ansvar för deras handlingar enligt såväl denna konvention som fjärde Genèvekonventionen.

Den rättsliga betydelsen är trefaldig:
\begin{itemize}
  \item Israel outsourcar dödligt våld till oreglerade aktörer under egen ledning.
  \item Det humanitära biståndet omvandlas till en dödsfälla snarare än livlina.
  \item Det skapas en juridiskt gråzon som underminerar internationell ansvarsutkrävning.
\end{itemize}

Att medvetet låta gängledare med koppling till IS, kända för tidigare grova brott (bl.a. mord, narkotikahandel, våldtäkter och plundring av FN-konvojer), agera som beväpnade verkställare i en belägrad civil zon, utgör ett brott mot flera konventioner – däribland artiklarna i Genèvekonventionerna, Romstadgan för ICC och folkmordskonventionen.

Sammantaget är detta inte isolerade övertramp. Det är en del av en medveten statlig strategi: att omvandla hunger till ett vapen, att outsourca våldet, och att undvika juridiskt ansvar genom användning av lokala klientstrukturer och internationella legosoldater. Sådana handlingar kräver omedelbar internationell utredning, rättsligt åtal och sanktioner mot de stater och bolag som medverkat.

\textit{Folkmordets teknokrati förfinas när den operativa kedjan fragmenteras – men det juridiska ansvaret kvarstår hos den ockuperande makten.}







\subsubsection*{Orientalismens raster – varför väst inte reagerar}
Den västerländska tolkningsramen reducerar ofta dessa brott till undantag eller nödvändiga konsekvenser av en \enquote{svår konflikt}. När medier rapporterar att \enquote{Israel bombade ett hus där en journalist bodde}, uppfattas detta av många européer som en enskild incident – inte som en del av en systematisk strategi.

I verkligheten är det i Gaza, liksom i stora delar kring Medelhavet och Mellanöstern, vanligt att flera generationer av samma familj lever tillsammans i flerplanshus byggda i armerad betong. Det är inte ovanligt med tre till fem våningar, där betongstommen med utstickande armeringsjärn möjliggör framtida påbyggnad. Familjens söner och deras respektive med barn bor på separata våningsplan, medan far- och morföräldrar ofta bor på bottenplan.

När ett sådant hus bombas innebär det alltså sällan ett angrepp på en individ – utan regelmässigt en massaker på en hel släkt. Trots detta beskrivs offren i västerländska medier ofta som att de \enquote{dött} snarare än \enquote{mördats}, och läsaren erbjuds den underförstådda eller ibland den explicita förklaringen att \enquote{minst en av dem måste ha varit terrorist}.

Koloniala föreställningar – där \enquote{de andra} betraktas som kollektivt ansvariga, mindre sörjbara och mer utbytbara – gör att denna brutalitet kan rationaliseras. Angrepp på ambulanser, sjukhus och flyktingläger framstår som förklarliga – inte på grund av bevis, utan för att bortförklaringarna tilltalar vår rasifierade förståelse av offren.

I sociala medier yttrar sig denna logik än mer öppet. Regeringens röstkader använder ibland termer som \enquote{sandnegrer}, \enquote{islamister} eller andra eufemismer för att avhumanisera och ursäkta dödandet. Det handlar inte om neutral analys, utan om ett internaliserat rasistiskt raster som gör det annars otänkbara acceptabelt.

Ett tydligt exempel är Charlie Weimers tweet där han gratulerade Libanon till att ha mottagit de två 1000-kilosbomber som raderade ut ett bostadskvarter i syfte att döda Hizbollahs andlige ledare Hassan Nasrallah. Detta trots att Hizbollah utgör en folkrörelse, ett av Libanons största politiska partier, och fram till nyligen hade parlamentsmajoritet.

Dels handlar det om en initialt bristfällig nyhetsbevakning av Gaza\footnote{\url{https://www.journalisten.se/debatt/rapporteringen-om-gaza-ar-ett-fatalt-misslyckande/}}, dels om en djupare strukturell rasism – ett kolonialt raster – som avgör vilka döda som sörjs, och vilka som bara bokförs.

\subsection{Ideologiska drivkrafter – folkmordets slutpunkt i ett sionistiskt ramverk}
%\addcontentsline{toc}{subsection}{Ideologiska drivkrafter – folkmordets slutpunkt i ett sionistiskt ramverk}

I varje folkmordsutredning är gärningsmannens intention central. Denna intention uppstår inte i ett vakuum, utan formas av ideologiska övertygelser, samhällsberättelser och historiska trauman. När det gäller Israels agerande i Gaza måste därför även den sionistiska ideologins långsiktiga mål och interna motsägelser undersökas.

Redan Theodor Herzl och senare Vladimir Jabotinsky – som i sin \textit{järnmur}-doktrin förordade militär övermakt som villkor för fred – såg införlivandet av hela det historiska Palestina som ett nationellt imperativ. I denna berättelse är palestiniernas närvaro inte bara ett praktiskt hinder, utan en existentiell motsägelse. Om de förblir, kollapsar myten om ett exklusivt judiskt hemland.

Det är mot denna bakgrund John Mearsheimers observation bör förstås: Israel befinner sig i ett ideologiskt dödläge. Tvåstatslösningen är politiskt död, och ett permanent apartheidtillstånd vore – enligt israeliska säkerhetsexperter själva – oförenligt med långsiktig stabilitet. Det återstår då endast ett alternativ: fördrivning. Inte som ett av flera val, utan som en oundviklig logisk konsekvens av en etnostats konstruktion.

Fördrivningen av Gazas befolkning är därmed inte ett plötsligt avsteg från Israels grundidé, utan dess slutpunkt. Genom att upprätta ett juridiskt vakuum – där inga civila erkänns, inga soldater lagförs och inga domstolsutslag efterlevs – skapas ett handlingsutrymme där ideologiska mål kan förverkligas med militära medel.

Detta skifte från diskriminering till eliminering är inte nytt. Det följer ett mönster som i folkmordsteorin ofta betecknas som \textit{den åttonde fasen} – det ögonblick då målet inte längre är kontroll, utan frånvaro. En permanent lösning på det palestinska problemet.

Att förstå folkmordets ideologiska motor är avgörande, inte för att förklara bort brottet, utan för att placera det i en historisk struktur där avsikt, metod och vision smälter samman till ett systematiskt mönster av utplåning.
 % Bevisning om Israels folkrättsbrott och krigsförbrytelser

%filnamn:folkratt_main.tex
% Genomgång av Sveriges folkrättsliga åtaganden och konventionsbundenhet

\section{Folkrättens bindande ramverk}

% Sammanfatta relevanta rättskällor: FN-stadgan, Folkmordskonventionen, Genèvekonventionerna, sedvanerätt.
% Förklara varför folkrätten inte kan kringgås av politiska hänsyn.
% Avsluta med Sveriges konstitutionella skyldigheter enligt Regeringsformen 10:1 och 13:3.


Regeringen skriver i sin kommuniké från den 27 maj 2025:\footnote{\url{https://www.government.se/statements/2025/05/statement-from-the-ministry-for-foreign-affairs-on-summoning-of-israeli-ambassador/}}

\begin{quote}
\textit{“When the Ambassador was summoned, it was stressed that Israel has the right to defend itself. However, that right must be exercised in accordance with international law.”}
\end{quote}

Vid en första anblick framstår detta som ett balanserat uttalande. Men den retoriska konstruktionen döljer en djupare förskjutning. Vad som presenteras som ett villkorat erkännande av folkrätten, fungerar i själva verket som en rättslig täckmantel för folkrättsbrott.

FN-stadgan artikel 51\footnote{\url{https://www.un.org/en/about-us/un-charter/full-text}} ger endast rätt till självförsvar vid ett väpnat angrepp – riktat mot en suverän medlemsstat. Rätten gäller alltså inte en ockupationsmakt som utövar kontroll över ett territorium och dess civilbefolkning.

\lagrum{Artikel 51, FN-stadgan\quad Nothing in the present Charter shall impair the inherent right of individual or collective self-defence if an armed attack occurs against a Member of the United Nations...}

Att hävda att Israel har rätt att försvara sig mot Gaza är därför inte enbart en politisk förenkling – det är en rättslig förfalskning av folkrättens grundprinciper. Det ger det strukturella våldet en rättslig fernissa, trots att folkrätten bygger på ansvar – inte på retoriska undantag.

Internationella domstolen (ICJ) fastslog den 19 juli 2024 att Israels ockupation av Gaza, Västbanken och östra Jerusalem är olaglig, och att den utgör apartheid.\footnote{\url{https://mondoweiss.net/2024/07/in-a-historic-ruling-icj-declares-israeli-occupation-unlawful-calls-for-settlements-to-be-evacuated-and-for-palestinian-reparations/}} Detta är inte en tolkning. Det är ett rättsligt bindande konstaterande.

Trots detta fortsätter den svenska regeringen att referera till Israels \enquote{rätt till självförsvar} som om Gaza vore en angripande stat, inte ett ockuperat territorium. Uttalandet neutraliserar den folkrättsliga kontexten och bekräftar därigenom en rättsvidrig världsbild.

Därmed gör sig regeringen skyldig till mer än en vilseledande formulering. Den medverkar – om än indirekt – till att undergräva den internationella rättsordningen. Och detta sker inte av juridisk okunskap, utan som del av en förutsägbar, politiskt motiverad ordkonstruktion:

\begin{quote}
Ett sätt att säga något – men samtidigt inte mena det.  
Detta är inte ansvarsutkrävande. Det är ett medvetet undandragande av ansvar, maskerat som neutral diplomati.
\end{quote}

\medskip

\textbf{Frågan är därför inte om Israel har rätt till självförsvar – utan var den rätten får utövas.}

För att bedöma giltigheten i regeringens påstående krävs en tydlig förståelse av folkrättens territoriella begränsning: självförsvar enligt FN-stadgan artikel 51 får endast utövas inom en stats eget internationellt erkända territorium. Ockupationsmaktens rättigheter är däremot reglerade av helt andra folkrättsliga instrument.


 % Genomgång av Sveriges folkrättsliga åtaganden och konventionsbundenhet

%filnamn:regeringens_agerande_main.tex
% Beskrivning av regeringens faktiska agerande eller passivitet

\section{Sveriges folkrättsliga ansvar}
% Redogör för Sveriges positiva skyldigheter enligt internationell rätt.
% Bevisa att tystnad, fortsatt vapenhandel eller selektiv diplomati utgör medansvar.
% Redogör för tidigare exempel där Sverige agerat kraftfullt mot folkrättsbrott (t.ex. Ryssland, Myanmar, Iran).

\subsection*{Israel saknar rätt till försvar utanför israelisk mark}
\addcontentsline{toc}{subsection}{Israel saknar rätt till försvar utanför israelisk mark}

Oavsett legaliteten i själva ockupationen kan en ockupationsmakt aldrig åberopa självförsvar mot den befolkning som står under dess kontroll. Enligt folkrätten får en ockupationsmakt endast vidta åtgärder som är strikt nödvändiga för att upprätthålla allmän ordning och säkerhet, i enlighet med artikel 43 i Haagreglementet:\footnote{\url{https://ihl-databases.icrc.org/en/ihl-treaties/hague-regulations-1899/article-43}}

\lagrum{Article 43\quad The authority of the legitimate power having in fact passed into the hands of the occupant, the latter shall take all the measures in his power to restore, and ensure, as far as possible, public order and safety, while respecting, unless absolutely prevented, the laws in force in the country.}

Samtidigt är ockupationsmakten skyldig att skydda civilbefolkningen från våld och hot enligt artikel 27 i den fjärde Genèvekonventionen:\footnote{\url{https://ihl-databases.icrc.org/en/ihl-treaties/gciv-1949/article-27}}

\lagrum{Article 27\quad Protected persons are entitled, in all circumstances, to respect for their persons, their honour, their family rights, their religious convictions and practices, and their manners and customs. They shall at all times be humanely treated, and shall be protected especially against all acts of violence or threats thereof.}

Därtill får ockupationsmakten inte ersätta det lokala rättssystemet, utan endast temporärt administrera det för upprätthållande av civil ordning, enligt artikel 64:\footnote{\url{https://ihl-databases.icrc.org/en/ihl-treaties/gciv-1949/article-64}}

\lagrum{Article 64\quad The penal laws of the occupied territory shall remain in force, with the exception that they may be repealed or suspended by the Occupying Power in cases where they constitute a threat to its security or an obstacle to the application of the present Convention...}

Att hävda rätt till militärt självförsvar inom ett territorium där man själv är ockupant innebär att man juridiskt förväxlar våld med skydd, och därigenom ger repressionen ett falskt moraliskt anspråk. Det är att kriminalisera motstånd – och samtidigt rättfärdiga fortsatt förtryck.

Självförsvar enligt FN-stadgans artikel 51 är territoriellt begränsat till internationellt erkänd, egen mark:

\lagrum{Article 51\quad Nothing in the present Charter shall impair the inherent right of individual or collective self-defence if an armed attack occurs against a Member of the United Nations...}

Detta innebär att självförsvar inte kan åberopas av en stat som befinner sig utanför sitt eget erkända territorium – särskilt inte mot den civilbefolkning som staten i egenskap av ockupationsmakt har skyldighet att skydda.

Detta är också exakt den folkrättsliga position Sverige intar gentemot Ryssland och som man framhåller som självklart i officiella sammanhang: Ryssland kan inte hävda självförsvar på ukrainsk mark, eftersom Ukraina inte är ryskt territorium.  

Att däremot medge Israel rätt till självförsvar på ockuperad palestinsk mark är ett flagrant avsteg från denna princip. Det är en inkonsekvent rättstillämpning – där likabehandlingsprincipen offras till förmån för politisk opportunism.


Att därtill tillfoga formuleringar som \enquote{men det måste ske i enlighet med internationell rätt} är utan betydelse. Ty det är själva användningen av begreppet \textit{självförsvar} i detta sammanhang som utgör ett folkrättsbrott.

När en regering offentligt erkänner rätten till självförsvar för en stat som agerar utanför sitt eget territorium – i strid med folkrätten – så förskjuts hela den rättsliga tyngdpunkten. Diskussionen handlar då inte längre om \textit{huruvida} våldet är tillåtet, utan \textit{hur mycket} våld som kan anses proportionerligt.

Detta medför två rättsliga och moraliska implikationer:

\begin{enumerate}
  \item \textbf{Konkludent handlande i folkrättslig mening.}  
  Genom att använda termen \enquote{självförsvar} utan förbehåll, agerar regeringen i strid med den folkrättsliga huvudprincipen att självförsvar enligt FN-stadgan endast får utövas på egen, internationellt erkänd mark.

  Det utgör ett tyst – och därmed konkludent – godkännande av ett pågående folkrättsbrott, såsom olaglig ockupation eller övervåld, eftersom formuleringen ger dessa handlingar en legal fernissa.

 \item \textbf{Medverkan eller stämpling i straffrättslig mening.}  
Om en regering genom officiella uttalanden uppmuntrar, normaliserar eller skyddar ett folkrättsbrott, kan detta – i ett internationellt rättsligt sammanhang – jämställas med det som i svensk straffrätt motsvarar stämpling till brott.

\lagrum{23 kap. 2 § 2 st BrB\quad  I de fall det särskilt anges döms för stämpling till brott. Med stämpling förstås att någon i samråd med någon annan beslutar gärningen eller att någon söker anstifta någon annan eller åtar eller erbjuder sig att utföra den.}

Ett exempel: Om Sveriges statsminister skulle säga \enquote{Ryssland har rätt att försvara sig på ukrainsk mark, men det måste ske enligt internationell rätt}, så har man redan legitimerat det första skottet. Diskussionen handlar därefter inte om \textit{att} Ryssland får använda våld – utan \textit{hur mycket} våld som är acceptabelt. På så sätt förskjuts skuldbedömningen från själva olagligheten till våldets omfattning, vilket innebär att man rättsligt och moraliskt gör sig medskyldig till brottet.
\end{enumerate}

\subsubsection*{Sammanfattning}
\noindent En ockupationsmakt har ingen rätt till självförsvar mot den befolkning som står under dess kontroll. Att erkänna sådan rätt är att förskjuta rättens tyngdpunkt från olagligt våld till "godtagbar nivå av våld", och därmed aktivt legitimera förtryck. 

Regeringens uttalanden står därmed i direkt motsättning till både Sveriges officiella folkrättsliga linje i andra konflikter och till grundläggande principer i internationell humanitär rätt.


\section{Regeringens tystnad, undfallenhet och rättsstridiga stöd}

\subsection*{Tystnaden inför apartheid är inte neutral}


Alla former av affärstransaktioner och diplomatiska förbindelser bidrar till att upprätthålla en apartheidregim.  
Det som i dag erkänns av Internationella domstolen (ICJ) och av världens främsta folkrättsliga institutioner benämns ännu inte av Sveriges regering\footnote{\url{https://www.amnesty.org/en/latest/news/2022/02/israels-apartheid-against-palestinians/}}\footnote{\url{https://www.hrw.org/news/2021/04/27/israel-apartheid-against-palestinians}}.  

Tystnaden är inte neutral – den är ett aktivt val.


\subsection*{Regeringens selektiva moral och dubbla måttstockar}
\addcontentsline{toc}{subsection}{Regeringens selektiva moral och dubbla måttstockar}

\subsubsection*{Regeringens uttalanden och brist på åtgärder}
% Texten om statsministerns uttalande, från "Regeringen skriver vidare..." till "...fortsatt handel och diplomati som om inget hänt."

Regeringen skriver vidare:

\begin{quote}
\textit{“The terrorist organisation Hamas bears heavy responsibility for the current situation. The hostages must be released – unconditionally and immediately.”}
\end{quote}

Regeringen anser sig inta det moraliska överläget men har aldrig sanktionslagt Israel för dess folkrättsbrott. Ingen konsekvens har utdelats trots årtionden av övergrepp. Istället har Sverige fortsatt handel och diplomati som om inget hänt och har därmed inte gjort sin bekärda andel för upprätthållandet av internationell lag och ordning som skulle kunnat ge regeringen den moraliska höjden att utpeka en folkligt förankrad befrielserörelse såsom terroristorganisation – så till vida att denna inte beaktat och respekterat regeringens försök att upprätthålla internationell lag och ordning.

Detta är inte enbart en moralisk inkonsekvens. Det är ett folkrättsligt avtalsbrott i förhållande till Sveriges åtaganden enligt bland annat folkmordskonventionen, FN-stadgan och sedvanerättens tvingande normer.

Genom ett historiskt mönster av passivitet och konkludent handlande har Sverige inte bara underlåtit att agera – utan de facto bidragit till att ge Israel immunitet. Oavsett regim har Sverige fortsatt diplomatiskt och ekonomiskt stöd utan att utkräva ansvar för de upprepade folkrättsbrotten: illegala bosättningar, bombningar av civila och systematisk rasdiskriminering har ursäktats eller ignorerats.

Den som konsekvent tolererar statsterror förlorar rätten att fördöma motstånd mot den. Sverige har därmed spelat en \textit{aiding and abetting}-roll – och saknar rättslig trovärdighet när det gäller att peka ut andra för terror.


\subsubsection*{Regeringens underminering av folkrättens institutioner}
\addcontentsline{toc}{subsubsection}{Regeringens underminering av folkrättens institutioner}

Regeringen har konsekvent agerat för att skydda Israel från rättsligt ansvarsutkrävande. Den svenska statsministern har öppet opponerat sig mot Internationella brottmålsdomstolens (ICC) arresteringsorder mot Israels premiärminister – ett beslut grundat i omfattande bevisning om brott mot mänskligheten. Denna hållning utgör inte bara ett politiskt ställningstagande, utan en direkt underminering av det mest centrala internationella rättsliga verktyget för ansvar efter folkrättsbrott.

Samtidigt har regeringen ignorerat den samstämmiga bedömningen från världens främsta människorättsorganisationer\footnote{\url{https://www.amnesty.org/en/documents/mde15/8668/2024/en/}},  
Human Rights Watch\footnote{\url{https://www.hrw.org/report/2024/12/19/extermination-and-acts-genocide/israel-deliberately-depriving-palestinians-gaza}},  
FN:s särskilda rapportör Francesca Albanese\footnote{\url{https://www.ohchr.org/en/documents/country-reports/ahrc5573-report-special-rapporteur-situation-human-rights-palestinian}},  
och ledande folkmordsforskare\footnote{\url{https://ifpnews.com/top-scholars-israel-genocide-gaza/}},  
vilka alla konstaterat att ett folkmord pågår i Gaza. 

Den svenska regeringen har därmed förbrukat varje anspråk på att vara en neutral aktör eller försvarare av folkrätten. Genom att inte agera i enlighet med sina skyldigheter under folkmordskonventionen och FN-stadgan har Sverige inte enbart misslyckats i sitt förebyggande ansvar – det har valt sida i ett pågående folkrättsbrott.

\lagrum{Artikel I, Folkmordskonventionen\quad De fördragsslutande parterna bekräftar att folkmord, vare sig det begås i fredstid eller krigstid, är ett brott enligt internationell rätt som de åtar sig att förebygga och bestraffa.\footnote{\url{https://www.ohchr.org/en/instruments-mechanisms/instruments/convention-prevention-and-punishment-crime-genocide}}}

Denna artikel innebär uttryckligen att:
\begin{itemize}
  \item Staten inte enbart förbjuds att själv begå folkmord,
  \item utan är skyldig att förebygga det – även utanför sitt eget territorium,
  \item och att underlåtenhet att ingripa kan medföra internationellt ansvar.
\end{itemize}

Detta fastslogs entydigt av Internationella domstolen (ICJ) i målet \textit{Bosnia and Herzegovina v. Serbia and Montenegro} (2007):\footnote{\url{https://www.icj-cij.org/public/files/case-related/91/091-20070226-JUD-01-00-EN.pdf}}

\begin{quote}
\textit{“A State may incur responsibility not only for its own acts but also by aiding or assisting another State in the commission of an internationally wrongful act.”}
\end{quote}

(se domens punkt 420 ff.)

Vidare fastslås i FN-stadgan att medlemsstater inte får vara passiva inför grova människorättsbrott:

\lagrum{Article 1(3), FN-stadgan\quad To achieve international co-operation in solving international problems of an economic, social, cultural, or humanitarian character, and in promoting and encouraging respect for human rights...\footnote{\url{https://www.un.org/en/about-us/un-charter/full-text}}}

\lagrum{Article 56, FN-stadgan\quad All Members pledge themselves to take joint and separate action in co-operation with the Organization for the achievement of the purposes set forth in Article 55.\footnote{\url{https://www.un.org/en/about-us/un-charter/full-text}}}

Regeringens systematiska ignorans, dess aktiva försvar av förövaren och dess tystnad inför samstämmiga larm från rättsliga och humanitära organ visar att alla gränser nu är passerade. Detta handlar inte längre om att "utreda" eller "bevaka utvecklingen". Sveriges regering har, genom sitt agerande och sin underlåtenhet, trätt över den gräns som skiljer neutralitet från medverkan.

Detta dokument är därför inte en förfrågan, utan en anmälan.

\subsubsection*{Underlåtenhet att förebygga folkrättsbrott}
\addcontentsline{toc}{subsubsection}{Underlåtenhet att förebygga folkrättsbrott}

Den svenska regeringen har inte enbart förhållit sig passiv till folkmordskonventionens krav. Den har aktivt brutit mot dess anda och bokstav. Genom att frysa finansieringen av UNRWA – det FN-organ som ansvarar för livsnödvändigt bistånd till palestinska flyktingar – i enlighet med Israels påtryckningar, har Sverige bidragit till att avväpna det internationella systemet för humanitärt skydd.

Istället har regeringen omdirigerat biståndet till en israeliskt kontrollerad distributionsstruktur, underställd militär logistik. Detta har lett till ett system där civila i Gaza tvingas hämta mat och mediciner inom snäva, av militären fastställda tidsfönster – under hot om beskjutning. Det är ett system designat för kontroll, inte skydd. Enligt Euro-Med Human Rights Monitor har minst 60 civila skjutits ihjäl vid dessa hjälppunkter under tre dagar.

Detta utgör inte en avvikelse – det är en följd. En följd av att Sveriges regering valt att medverka till ett militärt organiserat biståndssystem som i praktiken upphäver Genèvekonventionens grundprinciper om opartiskhet, humanitet och civilas särskilda skyddsbehov.

Regeringen har därmed aktivt undergrävt FN:s auktoritet, legitimerat ett dödligt distributionssystem och svikit sitt ansvar att stå upp för den humanitära rätten. Det handlar inte om missriktad välvilja. Det handlar om medverkan till ett systematiskt brott mot folkrätten.

Vidare föreligger trovärdiga rapporter om att Israel beväpnar och skyddar väpnade grupper i Gaza – grupper vars ledare tidigare fängslats av Hamas för bland annat narkotikabrott och terrorism. Syftet har varit att destabilisera samhället inifrån och sabotera hjälpsystemet. Att Sveriges regering förblir tyst trots vetskap om detta är inte längre moraliskt förkastligt – det är rättsligt förpliktigande.

Vi konstaterar:

\begin{itemize}
  \item Att regeringens agerande utgör brott mot Sveriges skyldigheter enligt folkmordskonventionen.
  \item Att Sveriges modell för biståndsdistribution i Gaza innebär ett medvetet avsteg från humanitär folkrätt.
  \item Att tystnad inför rapporter om israeliskt stöd till kriminella och jihadistiska grupper kan medföra medansvar.
\end{itemize}

Vi kräver därför:

\begin{enumerate}
  \item Att utrikesministern offentligt kommenterar uppgifterna om svenskt stöd till en folkrättsvidrig biståndsmodell.
  \item Att Konstitutionsutskottet omedelbart utreder huruvida Utrikesdepartementet fullgjort sina förpliktelser enligt folkmordskonventionen och FN-stadgan.
\end{enumerate}

Detta är inte en begäran om förklaring. Det är en formell anklagelse. Sverige har inte bara brutit mot sina skyldigheter – det har gjort det med öppen blick och kallt beräknande.

\subsubsection*{Varför Israel stödjer jihadistgrupper}
\addcontentsline{toc}{subsubsection}{Varför Israel stödjer jihadistgrupper}

Frågan varför Israel aktivt stödjer salafistiska och jihadistiska grupper i Gaza – inklusive element med koppling till ISIS – måste förstås utifrån en strategisk logik, inte som ett säkerhetspolitiskt misslyckande.

\textbf{1. Hamas har förändrats – och det hotar Israels narrativ}

Sedan åtminstone 2006 har Hamas genomgått en djupgående politisk omorientering. I sitt policyprogram från 2017 samt i det gemensamma avtalet med Fatah 2021 erkände rörelsen:

\begin{itemize}
  \item internationell rätt som ramverk,
  \item PLO:s överordnade roll som palestinskt paraplyorgan,
  \item en tvåstatslösning enligt 1967 års gränser, med östra Jerusalem som huvudstad,
  \item och ett fredligt, folkligt motstånd som metod.
\end{itemize}

Denna förändring dokumenteras bland annat av Wikipedia, som redogör för hur Hamas i sitt nya policydokument 2017 explicit accepterade en palestinsk stat inom 1967 års gränser och bygger vidare på tidigare initiativ såsom Prisoners’ Document (2006)\footnote{\url{https://en.wikipedia.org/wiki/Palestinian_Prisoners\%27_Document}}, Mecka-avtalet (2007)\footnote{\url{https://en.wikipedia.org/wiki/Fatah–Hamas_Mecca_Agreement}} och avtalet 2020\footnote{\url{https://en.wikipedia.org/wiki/2020_Palestinian_reconciliation_agreement}}.

Akademiskt har detta analyserats som en genuin förändring, inte endast en kosmetisk fasad. Professor Neve Gordon\footnote{\url{https://en.wikipedia.org/wiki/Neve_Gordon}} och professor Menachem Klein\footnote{\url{https://www.972mag.com/hamas-fatah-elections-israel-arrogance/}} menar att denna utveckling syftade till att uppnå internationell legitimitet, demokratisk försoning och en väg mot ett återförenat palestinskt ledarskap.

Men enligt Klein underminerades dessa fredssträvanden aktivt av Israel, som i stället destabiliserade processen för att kunna upprätthålla narrativet att ”det saknas en trovärdig fredspartner”.

\textbf{2. Israel behöver oresonliga fiender för att rättfärdiga sin politik}

Att Israel historiskt har funnit strategiskt värde i att understödja radikala element är väl dokumenterat. Tidigare premiärminister Ehud Barak har själv erkänt att Israel i decennier aktivt stött Hamas i syfte att försvaga PLO – en splittringsstrategi som nu återanvänds, med än farligare inslag: salafistiska och jihadistiska klaner, inklusive grupper ledda av personer tidigare fängslade av Hamas för narkotikabrott och religiös extremism.

\textbf{Syftet är inte att bekämpa terror, utan att bevara kaos.} Ett extremistdominerat Gaza fungerar som en permanent motbild till ”fred”, och förstärker den israeliska statens centrala narrativ: att det inte existerar någon legitim, förhandlingsbar palestinsk motpart. Detta narrativ möjliggör fortsatt ockupation, kollektiv bestraffning och territoriell expansion – under förevändning av säkerhetsbehov.

Professor Norman Finkelstein har i detta sammanhang lyft den israeliska administrationens begrepp \textit{“The Palestinian peace offensive”} – en intern varningssignal för när motståndet uppfattas som \textit{för rationellt}. Ett måttfullt Hamas som respekterar internationell rätt, erkänner 1967 års gränser och söker val utgör ett långt större hot mot den israeliska långsiktiga strategin än en beväpnad jihadist. Fienden får inte bli trovärdig – den måste vara grotesk.

\textbf{Därför understöds extremism – inte trots dess brutalitet, utan på grund av dess politiska användbarhet.} Detta är inte ett misstag, inte en olycklig följd. Det är en avsiktlig, väldokumenterad realpolitisk strategi med djup historisk kontinuitet.

\medskip

\textbf{Att Sveriges regering ignorerar detta mönster – och därmed i praktiken legitimerar det – är inte en fråga om tolkning, utan om ansvar.} Genom att ensidigt fördöma Hamas, utan att erkänna rörelsens dokumenterade förvandling till en potentiell politisk aktör, förstärker Sverige ett förljugat narrativ som aktivt motverkar fred.

Detta är inte en underrättelsemiss. Det är ett systemfel.

\textbf{Vi frågar därför: Hur kan Sveriges regering, med tillgång till all tillgänglig dokumentation, agera såsom om konflikten endast handlade om terrorbekämpning – och inte om ockupation, kontroll och medvetet sabotage av varje fredsinitiativ som hotar status quo?}

\subsubsection*{Sammanfattning: Sveriges folkrättsliga skyldigheter är bindande – inte valfria}
\addcontentsline{toc}{subsubsection}{Sammanfattning: Sveriges folkrättsliga skyldigheter är bindande – inte valfria}

Den svenska regeringens agerande måste nu bedömas i ljuset av sina folkrättsliga förpliktelser. Vad som ovan visats – ockupationens karaktär, sabotaget mot fredsprocesser, beväpning av extremistgrupper och vägran att ingripa – utgör inte enbart politiska eller moraliska tillkortakommanden. Det är fråga om konkreta rättsbrott genom underlåtenhet att uppfylla bindande konventionsplikt.

\medskip

Enligt artikel I i Konventionen om förebyggande och bestraffning av brottet folkmord är Sverige skyldigt att inte bara avstå från folkmord, utan också att aktivt förebygga och straffa det. Denna skyldighet har erkänts av Internationella domstolen (ICJ) som en \textit{erga omnes}-förpliktelse – det vill säga en skyldighet som gäller gentemot hela det internationella samfundet.

Därtill är Sverige bundet av sedvanerättens princip om \textit{non-assistance in wrongful acts}, som förbjuder:

\begin{itemize}
  \item att bistå aktörer som begår folkrättsbrott,
  \item att ekonomiskt eller politiskt dra nytta av sådana brott,
  \item att förhålla sig passiv när man har kännedom om brott och en rättslig skyldighet att agera.
\end{itemize}

Detta gäller i synnerhet vid:

\begin{itemize}
  \item folkmord (Genocide Convention),
  \item brott mot mänskligheten (Romstadgan),
  \item grova krigsbrott (Genèvekonventionerna),
  \item apartheid (FN:s apartheidkonvention).
\end{itemize}

Att i detta läge – där Israel systematiskt förvägrar Gazas befolkning skydd enligt humanitär rätt – kräva att det palestinska folket ska avstå från motstånd, utan att samtidigt ingripa mot förövaren, är inte en neutral hållning. Det är ett rättsbrott.

Att kalla varje handling av motstånd “terrorism” medan man själv skyddar, finansierar eller legitimerar ockupationsmakten är att delta i det rättsvidriga status quo.

\medskip

Endast den stat som själv uppfyller sina rättsliga skyldigheter kan moraliskt och juridiskt fördöma andras svar. Regeringens vägran att göra detta innebär att Sverige:

\begin{itemize}
  \item förnekar det palestinska folket varje legitimt alternativ till självförsvar,
  \item avstår från att använda diplomatiska, rättsliga och ekonomiska påtryckningsmedel för att förhindra folkrättsbrott,
  \item och därigenom gör sig medskyldig genom underlåtenhet att reagera.
\end{itemize}

Det är alltså inte den som slår tillbaka i desperation som bär huvudansvaret – utan den regering som, trots kännedom om brotten, vägrar att ingripa.

\bigskip

\subsubsection*{Avtalet med Elbit Systems – svensk medverkan}
\addcontentsline{toc}{subsubsection}{Avtalet med Elbit Systems – svensk medverkan}

Regeringens skuld stannar inte vid demoniseringen av en folkligt förankrad befrielserörelse. Den sträcker sig vidare genom total vägran att tillämpa internationell rätt gentemot Israel – trots överväldigande bevis på tidigare krigsbrott och folkrättsbrott (före den 7 oktober 2023) – och kulminerar i aktiv medverkan.

Den 27 oktober 2023 – samma dag som Israels markinvasion av Gaza inleddes – undertecknade Sveriges regering ett militärt samarbetsavtal med den israeliska vapentillverkaren Elbit Systems. Genom detta har Sverige aktivt bidragit till att legitimera och stödja en krigförande stats vapenindustri mitt under ett pågående folkmord.

Detta trots att det sedan länge är dokumenterat att Elbit och andra israeliska försvarskoncerner använder Gaza som testarena för nya vapensystem. Dessa vapen säljs därefter internationellt som “battle tested”.\footnote{\url{https://www.youtube.com/watch?v=78rs9_FrgmA}} Journalisten Yotam Feldman har visat detta i dokumentären \textit{The Lab}.

Enligt Euro-Med Human Rights Monitor hade Israel, redan inom den första månaden av angreppet, fällt en mängd sprängmedel över Gaza motsvarande två Hiroshimabomber.\footnote{\url{https://euromedmonitor.org/en/article/5908/Israel-hits-Gaza-Strip-with-the-equivalent-of-two-nuclear-bombs}} Trots detta valde Sveriges regering att investera i denna vapenapparat – samtidigt som den fördömde det palestinska motståndet och förteg ockupationsmaktens rättsbrott.

Detta agerande – att i affektens skugga inleda militärt samarbete med en regim som systematiskt bryter mot internationell rätt – utgör inte bara ett moraliskt svek. Det är en rättsstridig handling i sig. Det är medverkan till folkrättsbrott.


\subsubsection*{Sammanfattning: Sveriges folkrättsliga skyldigheter är bindande – inte valfria}
\addcontentsline{toc}{subsubsection}{Sammanfattning: Sveriges folkrättsliga skyldigheter är bindande – inte valfria}

Den svenska regeringens agerande måste nu bedömas i ljuset av sina folkrättsliga förpliktelser. Vad som ovan visats – ockupationens karaktär, sabotaget mot fredsprocesser, beväpning av extremistgrupper och vägran att ingripa – utgör inte enbart politiska eller moraliska tillkortakommanden. Det är fråga om konkreta rättsbrott genom underlåtenhet att uppfylla bindande konventionsplikt.

\medskip

Enligt artikel I i Konventionen om förebyggande och bestraffning av brottet folkmord är Sverige skyldigt att inte bara avstå från folkmord, utan också att aktivt förebygga och straffa det. Denna skyldighet har erkänts av Internationella domstolen (ICJ) som en \textit{erga omnes}-förpliktelse – det vill säga en skyldighet som gäller gentemot hela det internationella samfundet.

Därtill är Sverige bundet av sedvanerättens princip om \textit{non-assistance in wrongful acts}, som förbjuder:

\begin{itemize}
  \item att bistå aktörer som begår folkrättsbrott,
  \item att ekonomiskt eller politiskt dra nytta av sådana brott,
  \item att förhålla sig passiv när man har kännedom om brott och en rättslig skyldighet att agera.
\end{itemize}

Detta gäller i synnerhet vid:

\begin{itemize}
  \item folkmord (Genocide Convention),
  \item brott mot mänskligheten (Romstadgan),
  \item grova krigsbrott (Genèvekonventionerna),
  \item apartheid (FN:s apartheidkonvention).
\end{itemize}

Att i detta läge – där Israel systematiskt förvägrar Gazas befolkning skydd enligt humanitär rätt – kräva att det palestinska folket ska avstå från motstånd, utan att samtidigt ingripa mot förövaren, är inte en neutral hållning. Det är ett rättsbrott.

Att kalla varje handling av motstånd “terrorism” medan man själv skyddar, finansierar eller legitimerar ockupationsmakten är att delta i det rättsvidriga status quo.

\medskip

Endast den stat som själv uppfyller sina rättsliga skyldigheter kan moraliskt och juridiskt fördöma andras svar. Regeringens vägran att göra detta innebär att Sverige:

\begin{itemize}
  \item förnekar det palestinska folket varje legitimt alternativ till självförsvar,
  \item avstår från att använda diplomatiska, rättsliga och ekonomiska påtryckningsmedel för att förhindra folkrättsbrott,
  \item och därigenom gör sig medskyldig genom underlåtenhet att reagera.
\end{itemize}

Det är alltså inte den som slår tillbaka i desperation som bär huvudansvaret – utan den regering som, trots kännedom om brotten, vägrar att ingripa.

\bigskip



\subsection{Sveriges aktiva medverkan}
  \subsubsection*{Avtalet med Elbit Systems – svensk medverkan}
\addcontentsline{toc}{subsubsection}{Avtalet med Elbit Systems – svensk medverkan}

Regeringens skuld stannar inte vid demoniseringen av en folkligt förankrad befrielserörelse. Den sträcker sig vidare genom total vägran att tillämpa internationell rätt gentemot Israel – trots överväldigande bevis på tidigare krigsbrott och folkrättsbrott (före den 7 oktober 2023) – och kulminerar i aktiv medverkan.

Den 27 oktober 2023 – samma dag som Israels markinvasion av Gaza inleddes – undertecknade Sveriges regering ett militärt samarbetsavtal med den israeliska vapentillverkaren Elbit Systems. Genom detta har Sverige aktivt bidragit till att legitimera och stödja en krigförande stats vapenindustri mitt under ett pågående folkmord.

Detta trots att det sedan länge är dokumenterat att Elbit och andra israeliska försvarskoncerner använder Gaza som testarena för nya vapensystem. Dessa vapen säljs därefter internationellt som “battle tested”.\footnote{\url{https://www.youtube.com/watch?v=78rs9_FrgmA}} Journalisten Yotam Feldman har visat detta i dokumentären \textit{The Lab}.

Enligt Euro-Med Human Rights Monitor hade Israel, redan inom den första månaden av angreppet, fällt en mängd sprängmedel över Gaza motsvarande två Hiroshimabomber.\footnote{\url{https://euromedmonitor.org/en/article/5908/Israel-hits-Gaza-Strip-with-the-equivalent-of-two-nuclear-bombs}} Trots detta valde Sveriges regering att investera i denna vapenapparat – samtidigt som den fördömde det palestinska motståndet och förteg ockupationsmaktens rättsbrott.

Detta agerande – att i affektens skugga inleda militärt samarbete med en regim som systematiskt bryter mot internationell rätt – utgör inte bara ett moraliskt svek. Det är en rättsstridig handling i sig. Det är medverkan till folkrättsbrott.
 % Beskrivning av regeringens faktiska agerande eller passivitet




\section{Bevisning av folkmordsavsikt}
%\addcontentsline{toc}{section}{Bevisning av folkmordsavsikt}

%\section*{Om folkmord – när intentionen är uttalad}
%\addcontentsline{toc}{section}{Om folkmord – när intentionen är uttalad}

Utrikesdepartementet har i svar till svenska medborgare förklarat att det inte ankommer på regeringen att ta ställning till huruvida Israel begår folkmord, utan att sådana bedömningar måste invänta slutsatser från ICJ och ICC (Svar från ud.mena.brevsvar@gov.se):

\textit{Kraven på respekt för folkrätten inklusive den internationella humanitära rätten har varit – och fortsätter att vara – ett av regeringens nyckelbudskap. Dessa budskap framförs i våra egna kontakter med Israel och vi gör det tillsammans med andra EU-länder och likasinnade. Ibland sker det offentligt och ibland på annat sätt. Hur regler respekteras och om krigsförbrytelser begåtts måste bedömas från fall till fall utifrån den internationella humanitära rätten. Det är inte regeringens roll att göra sådana bedömningar. För regeringen är det är centralt att eventuellt överträdelser av den internationella humanitära rätten och möjliga krigsförbrytelser utreds och att ansvarsutkrävande säkerställs. Både ICC och ICJ har pågående utredningar om situationen i Palestina. Hittills har de kommit fram till att Israel måstes göra mer för att skydda den civila befolkningen. Det är något regeringen också har framfört till Israel.}



Utrikesdepartementets linje är att det inte är regeringens roll att ta ställning till huruvida Israel begår folkmord, utan att sådana bedömningar ska överlämnas till internationella domstolar. Men om detta vore regeringens övergripande princip, skulle Sverige exempelvis ha tvingats förhålla sig neutralt till Rysslands invasion av Ukraina – i väntan på en formell dom från ICJ. Det gjorde man inte. Redan efter några dagar fördömde Sverige invasionen som ett brott mot FN-stadgan och internationell rätt.  

Samma mönster gäller andra folkrättsöverträdelser. Sverige tog tydligt ställning mot USA:s invasion av Irak 2003. Man fördömde apartheidregimen i Sydafrika långt innan internationella domstolar hade fällt bindande utslag. Regeringen har också kritiserat förtryck i Syrien, Iran, Belarus och Kina, utan att hänvisa till att man måste “vänta på ICC”.

Det visar att Sverige mycket väl kan – och faktiskt redan gör – folkrättsliga och moraliska bedömningar i realtid. Men när det gäller Israel gör man plötsligt avkall på denna förmåga, som om det skulle krävas en internationell prästvigning för att skilja rätt från fel.

``Det är inte regeringens roll att göra sådana bedömningar'' - Att gömma sig bakom framtida domstolsprövningar är inget juridiskt ställningstagande – det är ett moraliskt abdikerande. Och det sker i strid med Sveriges skyldigheter enligt folkmordskonventionen, som inte bara kräver straff, utan uttryckligen kräver att stater \textit{förhindrar} folkmord innan det sker.




%\subsection*{“Frågan om folkmord är en komplex juridisk fråga”}
%\addcontentsline{toc}{subsection}{“Frågan om folkmord är en komplex juridisk fråga”}

En återkommande ursäkt för att undvika att benämna det som sker i Gaza som folkmord är påståendet att detta vore en särskilt svårbedömd brottsrubricering – som om dess juridiska status vore mer oklar än andra internationella förbrytelser. Denna hållning är vilseledande.

Folkmord är ett tydligt definierat brott i internationell rätt. Den rättsliga svårigheten ligger inte i att förstå vad folkmord \textit{är}, utan i att bevisa gärningsmannens \textit{avsikt att förinta} en folkgrupp i sin helhet eller delvis – den så kallade \textit{dolus specialis}.

\textbf{Artikel II i FN:s konvention om förebyggande och bestraffning av brottet folkmord (1948):}\\
\textit{“I denna konvention avses med folkmord någon av följande gärningar, begångna i avsikt att förinta, helt eller delvis, en nationell, etnisk, raslig eller religiös grupp som sådan.”}\footnote{\url{https://www.ohchr.org/en/instruments-mechanisms/instruments/convention-prevention-and-punishment-crime-genocide}}

Att bevisa denna särskilda avsikt – \textit{dolus specialis} – är i regel en av de mest utmanande delarna i ett folkmordsfall. Förövarna är sällan benägna att uttrycka sina intentioner öppet; istället förnekar de vanligtvis att någon sådan avsikt alls föreligger. 

I Israels fall gäller dock det omvända. Här har intentionen inte dolts – tvärtom har den uttryckts öppet, upprepat och från högsta ort.

\subsection{Internationella domstolen bekräftar folkmordsrisken}
%\addcontentsline{toc}{subsection}{Internationella domstolen bekräftar folkmordsrisken}

Just detta låg till grund för Sydafrikas stämning av Israel inför Internationella domstolen (ICJ) i Haag. I sitt beslut den 26 januari 2024 fann domstolens majoritet att det föreligger en \textit{“plausible risk”} för folkmord i Gaza och ålade Israel att omedelbart vidta åtgärder för att förhindra att brottet fullbordas.\footnote{\url{https://www.icj-cij.org/sites/default/files/case-related/192/192-20240126-ord-01-00-en.pdf}}

Domstolen uttryckte särskilt oro över officiella israeliska uttalanden som uppmanade till att “utrota” Gazas invånare, samt över militära åtgärder som tydligt antyder att sådana uttalanden inte är retorik utan strategi.

Detta rättsliga ställningstagande har därefter förstärkts av nya bevis och rapporter från FN-organ och människorättsorganisationer, vilka bekräftar att den israeliska offensiven fortsatt i strid med ICJ:s interimistiska föreläggande.

\subsection{Israels rättsstat ignorerar systematisk uppvigling till folkmord}
%\addcontentsline{toc}{subsection}{Israels rättsstat ignorerar systematisk uppvigling till folkmord}

Parallellt med den internationella kritik som riktats mot Israels politiska ledarskap, har även prominenta israeliska samhällsaktörer larmat om en inhemsk rättslig kollaps.

Den 3 januari 2024 publicerade \textit{The Guardian} ett öppet brev, undertecknat av flera av Israels mest respekterade akademiker, tidigare diplomater, journalister och parlamentariker. De anklagar landets justitieväsende för att ignorera vad de kallar en “utbredd och flagrant” uppvigling till folkmord och etnisk rensning, särskilt riktad mot Gazas civilbefolkning.\footnote{\url{https://web.archive.org/web/20250305043036/https://www.theguardian.com/world/2024/jan/03/israeli-public-figures-accuse-judiciary-of-ignoring-incitement-to-genocide-in-gaza}}

I brevet konstateras att:
\begin{quote}
\textit{“För första gången vi kan minnas har explicita uppmaningar att begå ohyggliga brott mot miljontals civila blivit en legitim och regelbunden del av det israeliska offentliga samtalet. Idag är sådana uttalanden vardagsmat i Israel.”}
\end{quote}

Under juridisk representation av människorättsadvokaten Michael Sfard listar de 11 sidor av uttalanden från regeringsmedlemmar, Knesset-ledamöter, journalister, militära befäl, opinionsbildare och kändisar – alltifrån krav på användning av kärnvapen till bibelbaserade anspelningar om att utplåna Gazas befolkning såsom “Amalek”.

Brevet konstaterar att myndigheterna inte agerat mot dessa brott, samtidigt som de drivit hundratals utredningar mot arabiska medborgare i Israel för tal som tolkats som stöd till Hamas – ofta från anonyma konton med begränsad räckvidd.

Sfard uttrycker särskild oro för att denna retorik har spridits från marginalerna till mitten av samhället:
\begin{quote}
\textit{“Jag kunde aldrig föreställa mig att jag skulle behöva skriva ett sådant brev. Denna typ av språk har lämnat ytterkanterna och blivit mainstream i en utsträckning som är ofattbar.”}
\end{quote}

Brevet sändes \textit{innan} Sydafrika lämnade in sin stämning till Internationella domstolen, men innehåller citat och exempel som sedermera införlivats i den sydafrikanska argumentationen om “incitement to genocide”.

\textit{“Detta är precis den jordmån där omoraliska monster växer – och växer.”}, avslutar signatärerna.


\subsection{Israelisk jurist lämnar in ICC-ärende om uppvigling till folkmord}
%\addcontentsline{toc}{subsection}{Israelisk jurist lämnar in ICC-ärende om uppvigling till folkmord}

Den 10 december 2024 lämnade den fransk-israeliske juristen Omer Shatz in en formell anmälan till Internationella brottmålsdomstolen (ICC) i Haag, med krav på att åtta högt uppsatta israeliska politiker, militärer och opinionsbildare ska åtalas för uppvigling till folkmord (\textit{incitement to genocide}).\footnote{\url{https://www.statewatch.org/news/2024/december/case-filed-at-icc-to-prosecute-israeli-officials-for-incitement-to-genocide/}}

Anmälan riktar sig bland annat mot premiärminister Benjamin Netanyahu, försvarsminister Yoav Gallant, president Isaac Herzog samt journalisten Zvi Yehezkeli. Den åberopar just den punkt där Internationella domstolen (ICJ) i januari 2024 konstaterade att Israel har en folkrättslig skyldighet att förhindra och bestraffa offentlig uppmaning till folkmord i Gaza.

Trots detta har den israeliska regeringens juridiska rådgivare meddelat Högsta domstolen i Israel att inga brottsutredningar kommer att inledas – i direkt trots mot ICJ:s föreläggande.

Juristens inlagor argumenterar därför att ICC är skyldig att agera i statens ställe. Enligt anmälan uppfyller ICJ:s bevisstandard om "plausible risk" samtidigt ICC:s tröskelvärde för arrestering: \textit{"reasonable grounds to believe"}.

De misstänkta namnges enligt följande:

\begin{itemize}
  \item Benjamin Netanyahu, premiärminister
  \item Yoav Gallant, tidigare försvarsminister
  \item Isaac Herzog, president
  \item Bezalel Smotrich, finansminister
  \item Itamar Ben-Gvir, säkerhetsminister
  \item Israel Katz, försvarsminister
  \item Giora Eiland, f.d. generalmajor
  \item Zvi Yehezkeli, journalist
\end{itemize}

Dokumentet understryker att brottet uppvigling till folkmord är självständigt enligt Romstadgan och inte kräver att ett faktiskt folkmord fullbordats.

\textit{Bilaga:} Den fullständiga stämningsansökan finns att läsa i originalversion här: \url{https://www.statewatch.org/media/4123/icc-communication-gaza-incitement-genocide-dec-2024.pdf}


\subsection{“A textbook case of genocide”}
%\addcontentsline{toc}{subsection}{“A textbook case of genocide”}

Just för att Israels intentioner inte bara är dokumenterade utan uttryckta offentligt – och från högsta politiska och militära nivå – fastslår den israelisk-amerikanske folkmordsforskaren Raz Segal att fallet Gaza utgör ett \textit{textbook case of genocide}, det vill säga ett skolboksexempel på folkmord.

\begin{quote}
“This is a textbook case of genocide. [...] Israel’s intent to commit genocide is not hidden. It’s completely out in the open, declared by high-level officials, ministers, the president, and military leaders.”
\end{quote}

\footnote{\url{https://www.youtube.com/watch?v=AUeEnjULHe0}; se även: \url{https://www.democracynow.org/2023/10/16/raz_segal_textbook_case_of_genocide}}

Detta är inte ett isolerat utlåtande från en enskild akademiker – det vilar på en bred och växande internationell forskarkonsensus.


\subsection{Konsenus råder därför bland folkmordsexperter världen över}
%\addcontentsline{toc}{subsection}{Konsenus råder därför bland folkmordsexperter världen över}


Konsenus råder därför bland folkmordsexperter världen över, inklusive israeliska som tidigare motsatt sig detta.\footnote{\url{https://ifpnews.com/top-scholars-israel-genocide-gaza/}}

Enligt en granskning från den nederländska dagstidningen \textit{NRC} råder idag närmast total enighet bland världens ledande folkmordsexperter om att Israels agerande i Gaza utgör folkmord. NRC:s undersökning bygger på intervjuer med sju framstående forskare från sex olika länder och kompletteras med en genomgång av den senaste forskningen inom området. 

Israels folkmordsforskare Raz Segal, verksam vid Stockton University i USA, uttalar sig utan förbehåll:

\begin{quote}
“Can I name someone whose work I respect who does not think it is genocide? No, there is no counterargument that takes into account all the evidence,” Israeli researcher Raz Segal told NRC.
\end{quote}

Segals uttalande är centralt: det handlar inte bara om hans egen ståndpunkt, utan om frånvaron av kvalificerad, evidensbaserad opposition inom det akademiska fältet.

Denna uppfattning bekräftas av professor Ugur Umit Ungor, verksam vid Amsterdams universitet och NIOD Institute for War, Holocaust and Genocide Studies:

\begin{quote}
“While there are certainly researchers who say it is not genocide, I don’t know them,” said Professor Ugur Umit Ungor.
\end{quote}

Ungor framhåller alltså inte bara konsensus bland kollegor, utan att han personligen inte känner till en enda sakkunnig inom fältet som motsätter sig bedömningen.

För att styrka detta empiriskt genomförde NRC en genomgång av 25 vetenskapliga artiklar publicerade i den ledande tidskriften \textit{Journal of Genocide Research}. Resultatet är tydligt:

\begin{quote}
“All eight academics from the field of genocide studies see genocide or at least genocidal violence in Gaza.”
\end{quote}

Detta är anmärkningsvärt i ett akademiskt fält som annars präglas av metodologisk försiktighet och där det ofta finns olika tolkningsmodeller. NRC:s artikel sammanfattar:

\begin{quote}
“Contrary to public opinion, leading genocide researchers are surprisingly unanimous: the Benjamin Netanyahu government, they say, is in that process – according to the majority, even in its final stages. That is why most researchers no longer speak only of ‘genocidal violence’, but of ‘genocide’.”
\end{quote}

Särskilt intressant är att även forskare som tidigare varit skeptiska till folkmordsbegreppet i detta sammanhang nu ändrat uppfattning. Ett exempel är Shmuel Lederman från Open University of Israel:

\begin{quote}
“Lederman initially opposed the use of the genocide label. However, following Prime Minister Benjamin Netanyahu’s dismissal of the ICJ’s ruling, the continued closure of land crossings to Gaza and a letter by 99 US health workers stating that the death toll in Gaza exceeded 100,000, he was convinced that Israel’s actions do in fact constitute genocide.”
\end{quote}

Det faktum att till och med tidigare kritiker har ändrat ståndpunkt understryker allvaret i utvecklingen – när de juridiska och humanitära indikatorerna blivit för starka för att ignoreras.

Slutligen bör ett särskilt fokus riktas mot Melanie O’Brien, ordförande för \textit{International Association of Genocide Scholars}. Det är inte bara hennes bedömning som är intressant, utan även hennes institutionella roll. Forskare i sådana ledarpositioner tillskrivs vanligen särskild integritet, metodologisk skärpa och fältets förtroende:

\begin{quote}
“Melanie O’Brien, president of the International Association of Genocide Scholars, told NRC that Israel’s deliberate denial of food, water, shelter and sanitation was the key factor in her determination that the military campaign was a genocide.”
\end{quote}

O’Brien lyfter här fram att även utan direkt dödande kan förstörelsen av livets nödvändiga förutsättningar – såsom vatten, föda, skydd och sanitet – utgöra folkmord i folkrättslig mening. Det handlar om att skapa levnadsförhållanden som syftar till att fysiskt förinta en grupp, helt eller delvis.

Sammanfattningsvis visar detta avsnitt att det inte längre råder någon meningsfull akademisk oenighet om huruvida Israels agerande i Gaza utgör folkmord. De mest framstående experterna i världen, inklusive israeliska forskare och ledande företrädare för internationella sammanslutningar, är överens om att kriterierna för folkmord enligt FN:s konvention är uppfyllda.

Men folkrättens ansvarsfördelning stannar inte vid beslutsfattare, soldater eller domstolens expertutlåtanden. Enligt folkrätten kan även stater hållas ansvariga för ett folkmord när det visar sig att hela samhällen – inte bara en ledning – aktivt eller passivt bär, sprider eller legitimerar tanken att en skyddad grupp ska förintas.


Det är därför folkrättsligt avgörande att undersöka vad det israeliska folket anser. 
\textbf{Om föreställningen om palestiniernas eliminering har förvandlats från marginell avvikelse till majoritetsuppfattning, medför det en förskjutning i folkrättens tillämpning från individuell till strukturell skuld.} 
Det implicerar inte att varje individ bär juridiskt ansvar – men att statens samhälleliga infrastruktur genomsyras av stöd för utplåningstanken.


Mot bakgrund av detta blir nästa fråga oundviklig: Har folkmordsideologin förankrats i samhället i stort? Om tanken på palestiniernas utplåning inte längre möter motstånd, utan snarare betraktas som legitim eller nödvändig, implicerar det en samhällsdiskurs som i sig bär folkmordsretorik – något som måste återspeglas i gallupundersökningar.


Det är i detta sammanhang vi nu vänder oss till de mest aktuella opinionsundersökningarna.


\subsection{Opinionsundersökningar visar utbrett stöd för eliminatorisk politik mot palestinier}
%\addcontentsline{toc}{subsection}{Opinionsundersökningar visar utbrett stöd för eliminatorisk politik mot palestinier}

Folkrättens förbud mot folkmord gäller inte enbart dem som bär vapen eller ger order. En stat kan hållas ansvarig när dess samhälle utvecklar en bred acceptans – eller till och med entusiasm – för systematiskt våld mot en skyddad grupp. Det är därför av avgörande betydelse att granska de samhälleliga attityderna. Om det visar sig att idén om palestiniernas utplåning inte längre är en marginalföreteelse, utan snarare en brett omfattad samhällsåsikt, förändras den folkrättsliga bedömningen i grunden.

Den israeliske journalisten Jonathan Ofir rapporterar i \textit{Mondoweiss} att eliminatoriskt tänkande – det vill säga uppfattningen att palestinier bör utplånas – inte längre är en ”fringe opinion”, utan i dag delas av en majoritet av Israels judiska befolkning.\footnote{\url{https://mondoweiss.net/2025/05/poll-shows-israeli-belief-that-palestinians-should-be-eradicated-is-no-longer-a-fringe-opinion/}}

Bakgrunden är en opinionsundersökning genomförd vid Penn State University, där 65\% av judiska israeler instämmer i att det finns en samtida inkarnation av det bibliska folket Amalek – den grupp som enligt Gamla testamentet skulle utplånas ”till sista barnet och sista oxen”. 

\begin{quote}
“One of the poll’s results shows that 65\% of Jewish Israelis agree that a present-day incarnation of the ‘Amalek’ exists.”
\end{quote}

Av dem som instämmer i detta menar 93\% att det även gäller dagens palestinier. Detta innebär att en absolut majoritet stöder förintande åtgärder i religiöst motiverad bemärkelse.

\begin{quote}
“93\% of those who believe in that ‘reincarnation’ of the Amalek also answer that ‘eradicating its memory’ also applies to Palestinians today.”
\end{quote}

Men stödet sträcker sig bortom religiösa referensramar. På frågan om huruvida Israels armé bör agera såsom israeliterna gjorde vid intagandet av Jeriko – där alla invånare dödades – svarade 47\% jakande. Än mer alarmerande är att 82\% stödjer tvångsfördrivning av hela Gazas befolkning, och 56\% vill även fördriva palestinska medborgare inom Israel.

Detta är inte enbart stöd för etnisk rensning – det är ett uttryck för etablerad beredskap för folkmord.

Ofir noterar att detta tankegods inte är begränsat till ultraortodoxa eller högerextrema grupper. Tvärtom är stödet mycket utbrett även bland sekulära judar:

\begin{quote}
“On the question of forced expulsion from the Gaza Strip, the percentage among seculars is 70\%. Among ‘traditional,’ it’s 91\%, and among the ‘religious’ ultra-orthodox, or Haredim, a whopping 97\%.”
\end{quote}

Särskilt oroande är att yngre judiska israeler uppvisar de mest folkmordstoleranta attityderna. Endast 9\% av israeler under 40 år förkastar idéerna om utplåning eller tvångsfördrivning av palestinier. Det är just denna åldersgrupp som också utgör ryggraden i det israeliska militärapparatet.

\begin{quote}
“Jewish Israelis under 40 are more genocidal. Only 9\% of those under 40 rejected the ideas of expulsion and extermination presented to them.”
\end{quote}

Med andra ord: i den demografiska grupp som idag bär vapen i Gaza, instämmer 91\% i idén att palestinierna bör utplånas.

Sammanfattningsvis visar dessa data att folkmordsideologin inte är isolerad till ledarskiktet – den delas av det israeliska samhället i stort. Den har stöd från både sekulära och religiösa grupper, och frodas bland unga. Detta implicerar inte att varje individ aktivt förespråkar folkmord, men det demonstrerar att de sociala, politiska och militära förutsättningarna för ett folkmord är uppfyllda. Det rör sig inte längre om en extrem minoritet – det är en dominerande samhällshållning.





\subsection{Exempel på offentlig uppvigling till folkmord enligt folkmordskonventionen från civila och politiska företrädare}
%\addcontentsline{toc}{subsection}{Exempel på offentlig uppvigling till folkmord enligt folkmordskonventionen från civila och politiska företrädare}

Som komplettering till de kvantitativa opinionsundersökningarna finns ett växande antal videoklipp och dokumenterade uttalanden, där israeler – både privatpersoner och folkvalda företrädare – uttrycker stöd för utplåning av Gaza eller hela det palestinska folket.

Dessa uttalanden är inte avgörande för att fastställa statsansvar i sig, men de utgör \textbf{empiriskt stöd} för att eliminatoriskt tänkande genomsyrar såväl civil som politisk diskurs. De visar på en folklig och institutionell acceptans av extrema åtgärder, vilket förstärker bedömningen att Israel som stat befinner sig i ett strukturellt folkrättsbrott\footnote{Se FN:s konvention om förebyggande och bestraffning av brottet folkmord (1948), särskilt artikel III c: “direct and public incitement to commit genocide”}.

\vspace{1em}
\textbf{Exempel (videoklipp och källor):}
\begin{itemize}
    \item \textbf{Ung kvinna i grupp som blockerar hjälpsändningar till Gaza:} \textit{"When there is a war it doesn't matter who your enemy is, you need to destroy their offspring to prevent them from creating more offspring."} — Uttalat i samband med en organiserad blockad där demonstranter hindrade lastbilar med förnödenheter från att nå Gaza. \footnote{\url{https://x.com/s_m_marandi/status/1925722444041502937}}

    \item \textbf{Revital "Tally" Gotliv, Knessetledamot (Likud):} \textit{"Doomsday weapon! [...] Crushing and flattening Gaza… without mercy!"} — Krävde kärnvapenanvändning mot Gaza. \footnote{\url{https://news.antiwar.com/2023/10/10/us-israeli-lawmakers-call-for-genocide-of-palestinians-in-gaza/}}

    \item \textbf{Israelisk kvinna i Rumble-intervju:} \textit{"The only innocent people that are in Gaza now are the 229 hostages... Once they go back to Israel, we will bomb Shifa Hospital... and kill them all."} — Uttalat inför sonens avfärd till Gaza. \footnote{\url{https://rumble.com/v3t0h2t-israeli-women-calls-for-genocide-with-warren-thornton.html}}

\item \textbf{Reshit Guela – ung kvinna i intervju med Sky News:} \textit{"We should kill them, every last one of them."} — Uttalande från en ung israelisk kvinna, dotter till en soldat som tjänstgjort i Gaza. Citatet fälldes under en Sky News-intervju i samband med en konferens om bosättning av Gaza, där hon även hänvisade till Toran och soldaters offer som skäl att ”fullfölja vad de startat”. Intervjun sändes i oktober 2024. \footnote{\url{https://www.informationliberation.com/?id=64705}}


    \item \textbf{Premiärminister Benjamin Netanyahu:} \textit{"You must remember what Amalek has done to you, says our Holy Bible, and we are acting accordingly."} — Referens till 1 Samuelsboken 15:3, där Gud befaller folkmord. \footnote{\url{https://x.com/kthalps/status/1723538810221338973}}

    \item \textbf{Moshe Feiglin, f.d. Knessetledamot:} \textit{"This is not about Hamas. This is a biblical moment – flatten Gaza entirely."} — Krävde fullständig utplåning av Gaza. \footnote{\url{https://x.com/KintsugiMuslim/status/1723144789292371971}}

    \item \textbf{Giora Eiland, f.d. nationell säkerhetsrådgivare:} I en artikel i \textit{Yedioth Ahronoth} föreslår Eiland att civilbefolkningen i Gaza ska tvingas på flykt genom att Gaza görs obeboeligt. Han skriver: \textit{“Gaza should be smaller – the only way to shorten the war is by creating a humanitarian crisis that will make residents flee.”} Han menar att Israel inte har något ansvar för befolkningens välfärd, föreslår att inga matleveranser släpps in, och avslutar: \textit{“The State of Israel must not supply Gaza with one watt of electricity, one drop of water or one liter of fuel.”} Uttalandet, som innebär kollektiv bestraffning och tvångsfördrivning, har citerats av flera människorättsorganisationer som exempel på krigsbrottsligt uppviglande. \footnote{\url{https://www.haaretz.com/israel-news/2023-10-12/ty-article-opinion/.premium/former-israeli-general-says-flatten-gaza-create-a-humanitarian-crisis/0000018b-9c66-d21e-ab8f-bef71a580000}}


    \item \textbf{Demonstrant i New York:} \textit{"We got to wipe them off... flatten them like a parking lot."} — Offentligt uttryckt i intervju. \footnote{\url{https://www.youtube.com/watch?v=qgAkv4SdCJ0}}

    \item \textbf{Lilia Sandler (sjuksköterska, DC):} \textit{"I love it! I don’t think they’re dying enough. Death to Gaza!"} — Filmas när hon river ner fredsbudskap. \footnote{\url{https://www.youtube.com/watch?v=nbkm2v5Hjos}}

    \item \textbf{President Isaac Herzog:} \textit{"It’s not true this rhetoric about civilians not aware..."} — Avvisar idén om civila som oskyldiga. \footnote{\url{https://web.archive.org/web/20231015190209/https://en-volve.com/2023/10/14/watch-israeli-president-claims-there-are-no-innocents-in-gaza-including-civilians/}}

    \item[] \textbf{Systematisk dokumentation – 107 fall:}
    Abu Bakr Hussain har sammanställt 107 fall av uppvigling till folkmord:
    \begin{itemize}
        \item \textbf{Del 1:} \url{https://x.com/KintsugiMuslim/status/1723144789292371971}
        \item \textbf{Del 2:} \url{https://x.com/KintsugiMuslim/status/1723145632594952606}
        \item \textbf{Samlad version:} \url{https://x.com/KintsugiMuslim/status/1723145848895230432}
    \end{itemize}

    \item \textbf{Florida-politiker:} \textit{"All of them."} — Svarade på fråga om hur många fler palestinier som bör dödas. \footnote{\url{https://www.youtube.com/watch?v=qdIoOh3S6PM}}

    \item \textbf{Tzipi Hotovely, Israels ambassadör i Storbritannien:} \textit{"I have zero empathy... they committed the crime of attacking Israel."} — Förnekade oskyldiga civila i Gaza. \footnote{\url{https://www.youtube.com/watch?v=n15yKYHTwSw}}

    \item \textbf{Demonstration i New York:} Citat från flera deltagare:
    \begin{quote}
    \textit{"Kill all Palestinians. Not one left."}\\
    \textit{"Wipe them flat off the map."}\\
    \textit{"I’m not stopping until all Arabs are wiped out."}
    \end{quote}
    \footnote{\url{https://www.youtube.com/watch?v=qgAkv4SdCJ0}}

    \item \textbf{Avi Dichter, jordbruksminister:} \textit{"We are now rolling out the Gaza Nakba."} — Bekräftade ny fördrivningskampanj. \footnote{\url{https://electronicintifada.net/blogs/ali-abunimah/we-are-now-rolling-out-gaza-nakba-israeli-minister-announces}}

    \item \textbf{Demonstration med barn:} Barn och vuxna ropar \textit{"En bra arab är en död arab"} — uttryck för folkmordsnormer i barnuppfostran. \footnote{\url{https://x.com/DrLoupis/status/1723268444236268025}}

    \item \textbf{Advokat Nauri Nilli:} \textit{"There are no innocent civilians in Gaza. Not even the children."} — I israelisk TV. \footnote{\url{https://x.com/StopZionistHate/status/1723474723672068318}}

    \item \textbf{Ezra Yachin, 95 år, IDF-reservist:} \textit{"Erase them, their families, mothers and children."} — Uttalande till soldater inför offensiv. \footnote{\url{https://www.naturalnews.com/2023-10-22-oldest-idf-reservist-wants-to-eradicate-palestinians.html}}

\item \textbf{Kanal 14: Patriotprogram med folkmordshets som underhållning.} I det israeliska tv-programmet ”Patrioten” på Kanal 14 tävlar deltagare i att uttrycka mest extrema åsikter om vad som bör hända med Gazas befolkning. Programmet präglas av ”rörande enighet” om eliminatoriska åtgärder. I ett klipp utropas: \textit{“Skjut civila! Självklart!!”}, vilket mottas med skratt och applåder.\footnote{\url{https://www.youtube.com/shorts/JcQhwT2kJ_4}}

\item \textbf{Kanal 13: Debatt om spädbarns oskuld.} I ett inslag från mars 2025 diskuterade programledare Eyal Berkovic och Moriah Asraf med f.d. försvarsminister Moshe Ya’alon huruvida även nyfödda barn i Gaza bör betraktas som legitima mål. Berkovic argumenterade: \textit{“Det finns inga oskyldiga i Gaza. Alla är terrorister.”} Ya’alon varnade för att denna inställning legitimerar folkmord. Debatten speglar hur eliminatoriskt tänkande institutionaliserats i israelisk massmedia.\footnote{\url{https://www.informationliberation.com/?id=64915}}

\item \textbf{Bezalel Smotrich:} Israels finansminister förklarade i ett uttalande att kriget mot Gaza inte ska avslutas ”förrän Amalek är fullständigt utrotat”, en teologisk kod som i sammanhanget tolkas som en uppmaning till total förintelse av palestinier. Liknande språk har använts av premiärminister Netanyahu och andra regeringsmedlemmar.\footnote{Referenser i tidigare fotnoter, samt \url{https://x.com/infolibnews/status/1725234779090563548?}}

\item \textbf{Knesset-ledamot försvarar tortyr och sexuellt våld:} I en debatt i Knesset svarar Hanoch Milwidsky (Likud) \textit{"Yes! If he is a Nukhba everything is legitimate to do him!"} på en fråga från MK Ahmad Tibi (TA'AL) om det är legitimt att föra in ett föremål i en palestinsk fånges ändtarm. Uttalandet gjordes i kontexten av en revolt vid det ökända fånglägret Sde Teiman, där soldater vägrade låta kollegor gripas för grova övergrepp mot en palestinsk fånge. Händelsen ledde till att högerpolitiker och reservister samlades för att stoppa gripandena, och Milwidsky svarade med att utlysa en "röststrejk" i protest mot rättsprocessen.\footnote{\url{https://x.com/ireallyhateyou/status/1817904053462196523}}

\item \textbf{Hög rabbin välsignar gruppvåldtäkt av palestinsk fånge:} Meir Mazuz, en av Israels mest inflytelserika rabbiner och nära allierad till Netanyahu och hans regering, kommenterade gruppvåldtäkt av en palestinsk fånge med orden: \textit{"You beat the enemy, so what? It's all good… Don't we have the right to do it?… In any other country, they'd get medals… Don't fear the goyim."} Uttalandet följde efter att soldater misstänkta för våldtäkt välsignats offentligt, och sammanföll med våldsamma protester mot gripanden vid lägret Sde Teiman. Uttalandet illustrerar hur även extrema övergrepp försvaras inom vissa religiösa kretsar i Israel.\footnote{\url{https://x.com/davidsheen/status/1832362225052307581}}

\item \textbf{Barnkör i Israel sjunger om total utplåning:} I en video uppladdad av Israels statliga TV-bolag \textit{Kan}, sjunger israeliska barn en så kallad vänskapssång med textrader som: \textit{"Inom ett år kommer vi att utplåna alla, och sedan återvänder vi för att plöja våra fält."} Videon togs senare bort efter massiv kritik, men arkiverades. Sången skrevs av Ofer Rosenbaum, en PR-expert känd för att förespråka etnisk rensning, och används för att stärka stödet för kriget mot Gaza. Barnens "änglalika röster" kombineras här med explicit folkmordsretorik, något som enligt internationell rätt kan utgöra ett brott i sig.\footnote{\url{https://electronicintifada.net/blogs/ali-abunimah/watch-israeli-children-sing-we-will-annihilate-everyone-gaza}}

\item \textbf{Ministern dansar till barnens utplåning:} I en video från en offentlig manifestation ses ungdomar jubla, dansa och sjunga ramsan: \textit{"Gaza is a graveyard, Gaza is a graveyard. There is no school in Gaza, because there are no children left in Gaza."} Israels säkerhetsminister Itamar Ben Gvir deltar\footnote{\url{https://x.com/AngDem29/status/1725262889936671035}}

\item \textbf{“There are no innocent Palestinians”:} Den amerikanske radioprofilen Mark Levine, en neokonservativ opinionsbildare med judisk trosbekännelse, skrev på X (tidigare Twitter): \textit{“There are no ‘innocent Palestinians’.”} Uttalandet gjordes i samband med pågående bombningar i Gaza och tusentals döda barn. Uttalandet väckte starka reaktioner och jämförelser med hur omvärlden skulle reagera om någon uttryckt sig på liknande sätt om judar efter ett massmord. Sådan generaliserande retorik kan utgöra en del av en folkmordsnormaliserande diskurs.\footnote{\url{https://x.com/jakeshieldsajj/status/1732995903966040435}}

\item \textbf{Viceborgmästare: “Begrav dem levande”} — Jerusalems viceborgmästare Aryeh Yitzhak King föreslog i ett inlägg på X (senare raderat) att palestinska civila fångar i Gaza borde begravas levande: \textit{“Om det vore upp till mig, skulle jag ha skickat D-9-bulldozrar och låtit täcka dessa hundratals myror medan de fortfarande levde.”} King kallade palestinierna för “subhumans” och anspelade, likt Netanyahu, på Amalek – en biblisk fiende vars fullständiga utplåning beordras i Första Samuelsboken 15:3. Uttalandet har väckt omfattande kritik som ett explicit folkmordsuttalande från en högt uppsatt politiker med administrativt ansvar över både västra och ockuperade östra Jerusalem. \footnote{\url{https://www.middleeasteye.net/news/israel-palestine-war-politician-calls-civilians-buried-alive}}

\item \textbf{Journalist deltar i beskjutning av Gaza.} — Rotem Achihun, en reporter för israelisk statlig TV, filmades medan hon aktivt deltog i artilleribeskjutning av Gaza. På kameran ses hon skriva “Hälsningar till Gazaborna” på granater som sedan avfyras. Hon kommenterar: \textit{“Jag känner för att göra något ont mot dem.”} Händelsen följer på att en radiovärd i Israel 103 FM öppet erkänt sitt deltagande i dödandet av palestinier samtidigt som han rapporterar om det. \footnote{\url{https://x.com/davidsheen/status/1732765077625827389}}

\item \textbf{F.d. Knessetledamot: Förinta hela Gaza.} — Advokat och tidigare parlamentsledamot Danny Neuman uttryckte i israelisk TV att hela Gaza bör "förintas", att området ska jämnas med marken och dess "damm rensas bort". Han kallade samtliga invånare för terrorister och föreslog att ett nytt, säkert område för Israel borde byggas i deras ställe. Uttalandet har väckt oro över de verkliga syftena med kriget i Gaza. \footnote{\url{https://www.middleeastmonitor.com/20231213-former-knesset-member-advocates-for-exterminating-all-of-gaza/}}

\item \textbf{Rabbin och professor Dov Fischer: Gazas civila är inte oskyldiga.} — I en krönika publicerad på Israel National News förnekar Rabbi Prof. Dov Fischer, juridikprofessor och ledande religiös företrädare, att någon i Gaza kan betraktas som civil eller oskyldig. Han hävdar att befolkningen röstade fram Hamas, aktivt stödjer attacker mot judar, och därmed har förverkat varje skydd. Artikeln uttrycker en total moraliskt-politisk diskvalificering av en hel civilbefolkning, vilket ger ideologiskt stöd till eliminatoriskt våld och utgör ett varnande exempel på hur akademiska och religiösa auktoriteter kan legitimera folkrättsbrott.\footnote{\url{https://www.israelnationalnews.com/news/382016}}

\item \textbf{”America’s genocide” – USA:s medansvar för folkmord.} Enligt en genomgång från World Socialist Web Site bidrar USA inte bara med vapen och finansiering, utan även med diplomatisk täckmantel och ideologiskt stöd till Israels militära offensiv i Gaza. Artikeln citerar bland annat israeliska politiker som uttryckt önskan att Gaza ska ”se ut som Auschwitz” och som öppet förespråkar etnisk rensning. Samtidigt försvarar USA:s utrikesminister Blinken Israels agerande som självförsvar och avfärdar krav på eldupphör. Artikeln menar att folkmordet i Gaza inte bara är Israels krig, utan också ett amerikanskt krig, drivet av imperiella intressen och repressiv inrikespolitik.\footnote{\url{https://www.wsws.org/en/articles/2023/12/22/vvtj-d22.html}}

\item \textbf{”There are no civilians in Gaza” – demonisering av hela befolkningen.} I en opinionsartikel i den judiska nyhetstidningen JNS hävdar den israelisk-amerikanske kolumnisten Daniel Greenfield att Hamas inte är en avgränsad terroristgrupp utan en kulturell och familjebaserad struktur djupt rotad i Gazas samhälle. Han menar att civila hushåll gömmer gisslan och att klanbaserade nätverk fungerar som logistisk och moralisk infrastruktur för Hamas. Artikeln avfärdar begreppet ”civil” i Gaza som irrelevant och föreslår att Israel bör definiera befolkningen utifrån lojalitet – som ”fiender” eller ”neutrala” – snarare än som stridande eller icke-stridande. Detta synsätt förkastar den internationella humanitärrättens grundläggande distinktioner och rättfärdigar därigenom kollektiv bestraffning.\footnote{\url{https://www.jns.org/there-are-no-civilians-in-gaza/}}


\end{itemize}


\subsection{Avslutande psykologisk bedömning}
%\addcontentsline{toc}{subsection}{Avslutande psykologisk bedömning}

Vad som i början av denna analys framstod som politiska och militära övertramp från ett enskilt statsledarskap har i takt med materialets omfattning visat sig vara något mycket mer djuptgående. Det rör sig inte om isolerade uttryck för extremism, utan om en genomgripande samhällelig förskjutning i synen på en skyddad folkgrupp. 

Ett växande antal individer med judisk identitet – inklusive politiker, journalister, akademiker och sjukvårdspersonal – uttrycker nu öppet stöd för eliminatoriska idéer. De säger det med sina riktiga namn, inför kamera, i nationell TV, i parlament, i tidningsspalter och på sociala medier. Detta sker utan skam, utan fördömande från deras egna institutioner och ofta utan några rättsliga eller sociala konsekvenser.

Detta avslöjar att folkmordsretoriken inte längre behöver maskeras. Den är inte längre skamfylld eller socialt sanktionerad. Tvärtom – den framförs som en rimlig, rationell och i vissa fall religiöst föreskriven ståndpunkt.


Det är dessutom tydligt att denna ideologi:
\begin{itemize}
    \item inte är bunden till Israel som geografisk plats,
    \item inte begränsas till militära eller teologiska kretsar,
    \item inte förutsätter låg utbildning eller social marginalisering – men vilar ofta på ideologiska villfarelser, såsom föreställningen om en bibliskt sanktionerad rätt till land eller existensen av en judisk ras i genetisk mening.
\end{itemize}


Tvärtom uttrycks uppmaningar till utrotning av välutbildade, resursstarka individer – med hög samhällsstatus – vilket visar att tankefiguren om palestiniernas obotliga ondska har internaliserats i breda samhällsskikt, oberoende av utbildningsnivå eller politisk gren.

Att sådana yttranden inte möts med institutionell bestraffning – som i fallet med senator Michelle Salzman i Florida – visar att tyst acceptans nu har blivit normen\footnote{Andrew Mitrovica, \textit{Getting away with a call to genocide in Gaza}, Al Jazeera, 20 nov 2023. \url{https://www.aljazeera.com/opinions/2023/11/20/all-of-them}}. Det är inte bara den som hatar som bär ansvar – det är också den som tiger.

I psykologiska termer handlar detta om moralisk desensibilisering och det Bandura kallat \textit{“moral disengagement”} – där förstörelse inte längre ses som ett moraliskt problem utan som ett nödvändigt “försvar”. Begrepp som “no ceasefire” fungerar som kodspråk för fortsatt förintelse utan samvetsansvar.

Det är också ett exempel på hur hatideologier institutionaliseras – när samhällets struktur (politik, religion, media och utbildning) inte längre bara tolererar, utan bekräftar och sprider eliminatoriska normer. När folkmordsideologi blir “förnuftig ståndpunkt” är folkmord inte längre bara en risk. Det är en process.

Det är därför denna promemoria inte bör tolkas som en moralisk appell, utan som ett folkrättsligt underlag för handling. Sveriges skyldigheter enligt folkmordskonventionen omfattar inte bara straff – utan \textbf{förhindrande}. Och det finns ingenting mer preventivt än att känna igen när ett samhälle befinner sig mitt i en psykologisk och retorisk normalisering av det mest förbjudna brottet i mänsklighetens historia.


\subsubsection*{En tweet som sammanfattar den psykologiska strukturen}
%\addcontentsline{toc}{subsubsection}{En tweet som sammanfattar den psykologiska strukturen}

Den israeliske journalisten Gideon Levy har i flera sammanhang, bland annat i föreläsningen \textit{The Zionist Tango}\footnote{\url{https://www.youtube.com/watch?v=JQS-_9K5-Dk&t=1022s}}, identifierat tre djupgående psykologiska föreställningar som bär upp den israeliska självbilden – men som, vilket denna analys visar, även förekommer globalt bland personer med stark judisk identitet oavsett geografisk hemvist eller religiös livsåskådning:

\begin{enumerate}
    \item \textbf{Föreställningen om att vara ett utvalt folk} – vilket ger en upplevd rätt att omdefiniera internationell rätt och etik.
    \item \textbf{Offerrollen som moralisk immunisering} – där även aggressivitet och dödande tolkas som defensiva handlingar.
    \item \textbf{Avhumanisering av palestinier} – som beskrivs som fundamentalt onda, irrationella och ovärdiga skydd.
\end{enumerate}

Ett tydligt exempel på hur dessa tre föreställningar samverkar finner vi i ett antal inlägg från den amerikanske barnläkaren Dr. Darren Klugman, verksam vid Johns Hopkins Hospital i Baltimore\footnote{Andrew Mitrovica, \textit{Getting away with a call to genocide in Gaza}, Al Jazeera, 20 nov 2023. \url{https://www.aljazeera.com/opinions/2023/11/20/all-of-them}}. På sociala medier beskrev han palestinier som:

\begin{quote}
\textit{“barbaric,” “savage,” and “blood thirsty, morally depraved animals who want nothing short of every inch of Israel and all Jews dead.”}
\end{quote}

Han fortsatte med att förorda massfördrivning:
\begin{quote}
\textit{“There is lots of sand for Palestinians in Sinai which Israel gave to Egypt.”}
\end{quote}

Och avslutade med ett religiöst färgat godkännande av folkmord:
\begin{quote}
\textit{“G-d willing.”}
\end{quote}

Här ryms hela Gideon Levys modell:

\begin{itemize}
    \item \textbf{Avhumaniseringen} är explicit i beskrivningen av palestinier som \textit{“morally depraved animals”}, ett klassiskt retoriskt mönster i folkmordspropaganda där den andra parten fråntas mänskliga egenskaper. Det möjliggör handlingar som annars vore otänkbara.
    
    \item \textbf{Utvaldheten} manifesteras i idén att Gaza kan “reclaimas” som en legitim rätt, trots att området är befolkat av andra människor. Att tala om “återtagande” av Gaza är inte en neutral territoriell term – det bygger på en föreställning om gudagiven rätt, historisk oförrätt och kollektiv återlösning. Det är inte bara en politisk anspråksrätt, utan en psykologisk upplevelse av exklusiv moralisk prioritet. Världen är inte bara tyst – den ska helst förstå att detta undantag är berättigat.

    \item \textbf{Den religiösa legitimeringen} förstärks av uttrycket \textit{“G-d willing”}, som inte bara fungerar som förstärkare, utan som moraliskt ankare: om Gud vill det, så kan det inte vara fel. Även om detta uttrycks i angliciserad, till synes lågintensiv form, är det en kodifiering av helgat våld.

    \item \textbf{Offerrollen} återfinns i den underliggande logiken: palestinierna är inte bara farliga – de vill ha “every inch of Israel and all Jews dead”. Det är alltså inte vi som angriper – vi försvarar vår existens. Det psykologiska syftet är att vända på rollerna: förövaren blir beskyddare, och offret blir det verkliga hotet.
\end{itemize}

Det är särskilt den andra pelaren i Levys modell – föreställningen om den exklusiva offerrollen – som förtjänar särskilt fokus. Levy framhåller att Israel är unikt i världshistorien: aldrig tidigare har en ockuperande makt lyckats etablera en självbild där den inte bara ser sig som ett offer bland andra, utan som det enda moraliskt relevanta offret. Inte bara i den pågående konflikten – utan i hela världen, genom hela historien. En sådan självbild lämnar inget utrymme för andras lidande att erkännas, än mindre att bemötas.


Detta synsätt är inte ett retoriskt knep utan en genuint internaliserad övertygelse. Det bygger på en djup historisk sedimentering där Förintelsen utgör inte bara ett trauma, utan ett absolut moraliskt paradigm. Inget lidande i världen har varit större, renare eller mer oförskyllt än det judiska lidandet under andra världskriget – och eftersom detta brott utfördes av omvärlden mot ”Guds egendomsfolk” kan ingen annan grupp i eftervärlden riktigt mäta sig med det judiska folkets sårbarhet. 

Därför blir det, i detta psykologiska raster, möjligt att döda barn i Gaza samtidigt som man ser sig själv som ytterst sårbar. Offerrollen har här inte blivit mindre med statsmakt, kärnvapen eller territoriell kontroll – tvärtom. Den har institutionaliserats som nationell identitet.

I fallet med Dr. Klugman är detta tydligt. Hans beskrivning av palestinier som bestialiska fiender som vill döda alla judar är inte en rationell säkerhetsbedömning – det är en spegling av en existentiell självbild där varje avvikelse från judisk dominans tolkas som hot om förintelse. För att det judiska folket ska existera tryggt, måste andra – i detta fall palestinierna – förnekas samma mänskliga status. Hans språkbruk är inte bara hatfyllt – det är psykologiskt ritualiserat, ett uttryck för att bekräfta vem som har rätt att finnas och vem som inte har det.

Det är därför Gideon Levy har rätt: så länge denna psykologiska struktur förblir intakt – utvaldhet, exklusiv offeridentitet, avhumanisering – kommer inga materiella förändringar eller förhandlingar att kunna leda till verklig fred. Och som denna promemoria visar: strukturen har inte bara förblivit intakt, den har blivit globaliserad.


Det mest alarmerande är inte att en enskild individ uttrycker detta, utan att han:
\begin{itemize}
    \item är högt utbildad och yrkesverksam som barnläkare vid ett av världens mest prestigefyllda sjukhus,
    \item gör uttalandena öppet, i sitt eget namn,
    \item inledningsvis inte möter någon bred offentlig fördömelse.
\end{itemize}

Detta visar att den psykologiska strukturen som Levy beskriver inte är beroende av israelisk nationalitet eller ortodox religiös tro. Den har blivit ett globalt, transnationellt tankesystem där sekulära, moderna, ofta liberala personer kan bära på samma kognitiva byggstenar som religiösa extremister – bara uttryckta i annan tonart.

Därmed har vi inte bara identifierat en ideologi. Vi har identifierat en inre struktur för hur denna ideologi rationaliseras och reproduceras. Det är denna struktur som måste exponeras, ifrågasättas och – i enlighet med folkmordskonventionens förebyggande syfte – desarmeras innan den fullbordar sin destruktiva logik.


\subsubsection*{Prof. Raz Segal: Från avsikt till akademisk legitimitet – ett ekosystem för undantagstillståndet}
%\addcontentsline{toc}{subsubsection}{Prof. Raz Segal: Från avsikt till akademisk legitimitet – ett ekosystem för undantagstillståndet}

I en uppmärksammad intervju med Owen Jones\footnote{\url{https://www.youtube.com/watch?v=VqHoRP9u5Bk}}, publicerad den 3 juni 2024 – åtta månader efter attackerna den 7 oktober – förklarar professor Raz Segal hur Israels språkbruk och agerande inte bara bär tecken på folkmordsavsikt enligt konventionens definition, utan dessutom faller in i ett historiskt mönster av kolonial dominans legitimerad genom rättssystem och forskning. Han noterar bland annat följande:


\begin{itemize}
\item Israels avsikt har uttryckts öppet sedan dag ett. Språket om “\textit{human animals}”, uppmaningar till “\textit{hell}” och kollektiv bestraffning genom belägring utgör inte bara retorik – utan en faktisk, planerad verkställighet.
\item Han pekar på hur internationell humanitär rätt vänds mot sin egen intention, genom att “säkra evakueringsvägar” används som måltavlor. Det är vad Segal kallar en “weaponization” av juridiken – ett brott med folkrättens anda.
\item Den inledande svälten, bombningarna, targeting av sjukhus och 15 000 döda barn utgör, enligt Segal, “förhållanden avsedda att förgöra gruppen i sin helhet eller delvis”, ett nyckelbegrepp i folkmordskonventionen.
\item Han kopplar också avsikten till fantasibilden om ett kolonialt krig: israeliska ledare föreställer sig att de utkämpar ett civilisatoriskt krig mot “barbarer”, “nazister” eller “djur” – vilket skapar kognitiv frihet att avvika från normala rättsliga och moraliska ramar.
\end{itemize}

Segal konstaterar därefter något som direkt återknyter till Gideon Levys psykologiska modell: Denna koloniala världsbild är inte bara ett inhemskt fenomen i Israel utan bekräftas och legitimeras i västvärlden genom två avgörande mekanismer:

\begin{enumerate}
\item Det internationella rättssystemet, inklusive ICC och ICJ, är fortfarande präglat av kolonial maktstruktur. Det har historiskt skyddat västmakters intressen, och är därför långsamt eller oförmöget att ingripa mot en “klientstat” som Israel.
\item Forskningen kring folkmord har strukturerats kring vad Segal tidigare kallat en “helig särställning för Förintelsen”. Namnvalet “\textit{Holocaust and Genocide Studies}” speglar inte en neutral akademisk kronologi, utan en ideologisk hierarki. Den bekräftar och förstärker uppfattningen att det judiska lidandet är historiskt unikt och utan jämförelse, vilket effektivt hindrar en vetenskaplig analys av Israels agerande inom samma kategori.
\end{enumerate}

\noindent
Precis som Gideon Levy beskrev i sin modell, så formas ett psykologiskt fundament där:
\begin{itemize}
\item Man är det utvalda folket,
\item Det enda verkliga offret i historien,
\item Motståndaren (palestinierna) är inte fullt mänsklig,
\item Och hela det västerländska juridiska och akademiska ramverket är konstruerat för att bekräfta denna självbild.
\end{itemize}

\noindent
Detta utgör vad som i vetenskapsteorin kallas en positiv återkopplingsslinga. När det akademiska fältet och den juridiska infrastrukturen bekräftar den egna särställningen, upphör all självkritik – vilket möjliggör systematisk grymhet utan skuldmedvetande. Det är detta som gör situationen så farlig.



\subsubsection*{Den exklusiva offerrollen som moraliskt imperativ}\footnote{Begreppet ”moraliskt imperativ” anspelar på Immanuel Kants kategoriska imperativ: ett moraliskt påbud som inte är beroende av syfte eller konsekvens, utan som gäller ovillkorligt. I denna kontext innebär det att offeridentiteten blir något absolut – inte ett val, utan en skyldighet.}
%\addcontentsline{toc}{subsubsection}{Den exklusiva offerrollen som moraliskt imperativ}

Det är särskilt den andra pelaren i Levys modell – föreställningen om den exklusiva offerrollen – som förtjänar särskilt fokus. Vad Levy säger, och vad Segal bekräftar från ett akademiskt perspektiv, är detta: Israel utgör ett historiskt unikum – aldrig tidigare har en ockupationsmakt lyckats etablera och exportera en självbild där den uppfattas som det enda moraliskt relevanta offret. Inte bara i den aktuella konflikten, utan i världshistorien.

Detta är inte ett retoriskt verktyg utan en psykologisk struktur, förankrad i en teologisk och historisk självförståelse där Förintelsen inte bara var det värsta brottet i mänsklighetens historia – utan det enda verkligt oförskyllt begångna. Det betraktas som utfört av yttervärlden mot 'Guds egendomsfolk' – och ses, enligt detta raster, som utan jämförelse. Därför, menar Levy, erkänns inget annat folks lidande fullt ut – och framför allt inte palestiniernas.

I detta raster är offerstatusen inte ett tillstånd, utan en moralisk rättighet. Den rättigheten måste försvaras, även med våld. Det är inte trots barnadödande som man förblir offer – utan genom det. För i den psykologiska logiken är det inte handlingen som definierar moralen, utan identiteten. Vem du är avgör om du har rätt att döda.

Detta får långtgående konsekvenser: varje jämförelse reduceras till antisemitism, varje palestinsk överlevare blir ett kognitivt hot, och varje försök att applicera folkrätt blir en kränkning. Vad vi står inför är därför inte bara en etisk kris – utan ett ideologiskt vakuum där all moralisk prioritet tilldelas en enda grupp. I en sådan struktur finns ingen fred att förhandla om – bara kapitulation.

\medskip
\paragraph*{Rättvisa som motstånd – ledarskapets ansvar}

Men att erkänna detta ideologiska vakuum innebär inte att ge upp på idén om rättvisa. Tvärtom. Det är inte ett uttryck för humanism att undvika konflikt eller att stryka en part medhårs. Verklig anständighet kräver vuxet ledarskap – ett ledarskap som inte väjer för obekväma sanningar, utan som orubbligt försvarar principen om lika värde och rätt för alla.

Att insistera på rättsstatens principer, även när det smärtar, är inte ett angrepp på judisk identitet – det är ett skydd för mänskligheten. Det är först när rättvisa blir universell som trygghet kan bli möjlig – även för dem som i dag identifierar sig genom den judiska erfarenheten. Det är just därför som folkmordskonventionen antogs – inte för att skydda en moralisk särställning, utan för att förhindra framtida brott, oavsett gärningsman.



 % Beviskedja för att folkmordsavsikt föreligger hos Israel – med fokus på intention

\newpage
\section{Om Hamas och den 7 oktober}


%filnmanmn. Hamas_overgripande.tex

\subsection{Hamas manifest 2017 – acceptans av en lösning i enlighet med folkrätten}

I sin politiska deklaration från maj 2017, \textit{A Document of General Principles and Policies}, uttrycker Hamas en fortsatt ståndpunkt att hela det historiska Palestina, från Jordanfloden till Medelhavet, utgör ockuperat territorium. Samtidigt slås det fast att:

\begin{itemize}
  \item Hamas inte erkänner staten Israel i formell mening (punkt 19–20),
  \item men att man \textbf{accepterar etablerandet av en fullständigt suverän palestinsk stat inom 1967 års gränser, med Östra Jerusalem som huvudstad}, som en formel för nationell konsensus.
\end{itemize}

\begin{quote}
"Without compromising its rejection of the Zionist entity and without relinquishing any Palestinian rights, Hamas considers the establishment of a fully sovereign and independent Palestinian state, with Jerusalem as its capital along the lines of the 4th of June 1967 [...] to be a formula of national consensus." (punkt 20)
\end{quote}

Detta ska inte förstås som ett undantag från folkrätten, utan som ett politiskt uttryck för folkrättslig efterlevnad – särskilt vad gäller FN:s säkerhetsrådsresolution 242 (1967), som fastslår principen om \textit{land mot fred}. Denna resolution förpliktar Israel att dra sig tillbaka från ockuperade områden och erkänna alla staters rätt till fredliga gränser – inklusive en palestinsk stat inom 1967 års gränser.

\vspace{0.5em}
\noindent
\textbf{Rättslig analys:}  
Den internationella rättsordningen kräver inte ett ömsesidigt erkännande för att etablera rättigheter enligt självbestämmanderätten. Hamas’ uttryckta acceptans av 1967 års gränser innebär därmed ett praktiskt ställningstagande i enlighet med folkrätten – medan Israels fortsatta vägran att erkänna en palestinsk stat utgör ett folkrättsbrott i den mening att det förnekar ett ockuperat folk dess självbestämmanderätt.





\subsection{Om erkännandekravet och karaktären av Israel som etnostat}

Hamas efterfrågar inte bilaterala förhandlingar med Israel, och erkänner varken staten Israel eller dess rätt att utöva kontroll över något territorium i historiska Palestina. Istället kräver Hamas att internationell rätt ska tillämpas fullt ut – i synnerhet FN:s säkerhetsrådsresolution 242, Genèvekonventionerna och principen om folkens rätt till självbestämmande. 

Israels vägran att erkänna denna ståndpunkt måste förstås i ljuset av landets återkommande krav på att bli erkänd som en \textit{judisk stat}. Detta krav innebär inte enbart ett erkännande av Israels existens som suverän stat, utan ett erkännande av dess \textbf{etniska karaktär} – som en stat där politisk suveränitet och territoriell legitimitet är reserverad för en specifik folkgrupp.

Samtidigt har Israel själv konsekvent vägrat att erkänna en palestinsk stat inom 1967 års gränser, trots att dessa gränser utgör utgångspunkt i FN:s resolutioner. I israelisk doktrin betraktas gränserna som preliminära och förhandlingsbara – ett synsätt som undergräver hela FN-systemets legitimitet. Medan Israel kräver erkännande av sin ideologiska självdefinition, förvägrar det palestinierna deras folkrättsligt erkända rätt till självbestämmande och statssuveränitet.


\subsubsection*{Definition av etnostat i folkrättsligt sammanhang}

Med en \textit{etnostat} avses i detta sammanhang en statsbildning där grundlag och författningsstruktur konstituerar politisk suveränitet, kollektiv identitet och rättighetstilldelning med utgångspunkt i etnisk tillhörighet. Detta skiljer sig principiellt från stater med en dominerande statsreligion, där medborgerlig likabehandling – åtminstone formellt – upprätthålls.

\begin{itemize}
  \item Hamas har konsekvent gjort åtskillnad mellan ett \textit{de facto}-accepterande av en tvåstatslösning baserad på FN:s säkerhetsrådsresolution 242, och ett \textit{de jure} erkännande av Israel som etnisk stat. 
  \item Ett formellt erkännande av Israel som ”judisk stat” skulle enligt Hamas och andra aktörer innebära att man accepterar en rättsordning där andra folkgrupper per definition är sekundära.
\end{itemize}

Det är samtidigt av vikt att understryka att Israel själv aldrig har erkänt en palestinsk stat inom 1967 års gränser, trots omfattande internationellt stöd för detta. Israel motsätter sig även fullt medlemskap för Palestina i FN, vilket ytterligare förstärker asymmetrin i erkännandeprocessen.

\subsubsection*{Konstitutionell lagstiftning och diskriminerande struktur}

År 2018 antogs den israeliska grundlagen \textbf{Basic Law: Israel as the Nation-State of the Jewish People}, vilken har betydande konsekvenser i rättslig mening:

\begin{itemize}
  \item Lagen slår fast att endast det judiska folket har nationell självbestämmanderätt i Israel, vilket utesluter andra befolkningsgrupper – inklusive över 20\% palestinska medborgare – från kollektiv suveränitet.
  \item Arabiska språket, tidigare ett av två officiella språk, nedgraderas till en ”speciell status”.
  \item Staten åläggs att främja judisk bosättning som ett nationellt intresse, utan motsvarande skyldigheter gentemot andra grupper.
\end{itemize}

Kritik mot lagen har framförts av såväl israeliska som internationella rättsorganisationer, däribland \textit{Adalah}, \textit{B’Tselem} och \textit{Human Rights Watch}, vilka samstämmigt pekat ut lagstiftningen som systematiskt diskriminerande. Den utgör enligt dessa bedömningar ett konstitutionellt uttryck för apartheid enligt internationell rätt, särskilt i ljuset av FN:s apartheiddefinition (International Convention on the Suppression and Punishment of the Crime of Apartheid, 1973).

\subsubsection*{Konsekvens i rättsligt perspektiv}

Symbolisk representation – såsom en arabisk domare i Högsta domstolen – förändrar inte den strukturella exkluderingen i den rättsordning som grundlagen etablerar. Sådana exempel representerar snarare ett minoritetsundantag inom en överordnad juridisk struktur som bekräftar etnisk prioritet. Folkrätten kräver icke-diskriminering, jämlik representation och rätt till självbestämmande för alla folk – villkor som i dagsläget inte uppfylls inom den israeliska rättsordningen.



\subsection{Hamas och atrocitypropaganda – vad har motbevisats av FN, rättsmedicin och åklagare?}

De påstådda massvåldtäkterna inte kunnat styrkas av vare sig israeliska åklagare, FN:s kommissioner eller oberoende rättsorgan. Den enda FN-rapport som hittills behandlat frågan – ett 17-sidigt dokument från Pramila Patten, generalsekreterarens särskilda sändebud för sexuellt våld i konflikter – klargör att uppdraget varken var \enquote{utredande till sin natur} eller byggde på direkt bevisning. Rapporten vilar nästan uteslutande på material tillhandahållet av israeliska myndigheter och medger uttryckligen att inga konkreta digitala eller forensiska bevis på våldtäkt kunde identifieras bland de över 5 000 granskade bilderna och 50 timmarna av videomaterial. Trots detta drar rapporten rättsligt laddade slutsatser om \enquote{rimlig grund att anta} att våldtäkt förekommit – utan att specificera hur många fall som avses. Professor Norman Finkelstein har i en detaljerad analys visat hur rapporten, trots sin icke-utredande karaktär och brist på substantiell bevisning, ändå bidragit till att legitimera omfattande demonisering av det palestinska motståndet.\footnote{Finkelstein, N. (2024). \textit{Pramila Patten’s Rape Fantasies: A Critical Analysis of the UN Report on Sexual Violence during the 7 October Attack}. In Gaza, 11 mars 2024.}

I en intervju med \textit{Yedioth Ahronoth} i januari 2025 medgav den israeliska chefsåklagaren Moran Gez att det – 15 månader efter händelserna – fortfarande inte finns en enda målsägande i något av fallen.\footnote{\url{https://electronicintifada.net/content/israel-still-cant-find-any-7-october-rape-victims-prosecutor-admits/39601}} FN har i två separata rapporter bekräftat att man, trots tillgång till omfattande bild- och videomaterial, inte funnit några konkreta tecken på våldtäkt eller sexuella övergrepp.\footnote{UN Human Rights Council, A/HRC/55/73 och A/HRC/55/75 (mars 2024).}
Detta innebär att ingen person har trätt fram som målsägande och därmed heller inte gjort anspråk på brottsofferersättning – något som är ytterst ovanligt vid grova våldsbrott om dessa faktiskt har ägt rum.

Samtidigt är det väldokumenterat att israeliska styrkor regelbundet utsätter palestinska fångar – inklusive kvinnor – för sexuellt våld, förnedring och tortyr, bland annat i det ökända Sde Teiman-lägret. Den israeliska journalisten David Sheen har redogjort för dessa övergrepp i flera rapporter, däribland vittnesmål om gruppvåldtäkter på palestinska män.\footnote{\url{https://mondoweiss.net/2024/02/new-reports-confirm-months-of-israeli-torture-abuse-and-sexual-violence-against-palestinian-prisoners/}}

Mot denna bakgrund framstår den svenska regeringens ställningstagande som särskilt anmärkningsvärt. Istället för att agera med försiktighet och efterfråga oberoende utredningar valde regeringen att kraftfullt bekräfta obestyrkta påståenden och fördöma palestinskt motstånd – samtidigt som man förteg Israels redan då väldokumenterade övergrepp.

Det är inte uteslutet att dessa atrocity-berättelser utnyttjats strategiskt – för att motivera fortsatt stöd till Israel, neutralisera kritik och ge opinionen en förevändning att bortse från folkrättsbrott. I detta klimat kunde den svenska regeringen den 27 oktober 2023 – alltså dagen för Israels markinvasion – underteckna ett militärt samarbetsavtal med det israeliska försvarsföretaget Elbit Systems.

Genom att i detta läge köpa in krigsmateriel från Elbit, med svenska skattemedel, har regeringen aktivt medverkat till att legitimera folkrättsbrott och ge ekonomiskt stöd till en ockupationsmakt under pågående folkmord.

\subsection*{Vem hittade på det? Militära talespersoner, ZAKA, United Hatzalah och David Ben Zion}
\addcontentsline{toc}{subsection}{Vem hittade på det? ZAKA, United Hatzalah och David Ben Zion}
\subsubsection*{Upphovsmannen till påståendet om halshuggna spädbarn}
\addcontentsline{toc}{subsubsection}{Upphovsmannen till påståendet om halshuggna spädbarn}

Ett av de mest spridda påståendena efter 7 oktober var att Hamas “halshögg 40 spädbarn”. Detta återgavs av bland andra USA:s president Joe Biden, Israels premiärminister Benjamin Netanyahu och den israeliska utrikesministern. Påståendet visade sig dock sakna bevis och har senare dementerats av såväl israeliska som internationella medier.

The Grayzone kunde identifiera den ursprungliga källan till detta påstående: David Ben Zion, en fanatisk bosättarledare och biträdande befälhavare i Israels armé. I en intervju den 10 oktober med den israeliska tv-kanalen i24 sade han: \textit{“De skar huvuden av barn. De skar huvuden av kvinnor.”} Ben Zion beskrev palestinierna som “djur” utan hjärta och publicerade timmar senare en video på sig själv där han log brett i byn Kfar Aza – platsen för den påstådda massakern. 

Ben Zion har tidigare uppmanat till att “utplåna” palestinska byar såsom Huwara, uttryckt stöd för deportation av palestinier och spridit rasistiska uttalanden om deras “barbariska DNA”. Han är kopplad till den apokalyptiska Tempelrörelsen som vill riva al-Aqsa-moskén och bygga ett tredje tempel. Under tidigare militära operationer mot Gaza har han öppet uttryckt stöd för total förintelse av området.

Det faktum att denna individ – med dokumenterad extremism och rasideologisk agenda – utgör ursprungskällan till ett av de mest spridda påståendena om Hamas övergrepp, underminerar ytterligare trovärdigheten i Israels atrocitynarrativ. Ändå har svenska och västerländska medier inte nämnt detta faktum, trots att Blumenthals granskning varit tillgänglig sedan oktober 2023.\footnote{\url{https://thegrayzone.com/2023/10/11/beheaded-israeli-babies-settler-wipe-out-palestinian/}}

\subsubsection*{Upphovsmännen till de värsta atrocityhistorierna}
%\addcontentsline{toc}{subsubsection}{Upphovsmännen till de värsta atrocityhistorierna}

En av de mest genomgripande granskningarna av atrocitynarrativet efter 7 oktober har genomförts av journalisten Max Blumenthal i The Grayzone.\footnote{\url{https://thegrayzone.com/2023/12/06/scandal-israeli-october-7-fabrications/}} Artikeln visar hur den israeliska organisationen ZAKA – utan medicinsk legitimation – stått bakom flera centrala påståenden, bland annat om halshuggna spädbarn, gravida kvinnor med utskurna foster, och barn som bränts i ugnar. Samtliga berättelser saknar belägg i form av kroppar, dokumentation eller vittnesmål.  

De mest groteska historierna har inte bara spridits av ZAKA:s ledarfigurer Yossi Landau och Simcha Dizingoff, utan även vidareförmedlats av amerikanska och israeliska toppolitiker. Bland annat återgav president Joe Biden och utrikesminister Antony Blinken en berättelse om en familj med två barn (6 och 8 år gamla) som enligt ZAKA skulle ha blivit torterade och mördade inför ögonen på varandra. Ingen död med dessa åldrar är registrerad i det aktuella kibbutzet (Beeri), och inga kroppar har återfunnits i det tillstånd som beskrivs.

Samtidigt har rivaliserande organisationer som United Hatzalah hittat på ännu mer extrema påståenden, såsom att ett spädbarn bakats i en ugn. Dessa berättelser har visat sig vara falska.


\subsection{Vad har bekräftats av Israeliska medborgare?}
\addcontentsline{toc}{subsection}erade på gissningar, missuppfattningar eller i vissa fall – enligt israeliska medier – direkt ekonomiskt motiverad bluff för att driva in donationer från utländska sponsorer.

Avslöjandena visar hur en mycket liten grupp individer har lyckats styra det internationella narrativet, trots att bevisen saknas eller motsägs av dödsregister, vittnesmål, rättsmedicinska uppgifter och israeler på plats.

\subsubsection*{Falska vittnesmål som militärstrategi: fallet med de påstått hängda bebisarna}
\addcontentsline{toc}{subsection}{Falska vittnesmål som militärstrategi: fallet med de påstått hängda bebisarna}

Ett av de mest groteska och spridda propagandapåståendena efter den 7 oktober var att Hamas skulle ha “hängt bebisar på en tvättlina”. Den israeliske journalisten Ishay Cohen publicerade ett videoklipp där han intervjuar en IDF-officer som hävdar detta. Kort därefter raderade Cohen videon efter massiv kritik – men först efter att den fått hundratusentals visningar via kontot “Mossad Commentary”.

Cohen erkände senare att intervjun förmedlats av israeliska arméns talesperson, att talespersonens representant var närvarande under hela inspelningen, och att syftet var att främja Israels “hasbara”, det vill säga propagandainsatser mot omvärlden.\footnote{\url{https://www.uncaptured.media/p/israeli-journalist-retracts-babies} Dan Cohen, “Israeli Journalist Retracts ‘Babies Hung on a Clothesline’”, *Uncaptured Media*, 30 november 2023.}

Officeren i fråga var Yaron Buskila, en reservöverste som tidigare varit verksam i IDF:s informationsavdelning. Buskila uppgav olika versioner i olika medier: i en intervju med Epoch Times hävdade han att han hört historien från en rabbin – medan han i intervjun med Cohen påstod sig ha sett det själv. Den senare versionen förnekades sedermera av Cohen, som även medgav att hans publicering skedde utan kontroll av uppgifternas riktighet: 

\begin{quote}
    “Jag medger att jag inte trodde det var nödvändigt att kontrollera sanningshalten i en berättelse från en överstelöjtnant, operationschef i Gaza-divisionen, och dessutom ackompanjerad av en talesperson från IDF.”
\end{quote}

Den israeliska militären har aldrig dementerat påståendet offentligt, och vissa högerextrema konton fortsätter att sprida det. Det är ett exempel på ett återkommande mönster: makaber propaganda med obestyrkta eller fabricerade påståenden som sprids snabbt, får global spridning, och därefter – när de faller – lämnar ett bestående intryck hos mottagaren trots att de tillbakavisats.

Detta påstående ansluter sig till samma propagandaform som “40 halshuggna spädbarn” – ett påstående som förnekats av både IDF och Vita huset, men som ändå återges i västvärldens ledande nyhetsförmedling.\footnote{Se t.ex. Max Blumenthal och Alexander Rubinstein, “Source of dubious ‘beheaded babies’ claim is Israeli settler leader who incited riots to ‘wipe out’ Palestinian village”, *The Grayzone*, 11 oktober 2023.}

\textbf{Sammanfattningsvis:} De falska påståendena om “hängda” eller “halshuggna” spädbarn är inte slumpmässiga misstag – de är delar av ett mönster. I detta mönster spelar den israeliska arméns talesperson en aktiv roll som innehållsproducent, inte enbart som verifierare. Propagandan bygger på chock, avhumanisering och en vägran att erkänna att informationen kan vara manipulerad, även av egna myndigheter.


\subsection*{Vad har tillbakavisats av civila israeliska medborgare?}
\addcontentsline{toc}{subsection}{Vad har tillbakavisats av israeliska medborgare?}

\subsubsection*{Yasmin Porat: Vi dödades av våra egna}
\addcontentsline{toc}{subsubsection}{Yasmin Porat: Vi dödades av våra egna}
Ett centralt vittnesmål kommer från Yasmin Porat\footnote{\url{https://electronicintifada.net/content/israeli-forces-shot-their-own-civilians-kibbutz-survivor-says/38861}} Ali Abunimah \& David Sheen, “Israeli forces shot their own civilians, kibbutz survivor says”, *The Electronic Intifada*, 16 oktober 2023.
, en israelisk kvinna som överlevde attacken på Kibbutz Be’eri. I israelisk statsradio berättar hon att hon och andra civila hölls som gisslan av Hamas i flera timmar, men att dessa behandlade dem “mycket humant”, gav dem vatten, försökte lugna dem, och klargjorde att de inte ämnade döda dem utan ta dem till Gaza.

Porat beskriver hur israeliska säkerhetsstyrkor anlände efter flera timmar, öppnade eld “med tusentals kulor och två tankskal” och dödade “alla” – inklusive de israeliska civila som hölls som gisslan. Hon säger uttryckligen att “de dödade alla, även gisslan, i mycket, mycket tung korseld.”

Porat intervjuades i flera medier – inklusive Kan, Maariv och Channel 12 – men hennes mest explosiva uttalanden klipptes bort i senare versioner och fanns inte kvar i publicerad arkivversion. De tillgängliggjordes först efter påtryckningar. Hon säger: “Jag är arg på staten, jag är arg på armén. Kibbutzen var övergiven i tio timmar.” 

Detta stödjer teorin att delar av förlustsiffrorna den 7 oktober orsakades av israelisk eldgivning, vilket ytterligare underminerar bilden av en enbart ensidig massaker. Vittnesmålet pekar även mot att Hannibal-doktrinen tillämpats mot civila.

\subsubsection*{Erez Tidhar: “En IDF-helikopter sköt in i kibbutzen”}
\addcontentsline{toc}{subsubsection}{Erez Tidhar: “En IDF-helikopter sköt in i kibbutzen”}

Erez Tidhar, en israelisk militärveteran och frivillig i Eitam-enheten under evakueringsinsatserna den 7 oktober, har i en intervju med den israeliska public service-kanalen Kann berättat att han såg en israelisk Apachehelikopter skjuta rakt in i Kibbutz Be’eri:

\begin{quote}
“Varje minut kommer en missil ner över dig, varje minut. Och plötsligt ser du en missil från en helikopter som skjuter in i kibbutzen. Du säger till dig själv, ‘Jag fattar inte. En IDF-helikopter skjuter in i en israelisk kibbutz.’ Och sen ser du en stridsvagn köra genom gatorna i kibbutzen, den flankerar kanonen och avfyrar en granat in i ett hus. Sånt kan man inte begripa.”\footnote{\url{https://www.uncaptured.media/p/israeli-volunteer-apache-helicopter}}
\end{quote}

Uttalandet är det första dokumenterade ögonvittnesmålet om raketbeskjutning från israeliskt stridsflyg in i ett israeliskt samhälle. Tidigare har Ha’aretz rapporterat om att en helikopter “uppenbarligen också träffade festivaldeltagare”, och Yedioth Ahronoth har avslöjat att 28 militärhelikoptrar öppnade eld mot mål i Israel under de första fyra timmarna av attacken. Piloterna uppges ha haft “stora svårigheter att skilja mellan terrorister, soldater och civila”.

Israels militär erkänner nu att omfattande vänskapseld förekom, men vägrar utreda dessa händelser med hänvisning till att omfattningen är för komplex och att en utredning skulle vara “moraliskt olämplig”.

\subsubsection*{Danielle Aloni och andra frigivna gisslan: “Vi behandlades humant”}
\addcontentsline{toc}{subsubsection}{Danielle Aloni och andra frigivna gisslan: “Vi behandlades humant”}

Flera frigivna israeler som hållits som gisslan av Hamas har beskrivit en oväntat human behandling under sin fångenskap. Mest uppmärksammat är uttalandet från Danielle Aloni, som i ett brev till sina fångvaktare tackade för deras “onaturliga mänsklighet” och “vänlighet trots förlusterna ni själva lidit”.\footnote{\url{https://thegrayzone.com/2023/11/27/israeli-tank-orders-fire-kibbutz/}}

Alonis vittnesmål har inte fått bred medial spridning i Israel, och regeringen har enligt flera rapporter hindrat återvändande gisslan från att tala fritt med medier. Israelen Gadi Peretz har offentligt vittnat om hur han efter hemkomsten blivit instruerad att inte berätta detaljerat om sin tid i fångenskap, särskilt inte om eventuella humanitära aspekter. Liknande rapporter har framkommit från andra frigivna.

Detta mönster tyder på att staten aktivt försökt kontrollera narrativet och tysta de röster som motsäger bilden av Hamas som obetingat brutal och sadistisk. I vissa fall har detta inkluderat direkt censur eller psykologiska utvärderingar innan de frigivna tillåtits återförenas med sin familj eller intervjuas offentligt.

\begin{itemize}
\item Video:Agam Goldstein-Almog\footnote{\url{https://x.com/LasseKaragiann5/status/1930654442912723269}}
\item Audio: Yasmin Porat\footnote{\url{https://x.com/LasseKaragiann5/status/1930656123540939100}}
\item Video:Fru Bibas med barn\footnote{\url{https://x.com/LasseKaragiann5/status/1930648270105432510}}
\item Video:Israelisk media:\enquote{Gisslan behandlades väl}\footnote{\url{https://x.com/LasseKaragiann5/status/1930629355568337170}}
\item Video:Mia Shem hävdade först att hon behandlats väl, därefter att hon inte bahandlats väl, att hon vaknat upp från narkos efter reparation, men senare att hon opererats utan narkos av veterinär\footnote{\url{https://x.com/LasseKaragiann5/status/1930628167342952537}}
\item Video: Avital Aldajem\footnote{\url{https://x.com/LasseKaragiann5/status/1930625119585534222}}
\item En video på tillfångatagandet av den kvinnliga soldaten Naama Levy spreds med påståenden från propagandister om att smuts eller fläckar på hennes byxor utgjorde absolut bevis för att hon blivit våldtagen\footnote{\url{https://x.com/LasseKaragiann5/status/1931025103992594943}}. Detta visade sig vara ännu en falsk anklagelse bland många. I en senare video syns hur hon och hennes kollegor släpps fria, och hur de inför Röda Korset gensvarar på Gazabornas stöd med leenden och vinkningar.\footnote{\url{https://x.com/LasseKaragiann5/status/1931015171381629296}}

\item Video: Gisslan i gott skick – inga tecken på misshandel eller rädsla.\footnote{\url{https://x.com/LasseKaragiann5/status/1930639636600131836}}
\item Video: Flickan Mia Leimberg med valpen från videon ovan\footnote{\url{https://x.com/MiddleEastMnt/status/1732547191204626473}}
\item Video: Gisslan och Hamas soldater vinkar varann avsked.\footnote{\url{https://x.com/LasseKaragiann5/status/1930638815787376920}} \footnote{\url{https://x.com/LasseKaragiann5/status/1930639247473619452}}\footnote{\url{https://x.com/LasseKaragiann5/status/1930637597597978909}}

\item Video: Gisslan high-five med Hamas soldater vid hemfärd\footnote{\url{https://x.com/LasseKaragiann5/status/1930631411238666463}}
\item Video: Yocheved Lifshitz \footnote{\url{https://x.com/LasseKaragiann5/status/1930596740261917095}}
\item Video: Maya Regev \footnote{\url{https://x.com/LasseKaragiann5/status/1930582894772158744}}
\item Video: En kompilering av kvinnliga vittnesbörd\footnote{\url{https://x.com/LasseKaragiann5/status/1930597653861007793}}
\end{itemize}

\textbf{Slutsats:} Vittnesmål från överlevare och frigivna gisslan motsäger i flera avseenden den officiella israeliska berättelsen. Tvärtom beskriver flera en förhållandevis human behandling, vilket direkt underminerar centrala element i det atrocity-narrativ som etablerades omedelbart efter den 7 oktober. I en global mediemiljö där råvideor snabbt kan spridas, verifieras och analyseras i detalj, har Israels informationsstrategi inte bara förlorat sin genomslagskraft – den har i många fall resulterat i ett fullständigt sammanbrott i trovärdighet.

Före detta sympatisörer uttrycker nu öppet antisionistiska ståndpunkter, och det offentliga samtalet domineras i allt högre grad av satir, memes och virala uttryck som anklagar israeliska talespersoner för att medvetet ljuga. Deras systematiska demonisering av palestinier upplevs av många som alltför grov för att vara trovärdig, och beskrivs ofta i termer av "gaslighting" – uppenbart falska påståenden som inte bara undergräver avsändarens trovärdighet, utan upplevs som en direkt förolämpning mot mottagarens intelligens.



\subsection*{Vad har bekräftats av Israeliska medborgare?}
\addcontentsline{toc}{subsection}{Vad har bekräftats av Israeliska medborgare?}

\subsubsection*{Israels egna styrkor orsakade civila dödsfall den 7 oktober}
\addcontentsline{toc}{subsubsection}{Israels egna styrkor orsakade civila dödsfall den 7 oktober}
Flera israeliska källor – inklusive överstelöjtnanten Nof Erez i intervju med Haaretz\footnote{\url{https://electronicintifada.net/blogs/asa-winstanley/we-blew-israeli-houses-7-october-says-israeli-colonel} Asa Winstanley, “We blew up Israeli houses on 7 October, says Israeli colonel”, *The Electronic Intifada*, 5 december 2023.}
 – har bekräftat att israeliska attackhelikoptrar och drönare under morgonen den 7 oktober öppnade eld mot mål inne i israeliska bosättningar, trots att man visste att det kunde innebära dödligt våld mot egna civila. Syftet var att förhindra kidnappningar enligt den så kallade Hannibal-doktrinen, ett militärt protokoll som godtar att israeliska gisslan dödas för att undvika fångutväxling.

Erez beskriver i detalj hur arméns kommandostruktur kollapsade och hur lokala miliser gav direktiv till helikopterpiloter via mobiltelefon om att bomba specifika bostäder där misstänkta palestinier befann sig – även om israeliska civila hölls gisslan där. Detta bekräftas av både piloter och drönaroperatörer, samt i israeliska militära utredningar. Ynet rapporterade redan i oktober att israeliska helikoptrar “sköt mot allt längs gränsstängslet” och att 300 mål attackerades inom de första fyra timmarna, “flertalet på israeliskt territorium.”

Denna systematiska användning av dödligt våld inne i egna bostadsområden är en djupt komprometterande omständighet – särskilt som den bidrog till flera av de dödsfall som sedermera tillskrevs Hamas i internationell propaganda. Ändå har ingen internationell granskning genomförts, och varken FN, EU eller svenska regeringen har efterfrågat ansvar för dessa händelser.

\subsubsection*{Israeler beordrades skjuta mot egna civila}
\addcontentsline{toc}{subsubsection}{Israeler beordrades skjuta mot egna civila}

Fler vittnesmål har framkommit som visar att israeliska soldater den 7 oktober beordrades skjuta in i egna bostadsområden. En granskning av Max Blumenthal i \textit{The Grayzone}\footnote{\url{https://thegrayzone.com/2023/11/27/israeli-tank-orders-fire-kibbutz/} Max Blumenthal, “Israeli tank gunner reveals orders to fire indiscriminately into kibbutz”, \textit{The Grayzone}, 27 november 2023.} redovisar hur unga kvinnliga soldater i en israelisk stridsvagnsenhet vittnar om att de beordrades öppna eld mot hus i Kibbutz Holit – oavsett om civila fanns där eller inte. 

En av soldaterna, endast identifierad som “Karni”, beskriver hur en panikslagen kollega pekade ut ett hus och skrek “bara skjut”. Hon frågade: “Men är det civila där?” och fick svaret “jag vet inte – skjut bara.” Karni valde att avstå från att skjuta med kanonen, men öppnade istället eld med kulspruta mot huset.

Flera av de dödade israeliska civila i Holit kan ha fallit offer för denna eld, enligt artikeln. Liknande händelser ska ha inträffat i Kibbutz Be’eri, där en israelisk stridsvagn öppnade eld mot ett hus och dödade tolv civila, inklusive Liel Hetzroni – ett av de barn som senare användes i propaganda mot Hamas.

Artikeln beskriver också hur israeliska säkerhetstjänster försökt tysta frigivna gisslan som inte bekräftar den officiella berättelsen. Flera frigivna israeler, bland andra Danielle Aloni, vittnar om human behandling i Hamas fångenskap. Aloni skrev ett brev till sina fångvaktare och tackade för deras “onaturliga mänsklighet” och “vänlighet trots förlusterna ni själva lidit”. Hennes offentliga uttalanden har dock förhindrats.

Flera andra gisslan har beskrivit liknande behandling, och israeliska medier har avslöjat att återvändande fångar förbjudits tala fritt. Detta tyder på att regeringen aktivt kontrollerar narrativet, även när det sker på bekostnad av de frigivnas röster.

\subsubsection*{Frigivna gisslan berättar om övergrepp – vad vet vi egentligen?}
\addcontentsline{toc}{subsubsection}{Frigivna gisslan berättar om övergrepp – vad vet vi egentligen?}

En artikel från den israeliska nyhetssajten \textit{All Israel News} återger vittnesmål från nyligen frigivna israeliska gisslan som beskriver sexuella övergrepp utförda av palestinska gärningsmän.\footnote{\url{https://allisraelnews.com/freed-hostages-reveal-the-scope-of-atrocities-and-abuse-against-women-by-hamas-terrorists}} Artikeln saknar dock identifierbara målsägande, konkreta uppgifter om tid och plats, samt oberoende bekräftelse från rättsmedicinsk eller juridisk instans. Enligt uppgift bygger vittnesmålen på samtal med psykologer och socialarbetare, men ingen rättslig förundersökning åberopas och inga anmälningar omnämns.

De berättelser som återges – exempelvis om våldtäkter som ska ha filmats – är i linje med det atrocity-narrativ som etablerades tidigt av israeliska talespersoner. Det bör påpekas att flera liknande påståenden, inklusive från militära källor, sedermera visat sig sakna stöd eller ha motbevisats (se ovan). I detta fall redovisas varken omständigheter, ansvariga grupper eller om några målsägande trätt fram offentligt.

Detta innebär att även denna källa måste bedömas kritiskt. Anonyma utsagor via sekundärkällor uppfyller inte grundläggande rättssäkerhetskrav. Samtidigt utesluts inte att övergrepp kan ha förekommit – särskilt i ett kaotiskt händelseförlopp med flera väpnade fraktioner och även civila inblandade. 

Även om en majoritet av gisslan i intervjuer vittnat om respektfull behandling av sina fångvaktare, vore det sannolikt orealistiskt att utgå från att inga övergrepp förekommit. Det skulle strida mot normalfördelningens förväntade spridning i en situation med låg kontroll, stark stresspåverkan och väpnade aktörer från olika grupper med varierande disciplinära strukturer.


\subsection*{Vad har avslöjats av oberoende röster i sociala medier?}
\addcontentsline{toc}{subsection}{Vad har avslöjats av oberoende röster i sociala medier?}

\subsection*{Propaganda som militärstrategi – atrocitynarrativ i krigföring}
\addcontentsline{toc}{subsection}{Propaganda som militärstrategi – atrocitynarrativ i krigföring}

Mänskliga sköldar, läkare är Hamas, skjuter innefårn sjukhuset, kommandocentraler inne i sjukhuset.





%Ny section definieras inne i filen:
%\section{Rättslig bedömning av Hamas status}



% Pröva om Hamas juridiskt kan klassificeras som terroristorganisation givet folkrättsbrottens art från Israel.
% Använd historisk analogi (ex: Nat Turner, slavuppror).
% Argumentera att terrorbegreppet förlorar legitimitet om det tillämpas selektivt.

\section{Rättslig bedömning av Hamas status}
Regeringen proklamerar:
\textit{The terrorist organisation Hamas bears heavy responsibility for the current situation. }\\

För resonemangets skull – och enbart som hypotetisk premiss – antar vi att samtliga israeliska uppgifter om övergrepp från Hamas är korrekta, trots att flera av dem senare visat sig vara obestyrkta eller direkt falska.
Låt oss dessutom anta att Hamas inte är en folkvald regering och dessutom inte är en folkföränkrad befrielserörelse.


Då regeringen inte uppfyller sina förpliktelser enligt ingångna folkrättsliga avtal – såsom att verka för respekt av FN:s resolutioner och folkrättens regler – utan istället tillerkänner Israel praktisk immunitet trots fortlöpande brott mot internationell rätt, uppstår ett ansvar enligt principerna om staters medverkan till folkrättsbrott.

Detta ansvar förstärks när Sverige inte enbart underlåter att ingripa, utan även aktivt karaktäriserar det motstånd som utövas av den ockuperade parten som ``terrorism'' – utan att samtidigt erkänna att rätten till beskydd enligt internationell humanitär rätt nekats den befolkningen.

I en sådan situation sker inte bara ett brott mot neutralitetsprincipen och ett passivt medansvar – utan även en retorisk och diplomatisk handling som riskerar att inflammera konflikten ytterligare. Denna eskalation medför ett förstärkt medansvar även för de illdåd som följer, om det kan visas att staten bidragit till att skapa förtvivlan eller rättslöshet hos den utsatta parten.

\lagrum{Artikel 16, ILC Articles on State Responsibility\quad 
A State which aids or assists another State in the commission of an internationally wrongful act by the latter is internationally responsible\footnote{\url{https://legal.un.org/ilc/texts/instruments/english/draft_articles/9_6_2001.pdf}}...}

\lagrum{Artikel 1, Genèvekonventionerna\quad 
The High Contracting Parties undertake to respect and to ensure respect for the present Convention in all circumstances\footnote{\url{https://ihl-databases.icrc.org/en/ihl-treaties/gciv-1949/article-1}}.}


Det är denna konstruktion som förklarar varför en betydande del av det internationella samfundet inte erkänner Hamas som en terroristorganisation, utan istället betraktar dem som en aktör i en asymmetrisk konflikt där de rättsliga definitionerna måste ses i ljuset av förvägrad rättsordning och ockupation.\footnote{\url{https://en.wikipedia.org/wiki/Hamas\#/media/File:International_views_on_Hamas.svg}}

Ur ett realpolitiskt perspektiv kan en regering i extrema undantagsfall avvika från ingångna förpliktelser under hot om nationell undergång. Men detta får aldrig ske i det tysta eller som medvetet hållen, långvarig politik. Principen om \textit{pacta sunt servanda} är inte förhandlingsbar. Att aktivt bryta mot egna åtaganden och samtidigt bidra till en situation där våldet förvärras, kan aldrig rättfärdigas inom ramen för en rättsstat.

\begin{tcolorbox}[title=Rättsfigurer som aktualiseras vid terroristutpekning, colback=gray!10, colframe=black, sharp corners=south]
\textbf{Följande rättsliga principer och figurer aktualiseras i samband med att Sverige benämner Hamas som terroristorganisation, utan att samtidigt erkänna Israels rättsbrott:}

\begin{itemize}
    \item \textbf{Statsansvar vid medverkan} – enligt \textit{ILC Articles on State Responsibility} (särskilt art. 16 och 41)\footnote{\url{https://legal.un.org/ilc/texts/instruments/english/draft_articles/9_6_2001.pdf}}, riskerar Sverige att bli folkrättsligt medansvarigt om det bistår eller möjliggör folkrättsbrott genom partiskhet.
    
    \item \textbf{Neutralitetsplikt} – enligt sedvanerätt och internationell humanitär rätt\footnote{\url{https://ihl-databases.icrc.org/en}}, får en icke-stridande stat inte selektivt stödja en konfliktpart som själv bryter mot Genèvekonventionerna\footnote{\url{https://ihl-databases.icrc.org/en/ihl-treaties/gciv-1949}}.
    
    \item \textbf{Förnekande av skydd} – att neka en befolkning, eller dess representanter, skydd under IHL utgör rättsstridigt undandragande av erkända rättigheter och kan betraktas som kollektiv bestraffning.
    
    \item \textbf{Culpa in contrahendo} – staten agerar i strid med lojalitetsplikten när den undertecknar avtal om humanitär rätt, men medvetet vägrar tillämpa dess principer konsekvent.
    
    \item \textbf{Pacta sunt servanda} – folkrättens kärnprincip: ingångna avtal ska hållas enligt Wienkonventionen om traktaträtten, artikel 26\footnote{\url{https://legal.un.org/ilc/texts/instruments/english/conventions/1_1_1969.pdf}}. Retorik och politik som strider mot denna princip försvagar hela det traktatbaserade folkrättssystemet.
    
    \item \textbf{Förstärkt medansvar} – genom att retoriskt utpeka en ockuperad befolkning som "terrorister", utan erkännande av rättslöshet och förtryck, bidrar staten aktivt till konfliktens eskalation och bär därmed del i det moraliska och folkrättsliga ansvaret.
\end{itemize}
\end{tcolorbox}

Det är i denna kontext som även språkbruket i sig får rättsverkan.

När en stat benämner en aktör i en ockuperad befolkning som “terroristorganisation”, utan att samtidigt erkänna det rättsliga sammanhanget – ockupationen, rättslösheten och vägran att ge skydd enligt IHL – så sker ett folkrättsligt konkludent handlande.

Formuleringen blir inte neutral, utan ger det sken av att det endast finns en skyldig part. Detta utgör i sig en tyst sanktionering av ockupationen och bidrar till att normalisera den rättsvidriga ordningen.

Att därefter tillfoga “men Israel måste följa internationell rätt” saknar betydelse om man redan i språkbruket legitimerat själva asymmetrin. Retoriken skapar en ny rättslig utgångspunkt: inte om förtrycket är lagligt, utan hur hårt den förtryckta får slå tillbaka.

Detta är vad som i folkrätten betraktas som ett konkludent godkännande – en handling med juridisk effekt, trots att den sker i form av språk och diplomatiskt ställningstagande.

Ytterligare en aspekt aktualiseras här – nämligen den folkrättsliga motsvarigheten till \textit{stämpling till brott}.

Att benämna ett motstånd inom en ockuperad befolkning som "terrorism", utan att samtidigt tillerkänna befolkningen det skydd som Genèvekonventionerna förutsätter, är inte bara en retorisk gärning. Det riskerar att uppfattas som att staten – i samråd eller i lojal sympati – legitimerar ockupationsmaktens rättsbrott. Detta kan liknas vid en form av diplomatisk eller språklig \textit{anstiftan}.

På motsvarande sätt aktualiseras även principen om \textit{underlåtenhet att avslöja eller förhindra brott}, så som den uttrycks i 23 kap. 6 § BrB. En stat som är bunden av folkrätten men underlåter att protestera mot grova rättsbrott – trots kännedom – uppfyller de moraliska och juridiska kriterierna för passivt medverkansansvar, särskilt om staten samtidigt förfogar över diplomatisk eller institutionell makt att påverka utvecklingen.

I ett internationellt sammanhang motsvaras dessa handlingar av de former för ansvar som regleras i \textit{ILC Articles on State Responsibility}, särskilt artiklarna 16 (aid or assistance), 41 (duty not to recognize the situation as lawful) och 40 (responsibility for serious breaches of peremptory norms).

Sammantaget uppstår ett mönster: staten utövar ett aktivt språkbruk som konkludent legitimerar en rättsvidrig ordning, underlåter att ingripa trots skyldighet, och bidrar därmed – med eller utan uppsåt – till att möjliggöra fortsatt rättsbrott. Detta skapar vad som i straffrättsliga termer motsvarar ett samverkansansvar.


\subsection{Från Nat Turner till Gaza: Vem bär ansvaret för våldets födelse?}
%\addcontentsline{toc}{subsection}{Från Nat Turner till Gaza: Vem bär ansvaret för våldets födelse?}

En slav utan tillgång till lag, som gör uppror, är inte en terrorist.

Nat Turner\footnote{\url{https://www.normanfinkelstein.com/nat-turner-in-gaza/}}, slavupprorsledaren som 1831 dödade oskyldiga vita i den största slavrevolten i USA:s historia, betraktades som ”terrorist” av dem som försvarade slaveriet. Abolitionisterna erkände våldets fasor – men vägrade fördöma upprorsmännen. De riktade sin vrede mot det system som födde våldet. I dag erkänns Turner som hjälte.

Här finns paralleller. Palestina har aldrig erbjudits rättvisa. Att fördöma dess motstånd, utan att erkänna den rättslösa omgivning det fötts i, är moraliskt absurt och juridiskt ohållbart.

I rättslig mening vilar det tyngsta ansvaret inte på den som slår tillbaka – utan på den som vägrar ingripa. Det är de som åtagit sig att vara systemets väktare, men som låter rättsordningen falla, som bär huvudansvaret.

En människa utan tillgång till rättsligt skydd, vars vädjan om beskydd ignoreras av världens självutnämnda ”guardians”, kan inte utan vidare stämplas som terrorist. Hon är, i sin desperation, inte ett hot mot rättsstaten – utan dess spegelbild när skyddet uteblir.

Nat Turner dödade inte i brist på moral, utan i brist på erkännande. Han förvägrades sin människovärdighet av ett samhälle som inte såg honom som människa. Hans uppror utmålades som religiös fanatism – men historien visade att det var systemet som var vansinnigt, inte han.

På samma sätt föddes våldet i Gaza inte ur hat – utan ur ett konsekvent nekande av lagligt skydd. De unga män som i oktober 2023 trängde ut genom murarna var födda i ett rättslöst fängelse – utan flyktvägar, utan framtid. Deras död var väntad. Deras motstånd förbjudet.

När Sverige då inte bara vägrar att ingripa, utan dessutom benämner deras handlingar som ”terrorism”, är det inte rättvisa man försvarar – utan status quo. Det är inte rättsstat. Det är moraliskt sammanbrott.

I detta rättsläge upplöses de juridiska strukturerna – och nya, desperata logiker tar vid:

\begin{quote}
Om skydd enligt internationell humanitär rätt konsekvent nekas oss, återstår endast den arkaiska rättsformen: \textit{öga för öga, tand för tand} – som en sista metod för avskräckning. Då vilar de oskyldigas blod inte på oss som slog tillbaka, utan på de ansvariga för systemets sammanbrott.
\end{quote}

\begin{quote}
Om lagsystemets väktare – dess guardians – förvägrar oss människostatus, om vi behandlas som djur, som restposter utan rättslig existens – varför skulle vi då låtsas följa mänsklighetens kodex? När vår mänsklighet förnekas, återstår bara djungelns lag – och den som behandlas som ett djur kommer till slut att slå som ett djur. Ingen kan kräva att vi följer lagar som ingen upprätthåller för oss.
\end{quote}

Detta är inte ett försvar av dödande. Det är ett åtal mot den värld som tvingar fram det.




\subsection{Pedagogisk liknelse – när uppstår skuld hos systemets väktare?}
\%addcontentsline{toc}{subsection}{Pedagogisk liknelse – när uppstår skuld hos systemets väktare?}

\begin{quote}
En polis iakttar hur grupp A, med hänvisning till en gudagiven rätt, fördriver grupp B från ett land där B bott i generationer. Grupp A hävdar att B är illegitima inkräktare. Grupp B vädjar om beskydd och påstår sig utsättas för etnisk rensning.
\end{quote}

\textbf{1. Den första vädjan:}
\begin{quote}
En representant för grupp B ber polismannen om hjälp. Polisen lyssnar – men ingriper inte. Han säger att konflikten är ”komplex” och att det inte är hans uppgift att ta ställning.
\end{quote}

\textit{Här uppstår den första formen av ansvar: underlåtenhet att skydda den som söker rättsskydd. Att inte ingripa trots skyldighet är ett brott mot skyddsprincipen – en grundpelare i såväl humanitär rätt som polisärt uppdrag.}
\lagrum{Jfr artikel 1 i Genèvekonventionen\footnote{\url{https://ihl-databases.icrc.org/en/ihl-treaties/gciv-1949/article-1}} och artikel 41.1 i ILC Articles\footnote{\url{https://legal.un.org/ilc/texts/instruments/english/draft_articles/9_6_2001.pdf}}}

\textbf{2. Det aktiva ställningstagandet:}
\begin{quote}
Snart slutar polisen inte bara att ignorera grupp B:s situation, utan inleder dessutom avtal och samarbete med grupp A – som fortsatt fördriver och dödar medlemmar ur grupp B.
\end{quote}

\textit{Här förstärks ansvaret: från passivitet till aktivt medansvar. Det är att bidra till rättsbrottet – både i folkrättslig mening (ILC:s artikel 16) och enligt Genèvekonventionernas skyldighet att ”respektera och säkerställa” att konventionerna efterlevs.}
\lagrum{Se ILC Articles, artikel 16\footnote{\url{https://legal.un.org/ilc/texts/instruments/english/draft_articles/9_6_2001.pdf}} och sedvanerättsregel 139 i ICRC:s sammanställning\footnote{\url{https://ihl-databases.icrc.org/en/customary-ihl/v1/rule139}}}

\textbf{3. Den institutionaliserade tystnaden:}
\begin{quote}
Under åren fortsätter övergreppen. Grupp B förlorar mark, försörjning och tillgång till rättsmedel. Trots återkommande larm hänvisar polisen till ”neutralitet” – men säger inget, gör inget.
\end{quote}

\textit{Här blir tystnaden ett strukturellt svek. Väktaren som inte skyddar offret utan upprätthåller gärningsmannens straffrihet sviker hela rättsordningen. Det är inte neutralitet – det är medverkan genom likgiltighet.}
\lagrum{Se artikel 40 och 41.2 i ILC Articles\footnote{\url{https://legal.un.org/ilc/texts/instruments/english/draft_articles/9_6_2001.pdf}}}

\textbf{4. Det desperata svaret:}
\begin{quote}
Till sist slår vissa ur grupp B tillbaka. De angriper civila från grupp A – i ett försök att spegla det lidande de själva utsatts för och tvinga fram internationell uppmärksamhet.
\end{quote}

\textit{Detta är inte lagligt. Men det är begripligt. Det är desperation i frånvaro av rättvisa. Det är rättsstatens sammanbrott i realtid.}
\lagrum{Se proportionalitetsprincipen i IHL och artikel 1(4) i Tilläggsprotokoll I\footnote{\url{https://ihl-databases.icrc.org/en/ihl-treaties/api-1977/article-1}}}

\textbf{5. Den slutgiltiga domen:}
\begin{quote}
Då först reagerar polisen. Han fördömer våldet från grupp B, kallar dem ”terrorister” och kräver att de ställs inför rätta. Han nämner aldrig sin egen roll.
\end{quote}

\textit{Här uppstår dubbel skuld:}
\begin{itemize}
    \item \textbf{Skuld gentemot grupp B:} för att han förvägrade dem skydd, tillät deras fördrivning, och förnekade dem status som skyddsberättigade.
    \lagrum{Brott mot skyldigheten att förhindra folkrättsbrott enligt Genèvekonventionerna\footnote{\url{https://ihl-databases.icrc.org/en/ihl-treaties/gciv-1949/article-1}} och artikel 16 i ILC Articles\footnote{\url{https://legal.un.org/ilc/texts/instruments/english/draft_articles/9_6_2001.pdf}}}

    \item \textbf{Skuld gentemot grupp A:} eftersom hans vägran att upprätthålla lag och rätt bidrog till att våld föddes ur desperation – och riktades mot oskyldiga även i grupp A.
    \lagrum{Indirekt ansvar enligt artikel 16 och 41 i ILC Articles – medverkansansvar för följdskador\footnote{\url{https://legal.un.org/ilc/texts/instruments/english/draft_articles/9_6_2001.pdf}}}
\end{itemize}

\textbf{Detta är Sveriges roll.}\\
Inte som neutral observatör. Inte som fredsbevarare.\\
Utan som den polis som vägrade skydda offret, slöt avtal med förövaren – och till sist fördömde det motstånd han själv möjliggjorde.

\vspace{1em}
\noindent\textit{Den rättsliga analysen är därmed fullbordad. Vad följer är en etisk och retorisk kommentar – inte som ersättning, utan som illustration av den rättsliga nödvändigheten.}



\subsection{Om gisslan och administrativt förvar}
%\addcontentsline{toc}{subsection}{Om gisslan och administrativt förvar}

Regeringen:\textit{“The terrorist organisation Hamas bears heavy responsibility for the current situation. The hostages must be released – unconditionally and immediately.”}\\

Regeringen kräver att Hamas ovillkorligen friger gisslan – ett legitimt krav enligt humanitär rätt. Men varför riktas inga motsvarande krav mot Israel, som systematiskt och i strid med folkrätten berövar tusentals palestinier friheten utan vare sig åtal, rättegång eller fastställd tidsgräns?

\textit{Så kallat “administrativt förvar” innebär att en individ frihetsberövas enbart på grundval av säkerhetstjänstens påståenden – utan insyn, utan beviskrav, utan rättslig prövning, utan möjlighet till försvar. Det är ett rättsligt undantagstillstånd som kan förlängas i månader, år – i praktiken utan slut.}

Barn hämtas nattetid av beväpnade soldater. Föräldrar får inga besked. Minderåriga grips, isoleras och förhörs – ibland under tortyrliknande former. Tusentals palestinier har hållits frihetsberövade under dessa förhållanden – utan att någonsin delges misstanke om brott.

\textit{Den israeliska människorättsorganisationen B’Tselem har i åratal dokumenterat denna praxis, som bryter mot både fjärde Genèvekonventionen och FN:s konvention om medborgerliga och politiska rättigheter.}\footnote{\url{https://www.btselem.org/administrative_detention}}

Att regeringen väljer att offentligt fördöma Hamas gisslantagning – men tiger om Israels institutionella massinternering av civila, inklusive barn – utgör en folkrättslig medverkan genom:

\begin{itemize}
  \item \textbf{Konkludent handlande:} Staten Sverige erkänner indirekt det ena brottet (gisslantagning) som avvikelse från rättsordningen, men det andra (administrativt förvar) som normalitet. Det är ett rättsligt ställningstagande genom underlåtenhet, i strid med \lagrum{artikel 1 i fjärde Genèvekonventionen}\footnote{\url{https://ihl-databases.icrc.org/en/ihl-treaties/gciv-1949/article-1}}.

  \item \textbf{Underlåtenhet att förhindra eller avslöja brott:} En stat som har insyn och möjlighet att agera, men inte gör det, bryter mot skyldigheten att ”respektera och säkerställa” konventionens efterlevnad (\lagrum{GC IV art. 1}) och \lagrum{ICCPR artikel 2(1)}\footnote{\url{https://www.ohchr.org/en/instruments-mechanisms/instruments/international-covenant-civil-and-political-rights}}, särskilt i relation till \lagrum{ICCPR artiklarna 9 och 14}\footnote{\url{https://www.ohchr.org/en/instruments-mechanisms/instruments/international-covenant-civil-and-political-rights}}.

  \item \textbf{Stämplingsliknande ansvar:} Genom att den ockuperande maktens brott systematiskt ursäktas, tonas ner eller förtigs, och endast motståndaren fördöms, legitimeras brotten. Detta bryter mot \lagrum{artikel 16 i ILC:s utkast till statsansvar}\footnote{\url{https://legal.un.org/ilc/texts/instruments/english/draft_articles/9_6_2001.pdf}} samt folkrättens tvingande normer (\textit{jus cogens}), där förbudet mot godtyckligt frihetsberövande och tortyr är absoluta och icke-derogabla.
\end{itemize}

\textit{Det är inte brottsligt att kräva frigivning av gisslan.}\\
\textit{Men det är rättsvidrigt att endast kräva det från den ena parten.}\\
\textit{Att tyst legitimera den andres systematiska och olagliga massfångenskap – det är medverkan.}


\subsection{Slutsats om Hamas juridiska status}

Hamas utgör sedan 2006 den folkvalda regeringen i Gaza och åtnjuter, enligt flera internationella
källor, fortsatt bred folklig förankring bland palestinier. Denna faktiska kontroll och lokala legitimitet
innebär att Hamas – oavsett västerländska klassificeringar – i folkrättslig mening är en icke-statlig väpnad
aktör med territoriell kontroll och därmed också folkrättsliga rättigheter och skyldigheter.

Det är därför juridiskt problematiskt att entydigt tillskriva Hamas fullt ansvar för alla civila dödsoffer
i Israel den 7 oktober 2023. Hamas politiska ledarskap, genom Moussa Abu Marzouk, har förnekat att kvinnor,
barn och civila utgjort mål för attacken och hänvisat till militära order från Qassam-brigadernas ledare
Mohamed el-Deif om att civila skulle skonas.\footnote{\url{https://www.the-star.co.ke/counties/nyanza/2023-11-07-hamas-leader-denies-killing-of-civilians-in-israel}} Marzouk hävdar att endast soldater och
reservister attackerats, samt att det bevismaterial som lagts fram i västliga medier inte nödvändigtvis
kan tillskrivas Hamas direkt, då flera fraktioner varit verksamma i Gaza vid tillfället.

Det bör även framhållas att väpnade grupper såsom Palestinska Islamiska Jihad verkar parallellt med Hamas,
och att vissa hämndaktioner kan ha genomförts av odisciplinerade enskilda individer, lokala miliser eller kriminella
nätverk i kaosets inledningsskede. Till detta kommer att Israel sedan tidigare erkänt sin tillämpning av det
s.k. Hannibal-direktivet – en militär doktrin som tillåter beskjutning av områden där egna soldater tagits som
gisslan, även om det riskerar deras död – vilket i sig kan ha lett till israeliska civila dödsoffer.

Det är inte folkrättsligt rimligt att kräva av en belägrad och ockuperad civilbefolkning – instängd bakom en
olaglig mur enligt ICJ:s rådgivande yttrande 2004 – att förutse, kontrollera eller isolera alla fraktioner
som agerar i ett plötsligt maktvakuum efter att israeliska befästningar slagits ut. Ett sådant krav skulle
etablera en doktrin där endast väpnat motstånd med perfektion i intern disciplin kan erkännas, vilket i praktiken
utesluter samtliga befrielserörelser ur den juridiska rättighetssfären. Att hävda detta i fallet Gaza, under
årtionden av blockad och övergrepp, skulle vara att tillämpa ett mått av ansvar som aldrig riktas mot de stater
som utövar ursprungligt våld.


Flera internationellt profilerade Palestinaaktivister som tidigare varit kritiska mot Hamas har sedan
oktober 2023 reviderat sina uppfattningar. Den kanadensiske advokaten Dimitri Lascaris, som driver bloggen
\textit{Reason2Resist}, är ett exempel på detta skifte.\footnote{\url{https://www.youtube.com/@reason2resist}}


I den amerikanska kontexten är det i praktiken omöjligt att offentligt ifrågasätta terroristklassificeringen
av Hamas utan att riskera social eller professionell utstötning. Istället uttrycks en växande förståelse för
hur israeliskt våld och blockadpolitik framkallar väpnat motstånd – inte som rättfärdigande, men som förklaring.
Bland dem som formulerat denna linje finns inte bara analytiker som Scott Ritter, utan också inflytelserika
högerprofiler som Elon Musk, Tucker Carlson och Candace Owens, vilka samstämmigt betonat att det är Israels
agerande som “skapar terrorister”.

Denna hållning – att se våldsutbrottet som en konsekvens av långvarig förnedring, instängning och statligt våld –
innebär inte nödvändigtvis stöd för Hamas metoder, men utmanar narrativet om ensidig ondska. Det faktum att denna
tolkning nu förs fram av personer inom USA:s konservativa mainstream visar att Israels agerande inte längre bara
kritiseras från vänster, utan även från ideologiskt oväntade håll.

Den amerikanske militäre analytikern och tidigare FN-vapeninspektören Scott Ritter intar en mer
oförsonlig hållning än många andra västliga Palestina-sympatisörer. I sina analyser har han riktat
skarp kritik mot vad han uppfattar som moraliskt hyckleri: individer som säger sig stödja det
palestinska folkets rättigheter, men inte vågar uttala stöd för Hamas väpnade kamp.

Ritter betraktar Hamas som en legitim motståndsrörelse mot ockupation och kolonial dominans,
och menar att det är inkonsekvent att erkänna palestiniernas rätt till självförsvar men samtidigt
förkasta de medel som i praktiken står dem till buds. Enligt Ritter är det inte Israel som
behöver förstås, utan det västerländska behovet av att kategorisera palestinskt våld som "terrorism"
– ett begrepp han menar används strategiskt för att avväpna all motståndsrätt.
För Ritter är det inte nog att förstå Hamas – man måste erkänna rörelsens väpnade
motstånd som legitimt, annars är man enligt honom delaktig i att förneka palestiniernas rätt till kamp.

Den brittiske politikern George Galloway har i detta sammanhang påpekat att det faktum att vissa handlingar
inom en befrielserörelse utgör terror inte innebär att hela rörelsen saknar folkrättslig relevans. Samma
principer som tillämpades på ANC under apartheidregimen måste, i konsekvensens namn, även kunna tillämpas
på Hamas.

Som vi visat ovan, är det folkrättsligt oacceptabelt för en regering att utan granskning beteckna väpnade
motståndsgrupper som "terrorister" samtidigt som man själv undandrar Israel varje form av ansvarsutkrävande
för krigsförbrytelser. Att beteckna samtliga motståndshandlingar från en ockuperad befolkning som terrorism,
utan att erkänna den grundläggande asymmetrin i våldsanvändningen, utgör i sig ett brott mot neutralitetsprincipen
och urholkar den internationella rättsordningen.

\subsection*{Slutlig juridisk bedömning: statens ansvar för båda sidors brott}

Även om enskilda palestinier den 7 oktober begått illdåd mot civila, även om Hamas skulle visa sig skyldiga till övergrepp, förändras inte det centrala folkrättsliga ansvaret: den svenska regeringen – liksom andra västliga regeringar – har under flera decennier vägrat att påföra Israel några som helst rättsliga, ekonomiska eller diplomatiska konsekvenser för landets systematiska brott mot internationell rätt.

Genom att konsekvent undandra Israel ansvar, har man erbjudit en **politisk och moralisk immunitet** som gjort det möjligt för våldet att fortgå. Genom att samtidigt benämna palestinskt motstånd som ”terrorism”, har man fråntagit det ockuperade folket både sin rättsliga status och sina legitima uttrycksformer.

Detta är inte passivitet. Det är **konkludent medverkan** i en rättsordnings kollaps – där staten ger stöd åt den ena parten, vägrar utreda dess brott, och fördömer det motstånd som uppstår.

I rättslig mening innebär detta att den svenska staten bär ansvar **för döda israeler** – eftersom den möjliggjort ett system som provocerat fram våld – och **för döda palestinier**, eftersom den hjälpt till att upprätthålla den straffrihet som gjort det möjligt att fortsätta döda dem.

Detta ansvar är inte sekundärt. Det är inte indirekt. Det är **djupare** – eftersom staten vet vad som sker, men väljer att upprätthålla tystnaden. Det är därför staten – inte Hamas, inte ens Israel – måste hållas till svars för att ha satt sig själv utanför den rättsordning den påstår sig försvara.





 % Analys av regeringens fokus på Hamas och kritik mot snedvriden ansvarsfördelning







\section{Begärda åtgärder}
% Lista konkreta krav: diplomatiska sanktioner, avbryt handel, rättsutredning, KU-granskning.
% Visa att dessa krav ligger inom ramen för Sveriges skyldigheter – inte önskemål.

%\section*{Regeringen har inga befogenheter att formulera egna rättsprinciper}
%\addcontentsline{toc}{section}{Regeringen har inga befogenheter att formulera egna rättsprinciper}

Det är djupt oroande att Sveriges regering i sitt officiella språkbruk tycks föreställa sig att den kan ”balansera” folkrätt, militär ockupation och systematiskt våld med egna moraliska avvägningar – som om Utrikesdepartementet vore en gymnasieklass i etik.

Men Sveriges regering är inte en moralfilosofisk tankesmedja. Den har inga befogenheter att formulera egna rättsprinciper – än mindre att ersätta gällande internationell rätt med känslomässiga symmetrier.

Utrikespolitiken är inte ett tyckande. Den är – eller borde vara – bunden av:

\begin{itemize}
  \item \textbf{Regeringsformen (RF) 1 kap. 10 §}, där det uttryckligen anges att Sverige ska respektera sina internationella åtaganden,
  \item \textbf{FN-stadgan}, inklusive förbudet mot anfallskrig och kollektiv bestraffning,
  \item \textbf{Internationella domstolens prejudikat}, som fastslår att Israel är ockupationsmakt – inte ett offer för yttre aggression.
\end{itemize}

Att i detta rättsläge hävda att Israel har ”rätt till självförsvar” gentemot en befolkning det själv ockuperar är inte bara juridiskt ohållbart – det är ett kompetensöverskridande.

Internationell rätt är inte ”vår tolkning”. Det är inte ett perspektiv bland andra. Det är ett förpliktande ramverk. Om regeringen börjar ersätta rätt med moral, har vi i praktiken avskaffat rättsstaten.

\vspace{1em}
\textit{Makten må utgå från folket – men rättsstatens giltighet gör det inte.}

Även om folkopinionen formats av desinformation, psykologisk krigsföring och inverterad verklighetsbeskrivning, får regeringen inte ta sin legitimitet ur folkets affekt. Den får det ur konstitutionen – och ur Sveriges internationella rättsliga åtaganden.

Inte ens om regeringen, på formellt laglig väg, skulle välja att utträda ur folkmordskonventionen och andra traktat, kvarstår det rättsliga faktum: dessa normer utgör \textit{jus cogens} – tvingande internationell rätt – och kan inte avtalas bort.

Om regeringen försöker, innebär det inte att rättsordningen förändrats. Det innebär att rättsstaten har övergetts.

Och när så sker – vilket humanistiska och juridiska eliter världen över förstår – uppstår ett nytt ansvar: att stoppa rättens sammanbrott.

\vspace{1em}
Regeringen kommer då att kalla dessa försök till försvar för ”extremism” eller ”terrorism” – men det är förväntat. Så reagerar alltid den makt som tappat sin juridiska legitimitet.






\section*{Statens skyldigheter enligt folkrätten}
\addcontentsline{toc}{section}{Statens skyldigheter enligt folkrätten}
\subsection*{Folkmordskonventionen väger tyngre än lojalitet mot EU}
\addcontentsline{toc}{subsection}{Folkmordskonventionen väger tyngre än lojalitet mot EU}

Att Sveriges regering hänvisar till framtida domstolsprövningar som ursäkt för sin passivitet är inte en neutral hållning – det är ett moraliskt svek.

Enligt artikel I i FN:s konvention om förebyggande och bestraffning av brottet folkmord (1948) är varje fördragsslutande stat skyldig att \textbf{förhindra} och \textbf{bestraffa} folkmord – oavsett om någon internationell domstol ännu fällt en dom.\footnote{\url{https://www.ohchr.org/en/instruments-mechanisms/instruments/convention-prevention-and-punishment-crime-genocide}}

Regeringen låtsas att inte se helheten. Den påstår att åtgärder måste ske ”tillsammans med andra EU-länder och likasinnade”, som om Sveriges rättsstatliga skyldigheter vore beroende av konsensus inom en politisk union. Men Sveriges folkrättsliga åtaganden står över lojaliteten till EU:s utrikespolitiska linje. Folkmordskonventionen utgör bindande internationell rätt – inte ett politiskt vägval.

Om regeringen föreställer sig att dess väljarmandat – eller dess alliansstrategi – ger rätt att ignorera folkmordskonventionen, har den allvarligt missförstått grundlagen.  
Regeringsformens portalparagraf (1 kap. 2 §) slår fast att Sverige ska respektera folkrätten. Detta gäller särskilt skyldigheten att agera preventivt vid risk för folkmord.

\lagrum{1 kap. 2 § 1 st. RF\quad Den offentliga makten ska utövas med respekt för alla människors lika värde och för den enskilda människans frihet och värdighet.}

Skyldigheten att respektera mänskliga rättigheter har även en grund i Europakonventionen, som har ställning som svensk lag.

\lagrum{2 kap. 19 § RF\quad Lag eller annan föreskrift får inte meddelas i strid med Sveriges åtaganden på grund av den europeiska konventionen angående skydd för de mänskliga rättigheterna och de grundläggande friheterna.}

\lagrum{1 § Lag (1994:1219)\quad Som lag här i landet ska gälla den europeiska konventionen den 4 november 1950 angående skydd för de mänskliga rättigheterna och de grundläggande friheterna – med de ändringar som gjorts genom ändringsprotokollen nr 11, 14 och 15 till konventionen, och med de tillägg som gjorts genom tilläggsprotokollen nr 1, 4, 6, 7, 13 och 16 till konventionen.}
Detta gäller särskilt skyldigheten att agera preventivt vid risk för folkmord.



\subsection*{Om folkrättens selektiva tillämpning och kritisk rättsteori}
\addcontentsline{toc}{subsection}{Om folkrättens selektiva tillämpning och kritisk rättsteori}

Folkrättens auktoritet vilar på dess anspråk på universell och likvärdig tillämpning. I teorin är dess normer opartiska. I praktiken är tillämpningen ofta asymmetrisk.

Modern forskning inom kritisk rättsteori har visat hur internationell rätt tenderar att tillämpas olika beroende på staters maktposition, allianser och – inte sällan – geopolitisk eller rasifierad identitet.  

Detta har beskrivits som en form av “rasialiserad folkrättstillämpning”\footnote{Se t.ex. Antony Anghie, \textit{Imperialism, Sovereignty and the Making of International Law}, Cambridge University Press, 2004.}, där vissa staters rätt till självförsvar eller territoriell kontroll aldrig ifrågasätts, medan andra stater och folk kontinuerligt misstänkliggörs. Koloniala mönster reproduceras i nutida diplomati och rättstillämpning – inte genom explicita hierarkier, utan genom selektiv tolerans mot folkrättsbrott beroende på avsändare.

I detta ljus blir Sveriges selektiva agerande i fallet Israel-Palestina svårt att försvara.

Samma regering som varit pådrivande i sanktioner mot Rysslands aggression mot Ukraina och i fördömanden av andra auktoritära stater, har valt att uttrycka sig med största försiktighet när det gäller en mångårig ockupationsmakt som är militärt och politiskt allierad med väst. 

Detta riskerar inte bara att skada Sveriges internationella anseende – det bidrar till att normalisera ett rättssystem där mänskliga rättigheter blir beroende av geostrategiska överväganden. En sådan ordning är oförenlig med den rättsstatliga principen om allas likhet inför lagen.



\vspace{1em}

Denna asymmetriska tillämpning är inte ett akademiskt problem – den får direkta konsekvenser när stater som Sverige avstår från att ingripa trots övertygande tecken på pågående folkrättsbrott.\footnote{Utöver Anghie har även forskare som Makau Mutua, Martti Koskenniemi och Sundhya Pahuja bidragit till förståelsen av hur internationell rätt används selektivt – ofta till förmån för globala maktcentra.}





\subsection*{Att vägra ingripa är inte neutralitet – det är medskyldighet}
\addcontentsline{toc}{subsection}{Att vägra ingripa är inte neutralitet – det är medskyldighet}

Det är därför inte möjligt att undandra sig konventionsansvaret genom att hänvisa till att ”dessa budskap framförs i våra egna kontakter med Israel och vi gör det tillsammans med andra EU-länder och likasinnade”.  
Det vore som om en polis, trots att han bevittnar ett pågående övergrepp, väljer att inte ingripa – med ursäkten att det vore ”opassande” att stöta sig med gärningsmannen, eftersom de tillhör samma yrkeskår.

På motsvarande sätt väljer Sveriges regering att inte använda sina statliga maktmedel – såsom ekonomiska sanktioner, vapenembargo eller diplomatiska markeringar – trots folkrättsbrott av högsta allvarlighetsgrad.

Detta är inte återhållsamhet. Det är \textbf{medskyldighet}.

Enligt artikel I i folkmordskonventionen\footnote{\url{https://www.ohchr.org/en/instruments-mechanisms/instruments/convention-prevention-and-punishment-crime-genocide}} är Sverige folkrättsligt skyldigt att inte bara bestraffa, utan även \textit{förhindra} folkmord – oavsett om någon dom ännu har fallit. Detta gäller i synnerhet när avsikten (dolus specialis) är öppet uttalad och inte hypotetisk.

Att regeringen trots detta avstår från handling får allvarliga konsekvenser:

\begin{itemize}
  \item Det underminerar den internationella rättsordningen och respekten för bindande konventioner.
  \item Det smutsar ner Sveriges rykte som humanitär rättsstat – särskilt i ögonen på det Globala Syd.
  \item Det skadar medborgarnas tilltro till demokratins möjligheter och till samhällets moraliska kompass.
  \item Det göder konspirationsteorier om internationell maktutövning – till exempel föreställningar om inflytande från mäktiga judiska nätverk.
  \item Det skapar en grogrund för radikalisering, polarisering och social fragmentering.
\end{itemize}

Om denna regering vill försvaga demokratin, urholka samhällsgemenskapen och dra Sveriges internationella anseende i smutsen – då är den inslagna vägen utan tvekan den rätta.






\subsection*{Från moralisk förebild till passiv möjliggörare – ekonomiska implikationer}
\addcontentsline{toc}{subsection}{Från moralisk förebild till passiv möjliggörare – ekonomiska implikationer}

Sverige vann under 1970- och 80-talen stor respekt i det Globala Syd genom sin kompromisslösa hållning mot apartheidregimen i Sydafrika. Genom att tydligt stödja ANC, kräva sanktioner och motsätta sig USA:s och Storbritanniens eftergiftspolitik, sågs Sverige som en röst för rättvisa, jämlikhet och folkrätt.

Detta moraliska kapital öppnade dörrar till affärsmöjligheter, biståndssamarbeten och diplomatiskt inflytande. Den svenska hållningen uppfattades som trovärdig – därför att vi var beredda att betala ett politiskt och ekonomiskt pris för våra principer.

I dag är situationen den omvända.

Varför skulle stater i Afrika, Asien eller Latinamerika vilja handla med Sverige – om vi uppfattas som en \textit{folkmordets medlöpare}?  
Om vi varken vågar använda våra maktmedel – eller ens vår röst – när ett folk systematiskt utplånas med uttalad avsikt, vad är då kvar av vår trovärdighet?

Varför skulle någon i det Globala Syd vilja samarbeta med ett europeiskt land som:

\begin{itemize}
  \item svikit den internationella rättsordningen,
  \item underminerat FN-systemets auktoritet, och
  \item rullat tillbaka de normer som syftade till att bygga en \textit{civiliserad} värld?
\end{itemize}

När kvalitet och pris är likvärdigt – varför välja Sverige, när Kina eller Brasilien erbjuder samma villkor utan moralisk dubbelmoral?

\textit{"Ni har redan visat oss att principer inte betyder något. Varför skulle vi då gynna er?"}



\section*{Krav på åtgärder}
\addcontentsline{toc}{section}{Krav på åtgärder}
\subsection*{Sanktioner mot Israel måste spegla Sveriges praxis mot andra stater}
\addcontentsline{toc}{subsection}{Sanktioner mot Israel måste spegla Sveriges praxis mot andra stater}

Om Sveriges regering menar allvar med sitt åtagande att försvara mänskliga rättigheter, krävs mer än halvhjärtade markeringar mot enskilda israeliska ministrar. Det räcker inte att – inom ramen för EU-samarbetet – diskutera selektiva sanktioner, samtidigt som man protesterar mot Internationella brottmålsdomstolens arresteringsorder mot Israels premiärminister.

Att motverka folkmord kräver samma kompromisslösa hållning som Sverige tidigare intog mot apartheidregimen i Sydafrika – och som i dag riktas mot Rysslands aggression mot Ukraina.

Om regeringen vill signalera att palestinska liv har samma människovärde som ukrainska, måste följande åtgärder vidtas utan dröjsmål:

\begin{itemize}
  \item Inför fullskaliga sanktioner mot Israel – inklusive vapenembargo, import- och exportförbud samt frysning av tillgångar för ansvariga befattningshavare.
  \item Avbryt statliga besök, samarbetsavtal och militära utbyten med Israel tills folkrätten respekteras.
  \item Förbjud israelisk statspropaganda i Sverige – inklusive sändningstillstånd, riktade kampanjer i sociala medier och pressutspel från IDF – på samma sätt som rysk propaganda via RT förbjöds.
  \item Stoppa legitimeringen av Israels narrativ genom krav på opartiskhet i public service. Sveriges Radio och andra statsunderstödda medier får inte okritiskt återge IDF:s kommunikéer som om de vore oberoende källor.
\end{itemize}

Regeringen kan inte hävda att den värnar rättsstat och mänskliga rättigheter om den samtidigt tillämpar dubbla måttstockar.

Det är inte neutralitet – det är hyckleri.\\
Det är inte återhållsamhet – det är moralisk feghet.\\
Det är inte diplomati – det är \textbf{tyst medverkan}.

Sverige ska inte dalta med en ockupationsmakt som kränker folkrätten – lika lite som vi gjorde det med apartheidregimen i Sydafrika eller Rysslands invasion av Ukraina.

\textbf{Att erkänna palestiniers människovärde kräver handling – inte retorik.}


Underlåtenhet att vidta dessa åtgärder riskerar att tolkas som konstitutionell underlåtenhet och som passiv medverkan i folkrättsbrott – med följd att regeringen kan komma att hållas ansvarig inför såväl riksdagens kontrollorgan som internationella rättsliga forum.

 % Förslag på rättsliga, politiska och diplomatiska åtgärder Sverige borde vidta

\include{sections/avslutning_section/avslutning_main} % Sammanfattning, slutsats, och begäran om rättslig prövning


\end{document}
